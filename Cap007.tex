% !TeX root = Harry.tex

\chapter{Reciprocità}
\label{capitolo:7}

\emph{«Tuo padre è quasi tanto fantastico quanto il mio.»}

~\\
~\\

Le labbra di Petunia Evans-Verres stavano tremando, e i suoi occhi riempiendosi di lacrime, mentre Harry l’abbracciava alla vita al Binario Nove della stazione di King’s Cross. «Sei sicuro di non volere che venga con te, Harry?»

Harry lanciò un’occhiata a suo padre Michael Verres-Evans, che sembrava lo stereotipo del severo-ma-fiero, e poi di nuovo a sua madre, che aveva davvero un aspetto piuttosto… non composto. «Mamma, so che il mondo dei maghi non ti piace molto. Non devi venire con me. Dico sul serio.»

Petunia fece una smorfia. «Harry, non dovresti preoccuparti per me, io sono tua madre e se hai bisogno di qualcuno con te –»

«Mamma, starò da solo a Hogwarts per \textit{mesi} e \textit{mesi}. Se non riesco a sopportare una piattaforma del treno da solo, meglio scoprirlo il più presto possibile in modo che possiamo annullare tutto.» Abbassò la voce a un sussurro. «E poi, mamma, tutti mi amano laggiù. Se avessi dei problemi, tutto quello che dovrò fare è togliermi la fascia», Harry toccò la fascetta ginnica che copriva la sua cicatrice, «e avrò \textit{molto} più aiuto di quanto io possa gestire.»

«Oh, Harry», sussurrò Petunia. Si inginocchiò e lo abbracciò forte, viso contro viso, le guance appoggiate l’una contro l’altra. Harry poté percepire il suo respiro irregolare, e poi sentì un singhiozzo soffocato sfuggirle. «Oh, Harry, ti voglio bene davvero, ricordalo sempre.»

\textit{È come se avesse paura di non vedermi mai più}, il pensiero nacque improvviso nella mente di Harry. Sapeva che il pensiero era vero, ma non sapeva perché la mamma avesse tanta paura.

Così tirò a indovinare. «Mamma, lo sai che non ho intenzione di trasformarmi in tua sorella solo perché sto imparando la magia, giusto? Farò qualsiasi magia mi chiederai — se posso, voglio dire — o se vorrai che io non usi nessuna magia a casa, farò anche questo, prometto che non lascerò mai che la magia si intrometta tra noi –»

Un forte abbraccio interruppe le sue parole. «Hai un buon cuore», sua madre gli sussurrò nell’orecchio. «Un cuore davvero buono, figlio mio.»

Harry rimase allora senza parole.

Sua madre lo lasciò e si alzò. Prese un fazzoletto dalla borsetta e con mano tremante tamponò il trucco che le colava intorno agli occhi.

Non c’erano dubbi sulla possibilità che suo padre lo accompagnasse dal lato magico della stazione di King’s Cross. Papà aveva problemi anche solo a guardare esplicitamente il baule di Harry. La magia scorreva nel sangue delle famiglie, e in Michael Verres-Evans non riusciva neppure a gocciolare.

Così invece il padre si schiarì appena la gola. «Buona fortuna per la scuola, Harry», disse. «Pensi che ti abbia comprato abbastanza libri?»

Harry aveva spiegato a suo padre di come avesse pensato che questa potesse essere la sua grande occasione per fare qualcosa di veramente rivoluzionario e importante, e il professor Verres-Evans aveva annuito e messo da parte la sua fitta agenda per due intere giornate allo scopo di dedicarsi al Più Grande Giro delle Librerie di Seconda Mano di Sempre, che aveva coperto quattro città e prodotto trenta casse di libri scientifici che ora giacevano nel livello sotterraneo del baule di Harry. La maggior parte dei libri erano andati via per una sterlina o due, ma alcuni di loro decisamente \textit{no}, come il più recente \textit{Manuale di Chimica e Fisica} o l’edizione completa del 1972 della \textit{Encyclopædia Britannica}. Suo padre aveva cercato di impedire a Harry di vedere gli schermi dei registratori di cassa, ma Harry aveva capito che doveva aver speso \textit{almeno} un migliaio di sterline. Harry aveva detto a suo padre che lo avrebbe ripagato non appena avesse capito come convertire l’oro dei maghi in denaro babbano, e suo padre gli aveva detto di andare a buttarsi in un lago.

E poi suo padre gli aveva chiesto: \textit{Pensi che ti abbia comprato abbastanza libri?} Era sufficientemente chiaro quale risposta Papà volesse sentire.

La voce di Harry era rauca, per qualche ragione. «Non è possibile avere abbastanza libri», disse recitando il motto della famiglia Verres, e suo padre si inginocchiò e gli diede un rapido e saldo abbraccio. «Ma \textit{certamente} ci hai provato», disse Harry, e si sentì soffocare di nuovo. «È stato davvero, \textit{davvero}, davvero un buon tentativo.»

Suo padre si raddrizzò. «Allora…», disse. «Lo vedi \textit{tu} il Binario Nove e Tre Quarti?»

La stazione di King’s Cross era enorme e trafficata, con pareti e pavimenti lastricati di ordinarie piastrelle macchiate di sporco. Era piena di persone ordinarie che si affrettavano nei loro affari ordinari, tenendo conversazioni ordinarie che causavano grandi quantità di rumore ordinario. La stazione di King’s Cross aveva un Binario Nove (su cui erano) e un Binario Dieci (proprio a fianco), ma non c’era niente tra il Binario Nove e il Binario Dieci, a parte un muro divisorio sottile e poco promettente. Un grande lucernario sulle loro teste lasciava entrare molta luce che illuminava la totale mancanza di qualunque Binario Nove e Tre Quarti.

Harry si guardò intorno fino a quando i suoi occhi non si inumidirono, pensando, \textit{forza, vista da mago, forza, vista da mago}, ma non apparve assolutamente nulla. Pensò di prendere la bacchetta e agitarla, ma la professoressa McGonagall lo aveva avvertito di non usarla. Inoltre, se ci fosse stata un’altra pioggia di scintille multicolori, l’avrebbero potuto arrestare per accensione di fuochi d’artificio all’interno di una stazione ferroviaria. E questo assumendo che la sua bacchetta non avesse deciso di fare qualcos’altro, come far saltare in aria l’intera King’s Cross. Harry aveva dato appena un’occhiata i suoi libri di scuola (anche se quell’occhiata era stata piuttosto bizzarra) nel rapido sforzo di determinare che tipo di libri di scienza avrebbe dovuto acquistare nelle 48 ore successive.

Bene, aveva — Harry guardò l’orologio — un’ora intera per capirlo, visto che doveva essere sul treno per le undici. Forse questo era l’equivalente di un test di qi e i bambini stupidi non potevano diventare maghi. (E la quantità di tempo in più che ti lasciavi avrebbe determinato la tua Coscienziosità, che era il secondo fattore più importante per il successo accademico.)

«Lo scoprirò», disse Harry ai suoi genitori in attesa. «Probabilmente è una specie di prova.»

Suo padre si accigliò. «Uhm… forse dovresti cercare per terra una traccia formata da impronte diverse, che porta da qualche parte che non sembra avere senso –»

«\textit{Papà!}» disse Harry. «Smettila! Non ho neppure \textit{provato} a capirlo da solo!» Era un ottimo suggerimento, persino, il che era peggio.

«Mi dispiace», si scusò il padre.

«Ah…», disse la madre di Harry. «Non credo che farebbero questo a uno studente, e tu? Sei sicuro che la professoressa McGonagall non ti abbia detto niente?»

«Forse è stata distratta», disse Harry senza pensare.

«\textit{Harry!}» sibilarono il padre e la madre all’unisono. «\textit{Che cosa hai fatto?}»

«Io, uhm –» Harry deglutì. «Guardate, non abbiamo tempo ora per questo –»

«\textit{Harry!}»

«Dico sul serio! Non abbiamo tempo ora per questo! Perché è una storia molto lunga e devo capire come arrivare a scuola!»

Sua madre aveva una mano sul viso. «Quanto è grave?»

«Io, ah», \textit{non posso parlarne per ragioni di sicurezza nazionale}, «circa la metà dell’Incidente con il progetto di scienze?»

«\textit{Harry!}»

«Io, ehm, oh guardate ci sono alcune persone con un gufo andrò a chiedergli come entrare!» e Harry fuggì via dai suoi genitori verso la famiglia dai capelli rosso fuoco, il suo baule che strisciò automaticamente dietro di lui.

La donna grassoccia lo guardò mentre si avvicinava. «Ciao, caro. Prima volta a Hogwarts? Anche Ron è nuovo, –» e poi lo scrutò attentamente. «\textit{Harry Potter?}»

Quattro ragazzi e una ragazza dai capelli rossi e un gufo si voltarono all’unisono e rimasero pietrificati sul posto.

«Oh, \textit{andiamo!}» protestò Harry. Aveva progettato di farsi passare per Harry Verres, almeno fino all’arrivo a Hogwarts. «Ho comprato una fascia e tutto il resto! Come fa a sapere chi sono?»

«Sì», disse il padre di Harry, giungendogli alle spalle con lunghi passi agili, «come \textit{fa} a sapere chi è?» La sua voce mostrava un certo timore.

«La tua foto era sui giornali», disse uno dei due gemelli di aspetto identico.

«\textsc{harry!}»

«\textit{Papà!} Non è come pensi! È perché ho sconfitto il Signore Oscuro Tu-Sai-Chi quando avevo un anno!»

«\textsc{cosa?}»

«La mamma può spiegare.»

«\textsc{cosa?}»

«Ah… Michael caro, ci sono alcune cose, ho pensato, con le quali sarebbe stato meglio non seccarti fino a ora –»

«Scusatemi», disse Harry alla famiglia dai capelli rossi che lo stava fissando al completo, «ma sarebbe davvero estremamente utile se poteste dirmi come arrivare al Binario Nove e Tre Quarti \textit{in questo preciso momento.}»

«Ah…» disse la donna. Alzò una mano e indicò il muro tra i binari. «Basta camminare dritto verso la barriera tra i Binari Nove e Dieci. Non fermarti e non aver paura di sbatterci contro, questo è molto importante. Meglio prendere un po’ di rincorsa, se sei nervoso.»

«E qualunque cosa tu faccia, non pensare a un elefante.»

«\textit{George!} Ignoralo, Harry caro, non c’è motivo per non pensare a un elefante.»

«Sono Fred, mamma, non George –»

«Grazie!» disse Harry e prese a correre verso la barriera –

Un attimo, non avrebbe funzionato \textit{se non ci avesse creduto?}

Era in momenti come quello che Harry odiava la sua mente, per il fatto che funzionasse tanto velocemente da rendersi conto che si trattava di un caso di «dubbio risonante», ovvero, se avesse pensato sin dall’inizio che sarebbe passato attraverso la barriera ce l’avrebbe fatta, solo che ora si chiedeva se credeva a sufficienza che sarebbe passato attraverso la barriera, il che significava che in realtà era preoccupato di schiantarsi contro di essa –

«\textit{Harry! Torna qui, devi darmi alcune spiegazioni!» Quello era suo padre.}

Harry chiuse gli occhi e ignorò tutto ciò che sapeva sulla credibilità giustificata e cercò solo di credere davvero tanto che sarebbe passato attraverso la barriera e –

– i suoni intorno a lui cambiarono.

Harry aprì gli occhi e incespicò fino a fermarsi, sentendosi vagamente insozzato dall’aver fatto uno sforzo deliberato di credere a qualcosa.

Si trovava su di una luminosa piattaforma all’aperto, vicino a un unico enorme treno, quattordici lunghe carrozze guidate da un’imponente macchina a vapore di metallo scarlatto con un’alta ciminiera che minacciava morte alla qualità dell’aria. La piattaforma era già leggermente affollata (anche se Harry era in anticipo di un’intera ora); dozzine di bambini e i loro genitori sciamavano attorno a panche, tavoli, e vari venditori ambulanti e bancarelle.

Era scontato dire che non c’era nessun posto del genere nella stazione di King’s Cross e nessuno spazio per nasconderlo.

\textit{Va bene, allora o (a) sono stato appena teletrasportato da tutt’altra parte, o (b) possono piegare lo spazio come nessun altro oppure (c) stanno semplicemente ignorando tutte le regole.}

Ci fu un rumore strisciante dietro di lui, e Harry si voltò a osservare il suo baule che l’aveva effettivamente seguito sui propri piccoli tentacoli artigliati. Apparentemente, per quanto riguardava la magia, anche il suo bagaglio era riuscito a credere con sufficiente intensità da passare attraverso la barriera. In realtà quel pensiero fu un po’ inquietante, quando Harry lo formulò.

Un attimo dopo, il ragazzo dai capelli rossi più giovane attraversò di corsa l’arco di ferro (arco di ferro?), tirandosi dietro il baule col guinzaglio, e per poco non si schiantò contro Harry. Harry, sentendosi stupido per essere rimasto in mezzo, iniziò ad allontanarsi rapidamente dalla zona di arrivo, e il ragazzo dai capelli rossi lo seguì, strattonando con forza il guinzaglio del baule per tenere il passo. Un attimo dopo, un gufo bianco svolazzò attraverso l’arco e andò a posarsi sulla spalla del ragazzo.

«Santo cielo», disse il ragazzo dai capelli rossi, «sei \textit{davvero} Harry Potter?»

\textit{Non di nuovo}. «Non ho un modo logico di saperlo con certezza. I miei genitori mi hanno cresciuto facendomi \textit{credere} che il mio nome sia Harry James Potter-Evans-Verres, e molte persone qui mi hanno detto che \textit{assomiglio} ai miei genitori, voglio dire agli altri miei genitori, ma», Harry aggrottò la fronte, quando capì le conseguenze, «per quanto ne so \textit{io}, potrebbero facilmente esistere incantesimi che mutano le sembianze di un bambino in un aspetto specifico –»

«Eh, che hai detto, fratello?»

\textit{Non sei destinato a Corvonero, deduco}. «Sì, sono Harry Potter.»

«Sono Ron Weasley», disse l’alto ragazzo magro, lentigginoso e dal naso lungo, e gli porse una mano, che Harry educatamente strinse mentre camminavano. Il gufo rivolse a Harry un grido stranamente misurato e cortese (più che altro un suono come eehhhhh, che sorprese Harry).

A questo punto Harry si rese conto della possibilità di una catastrofe imminente. «Solo un secondo», disse a Ron, e aprì uno dei cassetti del baule, quello che se ricordava correttamente era per l’abbigliamento invernale — lo era — e poi trovò la sciarpa più leggera che avesse, sotto il cappotto invernale. Harry si tolse la fascia, e altrettanto velocemente srotolò la sciarpa e la legò intorno al viso. Era un po’ calda, soprattutto d’estate, ma Harry poteva sopportarlo.

Poi chiuse quel cassetto e ne aprì un altro e tirò fuori le vesti nere da mago, che infilò sopra la testa, ora che era fuori dal territorio babbano.

«Fatto», disse Harry. Il suono uscì un po’ ovattato dalla sciarpa sul viso. Si rivolse a Ron. «Cosa sembro? Stupido, lo so, ma sono riconoscibile come Harry Potter?»

«Ehm», disse Ron. Chiuse la bocca, che era rimasta aperta. «Non proprio, Harry».

«Molto bene», disse Harry. «Tuttavia, per non vanificare il tutto, d’ora in poi ti rivolgerai a me chiamandomi», Verres poteva non funzionare più, «signor Spoo.»

«Okay, Harry», disse Ron incerto.

\textit{La Forza non scorre potente in lui}. «Devi… chiamarmi… signor… Spoo.»

«Va bene, signor Spoo –» Ron si fermò. «Non posso farlo, mi fa sentire stupido.»

\textit{Non è solo una sensazione}. «Va bene. Scegli \textit{tu} un nome.»

«Signor Cannon», disse Ron immediatamente. «Come i Chudley Cannons.»

«Ah…» Harry sapeva che si sarebbe terribilmente pentito di averlo chiesto. «Chi o cosa sono i Chudley Cannons?»

«\textit{Chi sono i Chudley Cannons?} Solo la più grande squadra di tutta la storia del Quidditch! Certo, sono finiti ultimi in classifica lo scorso anno, ma –»

«Cos’è il Quidditch?»

Anche quella domanda fu un errore.

«Allora fammi capire bene», disse Harry mentre sembrava che la spiegazione di Ron (con il relativo gesticolare) si stesse esaurendo. «Catturare il Boccino vale \textit{centocinquanta punti?}»

«Sì –»

«Quante segnature da dieci punti realizza di solito una squadra \textit{senza} contare il Boccino?»

«Forse quindici o venti nelle partite professionistiche –»

«Questo è completamente sbagliato. Vìola ogni possibile regola di progettazione dei giochi. Ascolta, il resto di questo gioco sembra poter avere un senso, una specie, per uno sport voglio dire, ma fondamentalmente stai dicendo che la cattura del Boccino ribalta quasi tutte le differenze di punti normali. I due Cercatori sono lassù che volano in giro alla ricerca del Boccino e di solito non interagiscono con nessun altro, avvistare per primo il Boccino è per lo più fortuna –»

«Non è fortuna!» protestò Ron. «Devi muovere gli occhi seguendo lo schema giusto –»

«Non c’è \textit{interazione}, non c’è confronto con l’altro giocatore e quanto è divertente guardare qualcuno incredibilmente bravo a muovere gli occhi? E poi qualunque Cercatore abbia un colpo di fortuna, piomba giù e afferra il Boccino e rende irrilevanti tutti gli sforzi degli altri. È come se qualcuno avesse preso un gioco vero e proprio e ci avesse innestato sopra questo inutile ruolo in più in modo da poter essere il Giocatore Più Importante, senza bisogno di farsi davvero coinvolgere o imparare il resto del gioco. Chi è stato il primo Cercatore, il figlio idiota del re che voleva giocare a Quidditch, ma non riusciva a capire le regole?» In realtà, ora che Harry ci pensava, sembrava un’ipotesi sorprendentemente buona. Mettetelo su un manico di scopa e ditegli di prendere la cosa scintillante…

Il viso di Ron era contratto in una smorfia. «Se non ti piace il Quidditch, non c’è bisogno di prenderlo in giro!»

«Se non puoi criticare, non puoi ottimizzare. Sto suggerendo come \textit{migliorare il gioco}. Ed è molto semplice. Sbarazzatevi del Boccino.»

«Non cambieranno il gioco solo perché lo dici \textit{tu!}»

«Io \textit{sono} il Ragazzo-Che-È-Sopravvissuto, sai. La gente mi ascolterà. E forse se riesco a convincerli a cambiare il gioco a Hogwarts, l’innovazione si diffonderà.»

Uno sguardo di orrore assoluto si stava diffondendo sul volto di Ron. «Ma, ma se ti liberi del Boccino, come sapremo quando il gioco finisce?»

«\textit{Comprate… un… orologio}. Sarebbe molto più giusto che far finire la partita a volte dopo dieci minuti e a volte continuare per ore, e il programma sarebbe molto più prevedibile anche per gli spettatori.» Harry sospirò. «Oh, smettila di rivolgermi quello sguardo di orrore assoluto, probabilmente non userò \textit{davvero} il mio tempo per distruggere questa patetica scusa di sport nazionale e rifarlo più forte e più intelligente a mia immagine. Ho cose molto, molto, molto più importanti di cui preoccuparmi.» Harry sembrò pensieroso. «Ma del resto, non ci vorrebbe molto tempo per scrivere le Novantacinque Tesi della Riforma Senza Boccino e inchiodarle al portale di una chiesa –»

«Potter», biascicò la voce di un giovane ragazzo, «\textit{che cosa} hai sul viso e \textit{che cosa} è in piedi accanto a te?»

Lo sguardo di orrore di Ron fu sostituito da un odio totale. «\textit{Tu!}»

Harry girò la testa; e infatti era Draco Malfoy, che poteva essere stato costretto a indossare le ordinarie vesti scolastiche, ma che se lo stava facendo perdonare con un baule che sembrava almeno altrettanto magico e molto più elegante di quello di Harry, decorato in argento e smeraldi e recante ciò che Harry immaginò fosse l’emblema della famiglia Malfoy, un bellissimo serpente artigliato sovrapposto a bacchette d’avorio incrociate.

«Draco!» disse Harry. «Ehm, o Malfoy, se preferisci, anche se questo mi suona come riferito a Lucius. Sono contento di vedere che tu stia così bene dopo, uhm, il nostro ultimo incontro. Questo è Ron Weasley. E sto cercando di restare in incognito, quindi chiamami, ehm», Harry si guardò le vesti, «signor Black.»

«\textit{Harry!}» sibilò Ron. «Non puoi usare \textit{quel} nome!»

Harry sbatté le palpebre. «Perché no?» \textit{Suonava} piacevolmente oscuro, come una figura misteriosa e straniera –

«Direi che è un nome \textit{distinto}», disse Draco, «ma appartiene alla Nobile e Antichissima Casa Black. Ti chiamerò signor Argento.»

«\textit{Tu} stai lontano da… dal signor Oro», Ron disse freddamente, e fece un passo avanti. «Non ha bisogno di parlare con gente come te!»

Harry alzò conciliante una mano. «Mi farò chiamare signor Bronzo, grazie per l’ispirazione. E, Ron, uhm», Harry faticò a trovare il modo di dirlo, «sono contento che tu sia così… entusiasta di proteggermi, ma non mi dispiace particolarmente parlare con Draco –»

Apparentemente quella fu l’ultima goccia per Ron, che si girò verso Harry con gli occhi ora ardenti di sdegno. «\textit{Cosa?} Tu \textit{sai} chi è?»

«Sì, Ron, ricorderai che l’ho chiamato Draco senza che lui abbia avuto la necessità di presentarsi.»

Draco ridacchiò. Poi i suoi occhi si illuminarono alla vista del gufo bianco sulla spalla di Ron. «Oh, che cos’è \textit{questo?}» Draco chiese con un tono ricco di malizia. «Dov’è il famoso ratto della famiglia Weasley?»

«Sepolto nel cortile di casa», disse Ron freddamente.

«Oh, che tristezza. Pot… ah, signor Bronzo, dovrei menzionare che è ampiamente accettato il fatto che la famiglia Weasley sia detentrice della \textit{migliore storia di sempre riguardo gli animali domestici.} Vuoi raccontarla, Weasley?»

Il volto di Ron si contorse. «Non lo troveresti divertente se fosse successo alla \textit{tua} famiglia!»

«Oh», fece le fusa Draco, «ma non sarebbe mai \textit{accaduto} ai Malfoy.»

Le mani di Ron si chiusero a pugno –

«Basta così», disse Harry, mettendo nella voce tutta l’autorevolezza composta che poté. Era chiaro che a qualunque cosa si riferisse, quello fosse un ricordo doloroso per il ragazzo dai capelli rossi. «Se Ron non vuole parlarne, non è obbligato a farlo, e ti devo chiedere di non parlarne neppure tu.»

Draco si voltò verso Harry con uno sguardo sorpreso e Ron annuì. «Giusto, Harry! Voglio dire signor Bronzo! Vedi che tipo di persona è? Ora digli di andare via!»

Harry contò mentalmente fino a dieci, cosa che per lui fu un rapido 12345678910 — una strana abitudine rimastagli dall’età di cinque anni, quando sua madre gli aveva insegnato a farlo per la prima volta, e Harry aveva ragionato che il suo modo era più veloce e sarebbe dovuto essere altrettanto efficace. «Non gli dirò di andare via», disse Harry con calma. «È libero di parlare con me, se vuole.»

«Beh, non ho intenzione di frequentare qualcuno che frequenta Draco Malfoy», Ron annunciò freddamente.

Harry scrollò le spalle. «Dipende da te. \textit{Io} non ho intenzione di permettere a nessuno di dirmi chi posso o non posso frequentare.» In silenzio recitò, \textit{per favore vai via, per favore vai via…}

Il volto di Ron si paralizzò per la sorpresa, come se si fosse aspettato che quella frase avrebbe davvero funzionato. Poi si girò di scatto, strattonò il guinzaglio della sua valigia e si precipitò giù per la piattaforma.

«Se non ti piace», chiese Draco con curiosità, «perché non te ne sei semplicemente andato?»

«Uhm… sua madre mi ha aiutato a capire come arrivare a questo binario dalla stazione di King’s Cross, quindi era un po’ difficile dirgli di sparire. E non è che io \textit{odi} questo Ron», disse Harry, «è solo che, che…» Cercò le parole.

«Non vedi alcun motivo per il quale dovrebbe esistere?» offrì Draco.

«Più o meno.»

«In ogni caso, Potter… se davvero sei stato cresciuto da Babbani –» Draco fece una pausa qui, come se si aspettasse una smentita, ma Harry non disse niente «– allora potresti non sapere cosa significa essere famosi. La gente vuole prendersi \textit{tutto} il nostro tempo. \textit{Devi} imparare a dire di no.»

Harry annuì, assumendo un’espressione pensierosa sul volto. «Sembra un buon consiglio.»

«Se cerchi di essere gentile, finirai per spendere più tempo con i più invadenti. Decidi con chi \textit{tu} vuoi trascorrere il tuo tempo e fai andar via tutti gli altri. Sei appena arrivato, Potter, quindi tutti ti giudicheranno in base a chi frequenti, e non vuoi essere visto con persone come Ron Weasley.»

Harry annuì di nuovo. «Se non ti dispiace, come hai fatto a riconoscermi?»

«\textit{Signor Bronzo}», disse Draco con voce strascicata, «ti avevo \textit{già} incontrato, ricordalo. Ho visto qualcuno che andava in giro con una sciarpa avvolta intorno alla testa, in maniera assolutamente ridicola. Così ho tirato a \textit{indovinare.}»

Harry chinò il capo, accettando il complimento. «Sono \textit{terribilmente} dispiaciuto per quello che è accaduto», disse Harry. «Il nostro primo incontro, voglio dire. Non volevo metterti in imbarazzo davanti a Lucius.»

Draco agitò la mano indirizzando a Harry uno sguardo strano. «Vorrei solo che mio Padre fosse entrato mentre \textit{tu} stavi lusingando \textit{me} –» Draco rise. «Ma grazie per ciò che \textit{tu} hai detto a mio Padre. Se non fosse stato per quello, sarebbe stato più difficile dare una spiegazione.»

Harry gli rivolse un inchino più profondo. «E grazie a \textit{te} per aver ricambiato con quello che hai detto alla professoressa McGonagall.»

«Non c’è di che. Anche se una delle assistenti deve aver chiesto alla sua migliore amica di giurare di mantenere il segreto assoluto, perché mio Padre dice che ci sono \textit{strane voci} in giro, come che tu e io ci saremmo azzuffati o qualcosa del genere.»

«Ahia», disse Harry, facendo una smorfia. «Sono \textit{davvero} dispiaciuto –»

«No, ci siamo abituati, Merlino sa che ci sono già parecchie voci sulla famiglia Malfoy.»

Harry annuì. «Sono contento di sentire che non sei nei guai.»

Draco fece un sorrisetto. «Mio Padre ha, ehm, un senso dell’umorismo \textit{sofisticato}, ma \textit{comprende} cosa significhi farsi degli amici. Lo comprende \textit{molto} bene. Me l’ha fatto ripetere ogni sera prima di andare a letto per tutto il mese scorso, ‘mi farò degli amici a Hogwarts’. Quando gli ho spiegato tutto e ha compreso ciò che stavo facendo, mi ha comprato un gelato.»

Harry rimase a bocca aperta. «\textit{Sei riuscito a metterla in modo che ti comprasse un gelato?}»

Draco annuì, sembrando tanto compiaciuto quanto quell’impresa meritava. «Beh, mio padre \textit{sapeva} quello che stavo facendo, ovviamente, ma è lui che mi ha insegnato \textit{come} farlo, e se sorrido nel modo giusto \textit{mentre} lo faccio, la rende una cosa tra padre e figlio e allora \textit{deve} comprarmi un gelato o gli metterò questa specie di faccia triste, come se pensassi di averlo deluso.»

Harry osservò calcolatore Draco, percependo la presenza di un altro maestro. «Hai avuto delle \textit{lezioni} su come manipolare le persone?»

«Naturalmente», disse Draco con orgoglio. «Sono un \textit{Malfoy}. Mio Padre mi ha pagato dei precettori.»

«Uau», disse Harry. Leggere \textit{Teoria e pratica della persuasione} di Robert Cialdini probabilmente non era allo stesso livello (anche se era comunque un gran bel libro). «Tuo padre è quasi tanto fantastico quanto il mio.»

Le sopracciglia di Draco si sollevarono altezzosamente. «Oh? E che cosa fa tuo padre?»

«Mi compra libri.»

Draco valutò la risposta. «Non mi sembra molto impressionante.»

«Dovevi essere presente. Comunque, sono felice di sentire tutto questo. Dal modo in cui Lucius ti stava guardando, ho pensato che stesse per c-crocifiggerti.»

«Mio padre mi ama davvero», disse Draco con fermezza. «Non l’avrebbe mai fatto.»

«Ehm…» disse Harry. Ricordava l’elegante figura dalle vesti nere e i capelli bianchi che era entrata come una furia da Madam Malkin, brandendo quel bellissimo e micidiale bastone dal manico d’argento. Non era facile immaginarselo come un padre amorevole. «Non prenderla a male, ma come fai a \textit{saperlo?}»

«Eh?» Era chiaro che si trattava di una domanda che Draco solitamente non si poneva.

«Sto ponendo la domanda fondamentale della razionalità: perché credi in ciò in cui credi? Cosa pensi di sapere e come pensi di saperlo? Cosa ti fa pensare che Lucius non ti sacrificherebbe nello stesso modo in cui sacrificherebbe ogni altra cosa per il potere?»

Draco indirizzò a Harry un altro strano sguardo. «Cosa sai esattamente \textit{tu} di mio Padre?»

«Uhm… membro del Wizengamot, membro del Consiglio Direttivo di Hogwarts, incredibilmente ricco, conosce il Ministro Fudge, ha la fiducia del Ministro Fudge, probabilmente ha alcune foto molto imbarazzanti del Ministro Fudge, è il purista del sangue più prominente ora che il Signore Oscuro è andato, ex-Mangiamorte che è stato trovato col Marchio Oscuro, ma che se l’è cavata affermando di essere stato sotto l’Incantesimo Imperius, cosa ridicolmente inverosimile e praticamente tutti lo sapevano… malvagio con la ‘\textsc{m}’ maiuscola e assassino nato… credo sia tutto.»

Gli occhi di Draco si ridussero a due fessure. «Te l’ha detto McGonagall, giusto?»

«No, non mi ha detto \textit{niente} su Lucius dopo, se non di stare lontano da lui. Quindi durante l’Incidente al negozio di pozioni, mentre la professoressa McGonagall era occupata a urlare contro il negoziante e a cercare di riportare tutto sotto controllo, ho preso da parte uno dei clienti e ho chiesto a \textit{lui} di Lucius.»

Gli occhi di Draco furono nuovamente spalancati. «L’hai fatto \textit{davvero?}»

Harry indirizzò a Draco uno sguardo perplesso. «Se ho mentito la prima volta, non ti dirò la verità solo perché me lo chiedi la seconda.»

Ci fu una certa pausa mentre Draco stava assimilando la risposta.

«Finirai certamente a Serpeverde.»

«Finirò certamente a Corvonero, grazie tante. Voglio il potere solo perché così posso ottenere i libri.»

Draco ridacchiò. «Sì, proprio. Comunque… per rispondere alla tua domanda…» Draco fece un respiro profondo, e il suo viso si fece serio. «Mio Padre una volta ha mancato un voto del Wizengamot per me. Ero su una scopa e sono caduto e mi sono rotto molte costole. Faceva davvero male. Non mi ero mai fatto così male prima e ho pensato di stare per morire. Così mio Padre ha disertato questo voto davvero importante, perché era lì accanto al mio letto al St. Mungo’s, tenendomi le mani e promettendomi che tutto sarebbe andato bene.»

Harry distolse lo sguardo per il disagio, poi, con uno sforzo, si costrinse a guardare di nuovo Draco. «Perché mi stai raccontando \textit{questo?} Sembra piuttosto… privato…»

Draco rivolse a Harry uno sguardo serio. «Uno dei miei insegnanti una volta ha detto che le persone formano amicizie strette conoscendo cose private l’una dell’altra, e la ragione per cui la maggior parte delle persone non hanno amici intimi è perché sono troppo imbarazzate per condividere qualcosa di veramente importante di sé stesse.» Draco girò in su i palmi delle mani, in un gesto di invito. «Tocca a te.»

Sapere che l’espressione speranzosa di Draco gli era stata probabilmente inculcata da mesi di pratica non la rese meno efficace, osservò Harry. In realtà la \textit{rendeva} un po’ \textit{meno} efficace, ma purtroppo non \textit{inefficace}. Lo stesso si poteva dire dell’uso intelligente da parte di Draco della pressione di ricambiare un regalo non richiesto, una tecnica che Harry aveva letto nei suoi libri di psicologia sociale (un esperimento aveva dimostrato che un dono incondizionato di \$5 era due volte più efficace di un’offerta condizionata di \$50 per convincere la gente a compilare dei sondaggi). Draco aveva fatto il dono non richiesto di una confidenza, e ora aveva invitato Harry a offrire una confidenza in cambio… e la questione era che Harry \textit{si sentiva} pressato. Un rifiuto, Harry era certo, sarebbe stato accolto con uno sguardo di triste delusione, e forse una piccola quantità di disprezzo che avrebbe indicato che Harry aveva perso punti.

«Draco, giusto per la cronaca, riconosco esattamente quello che stai facendo in questo momento. I miei libri la chiamano \textit{reciprocità} e spiegano come dare a qualcuno un regalo diretto di due sicli è risultato essere due volte più efficace che offrirgli venti sicli al fine di convincerlo a fare quello che vuoi…» Harry si interruppe.

Draco sembrava triste e deluso. «Non è un trucco, Harry. È un gesto autentico per diventare amici.»

Harry alzò una mano. «Non ho detto che non avevo intenzione di rispondere. Ho solo bisogno di tempo per trovare qualcosa che sia privato, ma altrettanto non dannoso. Come dire… volevo farti sapere che non mi si può mettere fretta.» Una pausa di riflessione poteva fare molto per disinnescare l’efficacia di molte tecniche che si basavano sulla spinta al conformismo, una volta che si avesse imparato a riconoscerle per quello che erano.

«Va bene», disse Draco. «Aspetterò mentre pensi a qualcosa. Oh, e ti prego di togliere la sciarpa mentre lo dici.»

\textit{Semplice ma efficace.}

E Harry non poté fare a meno di notare quanto goffo, impacciato e sgraziato fosse apparso il suo tentativo di resistere alla manipolazione / salvare la faccia / mettersi in mostra rispetto a Draco. \textit{Ho bisogno di quei precettori.}

«Va bene», disse Harry dopo un po’. «Ecco il mio.» Si guardò intorno e poi srotolò la sciarpa dal volto, scoprendo tutto tranne la cicatrice. «Uhm… sembra che tu possa davvero contare su tuo padre. Voglio dire… se gli parli seriamente, lui ti starà sempre ad ascoltare e ti prenderà sul serio.»

Draco annuì.

«A volte», disse Harry, e deglutì. Questo era sorprendentemente difficile, ma del resto doveva esserlo. «A volte vorrei che mio padre fosse come il tuo.» Gli occhi di Harry si ritrassero dal viso di Draco, più o meno automaticamente, e allora Harry si costrinse a guardare di nuovo Draco.

Poi Harry fu colpito da \textit{cosa diavolo aveva appena detto}, e si affrettò ad aggiungere, «Non è che vorrei che il mio Papà fosse un impeccabile strumento di morte come Lucius, mi riferivo semplicemente al prendermi sul serio –»

«Capisco», disse Draco con un sorriso. «Ecco… ora non sembra come se fossimo un po’ più vicini a essere amici?»

Harry annuì. «Già. Così sembra, in realtà. Ehm… senza offesa, ma ho intenzione di rimettermi il travestimento, \textit{davvero} non vorrei dover avere a che fare con –»

«Capisco.»

Harry abbassò nuovamente la sciarpa sul viso.

«Mio padre prende sul serio tutti i suoi amici», disse Draco. «È per questo che ha molti amici. Dovresti incontrarlo.»

«Ci penserò», disse Harry con voce neutra. Scosse la testa per la meraviglia. «Dunque tu sei davvero il suo unico punto debole. Eh.»

Ora Draco stava rivolgendo a Harry uno sguardo \textit{realmente} strano. «Vuoi andare a prendere qualcosa da bere e trovare un posto per sederci?»

Harry si rese conto che era rimasto in piedi nello stesso luogo troppo a lungo, e si stiracchiò, cercando di sbloccare la schiena. «Certo.»

Ormai la piattaforma stava iniziando a riempirsi, ma vi era ancora una zona più tranquilla sul lato lontano dal motore a vapore rosso. Lungo la strada passarono davanti a una bancarella che ospitava un uomo calvo e barbuto che offriva giornali e fumetti e lattine verde fluorescente impilate.

Il venditore stava, in effetti, sdraiato a bersi una delle lattine verde fluorescente nel momento esatto in cui vide il raffinato ed elegante Draco Malfoy che si avvicinava con un misterioso ragazzo dall’aspetto incredibilmente stupido, con una sciarpa avvolta sul viso, e questo causò al venditore un attacco di tosse improvvisa mentre beveva e gli fece rovesciare una grande quantità di liquido verde fluorescente sulla barba.

«Scusi», disse Harry, «che cos’è questa roba, esattamente?»

«SpiriTè», disse il venditore. «Se lo bevi, è destino che accada qualcosa di sorprendente che te lo fa rovesciare addosso o su qualcun altro. Ma è incantato per svanire pochi secondi più tardi –» Infatti la macchia sulla sua barba era già scomparsa.

«Che cosa divertente», disse Draco. «Che cosa davvero, davvero divertente. Venga, signor Bronzo, andiamo a cercare un altro –»

«Aspetta», disse Harry.

«\textit{Oh, andiamo!} Questo è assolutamente, assolutamente \textit{puerile!}»

«No, mi dispiace Draco, devo indagare. Cosa succede se bevo lo SpiriTè, mentre faccio del mio meglio per mantenere la conversazione del tutto seria?»

Il venditore sorrise misteriosamente. «Chi lo sa? Un amico arriva in un costume da rana? Qualcosa di inaspettato è destinato ad accadere –»

«No. Mi dispiace. Proprio non ci credo. Questo viola la mia tanto abusata sospensione dell’incredulità su così tanti livelli che non ho nemmeno le parole per descriverlo. Non esiste, non c’è proprio \textit{nessun modo} in cui una dannata \textit{bevanda} possa manipolare la realtà per produrre \textit{situazioni comiche}, oppure mollo tutto e mi ritiro alle Bahamas –»

Draco gemette. «Dobbiamo farlo \textit{veramente?}»

«Tu non devi berlo ma io \textit{devo} indagare. \textit{Devo}. Quanto?»

«Cinque zellini la lattina», disse il venditore.

«\textit{Cinque zellini?} Può vendere bevande gassate che manipolano la realtà per \textit{cinque zellini la lattina?}» Harry si mise una mano nella borsa, disse «quattro sicli, quattro zellini», e li batté sul bancone. «Due dozzine di lattine per favore.»

«Ne prendo una anch’io», sospirò Draco, e iniziò a mettere le mani nelle tasche.

Harry scosse la testa rapidamente. «Lascia stare, ci penso io, non conta neppure come favore, voglio vedere se funziona anche per te.» Prese una lattina dalla pila ora collocata sul bancone e la lanciò a Draco, poi iniziò a nutrire la propria borsa. Il Bordo Allargante della borsa ingoiò le lattine accompagnandole con ruttini, cosa che non stava proprio aiutando a ripristinare la fede di Harry nella possibilità di scoprire, un giorno, una spiegazione ragionevole per tutto ciò.

Ventidue rutti dopo, Harry teneva in mano l’ultima lattina acquistata, Draco lo guardava con un’aria di attesa, ed entrambi tirarono l’anello contemporaneamente.

Harry srotolò la sciarpa per liberare la bocca, e piegò indietro la testa per bere lo SpiriTè.

In qualche modo \textit{sapeva} di verde brillante — extra-frizzante e più «limonoso» del limone.

A parte ciò, null’altro accadde.

Harry guardò il venditore, che li stava osservando con benevolenza.

\textit{Va bene, se questo tizio ha appena approfittato di un incidente naturale per vendermi ventiquattro lattine di niente, applaudirò il suo spirito imprenditoriale creativo e poi lo ucciderò.}

«Non sempre accade subito», disse il venditore. «Ma è garantito che accada una volta per lattina, o sarete rimborsati.»

Harry bevette un altro lungo sorso.

Ancora una volta, non accadde nulla.

\textit{Forse dovrei solo tracannare tutto il più velocemente possibile… e sperare che il mio stomaco non esploda per tutto il biossido di carbonio, o che io non faccia un rutto mentre lo bevo…}

No, poteva permettersi di avere un po’ di pazienza. Ma in tutta onestà, Harry non vedeva come questo avrebbe potuto funzionare. Non potevi andare da qualcuno e dirgli «Ora ho intenzione di sorprenderti» oppure «E ora ho intenzione di dirti la battuta finale della barzelletta, e sarà davvero divertente.» Avresti rovinato l’effetto sorpresa. Nello stato di preparazione mentale di Harry, Lucius Malfoy sarebbe potuto passare in abiti da ballerina e non gli avrebbe causato alcun rigurgito. Con che tipo di bizzarro numero comico l’universo se ne sarebbe uscito ora?

«In ogni caso, sediamoci», disse Harry. Si preparò a bere un altro po’ di liquido e si mosse verso la distante zona d’attesa, cosa che lo mise nell’angolazione giusta per gettare uno sguardo indietro e vedere la parte del chiosco che conteneva la rastrelliera dedicata a un giornale chiamato Il Cavillo, che mostrava il seguente titolo:

\textsc{ragazzo-che-è-sopravvissuto}

\textsc{mette incinto draco malfoy}

«Gah!» urlò Draco mentre il liquido verde brillante gli fu spruzzato addosso dalla direzione di Harry. Draco si voltò verso Harry con il fuoco negli occhi e strinse la propria lattina. «Figlio di un sanguemarcio! Vediamo se a \textit{te} piace essere sputato addosso!» e diede un sorso deliberato alla lattina proprio mentre i suoi occhi scorsero il titolo.

Per puro riflesso, Harry cercò di proteggere il proprio viso mentre il getto di liquido volò nella sua direzione. Purtroppo si protesse con la mano che reggeva lo SpiriTè, mandando il resto del liquido verde a schizzare sulla propria spalla.
Harry fissò la lattina che aveva in mano anche mentre continuava a soffocare e sputacchiare e il colore verde iniziò a svanire dalla veste di Draco.


Poi alzò gli occhi e fissò il titolo del giornale.

\textsc{ragazzo-che-è-sopravvissuto}

\textsc{mette incinto draco malfoy}

Le labbra di Harry si aprirono e dissero: «ma-ma-ma-ma…»

Troppe obiezioni in concorrenza tra loro, quello era il problema. Ogni volta che Harry cercava di dire «Ma siamo solo undicenni!» l’obiezione «Ma gli uomini non possono restare incinti!» esigeva la priorità ed era poi travolta da «Ma non c’è niente tra noi, davvero!»

Poi Harry guardò nuovamente giù, verso la lattina nella sua mano.

Sentiva un profondo desiderio di fuggire via urlando a squarciagola fino a crollare per carenza d’ossigeno, e l’unica cosa che glielo impedì fu che una volta aveva letto che il panico completo era il segno di un problema scientifico \textit{veramente} importante.

Harry ringhiò, gettò con forza la lattina in un vicino cestino della spazzatura, e tornò di nuovo verso la bancarella. «Una copia de \textit{Il Cavillo}, prego.» Harry pagò altri quattro zellini, recuperò un’altra lattina di SpiriTè dalla borsa, e poi si diresse verso l’area picnic con il ragazzo dai capelli biondi, il quale fissava la propria lattina con un’espressione di sincera ammirazione.

«Ritiro tutto», disse Draco, «è stato davvero notevole.»

«Ehi, Draco, sai cosa scommetto che sia ancor meglio per diventare amici dello scambiarsi i segreti? Commettere un omicidio.»

«Ho un precettore che la pensa così», ammise Draco. Portò la mano sotto le vesti e si grattò con un movimento naturale e tranquillo. «Chi hai in mente?»

Harry sbatté \textit{Il Cavillo} giù duro sul tavolo da picnic. «Il tipo che si è inventato questo titolo.»

Draco gemette. «Non un tipo. Una tipa. Una tipa di \textit{dieci anni}, riesci a crederci? È impazzita dopo che sua madre è morta e suo padre, che possiede questo giornale, è \textit{convinto} che lei sia una veggente, così quando non sa qualcosa chiede a Luna Lovegood e crede a \textit{tutto} ciò che dice.»

Senza pensarci, Harry tirò l’anello della successiva lattina di SpiriTè e si preparò a bere. «Mi stai prendendo in giro? Questo è anche peggio del giornalismo babbano, cosa che avrei pensato fosse fisicamente impossibile.»

Draco ringhiò. «Lei ha un’ossessione perversa per i Malfoy, per di più, e suo padre ci è politicamente avverso, quindi ne pubblica ogni parola. Non appena sarò abbastanza grande ho intenzione di violentarla.»

Liquido verde schizzò fuori dalle narici di Harry, bagnando la sciarpa che ancora copriva quella parte. SpiriTè e polmoni non andavano d’accordo, e Harry trascorse i successivi secondi tossendo freneticamente.

Draco lo guardò attentamente. «Qualcosa non va?»

Fu allora che Harry giunse all’improvvisa realizzazione che (a) i suoni provenienti dal resto della piattaforma del treno si erano trasformati in un rumore bianco indistinto all’incirca quando Draco aveva messo la mano sotto le vesti, e (b) quando aveva parlato del commettere un omicidio come un metodo per stringere i legami, vi era stata esattamente una persona nella conversazione che aveva pensato che stessero scherzando.

\textit{Già. Perché lui sembrava un ragazzo tanto normale. Ed era un ragazzo normale, era proprio quello che ci si aspetterebbe che fosse un bambino medio se Darth Vader fosse il suo padre affettuoso.}

«Sì, beh», Harry tossì, oh dio, come avrebbe fatto a uscire da questo vicolo cieco colloquiale, «ero solo sorpreso di quanto fossi disposto a discuterne così apertamente, non sembravi preoccupato di essere arrestato o altro.»

Draco sbuffò. «Stai scherzando? La parola di \textit{Luna Lovegood} contro la mia?»

Porca miseriaccia. «Non esiste una cosa come il rilevamento magico della verità, ne deduco?» \textit{O il test del \textsc{dna}… per ora.}

Draco si guardò intorno. I suoi occhi si strinsero. «Già, tu non ne sai niente. Senti, ti spiegherò come stanno le cose, voglio dire il modo in cui funzionano davvero, proprio come se tu fossi già a Serpeverde e mi avessi fatto la stessa domanda. Ma devi giurare di non dire nulla a riguardo.»

«Lo giuro», disse Harry.

«I tribunali usano il Veritaserum, ma è davvero uno scherzo, basta farti Obliare prima di testimoniare e poi sostenere che l’altra persona abbia ricevuto un ricordo falso con l’Incantesimo di Memoria. Naturalmente se sei una persona comune, i giudici propendono per l’Obliazione, non per un Incantesimo del Falso Ricordo. Ma la corte ha potere discrezionale, e se fossi coinvolto \textit{io} allora andrebbe a ledere l’onore di una Nobile Casa, così si andrebbe al Wizengamot, dove mio Padre ha la maggioranza. Dopo che fossi dichiarato innocente, la famiglia Lovegood dovrebbe pagare un risarcimento per aver infangato il mio onore. E sanno fin dall’inizio che è così che andrà, così tengono semplicemente la bocca chiusa.»

Un brivido freddo discese su Harry, un freddo che venne con l’istruzione di mantenere il viso e la voce normali. \textit{Nota a me stesso: rovescia il governo della Gran Bretagna magica alla prima occasione.}

Harry tossì di nuovo per schiarirsi la voce. «Draco, ti prego ti prego \textit{ti prego} di non prenderla nel modo sbagliato, la mia parola mi impegna, ma come hai detto tu potrei finire a Serpeverde e voglio chiedertelo a scopo informativo, cosa succederebbe \textit{teoricamente parlando}, se \textit{io} testimoniassi che ti ho sentito progettarlo?»

«Allora, se fossi chiunque altro eccetto un Malfoy, sarei nei guai», Draco rispose con aria di sufficienza. «Dato che \textit{sono} un Malfoy… mio Padre ha la maggioranza. E poi ti schiaccerebbe… beh, credo non facilmente, perché tu \textit{sei} il Ragazzo-Che-È-Sopravvissuto, ma mio Padre è abbastanza bravo in questo genere di cose.» Draco aggrottò le sopracciglia. «E poi, \textit{tu} hai parlato di ucciderla, perché non eri preoccupato che \textit{io} testimoniassi dopo che fosse stata trovata morta?»

\textit{Come, come ha fatto questa giornata ad andare così storta?} La bocca di Harry si stava già muovendo più velocemente di quanto potesse pensare. «In quel momento pensavo che fosse più vecchia! Non so come funziona qui, ma nella Gran Bretagna babbana i tribunali sarebbero parecchio più sconvolti se qualcuno uccidesse un bambino –»

«Capisco», disse Draco, ancora un po’ sospettoso. «Ma in ogni caso, è sempre meglio se non si arriva affatto agli Auror. Se stessimo attenti a fare solo le cose che un Incantesimo di Guarigione possa rimarginare, potremmo semplicemente Obliarla in seguito e poi fare tutto da capo la settimana successiva.» Poi il ragazzo biondo ridacchiò, un acuto suono giovanile. «Anche se, immaginati solo che dica di essere stata sbattuta da Draco Malfoy e dal Ragazzo-Che-È-Sopravvissuto, nemmeno \textit{Silente} le crederebbe.»

\textit{Ho intenzione di frantumare il tuo patetico minuscolo residuo magico del Medioevo in pezzi più piccoli dei suoi atomi costituenti.} «In realtà, possiamo sospendere la faccenda? Dopo che ho scoperto che quel titolo è venuto da una ragazza di un anno più giovane di me, ho avuto un’idea diversa per la mia vendetta.»

«Eh? Racconta», disse Draco, e iniziò a bere un altro sorso del suo SpiriTè.

Harry non sapeva se l’incantesimo funzionasse più di una volta per lattina, ma sapeva che poteva evitare di assumersene la colpa, quindi fu attento a dirlo esattamente al momento giusto:

«Stavo pensando che \textit{un giorno sposerò quella donna.}»

Draco fece un suono orrido e perse del liquido verde dagli angoli della bocca come il radiatore rotto di un’automobile. «\textit{Sei impazzito?}»

«Al contrario, sono così sano di mente che brucia come il ghiaccio.»

«Hai gusti più strani di un Lestrange», disse Draco, sembrando per metà ammirato. «E suppongo che la vuoi tutta per te, eh?»

«Già. Potrei doverti un favore per questo –»

Draco agitò la mano. «Nah, questa è gratis.»

Harry fissò la lattina nella propria mano, la freddezza che si diffondeva nel sangue. Affascinante, felice, generoso di favori con i suoi amici, Draco \textit{non era} uno psicopatico. Quella era la parte triste e terribile, conoscere la psicologia umana abbastanza bene da sapere che Draco non era un mostro. C’erano state diecimila società nel corso della storia del mondo in cui quella conversazione si sarebbe potuta svolgere. No, il mondo sarebbe stato un posto molto diverso, se solo un \textit{mutante malvagio} avesse potuto dire ciò che Draco aveva detto. Era molto semplice, molto umano, era l’opinione normale se nient’altro interferiva. Per Draco, i suoi nemici non erano persone.

E nel tempo rallentato di quel paese rallentato, là e allora come nel buio-prima-dell’alba precedente l’Età della Ragione, il figlio di un nobile sufficientemente potente dava semplicemente per scontato di essere al di sopra della legge, almeno quando si trattava di una contadina qualsiasi. C’erano posti nelle terre babbane in cui funzionava ancora nello stesso modo, paesi in cui quel tipo di nobiltà esisteva ancora e ancora pensava in quel modo, o terre ancora peggiori in cui non si trattava solo della nobiltà. Era stato così in ogni luogo e tempo che non discendessero direttamente dall’Illuminismo. Una linea di discendenza, sembrava, che non includeva del tutto la Gran Bretagna magica, malgrado tutte le contaminazioni cross-culturali di cose come le lattine ad anello.

\textit{E se Draco non cambiasse idea sul suo desiderio di vendetta, e io non buttassi via la mia speranza di una vita felice per sposare una povera ragazza folle, allora tutto quello che avrei appena guadagnato è del tempo, e neppure molto…}

Per una ragazza. Non per le altre.

\textit{Mi chiedo quanto sarebbe difficile fare una lista di tutti i principali puristi del sangue e ucciderli.}

Avevano provato a fare proprio quello durante la Rivoluzione Francese, più o meno — compilare una lista di tutti i nemici del Progresso e togliere loro ogni cosa sopra al collo — e non aveva funzionato bene da quello che Harry ricordava. Forse aveva bisogno di rispolverare alcuni di quei libri di storia che suo padre gli aveva comprato, e vedere se quello che era andato storto con la Rivoluzione francese fosse qualcosa di facile da risolvere.

Harry alzò gli occhi al cielo, e alla forma pallida della Luna, visibile quella mattina nel cielo senza nuvole.

\textit{Così il mondo è guasto e imperfetto e folle, e crudele e sanguinoso e oscuro. Questa è una notizia? L’hai sempre saputo, comunque…}

«Sembri così serio», disse Draco. «Fammi indovinare, i tuoi genitori babbani ti hanno detto che questo genere di cose è malvagio.»

Harry annuì, non fidandosi molto della propria voce.

«Beh, come dice mio Padre, ci saranno pure quattro case, ma alla fine tutti appartengono o a Serpeverde o a Tassofrasso. E, francamente, non appartieni a Tassofrasso. Se decidessi di schierarti con i Malfoy sottobanco… il nostro potere e la tua reputazione… potresti cavartela per cose che neppure \textit{io} potrei fare. Vuoi \textit{provarci} per un po’? Vedere cosa vuol dire?»

\textit{Che astuto serpentello. Undici anni e già lusinghi la tua preda per farla uscire allo scoperto…}

Harry pensò, valutò, e scelse la sua arma. «Draco, mi spiegheresti la faccenda della purezza del sangue? Mi è nuova.»

Un ampio sorriso attraversò il volto di Draco. «Dovresti davvero incontrare mio Padre e chiedere a \textit{lui}, sai, è il nostro capo.»

«Dammi un accenno.»

«Va bene», disse Draco. Fece un respiro profondo, la sua voce divenne appena più grave e assunse una cadenza. «I nostri poteri sono divenuti più deboli, di generazione in generazione, mentre la contaminazione dei sanguemarcio aumenta. Mentre Salazar e Godric e Rowena e Helga fecero sorgere Hogwarts col loro potere, creando il Medaglione e la Spada e il Diadema e la Coppa, nessun mago di questi giorni sbiaditi è assurto a rivaleggiare con loro. Ci stiamo affievolendo, tutti ci stiamo affievolendo e stiamo diventando Babbani mentre ci incrociamo con la loro progenie e permettiamo ai nostri Maghinò di vivere. Se la contaminazione non sarà tenuta sotto controllo, presto le nostre bacchette si romperanno e tutte le nostre arti cesseranno, la stirpe di Merlino finirà e il sangue di Atlantide svanirà. I nostri figli non potranno che grattare la terra per sopravvivere come semplici Babbani, e l’oscurità coprirà tutto il mondo per sempre.» Draco sorbì un altro sorso dalla sua lattina, sembrando soddisfatto; quella sembrava essere l’intera argomentazione, per quanto lo riguardava.

«Persuasivo», disse Harry, intendendolo dal punto di vista descrittivo piuttosto che normativo. Era un modello consueto: la Caduta, la necessità di custodire ciò che era rimasto della purezza contro la contaminazione, il passato che si ergeva in alto e il futuro che scivolava verso il basso. E quello schema ora aveva anche una \textit{risposta}… «Devo correggerti su di un fatto, però. Le vostre informazioni sui Babbani sono un po’ arretrate. Noi non stiamo più esattamente grattando la terra.»

La testa di Draco si girò di scatto. «\textit{Cosa?} Che intendi dire con ‘\textit{noi}’?»

«Noi. Gli scienziati. La stirpe di Francesco Bacone e il sangue dei Lumi. I Babbani non si sono limitati a starsene seduti e a lamentarsi di non avere bacchette, abbiamo i \textit{nostri} poteri ora, con o senza la magia. Se tutti i vostri poteri fallissero, allora noi tutti avremmo perso qualcosa di molto prezioso, perché la vostra magia è l’unico indizio che abbiamo su come l’universo deve funzionare \textit{davvero} — ma non sareste costretti a grattare la terra. Le vostre case sarebbero ancora fresche d’estate e calde d’inverno, ci sarebbero ancora medici e medicine. La scienza può mantenervi in vita se la magia fallisse. Sarebbe una tragedia, ma non letteralmente la fine di tutta la luce del mondo. Tutto qui.»

Draco era arretrato di alcuni metri e il suo volto era pieno di paura mista a incredulità. «\textit{Nel nome di Merlino di cosa stai parlando, Potter?}»

«Ehi, ho ascoltato la \textit{tua} storia, non vuoi ascoltare la mia?» \textit{Goffo}, Harry si rimproverò, ma Draco smise realmente di arretrare e sembrò ascoltare.

«Ad ogni modo», disse Harry, «sto dicendo che pare che non abbiate fatto molta attenzione a ciò che accade nel mondo dei Babbani.» Probabilmente perché l’intero mondo magico sembrava considerare il resto della Terra come un bassofondo, meritevole della stessa copertura mediatica che il \textit{Financial Times} dedicava alle consuete agonie del Burundi. «D’accordo. Rapido controllo. I maghi sono mai stati sulla Luna? Sai, quella cosa li?» Harry indicò quel globo enorme e distante.

«\textit{Che cosa?}» disse Draco. Era abbastanza chiaro che il pensiero non era mai venuto al ragazzo. «\textit{Andare} su — è solo una –» Il suo dito puntava al coso pallido nel cielo. «Non ci si può Materializzare da qualche parte dove non si è mai stati e come potrebbe qualcuno arrivare sulla Luna, tanto per \textit{cominciare?}»

«Aspetta», disse Harry a Draco, «vorrei mostrarti un libro che ho portato con me, credo di ricordare in che scatola è.» E Harry si alzò e si inginocchiò ed estrasse le scale per il livello sotterraneo del suo baule, poi si precipitò giù per le scale e gettò via una scatola da sopra un’altra scatola, giungendo pericolosamente vicino a trattare i propri libri senza rispetto, e strappò via il coperchio della scatola e in fretta, ma con attenzione, diede un’occhiata alla pila di libri –

(Harry aveva ereditato la capacità quasi-magica dei Verres di ricordare dov’erano tutti i suoi libri, anche dopo averli visti solo una volta, cosa che era piuttosto misteriosa considerando la mancanza di qualsiasi collegamento genetico.)

E Harry corse su per la scala e la spinse di nuovo nel baule con il tallone, e, ansimando, girò le pagine del libro fino a trovare la fotografia che voleva mostrare a Draco.
Quella con il terreno bianco, secco, pieno di crateri, e le persone con le tute, e il globo bianco e blu che incombeva su tutto.


Quella fotografia.

\textit{La} fotografia, se una sola immagine in tutto il mondo fosse sopravvissuta.

«Così», disse Harry, la voce tremante perché non riusciva a nascondere completamente l’orgoglio, «è come la Terra appare dalla Luna.»

Draco si chinò lentamente. C’era una strana espressione sul suo giovane volto. «Se questa è una fotografia \textit{vera}, perché non si muove?»

\textit{Muoversi?} Oh. «I Babbani possono realizzare immagini in movimento, ma hanno bisogno di una scatola più grande per mostrarle, non possono ancora farle stare sulle singole pagine di un libro.»

Il dito di Draco si spostò su una delle tute. «Che cosa sono quelli?» La sua voce cominciava a vacillare.

«Quelli sono esseri umani. Indossano tute che coprono tutto il corpo per fornire loro l’aria, perché non c’è aria sulla Luna.»

«È impossibile», sussurrò Draco. C’era terrore nei suoi occhi, e confusione totale. «Nessun Babbano potrebbe mai farlo. \textit{Come}…»

Harry riprese il libro, sfogliò le pagine finché non trovò ciò che cercava. «Questo è un razzo che sale. Il fuoco lo spinge sempre più in alto, fino a quando non arriva alla Luna.» Girò ancora le pagine. «Questo è un razzo a terra. Questo minuscolo granello accanto a lui è una persona.» Draco rimase a bocca aperta. «Andare sulla Luna è costato l’equivalente di… probabilmente circa mille milioni di galeoni.» A Draco mancò il respiro. «E ci sono voluti gli sforzi di… probabilmente più persone di quante vivano in tutta la Gran Bretagna magica.» \textit{E quando sono arrivati, hanno lasciato una targa che diceva, ‘Siamo venuti in pace, per tutta l’umanità’. Anche se non sei ancora pronto per sentire queste parole, Draco Malfoy…}

«Stai dicendo la verità», disse Draco lentamente. «Non falsificheresti un libro intero solo per questo — e posso sentirlo nella tua voce. Ma… ma…»

«Come, senza bacchette o magia? È una lunga storia, Draco. La scienza non funziona agitando bacchette e recitando incantamenti, funziona sapendo come l’universo opera a un livello così profondo che si sa esattamente cosa fare per fargli fare quello che si vuole. Se la magia è come lanciare \textit{Imperio} su qualcuno per fargli fare quello che vuoi, allora la scienza è come conoscerlo così bene da poterlo convincere che sia sempre stata una sua idea. È molto più difficile che muovere una bacchetta, ma funziona quando le bacchette falliscono, proprio come, quando l’\textit{Imperius} fallisce, puoi ancora tentare di persuadere la persona. E la Scienza cresce di generazione in generazione. Devi \textit{veramente} sapere quello che stai facendo per fare scienza — e quando capisci davvero qualcosa, puoi spiegarla a qualcun altro. I più grandi scienziati di un secolo fa, le personalità più brillanti di cui si parla ancora con riverenza, i loro poteri non sono \textit{nulla} per i più grandi scienziati di oggi. Non vi è alcun equivalente nella scienza delle vostre arti perdute che hanno fondato Hogwarts. Nella scienza i nostri poteri aumentano ogni anno. E stiamo cominciando a capire e svelare i segreti della vita e dell’ereditarietà. Saremo in grado di guardare lo stesso sangue di cui hai parlato, e capire che cosa fa di te un mago, e tra una o due generazioni, saremo in grado di convincere quel sangue a rendere anche tutti i tuoi figli dei potenti maghi. Capisci ora, il tuo problema non è così grave come sembra, perché in un paio di decenni la scienza sarà in grado di risolverlo per te.»

«Ma…» disse Draco. La sua voce stava tremando. «Se i Babbani hanno questo tipo di potere… allora… che cosa siamo \textit{noi?}»

«No, Draco, non è così, non capisci? La scienza attinge al potere dell’intelletto umano di guardare il mondo e comprendere come funziona. Non può venir meno senza che venga meno l’umanità stessa. La tua magia potrebbe abbandonarti, e questo lo odieresti, ma tu saresti ancora \textit{tu}. Saresti ancora vivo per rammaricartene. Ma poiché la scienza si basa sulla mia intelligenza umana, è il potere che non può essere rimosso da me senza rimuovere \textit{me}. Anche se le leggi dell’universo cambiassero attorno a me, in modo che tutta la mia conoscenza non fosse più valida, mi limiterei a scoprire le nuove leggi, come è stato fatto in passato. Non è una caratteristica \textit{babbana}, è una caratteristica \textit{umana}, semplicemente affina e allena il potere che utilizzi ogni volta che guardi qualcosa che non comprendi e ti chiedi ‘Perché?’ Sei di Serpeverde, Draco, non capisci le implicazioni?»

Draco alzò lo sguardo dal libro di Harry. Il suo volto mostrava una comprensione nascente. «I Maghi possono imparare a usare questo potere.»

Con molta cautela, ora… l’esca è pronta, adesso l’amo… «Se riesci a imparare a pensare a te stesso come a un \textit{essere umano} invece che come a un \textit{mago}, allora puoi addestrare e perfezionare i tuoi poteri da essere umano.»

E se \textit{quell}’insegnamento non era in \textit{ogni} programma di scienze, Draco non aveva bisogno di saperlo, giusto?

Lo sguardo di Draco era pensieroso. «Tu l’hai… già fatto?»

«In un certo senso», ammise Harry. «La mia formazione non è completa. Non a undici anni. Ma — mio padre mi ha pagato \textit{anche} dei precettori, sai.» Certo, erano stati studenti universitari affamati, ed era stato solo perché Harry dormiva con un ciclo di 26 ore, ma meglio mettere tutto ciò da parte, per ora…

Lentamente, Draco annuì. «Pensi di poter padroneggiare \textit{entrambe} le arti, sommare i poteri insieme, e…» Draco fissò Harry. «Farti Signore dei due mondi?»

Harry fece una risata malvagia, sembrò venirgli naturale a quel punto. «Devi capire, Draco, che l’intero mondo che conosci, tutta la Gran Bretagna magica, è solo una casella di un tabellone da gioco molto più grande. Un tabellone che comprende luoghi come la Luna, e le stelle nel cielo notturno, che sono luci proprio come il Sole solo incredibilmente lontane, e cose come le galassie che sono enormemente più grandi rispetto alla Terra e al Sole, cose così grandi che solo gli scienziati possono vederle e che non sai neppure che esistono. Ma \textit{sono} realmente un Corvonero, sai, non un Serpeverde. Io non voglio dominare l’universo. Penso solo che potrebbe essere organizzato in maniera più sensata.»

C’era stupore sul volto di Draco. «Perché stai dicendo queste cose \textit{a me?}»

«Oh… non ci sono molte persone che sappiano fare \textit{vera} scienza — comprendere qualcosa per la prima volta, anche se ti confonde incredibilmente. Un aiuto sarebbe utile.»

Draco fissò Harry a bocca aperta.

«Ma non t’ingannare, Draco, la vera scienza \textit{non} è come la magia, non puoi praticarla e andartene via incontaminato come fai quando impari a pronunciare le parole di un nuovo incantesimo. Il potere ha un suo costo, un costo così alto che la maggior parte delle persone si rifiuta di pagarlo.»

Draco annuì in risposta, come se, finalmente, avesse sentito qualcosa che poteva comprendere. «E quel costo è… ?»

«Imparare ad ammettere di avere torto.»

«Uhm», Draco disse dopo che la pausa drammatica si era prolungata per un po’. «Hai intenzione di spiegarlo?»

«Quando si prova a immaginare in che modo qualcosa funziona a un livello così profondo, le prime novantanove spiegazioni che trovi sono sbagliate. La centesima è giusta. Quindi devi imparare ad ammettere che ti sbagli, più e più e più volte. Non sembra molto, ma è così difficile che gran parte delle persone non può fare scienza. Sempre mettersi in discussione, sempre ricontrollare le cose che hai dato per scontate», come avere un Boccino a Quidditch, «e ogni volta che cambi idea, cambi te stesso. Ma sto precorrendo i tempi. Li sto precorrendo parecchio. Voglio solo che tu sappia… che ti sto offrendo di condividere alcune delle mie conoscenze. Se vuoi. C’è solo una condizione.»

«Ah-a», fece Draco. «Sai, secondo mio Padre quando qualcuno ti dice così, non è mai un buon segno, mai.»

Harry annuì. «Ora, non fraintendermi pensando che stia cercando di scavare un solco tra te e tuo padre. Non si tratta di questo. È solo che voglio avere a che fare con qualcuno della mia età, e non che questo sia tra me e Lucius. Credo che anche tuo padre sarebbe d’accordo con ciò, lui sa che devi crescere prima o poi. Ma le tue mosse nella nostra partita devono essere tue. Questa è la mia condizione — che abbia a che fare con te, Draco, non con tuo padre.»

«Devo andare», disse Draco. Si alzò in piedi. «Devo andare via e rifletterci.»

«Prenditi il tuo tempo», disse Harry.

I suoni della piattaforma del treno mutarono da rumori indistinti a mormorii mentre Draco si allontanava.

Harry espirò lentamente l’aria che aveva trattenuto senza neppure rendersene conto, e poi guardò l’orologio che aveva al polso, un semplice modello meccanico che suo padre gli aveva comprato nella speranza che avrebbe funzionato in presenza di magia. La lancetta dei secondi stava ancora ticchettando, e se la lancetta dei minuti era corretta, allora non erano ancora le undici. Probabilmente sarebbe dovuto salire sul treno presto e iniziò la ricerca di comesichiama, ma gli parve che valesse la pena prendersi prima un paio di minuti per fare alcuni esercizi di respirazione e vedere se il suo sangue si sarebbe nuovamente scaldato.

Ma quando Harry alzò lo sguardo dal suo orologio, vide avvicinarsi due figure, assolutamente ridicole coi loro volti nascosti da sciarpe invernali.

«Salve, signor Bronzo», disse una delle figure mascherate. «Possiamo convincerla ad aderire all’Ordine del Caos?»

\begin{figure}[h]
	\includegraphics[scale=0.4]{boccino.png}
	\centering
\end{figure}

\subsubsection{Conseguenze}

Non molto tempo dopo, quando tutto il trambusto di quel giorno si fu finalmente placato, Draco stava chino sulla scrivania con la penna in mano. Aveva una stanza privata nei sotterranei di Serpeverde, con una scrivania propria e un proprio camino — purtroppo neppure \textit{lui} aveva diritto al collegamento al sistema via camino, ma almeno Serpeverde non aveva adottato quell’assurdità totale di far dormire \textit{tutti} nei dormitori. Non c’erano molte stanze private, bisognava essere il meglio \textit{in assoluto} all’interno della Casa dei migliori, ma questo poteva essere dato per scontato con Casa Malfoy.

\vspace{1em}
\begin{addmargin}[3em]{3em}% 1em left, 2em right
\begin{itpars}
Caro Padre, scrisse Draco.
\end{itpars}
\end{addmargin}
\vspace{1em}

E poi si fermò.

L’inchiostro gocciolò lentamente dalla sua penna, macchiando la pergamena vicino le parole.

Draco non era stupido. Era giovane, ma i suoi precettori l’avevano addestrato bene. Sapeva che probabilmente Potter si sentiva molto più vicino alla fazione di Silente di quanto non lasciasse intendere… anche se Draco pensava che Potter potesse essere tentato. Ma era chiaro che Potter stava cercando di tentare Draco proprio come Draco stava cercando di tentare lui.

Ed era anche chiaro che Potter era brillante, e ben più che un po’ matto, e stava giocando una vasta partita che Potter stesso per lo più non capiva, improvvisando a velocità massima con la sottigliezza di un Nundu infuriato. Ma Potter era riuscito a scegliere una tattica che Draco non poteva semplicemente ignorare. Aveva offerto a Draco una parte del proprio potere, puntando sul fatto che Draco non potesse usarlo senza diventare più simile a lui. Suo padre l’aveva definita una tecnica avanzata, e aveva avvertito Draco che spesso non funzionava.

Draco sapeva di non aver capito tutto quanto era accaduto… ma Potter aveva offerto a \textit{lui} la possibilità di giocare e in quel momento era sua. E se avesse spifferato tutto, sarebbe diventata di suo Padre.

Alla fine era così semplice come sembrava. Le tecniche minori richiedevano l’inconsapevolezza della vittima, o almeno la sua incertezza. L’adulazione doveva essere plausibilmente mascherata da ammirazione. («Saresti dovuto essere in Serpeverde» era un vecchio classico, altamente efficace su un certo tipo di persone che non se l’aspettavano, e se funzionava lo potevi ripetere.) Ma quando scovavi la leva decisiva di qualcuno, non importava che sapesse che tu sapevi. Potter, nella sua frenesia, aveva intuito la chiave per l’anima di Draco. E se Draco sapeva che Potter sapeva — anche se era stata un tipo di ipotesi scontata — questo non cambiava nulla.

Così ora, per la prima volta in vita sua, aveva dei veri segreti da mantenere. Stava giocando la propria partita. C’era un oscuro dolore in quello, ma sapeva che suo Padre sarebbe stato fiero, e quello rendeva tutto appropriato.

Lasciando le macchie di inchiostro al loro posto — c’era un messaggio in esse, e uno che suo padre avrebbe capito, perché avevano giocato il gioco delle sottigliezze più di una volta — Draco mise per iscritto l’unica domanda che l’aveva davvero tormentato in tutta la vicenda, la parte che sembrava che avrebbe \textit{dovuto} capire, ma non capiva, non del tutto.

\vspace{1em}
\begin{addmargin}[3em]{3em}% 1em left, 2em right
\begin{itpars}
Caro Padre,

immaginate che vi dica che ho incontrato uno studente di Hogwarts, non ancora membro della nostra cerchia di conoscenze, che vi definisce ‘un impeccabile strumento di morte’ e ha detto che sono il vostro ‘unico punto debole’. Cosa direste di lui?
\end{itpars}
\end{addmargin}
\vspace{1em}

Non ci volle molto tempo prima che la civetta di famiglia portasse la risposta.

\vspace{1em}
\begin{addmargin}[3em]{3em}% 1em left, 2em right
\begin{itpars}
Mio amato figlio,

direi che sei stato così fortunato da incontrare qualcuno che gode della fiducia intima del nostro amico e prezioso alleato, Severus Snape.
\end{itpars}
\end{addmargin}
\vspace{1em}

Draco fissò la lettera per un po’, e infine la gettò nel fuoco.
