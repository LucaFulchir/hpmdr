% !TeX root = Harry.tex

\chapter{Una giornata molto poco probabile}
\label{capitolo:1}

\emph {Al chiaro di luna risplende un piccolo frammento d’argento, una frazione di una linea…\\
(vesti neri, che cadono)\\
… sangue fuoriesce a litri, e qualcuno urla una parola.}

~\\


Ogni centimetro delle pareti è coperto da una libreria. Ogni libreria ha sei scaffali, che arrivano quasi fino al soffitto. Su alcuni scaffali sono impilati fino alla sommità libri dalla copertina rigida: scienza, matematica, storia, e tutto il resto. Altri scaffali hanno due file di fantascienza brossurata, con la fila di libri posteriore rialzata su vecchie scatole di fazzoletti o listelli di legno, così che sia possibile vederla sopra i libri davanti a essa. E non è ancora abbastanza. Libri traboccano dai tavoli e dalle poltrone e formano dei mucchietti sotto alle finestre.

Questo è il salotto della casa occupata dall’eminente professore Michael Verres-Evans e da sua moglie, la signora Petunia Evans-Verres, e dal loro figlio adottivo, Harry James Potter-Evans-Verres.

C’è una lettera sul tavolo del soggiorno, e una busta di pergamena giallastra senza francobollo, indirizzata al \emph{signor H. Potter} in inchiostro verde smeraldo.

Il professore e sua moglie parlano l’uno con l’altra molto bruscamente, ma non stanno urlando. Il professore ritiene che urlare sia incivile.

«Stai scherzando», disse Michael a Petunia. Il suo tono indicava che aveva una gran paura che dicesse sul serio.

«Mia sorella era una strega», ripeté Petunia. Sembrava spaventata, ma tenne duro. «Suo marito era un mago.»

«Questo è assurdo!» affermò Michael con decisione. «Erano al nostro matrimonio — sono venuti da noi a Natale –»

«Avevo detto loro che tu dovevi essere tenuto all’oscuro», sussurrò Petunia. «Ma è vero, ho visto cose –»

Il professore alzò gli occhi al cielo. «Cara, capisco che tu non abbia familiarità con la letteratura scettica. Potresti non comprendere come sia facile per un prestigiatore esperto simulare ciò che è apparentemente impossibile. Ricordi come insegnai a Harry a piegare i cucchiai? Se ti sembrava che potessero sempre indovinare ciò che stavi pensando, sappi che si chiama ‘lettura a freddo’ –»

«Non si trattava di cucchiai piegati –»

«Cos’era, allora?»

Petunia si morse il labbro. «Non posso dirtelo. Penseresti che sono –» Deglutì. «Ascolta, Michael. Non sono — sempre stata così –» Fece un gesto verso sé stessa, come a indicare la propria figura flessuosa. «Lily ha fatto questo. Perché — perché l’ho \emph{implorata}. Per anni, l’ho implorata. Lily era \emph{sempre} stata più bella di me, e io… ero stata cattiva con lei, per questo motivo, e quando ebbe anche la \emph{magia}, puoi immaginare come mi sono sentita? E l’ho \emph{implorata} di usare un po’ di quella magia su di me in modo che potessi essere bella anch’io, anche se non avrei potuto avere la sua magia, almeno sarei potuta essere bella.»

Delle lacrime stavano salendo agli occhi di Petunia.

«E Lily mi rispondeva di no, e si inventava le scuse più ridicole, come che il mondo sarebbe finito se fosse stata gentile con sua sorella, o che un centauro le aveva detto di non farlo — le scuse più ridicole, e la odiavo per questo. E quando mi ero appena laureata, uscivo con questo ragazzo, Vernon Dursley, era grasso ed era l’unico ragazzo che mi rivolgesse la parola. E disse che voleva dei bambini, e che il suo primo figlio si sarebbe chiamato Dudley. E dissi a me stessa, \emph{che razza di genitore chiamerebbe suo figlio Dudley Dursley}? Fu come veder scorrere tutto il mio futuro scorrere davanti ai miei occhi, e non potei sopportarlo. E scrissi a mia sorella e le dissi che se non mi avesse aiutata avrei preferito –»
Petunia si fermò.

«Ad ogni modo», riprese con voce sommessa, «cedette. Mi disse che sarebbe stato pericoloso, e risposi che non mi importava più, e bevetti quella pozione e caddi ammalata per settimane, ma quando tornai in salute la mia pelle si era schiarita e la mia figura si era finalmente riempita e io… ero bellissima, le persone erano \emph{gentili} con me», la sua voce si incrinò, «e dopo di questo non potei più odiare mia sorella, specie quando scoprii a cosa l’aveva portata infine la sua magia –»

«Cara», disse Michael con gentilezza, «ti sei ammalata, hai preso peso mentre riposavi a letto, e la tua pelle si è schiarita naturalmente. O ammalarti ti ha costretta a cambiare dieta –»

«Era una strega», ripeté Petunia. «L’ho visto.»

«Petunia», disse Michael. Il fastidio si stava lentamente insinuando nella sua voce. «Tu \emph{sai} che questo non può essere vero. Devo realmente spiegarti perché?»

Petunia si tormentava le mani. Sembrava essere sul punto di piangere. «Amore mio, so che non posso vincere una discussione con te, ma ti prego, su questo devi fidarti di me –»

«\emph{Papà! Mamma!}»

I due si fermarono e guardarono Harry come se si fossero dimenticati che c’era una terza persona nella stanza.

Harry fece un respiro profondo. «Mamma, i \emph{tuoi} genitori non usavano la magia, vero?»

«No», disse Petunia, l’espressione perplessa.

«Allora nessuno nella tua famiglia sapeva niente della magia quando Lily ricevette la sua lettera. Come si convinsero, \emph{loro}?»

«Ah…» fece Petunia. «Non ci mandarono solo una lettera. Inviarono un professore da Hogwarts. Egli –» Gli occhi di Petunia guizzarono in direzione di Michael. «Egli ci mostrò un po’ di magia.»

«Allora non è necessario che litighiate per questo», disse Harry con sicurezza. Sperando che questa volta, almeno questa volta, lo ascoltassero. «Se fosse vero, potremmo far venire qui un professore di Hogwarts e osservare la magia coi nostri occhi, e Papà dovrà ammettere che è vera. E se non fosse così, Mamma dovrà ammettere che è falsa. È per questo che esiste il metodo sperimentale, per non dover prendere decisioni solo con le discussioni.»

Il professore si voltò e guardò in basso verso di lui, sdegnoso come al solito. «Oh, andiamo, Harry. Sul serio, \emph{magia}? Pensavo che almeno \emph{tu} fossi abbastanza saggio da non prendere questa cosa sul serio, figliolo, anche se hai solo dieci anni. La magia è praticamente la cosa meno scientifica che ci sia!»

La bocca di Harry si contorse per l’amarezza. Era trattato bene, probabilmente meglio di quanto la maggior parte dei padri genetici trattassero i propri figli. Era stato mandato nelle migliori scuole elementari — e quando quello non aveva funzionato, aveva avuto precettori provenienti dall’inesauribile riserva degli studenti squattrinati. Era stato sempre incoraggiato a studiare qualunque cosa catturasse la sua attenzione, aveva ricevuto tutti i libri che avessero acceso la sua fantasia, era stato sostenuto in qualunque competizione di matematica o di scienza alla quale si fosse iscritto. Gli era stata data qualsiasi cosa ragionevole desiderasse, tranne, forse, un minimo di rispetto. Difficilmente si poteva pretendere che un Dottore di ricerca che insegnava biochimica a Oxford potesse ascoltare i consigli di un ragazzino. Avrebbe ascoltato in modo da Mostrare Interesse, ovviamente; quello era ciò che un Buon Genitore avrebbe dovuto fare, e in quel modo, se avesse pensato di essere un Buon Genitore, avrebbe agito. Ma prendere un bambino di dieci anni sul serio? Impossibile.

A volte Harry avrebbe voluto urlare contro suo padre.

«Mamma», disse Harry. «Se vuoi vincere questa discussione con Papà, guarda nel secondo capitolo del primo libro de \emph{La fisica di Feynman}. Lì c’è una citazione secondo cui i filosofi dicono molto su ciò di cui la scienza avrebbe assolutamente bisogno, ed è tutto sbagliato, perché nella scienza l’unica regola è che l’osservazione è l’arbitro finale – che devi solo osservare il mondo e riferire ciò che vedi. Uhm… a memoria non riesco a ricordare dove trovare qualcosa che dica che l’ideale della scienza sia di risolvere le cose con gli esperimenti anziché con le discussioni –»

Sua madre guardò giù verso di lui e sorrise. «Grazie, Harry. Ma –» la sua testa si rialzò per fissare suo marito. «Non voglio vincere una discussione con tuo padre. Voglio che mio marito ascolti sua moglie che lo ama, e che abbia fiducia in lei almeno per questa volta –»

Harry chiuse brevemente gli occhi. \emph{Senza speranza}. Erano entrambi semplicemente senza speranza.

Ora i suoi genitori si stavano infilando di nuovo in una di \emph{quelle} discussioni, in cui sua madre cercava di far sentire suo padre in colpa, e suo padre cercava di far sentire sua madre stupida.

«Andrò nella mia stanza», annunciò Harry. La sua voce tremava un po’. «Per favore, cercate di non litigare troppo per questa faccenda, Mamma, Papà, sapremo presto di che si tratta, giusto?»

«Certo, Harry», disse il padre, e sua madre gli diede un bacio rassicurante, e poi continuarono a litigare mentre Harry saliva le scale diretto alla sua stanza.

Si chiuse la porta alle spalle e cercò di pensare.

La cosa divertente era che \emph{sarebbe dovuto} essere d’accordo con Papà. Nessuno aveva mai visto alcuna prova della magia, e, secondo la Mamma, c’era un intero mondo magico là fuori. Come si poteva tenere segreta una cosa simile? Con altra magia? Sembrava una scusa piuttosto sospetta.

Sarebbe dovuto essere evidente che la Mamma stesse scherzando, mentendo o fosse diventata pazza, in ordine crescente di orrore. Se Mamma stessa avesse scritto la lettera, questo avrebbe spiegato come fosse arrivata nella cassetta delle lettere senza francobollo. Un po’ di follia era molto, molto meno improbabile dell’eventualità che l’universo funzionasse davvero in quel modo.

Solo che una parte di Harry era assolutamente convinta che la magia fosse reale, e lo era stata sin dall’istante in cui aveva visto la presunta lettera della Scuola di Magia e Stregoneria di Hogwarts.

Harry si grattò la fronte, facendo una smorfia. \emph{Non credere a tutto quello che pensi}, aveva detto uno dei suoi libri.

Ma questa bizzarra certezza… Harry stava scoprendo di aspettarsi proprio che, insomma, un professore di Hogwarts si presentasse e agitasse una bacchetta e della magia ne uscisse fuori. La strana certezza non stava facendo alcuno sforzo per premunirsi contro la propria falsificazione — non stava inventando delle scuse anticipate per spiegare il motivo per cui non ci sarebbe stato alcun professore, o per cui il professore sarebbe stato in grado solo di piegare i cucchiai.

\emph{Da dove vieni, piccola strana previsione}? Harry indirizzò il pensiero al proprio cervello. \emph{Perché credo a ciò a cui credo}?

Di solito Harry era abbastanza bravo a rispondere a quella domanda, ma in quel caso particolare, non aveva alcuna \emph{idea} di che cosa il suo cervello stesse pensando.

Mentalmente, Harry scrollò le spalle. Una piastra di metallo applicata a una porta permetteva di spingerla, e una maniglia su una porta permetteva di tirarla, e la cosa da fare con un’ipotesi verificabile era metterla alla prova.

Prese un foglio di carta a righe dalla sua scrivania e cominciò a scrivere.

\vspace{1em}
\begin{addmargin}[3em]{3em}% 1em left, 2em right
\begin{itpars}
Gentile Vicepreside
\end{itpars}
\end{addmargin}
\vspace{1em}

Harry si fermò, riflettendo; poi sostituì il foglio con un altro, appuntendo di un altro millimetro la mina in grafite della sua portamina. Tutto ciò richiedeva una calligrafia accurata.

\vspace{1em}
\begin{addmargin}[3em]{3em}% 1em left, 2em right
\begin{itpars}
Gentile Vicepreside Minerva McGonagall,

O chiunque sia competente:

Ho recentemente ricevuto la sua lettera di accettazione a Hogwarts, indirizzata al signor H. Potter. Potrebbe non essere al corrente del fatto che i miei genitori genetici, James Potter e Lily Potter (già Lily Evans) sono morti. Sono stato adottato dalla sorella di Lily, Petunia Evans-Verres, e da suo marito, Michael Verres-Evans.

Sono estremamente interessato a frequentare Hogwarts, a condizione che tale posto esista realmente. Soltanto mia madre dice di essere a conoscenza della magia, e non può usarla lei stessa. Mio padre è estremamente scettico. Io stesso sono incerto. Non so neppure dove procurarmi nessuno dei libri o degli oggetti elencati nella sua lettera di accettazione.

Mia madre ha menzionato il fatto che mandaste un rappresentante di Hogwarts da Lily Potter (all’epoca Lily Evans) allo scopo di dimostrare alla sua famiglia che la magia è reale, e, presumo, per aiutare Lily a procurarsi il suo corredo scolastico. Se potesse fare la stessa cosa per la mia famiglia, sarebbe estremamente d’aiuto.

Cordialmente,

Harry James Potter-Evans-Verres.
\end{itpars}
\end{addmargin}
\vspace{1em}


Harry aggiunse il loro indirizzo attuale, poi ripiegò la lettera e la mise in una busta, che indirizzò a Hogwarts. Un’ulteriore riflessione lo portò a procurarsi una candela e a far gocciolare della cera sul lembo della busta, e su di essa, utilizzando la punta di un temperino, incise le iniziali \textsc{h.j.p.e.v.} Se doveva sprofondare in quella follia, l’avrebbe fatto con stile.

Poi aprì la porta e tornò al piano di sotto. Suo padre era seduto in salotto e leggeva un libro di matematica avanzata per dimostrare quanto fosse intelligente; e sua madre era in cucina a preparare uno dei piatti preferiti di suo padre per dimostrare quanto fosse amorevole. Sembrava che non stessero affatto parlandosi. Per quanto spaventosi potessero essere i litigi, \emph{non litigare} era in qualche modo peggio.

«Mamma», disse Harry in quel silenzio snervante, «ho intenzione di mettere alla prova l’ipotesi. Secondo la tua teoria, come mando un gufo a Hogwarts?»

Sua madre si voltò dal lavandino per fissarlo, sembrando turbata. «Non — non lo so, penso che tu debba semplicemente avere un gufo magico.»

Quello sarebbe dovuto sembrare molto sospetto, \emph{oh, quindi non c’è modo di verificare la tua teoria, allora}, ma la bizzarra certezza dentro Harry sembrò essere disposta a rischiare ancora di più.

«Beh, in qualche modo la lettera è arrivata qui», disse Harry, «quindi la porterò fuori e griderò ‘lettera per Hogwarts!’ e vedrò se un gufo la preleva. Papà, vuoi venire a vedere?»

Suo padre scosse appena la testa e continuò a leggere. \emph{Naturalmente}, Harry pensò rivolto a sé stesso. La magia era qualcosa di disonorevole in cui credevano solo le persone stupide; se suo padre fosse arrivato anche solo a \emph{mettere alla prova} l’ipotesi, o persino a \emph{osservare} mentre era messa alla prova, gli sarebbe parso di \emph{associarsi} ad essa…

Solo mentre usciva con passo pesante dalla porta posteriore, diretto al giardino sul retro, a Harry sovvenne che se un gufo fosse davvero sceso dal cielo ad afferrare la lettera, avrebbe avuto qualche problema a raccontarlo a suo padre.

\emph{Ma} — beh — \emph{questo non può accadere sul serio, giusto? Non importa ciò che il mio cervello sembra credere. Se un gufo dovesse realmente scendere a prendere questa busta, avrei problemi molto più gravi di ciò che Papà potrebbe pensare.}

Harry fece un respiro profondo, e alzò per aria la busta.

Deglutì.

Gridare \emph{Lettera per Hogwarts!} mentre teneva una busta per aria nel mezzo del suo giardino sarebbe stato… in effetti piuttosto imbarazzante, ora che ci pensava.

\emph{No. Sono migliore di Papà. Userò il metodo scientifico anche se mi fa sentire stupido.}

«Lettera –» disse Harry, ma in realtà gli venne più simile a un gracidio sussurrato.

Harry rafforzò la propria volontà e gridò al cielo vuoto, «\emph{Lettera per Hogwarts! Posso avere un gufo?}»

«Harry?» chiese la voce sconcertata di una donna, uno dei suoi vicini.

Harry tirò giù la mano come se si fosse scottato e nascose la busta dietro la schiena, come se fosse denaro illegale. Il suo viso bruciava tutto per la vergogna.

Il volto dell’anziana signora fece capolino sopra la vicina staccionata, capelli grigi brizzolati che sfuggivano dalla retina. La signora Figg, occasionalmente sua bambinaia. «Cosa stai facendo, Harry?»

«Nulla», disse Harry con voce strozzata. «Solo — mettendo alla prova una teoria molto stupida –»

«Hai ricevuto la tua lettera di accettazione da Hogwarts?»

Harry rimase paralizzato.

«Sì», dissero le labbra di Harry poco dopo. «Ho ricevuto una lettera da Hogwarts. Dicono che vogliono il mio gufo entro il 31 luglio, ma –»

«Ma tu \emph{non hai} un gufo. Povero ragazzo! Non posso immaginare \emph{cosa} una certa persona deve aver pensato, per mandarti solo la lettera ordinaria.»

Un braccio rugoso si allungò da sopra la staccionata, e aprì una mano in attesa. A quel punto Harry era a malapena in grado di pensare, e consegnò la sua busta.

«Lascia fare a me, caro», disse la signora Figg, «e in un attimo o due farò venire qualcuno.»

E il suo viso scomparve dietro la staccionata.

Ci fu un lungo silenzio nel giardino.

E poi la voce di un ragazzo disse, calma e pacata, «Cosa.»
