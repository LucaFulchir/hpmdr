% !TeX root = Harry.tex

\chapter{L’errore di pianificazione}
\label{capitolo:6}

\emph{Pensi che la tua giornata sia stata surreale? Prova la mia.}

~\\
~\\

Alcuni bambini avrebbero atteso fino a \textit{dopo} il loro primo viaggio a Diagon Alley.

«Sacchetto di elemento 79», disse Harry, e ritirò la mano, vuota, dalla borsa mokeskin.

La maggior parte dei bambini avrebbe almeno aspettato di ricevere la propria \textit{bacchetta}, prima.

«Sacchetto di \textit{okane}», disse ancora Harry. Il pesante borsello d’oro gli comparve in mano.

Harry estrasse il sacchetto, poi lo mise nuovamente nella borsa mokeskin. Tirò fuori la mano, la rimise dentro e disse: «Sacchetto di strumenti di scambio economico.» Questa volta la sua mano uscì vuota.

«Dammi il sacchetto che ho appena messo dentro.» Il sacchetto d’oro apparve, ancora una volta.

Harry James Potter-Evans-Verres aveva messo le mani su almeno un oggetto magico. Perché aspettare?

«Professoressa McGonagall», disse Harry alla strega sconcertata che passeggiava accanto a lui, «mi può dire due parole, una che significhi oro e una qualcos’altro che non sia denaro, in una lingua che non conosco? Ma non mi dica quale delle due significa oro e quale no.»

«\textit{Ahava} e \textit{zahav}», disse la professoressa McGonagall. «È ebraico, l’altra parola significa amore.»

«Grazie, professoressa. Sacchetto di \textit{ahava}.» Vuota.

«Sacchetto di \textit{zahav}.» E gli apparve in mano.

«\textit{Zahav} è oro?» domandò Harry, e la professoressa McGonagall annuì.

Harry rifletté sui dati sperimentali che aveva raccolto. Era solo un tentativo incompleto e preliminare, ma era stato sufficiente a giungere almeno a una conclusione:

«\textit{Aaaaaaarrrgh tutto questo non ha alcun senso!}»

La strega accanto a lui alzò un sopracciglio altero. «Problemi, signor Potter?»

«Ho appena falsificato ogni singola ipotesi che avevo! Come può sapere che ‘borsa di 115 galeoni’ va bene, ma non ‘borsa di 90 galeoni più 25’? Può \textit{contare}, ma non può \textit{sommare}? Può capire i nomi, ma non i sintagmi nominali che significano la stessa cosa? Le persone che l’hanno realizzata probabilmente non parlano giapponese e \textit{io} non parlo ebraico, quindi non sta usando la \textit{loro} conoscenza e non sta usando la \textit{mia} conoscenza –» Harry agitò una mano impotente. «Le regole sembrano \textit{all’incirca} coerenti ma non significano niente! Non sto neppure chiedendo come una \textit{borsa} finisca per avere il riconoscimento vocale e la comprensione del linguaggio naturale, quando i migliori programmatori di Intelligenza Artificiale non riescono a far sì che i super-computer più veloci ce l’abbiano dopo 35 anni di duro lavoro», Harry annaspò, «ma che \textit{cosa} sta \textit{succedendo}?»

«Magia», disse la professoressa McGonagall.

«È solo una \textit{parola!} Anche dopo che me la dice, non posso fare nessuna nuova predizione! È esattamente come dire ‘flogisto’ o ‘\textit{elan vital}’ o ‘emergenza’ o ‘complessità’!»

La strega vestita di nero si mise a ridere forte. «Ma \textit{è} magia, signor Potter.»

Harry si abbatté un po’. «Con tutto il rispetto, professoressa McGonagall, non sono molto sicuro che capisca quello che sto cercando di fare.»

«Con tutto il rispetto, signor Potter, io sono abbastanza sicura di non capirlo. A meno che — questa è solo una supposizione, badi bene — lei stia cercando di conquistare il mondo?»

«No! Voglio dire sì — beh, \textit{no!}»

«Penso che forse dovrei essere allarmata dal fatto che abbia difficoltà a rispondere alla domanda.»

Harry considerò tristemente la Conferenza di Dartmouth sull’Intelligenza Artificiale nel 1956. Era stata la prima conferenza sul tema, quella che aveva coniato la frase «Intelligenza Artificiale». Avevano individuato i problemi chiave, come far capire ai computer il linguaggio naturale, come insegnare loro a imparare e come migliorare sé stessi. Avevano indicato, in assoluta serietà, che progressi significativi su tali problemi sarebbero potuti essere ottenuti da dieci scienziati che avessero lavorato insieme per due mesi.

\textit{No. Tirati su. Sei solo all’inizio dell’impresa di svelare tutti i segreti della magia. Non sai realmente se sarà troppo difficile farlo in due mesi.}

«E davvero non ha sentito parlare di altri maghi che si siano posti questo tipo di domande o abbiano fatto questo tipo di esperimenti scientifici?» Harry chiese di nuovo. Gli pareva così \textit{ovvio}.

Ma del resto, c’erano voluti più di duecento anni dopo l’invenzione del metodo scientifico, prima che gli scienziati babbani avessero pensato di indagare sistematicamente quali frasi un \textit{essere umano di quattro anni} poteva o non poteva capire. La psicologia dello sviluppo della linguistica sarebbe potuta essere stata scoperta nel \textsc{xviii} secolo, in linea di principio, ma nessuno aveva pensato di studiarla fino al \textsc{xx}. Così non potevi davvero dare la colpa al molto più piccolo mondo dei maghi per non aver indagato l’Incantesimo di Recupero.

La professoressa McGonagall strinse le labbra, poi scrollò le spalle. «Non sono ancora sicura di cosa intenda per ‘sperimentazione scientifica’, signor Potter. Come le ho detto, ho visto studenti babbani cercare di far funzionare la scienza babbana all’interno di Hogwarts, e la gente inventa nuovi incantesimi e pozioni ogni anno.»

Harry scosse la testa. «Tecnologia e scienza non sono affatto la stessa cosa. E provare molti modi diversi di fare qualcosa non è lo stesso che sperimentare per capire le regole.» C’erano state molte persone che avevano cercato di inventare macchine volanti provando un sacco di cose-con-le-ali, ma solo i fratelli Wright avevano costruito una galleria del vento per misurare la portanza… «Ehm, quanti bambini cresciuti da Babbani \textit{avete} a Hogwarts ogni anno?»

«Forse dieci o giù di lì?»

Harry inciampò sui propri passi e quasi cadde. «\textit{Dieci?}»

Il mondo babbano aveva una popolazione di sei miliardi e in crescita. Se eri uno su un milione, ce n’erano sette come te a Londra e un migliaio in più in Cina. Era inevitabile che la popolazione babbana producesse \textit{alcuni} undicenni che conoscessero il calcolo infinitesimale — Harry sapeva di non essere il solo. Aveva incontrato altri prodigi nelle competizioni matematiche. In realtà era stato meticolosamente surclassato da concorrenti che probabilmente passavano letteralmente \textit{tutto il giorno} esercitandosi su problemi di matematica, che non avevano \textit{mai} letto un libro di fantascienza, che si sarebbero bruciati \textit{completamente} prima della \textit{pubertà} e non avrebbero \textit{mai} concluso \textit{nulla} nella loro vita futura, perché si erano semplicemente esercitati in tecniche \textit{conosciute}, invece di imparare a pensare in modo \textit{creativo}. (Harry era un perdente piuttosto permaloso.)

Ma… nel mondo dei maghi…

Dieci bambini allevati da Babbani all’anno, tutti che avevano terminato la loro istruzione babbana all’età di undici anni? E la professoressa McGonagall poteva essere di parte, ma aveva sostenuto che Hogwarts era la più grande e la più eminente scuola di magia nel mondo… e istruiva solo fino all’età di diciassette anni.

La professoressa McGonagall conosceva indubbiamente ogni dettaglio di come ci si trasformava in un gatto. Ma sembrava non aver mai letteralmente sentito parlare del metodo scientifico. Per lei era solo magia babbana. E non sembrava neppure \textit{curiosa} di quali segreti potessero nascondersi dietro la comprensione del linguaggio naturale dell’Incantesimo di Recupero.

Il che lasciava solo due possibilità.

Possibilità uno: la magia era così incredibilmente opaca, contorta e impenetrabile, che anche se maghi e streghe avevano fatto del loro meglio per capirla, avevano ottenuto risultati scarsi o nulli e alla fine avevano rinunciato; e Harry non avrebbe fatto di meglio.

Oppure…

Harry fece scrocchiare le nocche in segno di determinazione, ma produssero solo un clic sommesso, invece che riecheggiare sinistramente dai muri di Diagon Alley.

Possibilità due: egli avrebbe conquistato il mondo.

Prima o poi. Forse non subito.

Quel genere di imprese richiedevano \textit{talvolta} più di due mesi. La scienza babbana non era andata sulla Luna entro la prima settimana dopo Galileo.

Ma Harry non riuscì ugualmente a fermare l’ampio sorriso che deformava così tanto le sue guance, che gli stavano iniziando a fare male.

Harry aveva sempre avuto paura di finire come uno di quei bambini prodigio che non avevano mai concluso nulla, e che trascorrevano il resto della loro vita a vantarsi di quanto fossero stati bravi all’età di dieci anni. Ma del resto, la maggior parte dei geni adulti non concludeva ugualmente nulla. C’erano probabilmente un migliaio di persone intelligenti come Einstein per ciascun Einstein vero e proprio, nella storia. Perché quegli altri geni non avevano messo le mani sulla cosa di cui avevano assolutamente bisogno per raggiungere la grandezza. Non avevano mai trovato un problema importante.

\textit{Adesso siete miei}, pensò Harry rivolto ai muri di Diagon Alley, e a tutti i negozi e gli oggetti e a tutti i negozianti e i clienti; e a tutte le terre e le persone della Gran Bretagna magica, e a tutto il resto del mondo dei maghi; e a tutto il grande universo di cui gli scienziati babbani comprendevano molto meno di quanto credessero. \textit{Io, Harry James Potter-Evans-Verres, rivendico ora questo territorio in nome della Scienza.}

Fulmini e tuoni omisero completamente di illuminare e rimbombare nel cielo terso.

«Per quale motivo sta sorridendo?» domandò la professoressa McGonagall, con circospezione e un’aria stanca.

«Mi chiedo se c’è un incantesimo per far comparire un lampo sullo sfondo ogni volta che formulo una risoluzione sinistra», spiegò Harry. Memorizzò attentamente le esatte parole della sua risoluzione sinistra, in modo che i futuri libri di storia avrebbero potuto riportarla correttamente.

«Ho la netta sensazione che dovrei fare qualcosa a riguardo», sospirò la professoressa McGonagall.

«La ignori, passerà. Ooh, carino!» Harry mise i suoi pensieri di conquista del mondo temporaneamente in attesa e scattò verso un negozio con un espositore esterno, e la professoressa McGonagall lo seguì.

\begin{figure}[h]
\includegraphics[scale=0.4]{boccino.png}
\centering
\end{figure}

Harry aveva ora acquistato gli ingredienti per le pozioni e il calderone, e, oh, un paio di accessori. Oggetti che sembravano le cose giuste da portare nella Borsa Conservante di Harry (alias Moke Super Pouch \textsc{qx}31 con Incantesimo dell’Estensione Impercettibile, Incantesimo di Recupero e Bordo Allargante). Acquisti intelligenti e sensati.

Harry sinceramente non capiva perché la professoressa McGonagall sembrasse così \textit{sospettosa}.

In quel momento, Harry era in un negozio abbastanza costoso da essere posto lungo la tortuosa strada principale di Diagon Alley. Il negozio aveva una parte anteriore aperta, con la merce disposta su scaffalature di legno inclinate, e protetta solo da tenui bagliori grigi e da una commessa giovanile in una versione molto ridotta delle vesti da strega, che mostrava ginocchia e gomiti.

Harry stava esaminando l’equivalente magico di un kit di pronto soccorso, il Pacchetto Extra di Guarigione di Emergenza. C’erano due lacci emostatici auto-serranti. Una Pozione di Stabilizzazione, che avrebbe rallentato la perdita di sangue e prevenuto il trauma. Una siringa di quello che sembrava fuoco liquido, che avrebbe dovuto ridurre drasticamente la circolazione nella zona trattata, mantenendo al contempo l’ossigenazione del sangue per un massimo di tre minuti, se fosse stato necessario evitare che un veleno si diffondesse attraverso il corpo. Un panno bianco che poteva essere avvolto su di una parte del corpo per renderla temporaneamente insensibile al dolore. Più un numero imprecisato di altri gingilli che Harry era stato totalmente incapace di comprendere, come il «Trattamento contro l’Esposizione a un Dissennatore», che aveva l’aspetto e l’odore del normale cioccolato. Oppure l’«Antidoto per Confondimorso», che sembrava un piccolo uovo vibrante ed era accompagnato da un cartellino che mostrava come incastrarlo su per la narice di qualcuno.

«Per cinque galeoni è un acquisto certo, non è d’accordo?» chiese alla professoressa McGonagall, e la commessa adolescente che si librava là vicino annuì con entusiasmo.

Harry si era aspettato che la professoressa gli indirizzasse una qualche sorta di commento di approvazione a proposito della sua prudenza e preparazione.

Ciò che invece stava ricevendo poteva essere descritto solo come il Malocchio.

«Ed esattamente \textit{per quale motivo}», chiese la professoressa McGonagall con forte scetticismo, «si aspetta di \textit{aver bisogno} di un kit di guarigione, giovanotto?» (Dopo lo sfortunato incidente al negozio di Pozioni, la professoressa McGonagall stava cercando di evitare di dire «signor Potter», se qualcuno era nelle vicinanze.)

La bocca di Harry si aprì e si richiuse. «Non mi \textit{aspetto} di averne bisogno! È solo nell’eventualità!»

«Solo nell’eventualità di \textit{che cosa?}»

Harry spalancò gli occhi. «Lei pensa che stia \textit{progettando} di fare qualcosa di pericoloso ed è \textit{per questo} che voglio un kit medico?»

Uno sguardo di torvo sospetto e d’ironica incredulità fu la risposta.

«Grande Giove!» disse Harry. (Quella era un’espressione che aveva imparato dalla scienziato pazzo Doc Brown di \textit{Ritorno al futuro}). «Lo stava pensando anche quando ho comprato la Pozione Caduta-Piuma, l’Algabranchia, e le Pillole del Cibo e dell’Acqua?»

«Sì.»

Harry scosse la testa per lo stupore. «E che genere di piano pensa che stia \textit{tramando}, ora?»

«Non lo so», disse la professoressa McGonagall cupamente, «ma finisce con lei che consegna una tonnellata di argento a Gringotts, o nella conquista del mondo.»

«Conquista del mondo è una frase così brutta. Preferisco chiamarla ottimizzazione del mondo.»

Questa battuta esilarante non riuscì a tranquillizzare la strega, che stava rivolgendogli lo Sguardo del Fato.

«Uau», disse Harry, quando si rese conto che era seria. «Lo pensa davvero. Pensa davvero che stia progettando di fare qualcosa di pericoloso.»

«Sì.»

«Come se questo fosse \textit{l’unico} motivo per comprare un kit di pronto soccorso? Non la prenda male, professoressa McGonagall, ma \textit{con che razza di bambini folli è abituata a trattare?}»

«Grifondoro» esclamò la professoressa McGonagall, e la parola portò con sé un carico di amarezza e disperazione che ricadde come una maledizione eterna su ogni entusiasmo e buon umore giovanile.

«Vicepreside professoressa Minerva McGonagall», disse Harry, mettendo le mani sui fianchi in un atteggiamento severo. «Non ho intenzione di andare in Grifondoro –»

A questo punto la Vicepreside s’intromise dicendo qualcosa del tipo che se quello \textit{l’avesse fatto}, ella avrebbe scoperto come uccidere un cappello, una strana osservazione che Harry lasciò passare senza commenti, anche se la commessa sembrò avere un attacco di tosse improvvisa.

«– ho intenzione di andare in Corvonero. E se davvero pensa che stia progettando di fare qualcosa di pericoloso, allora, onestamente, non mi ha capito per \textit{niente}. Non mi \textit{piace} il pericolo, è \textit{spaventoso}. Sono \textit{prudente}. Sono \textit{cauto}. Mi sto preparando per \textit{eventualità impreviste}. Come i miei genitori cantavano per me: \textit{Siate pronti! Questa è la canzone di marcia dei Boy Scout! Siate pronti! Mentre marciate attraverso la vita! Non siate nervosi, non siate agitati, non siate timorosi — siate pronti!}»

(I genitori di Harry gli avevano cantato in effetti sempre e solo quei versi in particolare della canzone di Tom Lehrer, e Harry era beatamente inconsapevole di tutto il resto.)

La posizione della professoressa McGonagall si ammorbidì leggermente – anche se per lo più ciò avvenne quando Harry disse che era destinato a Corvonero. «Per che genere di \textit{eventualità} immagina che questo kit potrebbe \textit{prepararla}, giovanotto?»

«Una delle mie compagne di classe viene morsa da un mostro orribile, e mentre rovisto freneticamente nella mia borsa mokeskin in cerca di qualcosa che potrebbe aiutarla, lei mi guarda tristemente e con il suo ultimo respiro dice, ‘\textit{Perché non eri preparato?}’ E poi muore, e mentre i suoi occhi si chiudono so che non mi perdonerà mai –»

Harry udì gemere la commessa, e alzò lo sguardo per vederla mentre lo fissava con le labbra serrate. Poi la giovane donna si voltò e fuggì nei recessi più profondi del negozio.

\textit{Che cosa…?}

La professoressa McGonagall si chinò e prese la mano di Harry tra le proprie, delicatamente ma con fermezza, e tirò Harry fuori dalla strada principale di Diagon Alley, portandolo in un vicolo tra due negozi che era pavimentato in mattoni sporchi e terminava in un muro di solida terra nera.

L’alta strega puntò la bacchetta verso la strada principale e parlò, «\textit{Quietus}» disse, e uno schermo di silenzio scese intorno a loro, bloccando tutti i rumori della strada.

\textit{Che cosa ho fatto di sbagliato…}

La professoressa McGonagall si voltò a osservare Harry. Non aveva la completa Espressione da Rimprovero da adulto, ma la sua espressione era piatta, controllata. «Deve ricordare, signor Potter», disse, «che c’è stata una guerra in questo Paese non più di dieci anni fa. Tutti hanno perso qualcuno, e parlare di amici che muoiono tra le tue braccia — non va fatto con leggerezza.»

«Io — io non volevo –» L’implicazione cadde come un sasso nell’immaginazione eccezionalmente vivida di Harry. Aveva parlato di qualcuno che esalava il suo ultimo respiro — e poi la commessa era fuggita — e la guerra era finita dieci anni prima, quindi la ragazza avrebbe avuto al massimo di otto o nove anni, quando, quando, «Mi dispiace, io non volevo…» Harry si sentì soffocare, e si voltò per sfuggire allo sguardo dell’anziana strega, ma c’era un muro di terra che bloccava la sua strada e non aveva ancora la sua bacchetta. «Mi dispiace, mi dispiace, mi \textit{dispiace!}»

Ci fu un profondo sospiro alle sue spalle. «So che è così, signor Potter.»

Harry osò sbirciare dietro di sé. La professoressa McGonagall sembrava solo triste, ora. «Mi dispiace», disse Harry di nuovo, sentendosi miserabile. «Qualcosa di simile è accaduto a –» e poi Harry chiuse le labbra e mise una mano sulla bocca per sicurezza.

Il viso dell’anziana strega divenne un po’ più triste. «Deve imparare a pensare prima di parlare, signor Potter, oppure passerà la sua vita senza molti amici. Questo è stato il destino di molti Corvonero, e spero che non sia il suo.»

Harry voleva solo fuggire via. Voleva tirar fuori una bacchetta e cancellare l’intera faccenda dalla memoria della professoressa McGonagall, tornare con lei nuovamente fuori dal negozio, \textit{e fare in modo che non fosse mai accaduto} –

«Ma per rispondere alla sua domanda, signor Potter, no, niente \textit{del genere} è mai successo a me. Certo, ho visto un amico esalare il suo ultimo respiro, una o sette volte. Ma nessuno di loro mi ha mai maledetto mentre moriva, e non ho mai pensato che non mi avrebbe perdonato. Perché lei dovrebbe \textit{dire} una cosa del genere, signor Potter? Perché dovrebbe persino pensarla?»

«Io, io, io», Harry deglutì. «È solo che cerco sempre di immaginare la cosa peggiore che potrebbe accadere», e forse aveva anche scherzato un po’, ma avrebbe preferito mordersi la lingua piuttosto che dirlo ora.

«Che cosa?» disse la professoressa McGonagall. «Ma \textit{perché?}»

«Così posso impedire che accada!»

«Signor Potter…» la voce della vecchia strega si affievolì. Poi sospirò, e si inginocchiò accanto a lui. «Signor Potter», disse ora con dolcezza, «non è responsabilità sua prendersi cura degli studenti di Hogwarts. È mia. Non lascerò che accada nulla di male a lei o a chiunque altro. Hogwarts è il posto più sicuro per i maghi bambini in tutto il mondo della stregoneria, e Madam Pomfrey ha un’intera infermeria per le guarigioni. Non avrà affatto bisogno del kit di guarigione, figuriamoci di uno da cinque galeoni.»

«Invece \textit{sì!}» proruppe Harry. «\textit{Nessun luogo} è perfettamente sicuro! E se i miei genitori avessero un attacco di cuore o fossero coinvolti in un incidente quando tornerò a casa per Natale – Madam Pomfrey non sarà lì, avrò bisogno di un kit di guarigione tutto mio –»

«Ma in nome di Merlino, \textit{cosa…}» disse la professoressa McGonagall. Si alzò e guardò giù verso Harry, con un’espressione divisa tra il fastidio e la preoccupazione. «Non c’è bisogno di pensare a queste cose terribili, signor Potter!»

L’espressione di Harry si contorse diventando amarezza, a sentire quelle parole. «\textit{Sì} che c’è! Se non ci pensa, non solo si farà male, finirà per fare male ad altre persone!»

La professoressa McGonagall aprì la bocca, poi la richiuse. La strega si massaggiò la radice del naso, l’espressione pensierosa. «Signor Potter… se dovessi offrirle di stare ad ascoltarla per un po’… ci sarebbe qualcosa che desidererebbe raccontarmi?»

«A proposito di cosa?»

«A proposito del perché lei sia convinto di dover stare sempre in guardia contro le cose terribili che potrebbero accaderle.»

Harry la guardò perplesso. Si trattava di un assioma auto-evidente. «Beh…» iniziò Harry lentamente. Cercò di organizzare i pensieri. Come \textit{poteva} spiegarsi con una professoressa-strega, quando ella non conosceva neppure le basi? «Ricercatori babbani hanno scoperto che le persone sono sempre molto ottimiste, rispetto alla realtà. Del tipo, dicono che ci vorranno due giorni per terminare qualcosa e ci vogliono dieci giorni, oppure dicono che ci vorranno due mesi e ci vogliono più di 35 anni. Ad esempio, in un esperimento hanno chiesto a degli studenti di stimare il tempo entro cui erano sicuri al 50\%, al 75\% e al 99\% di completare i loro compiti, e solo il 13\%, il 19\% e il 45\% degli studenti terminò entro quelle stime. E hanno scoperto che la ragione era che quando chiesero a un gruppo la stima per il caso migliore, quello in cui tutto andava nel migliore dei modi, e a un altro gruppo la stima del caso medio, quello in cui tutto andava come al solito, ottenevano risposte che erano statisticamente indistinguibili. Vede, se chiede a qualcuno ciò che si aspetta nel caso \textit{normale}, egli immaginerà quella che sembra la linea di massima probabilità a ciascun passo lungo il percorso — tutto va secondo i piani, senza sorprese. Ma effettivamente, dal momento che più della metà degli studenti non rispettò la scadenza che erano sicuri al 99\% di rispettare, la realtà offre solitamente risultati un po’ peggiori rispetto allo ‘scenario peggiore’. Si chiama errore di pianificazione, e il modo migliore per correggerlo è quello di chiedere quanto tempo c’è voluto l’ultima volta che si è provato. Questo si chiama usare il punto di vista esterno invece di quello interno. Ma quando stiamo per fare qualcosa di nuovo e non possiamo procedere in questo modo, dobbiamo semplicemente essere molto, molto, molto pessimisti. Come dire, così pessimisti che la realtà si dimostra effettivamente \textit{migliore} di quanto ci aspettavamo più o meno tanto spesso e con uguale intensità di quando si dimostra peggiore. In realtà è \textit{talmente difficile} essere così pessimisti, che c’è una possibilità decente di \textit{sottostimare} la vita reale. Come dire, faccio questo grande sforzo di essere pessimista e immagino che uno dei miei compagni di classe venga morso, ma quello che succede in realtà è che i Mangiamorte sopravvissuti attaccano tutta la scuola per arrivare a me. Ma da un punto di vista più allegro –»

«Si fermi», disse la professoressa McGonagall.

Harry si fermò. Era appena stato sul punto di sottolineare che almeno sapeva che il Signore Oscuro non avrebbe attaccato, in quanto era morto.

«Penso di non essermi spiegata abbastanza chiaramente», disse la strega, il suo nitido accento scozzese che sembrò ancora più attento. «È successo qualcosa a \textit{lei personalmente} che l’ha spaventata, signor Potter?»

«Quello che è successo a me personalmente è solo una prova aneddotica», spiegò Harry. «Non ha lo stesso peso di un articolo scientifico replicato e revisionato a proposito di uno studio controllato con assegnazione casuale, molti soggetti, ampia \textit{effect size} e forte significatività statistica.»

La professoressa McGonagall strinse con le dita la radice del naso, inspirò e espirò. «Vorrei saperlo comunque», disse.

«Uhm…» iniziò Harry. Fece un respiro profondo. «C’erano stati alcuni scippi nel nostro quartiere, e mia madre mi chiese di restituire un tegame che aveva preso in prestito da una vicina a due strade di distanza, e risposi che non volevo, perché avrebbero potuto aggredirmi, e lei mi disse, ‘Harry, non dire cose del genere!’ Come se pensarci lo \textit{facesse} succedere, dunque se non ne avessi parlato, sarei stato al sicuro. Cercai di spiegare perché non ne ero rassicurato, e lei mi fece portare la pentola comunque. Ero troppo giovane per sapere quanto fosse statisticamente improbabile che uno scippatore mi scegliesse come vittima, ma ero abbastanza grande da sapere che non pensare a qualcosa non impedisce che accada, quindi ero davvero spaventato.»

«Nient’altro?», chiese la professoressa McGonagall dopo una pausa, quando divenne chiaro che Harry aveva finito. «Non c’è niente \textit{altro} che le è successo?»

«So che non \textit{sembra} gran che», si difese Harry. «Ma fu uno di quei momenti cruciali di una vita, sa? Voglio dire, \textit{sapevo} che non pensare a qualcosa non impediva che accadesse, lo \textit{sapevo}, ma mi resi conto che la mamma la pensava davvero in quel modo.» Harry si fermò, alle prese con la rabbia che stava iniziando a rimontare ora che ci ripensava. «\textit{Non mi voleva ascoltare}. Cercai di dirglielo, la pregai di non farmi uscire, e lei \textit{la mise sul ridere}. Tutto ciò che dicevo, lei lo considerava come una specie di scherzo…» Harry costrinse la rabbia nera a ritirarsi nuovamente. «In quel momento mi resi conto che tutti quelli che avrebbero dovuto proteggermi erano in realtà folli, e che non mi avrebbero ascoltato, non importava quanto li pregassi, e che non avrei potuto mai fare affidamento su di loro per fare qualcosa nel modo giusto.» A volte non era sufficiente essere bene intenzionati, a volte si doveva essere sani di mente…

Ci fu un lungo silenzio.

Harry si prese il tempo per respirare profondamente e calmarsi. Non c’era motivo di arrabbiarsi. Non c’era motivo di arrabbiarsi. \textit{Tutti} i genitori erano così, \textit{nessun} adulto si sarebbe abbassato abbastanza da mettersi sullo stesso piano di un bambino e ascoltarlo, i suoi genitori genetici non sarebbero stati diversi. La sanità mentale era una piccola scintilla nella notte, un’eccezione infinitamente rara alla regola della follia, quindi non c’era motivo di arrabbiarsi.

Harry non si piaceva quando era arrabbiato.

«Grazie per averlo condiviso, signor Potter», disse la professoressa McGonagall dopo un po’. C’era un’espressione distratta sul suo viso (quasi esattamente la stessa espressione che era apparsa sul volto di Harry durante l’esperimento con la borsa, se solo Harry si fosse visto in uno specchio). «Dovrò rifletterci.» Si voltò verso l’imbocco del vicolo, e alzò la bacchetta –

«Uhm» disse Harry, «possiamo andare a prendere il kit di guarigione, ora?»

La strega si fermò e lo guardò fermamente. «E se dicessi di no — che è troppo costoso e non ne avrà bisogno — allora che succederebbe?»

L’espressione di Harry si contorse diventando amara. «Esattamente quello che sta pensando, professoressa McGonagall. \textit{Esattamente} quello che sta pensando. Concluderei che lei è un altro folle adulto al quale non posso parlare, e comincerei a progettare un piano per mettere comunque le mani sul kit di guarigione.»

«Sono il suo tutore in questo viaggio», disse la professoressa McGonagall con una sfumatura di pericolosità. «\textit{Non le permetterò} di menarmi per il naso.»

«Capisco», disse Harry. Tenne il risentimento fuori dalla sua voce, e non disse nessuna delle altre cose che gli erano venute in mente. La professoressa McGonagall gli aveva detto di pensare prima di parlare. Probabilmente non se lo sarebbe ricordato domani, ma poteva almeno ricordarlo per cinque minuti.

La bacchetta nella mano della strega disegnò una forma circolare, e i rumori di Diagon Alley tornarono. «Bene, giovanotto», disse. «Andiamo a prendere quel kit di guarigione.»

Harry rimase a bocca aperta per la sorpresa. Poi si affrettò dietro di lei, quasi inciampando nella sua corsa improvvisa.

\begin{figure}[h]
	\includegraphics[scale=0.4]{boccino.png}
	\centering
\end{figure}

Il negozio era come l’avevano lasciato, articoli riconoscibili e irriconoscibili ancora disposti sull’espositore di legno inclinato, il bagliore grigio ancora a proteggerli e la commessa tornata alla sua postazione. La ragazza alzò lo sguardo mentre si avvicinavano, e il suo viso ne mostrò la sorpresa.

«Mi dispiace», disse lei mentre si fecero più vicini, e Harry parlò quasi nello stesso momento, «Chiedo scusa per –»

Si interruppero e si guardarono l’un l’altra, e poi la commessa rise un po’. «Non avevo intenzione di metterti nei guai con la professoressa McGonagall», disse lei. La sua voce si abbassò in maniera cospiratoria. «Spero che non sia stata troppo terribile con te.»

«\textit{Della!}» scattò la professoressa McGonagall, l’espressione scandalizzata.

«Sacchetto d’oro», disse Harry alla sua borsa, e poi tornò a guardare la commessa mentre contava cinque galeoni. «Non ti preoccupare, ho capito che è terribile con me solo perché mi vuole bene.»

Contò i cinque galeoni alla commessa, mentre la professoressa McGonagall stava farfugliando qualcosa di poco importante. «Un Pacchetto Extra di Guarigione di Emergenza, per favore.»

Era davvero piuttosto inquietante vedere come il Bordo Allargante ingoiasse il kit medico delle dimensioni di una valigetta. Harry non poté fare a meno di chiedersi cosa sarebbe successo se avesse cercato di entrare egli stesso nella borsa mokeskin, dato che solo la persona che aveva messo dentro qualcosa sarebbe dovuta essere in grado di riprenderla.

Quando la borsa ebbe finito di… mangiare… il suo acquisto conquistato con molta fatica, Harry poté giurare di averle sentito fare un ruttino. Quello \textit{doveva} essere stato inserito volutamente. L’ipotesi alternativa era troppo orribile da contemplare… infatti Harry non riusciva nemmeno a pensare a qualunque ipotesi alternativa. Tornò a guardare la professoressa, mentre ripresero a passeggiare per Diagon Alley ancora una volta. «Dove si va?»

La professoressa McGonagall indicò un negozio che sembrava fosse stato costruito con della carne, invece che con i mattoni, e rivestito di pelliccia, invece che di pittura. «Gli animali domestici di piccola taglia sono ammessi a Hogwarts — potrebbe prendere un gufo per inviare le lettere, per esempio –»

«Posso pagare uno zellino o qualcosa del genere e \textit{affittare} un gufo quando ho bisogno di spedire la posta?»

«Sì», disse la professoressa McGonagall.

«Allora credo enfaticamente di \textit{no.}»

La professoressa McGonagall annuì, come se stesse spuntando un elemento di una lista. «Posso chiedere perché no?»

«Ho avuto una pietra domestica, una volta. È morta.»

«Non pensa di essere in grado di avere cura di un animale domestico?»

«\textit{Potrei}», disse Harry, «ma finirei per ossessionarmi tutto il tempo chiedendomi se mi fossi ricordato di nutrirlo, quel giorno, o se stesse lentamente morendo di fame nella sua gabbia, chiedendosi dove sia il suo padrone e perché non ci sia del cibo.»

«Povero gufo» disse l’anziana strega con voce dolce. «Abbandonato così. Mi chiedo che cosa farebbe.»

«Beh, mi aspetterei che avesse veramente fame e iniziasse a provare a uscire dalla gabbia o dalla scatola o da qualsiasi altra cosa, anche se probabilmente non avrebbe molta fortuna –» Harry si fermò di colpo.

La strega continuò, ancora con quella voce morbida. «E che cosa gli accadrebbe dopo?»

«Mi scusi» disse Harry, e afferrò la mano della professoressa McGonagall, delicatamente ma con fermezza, e la condusse in un altro vicolo; dopo aver schivato tanti ammiratori, quella precauzione era diventata quasi automatica. «La prego, lanci l’incantesimo del silenzio.»

«\textit{Quietus.}»

La voce di Harry stava tremando. «Quel gufo \textit{non} rappresenta me, i miei genitori non mi hanno mai chiuso in uno stanzino a morire di fame, \textit{non ho} paura di essere abbandonato e \textit{non mi piace la direzione che i suoi pensieri hanno preso, professoressa McGonagall!}»

La strega lo guardò seriamente. «E che pensieri sarebbero, signor Potter?»

«Lei pensa che io abbia» Harry ebbe dei problemi a dirlo, «che abbia subito degli \textit{abusi?}»

«Li ha subiti?»

«\textit{No!}» gridò Harry. «No, mai! Pensa che sia \textit{stupido?} Conosco il concetto di abuso su minori, so cosa significa toccare in maniera inappropriata e tutto il resto e se qualcosa di simile fosse capitato avrei chiamato la polizia! E l’avrei segnalato al dirigente scolastico! E avrei cercato i servizi sociali sull’elenco del telefono! E l’avrei detto al Nonno e alla Nonna e alla signora Figg! Ma i miei genitori non hanno \textit{mai} fatto niente di simile, mai mai \textit{mai!} Come \textit{osa} insinuare una cosa del genere!»

L’anziana strega lo guardò con fermezza. «È mio dovere di Vicepreside indagare possibili segni di abusi nei bambini affidati alle mie cure.»

La rabbia di Harry stava crescendo fuori controllo, trasformandosi in una furia pura e assoluta. «Non si \textit{azzardi} mai a riferire una sola parola di queste, queste \textit{insinuazioni} a nessun’altro! \textit{Nessuno}, mi ha capito, McGonagall? Un’accusa simile può rovinare le persone e distruggere le famiglie, anche quando i genitori sono completamente innocenti! Ho letto di casi simili sui giornali!» La voce di Harry stava salendo a un grido acuto. «Il \textit{sistema} non sa come \textit{fermarsi}, non crede ai genitori \textit{né} ai figli quando dicono che non è successo niente! \textit{Non si azzardi a minacciare la mia famiglia con questa storia! Non le lascerò distruggere casa mia!}»

«Harry», disse dolcemente la vecchia strega, e allungò una mano verso di lui –

Harry fece un brusco passo indietro, e la sua mano si alzò di scatto e allontanò quella di lei.

McGonagall si immobilizzò, poi ritirò la mano e fece un passo indietro. «Harry, va tutto bene», disse. «Ti credo.»

«\textit{Davvero}», sibilò Harry. La furia ruggiva ancora nel suo sangue. «O sta solo aspettando di allontanarsi da me in modo da poter compilare i documenti?»

«Harry, ho visto la tua casa. Ti ho visto con i tuoi genitori. Ti amano. Tu li ami. Ti credo quando dici che i tuoi genitori non stanno abusando di te. Ma ho \textit{dovuto} chiedere, perché c’è qualcosa che non quadra.»

Harry la guardò con freddezza. «Del tipo?»

«Harry, ho visto molti bambini maltrattati durante la mia permanenza a Hogwarts, ti spezzerebbe il cuore sapere quanti. E, quando sei felice, non ti comporti come uno di quei bambini, per \textit{niente}. Sorridi agli sconosciuti, abbracci le persone, ti ho messo la mano sulla spalla e non hai battuto ciglio. Ma a volte, solo a volte, dici o fai qualcosa che sembra \textit{molto} simile… a quello che farebbe qualcuno che ha trascorso i suoi primi undici anni chiuso in una cantina. Non nella famiglia amorevole che ho conosciuto.» La professoressa McGonagall inclinò la testa, la sua espressione divenne ancora una volta perplessa.

Harry prese quelle parole e le elaborò. La rabbia nera cominciò a defluire, mentre si rendeva conto che era ascoltato con rispetto, e che la sua famiglia non era in pericolo.

«E come \textit{spiega} le sue osservazioni, professoressa McGonagall?»

«Non lo so», disse lei. «Ma è possibile che le sia accaduto qualcosa che non ricorda.»

La furia esplose nuovamente in Harry. Sembrava tutto troppo simile a quello che aveva letto nei resoconti dei giornali sulle famiglie spezzate. «La soppressione della memoria è \textit{pseudoscienza!} La gente \textit{non reprime} i ricordi traumatici, se li ricorda fin \textit{troppo} bene per il resto della sua vita!»

«No, signor Potter. C’è un Incantesimo chiamato Obliazione.»

Harry si bloccò sul posto. «Un incantesimo che cancella i ricordi?»

L’anziana strega annuì. «Ma non tutte le conseguenze di quell’esperienza, se capisce quello che intendo, signor Potter».

Un brivido scese lungo la schiena di Harry. \textit{Quella} ipotesi… \textit{non poteva} essere facilmente confutata. «Ma i miei genitori non possono farlo!»

«In effetti no», concesse la professoressa McGonagall. «Ci sarebbe voluto qualcuno del mondo dei maghi. Non c’è… non c’è modo di esserne certi, temo.»

Le capacità razionaliste di Harry presero a funzionare nuovamente. «Professoressa McGonagall, quanto è certa delle sue osservazioni, e quali spiegazioni alternative potrebbero anche esserci?»

La strega aprì le mani, come per mostrare che erano vuote. «Sicura? Sono sicura di \textit{niente}, signor Potter. In tutta la mia vita non ho mai incontrato qualcuno come lei. A volte non sembra un undicenne e neppure tanto \textit{umano}.»

Le sopracciglia di Harry salirono verso il cielo –

«Mi dispiace!» la professoressa McGonagall disse in fretta. «Mi dispiace molto, signor Potter. Stavo cercando fare un discorso e ho paura che sia venuto fuori diverso da quello che avevo in mente –»

«Al contrario, professoressa McGonagall», disse Harry, e sorrise lentamente. «Lo prendo come un grandissimo complimento. Ma le dispiacerebbe se le offrissi una spiegazione alternativa?»

«Prego.»

«I bambini non sono fatti per essere troppo più intelligenti dei loro genitori», disse Harry. «O troppo più sani di mente, forse — mio padre potrebbe probabilmente superarmi in arguzia se lui, sa, \textit{provasse} a farlo, invece di usare la sua intelligenza da adulto principalmente per scovare nuove ragioni per non cambiare idea –» Harry si fermò. «Sono troppo intelligente, Professoressa. Non ho nulla da dire ai bambini normali. Gli adulti non mi rispettano a sufficienza per parlare veramente con me. E, francamente, anche se lo facessero, non sembrerebbero intelligenti come Richard Feynman, quindi tanto vale leggere qualcosa che Richard Feynman ha scritto, invece. Sono \textit{isolato}, professoressa McGonagall. Sono stato isolato per tutta la mia vita. Forse questo ha alcuni degli stessi effetti del rimanere rinchiuso in una cantina. E sono troppo intelligente per guardare ai miei genitori nel modo in cui i bambini sono progettati per fare. I miei genitori mi amano, ma non si sentono obbligati a rispondere alla ragione, e qualche volta mi sento come se loro fossero i bambini — bambini che \textit{non vogliono ascoltare} e che hanno un’autorità assoluta su tutta la mia esistenza. Cerco di non essere troppo amareggiato per questo, ma cerco anche di essere onesto con me stesso, quindi, sì, sono amareggiato. E ho anche un problema di gestione della rabbia, ma ci sto lavorando. Questo è tutto.»

«\textit{Questo è tutto?}»

Harry annuì con decisione. «Questo è tutto. Sicuramente, professoressa McGonagall, anche nella Gran Bretagna magica, la spiegazione normale è sempre degna di essere \textit{considerata}, no?»

\begin{figure}[h]
	\includegraphics[scale=0.4]{boccino.png}
	\centering
\end{figure}

Più tardi, quello stesso giorno, il sole stava scendendo nel cielo estivo e gli acquirenti cominciavano a scemare via dalle strade. Alcuni negozi avevano già chiuso; Harry e la professoressa McGonagall avevano comprato i suoi libri di testo da Flourish and Blotts appena prima del termine. Con appena una leggera esplosione quando Harry era andato dritto verso la sezione «Aritmanzia» e aveva scoperto che i libri di testo del settimo anno non richiedevano niente di più matematicamente avanzato della trigonometria.

In quel momento, però, i sogni di cogliere i risultati più facili della ricerca erano lontani dalla mente di Harry.

In quel momento, essi stavano uscendo da Ollivander’s, e Harry stava fissando la propria bacchetta. L’agitò, e produsse scintille multicolori, cosa che in realtà non avrebbe dovuto essere così sconvolgente dopo tutto quello che aveva già visto, ma in qualche modo –

\textit{Posso compiere magie.}

\textit{Io. Voglio dire, io personalmente. Sono una creatura magica; sono un mago.}

Aveva \textit{sentito} la magia scorrere su per il suo braccio, e in quell’istante aveva realizzato che aveva sempre avuto quella sensazione, che l’aveva posseduto tutta la vita, quel senso che non era vista, udito, olfatto, gusto o tatto, ma semplicemente magia. Come avere gli occhi ma tenerli sempre chiusi, così da non accorgersi neppure di vedere l’oscurità; e poi un giorno gli occhi si aprivano e vedevano il mondo. Lo sconvolgimento si era riversato attraverso di lui, toccando parti del suo sé, svegliandole, e poi si era andato a spegnere nel giro di pochi secondi; lasciando solo la conoscenza certa che ora era un mago, e lo era sempre stato, e persino che, in qualche modo particolare, l’aveva sempre saputo.

E –

«\textit{È davvero curioso che lei fosse predestinato a questa bacchetta quando la sua gemella, beh, la sua gemella le ha procurato quella cicatrice.}»

Non era possibile che fosse una coincidenza. C’erano state \textit{migliaia} di bacchette in quel negozio. Beh, va bene, in effetti \textit{poteva} essere una coincidenza, c’erano sei miliardi di persone nel mondo e coincidenze del tipo una-su-mille si avveravano ogni giorno. Ma il Teorema di Bayes diceva che ogni ipotesi ragionevole che rendesse \textit{più} probabile di mille a uno che egli entrasse in possesso della bacchetta gemella di quella del Signore Oscuro era avvantaggiata.

La professoressa McGonagall aveva semplicemente detto \textit{molto peculiare} e l’aveva finita lì, cosa che aveva sconvolto Harry per la pura e preponderante \textit{mancanza di curiosità} dei maghi e delle streghe. In nessun mondo \textit{concepibile} Harry avrebbe semplicemente fatto «Uhm» e se ne sarebbe uscito dal negozio senza neppure \textit{provare} a formulare un’ipotesi per ciò che stava accadendo.

La sua mano sinistra andò a toccare la sua cicatrice.

Cosa… \textit{esattamente…}

«Ora è un mago completo», disse la professoressa McGonagall. «Congratulazioni.»

Harry annuì.

«E cosa pensa del mondo dei maghi?»

«È strano», disse Harry. «Dovrei pensare a tutto ciò che ho visto della magia… a tutto ciò che ora so essere possibile, e a tutto ciò che ora so essere una bugia, e tutto il lavoro davanti a me per comprenderlo. Eppure mi trovo distratto da banalità relative come», Harry abbassò la voce, «la faccenda del Ragazzo-Che-È-Sopravvissuto.» Non sembrava ci fosse nessuno vicino, ma era inutile tentare il destino.

La professoressa McGonagall si schiarì la voce. «Davvero? Sono sorpresa.»

Harry annuì. «Sì. È semplicemente… \textit{strano}. Scoprire di essere stato parte di questa grande storia, l’impresa di sconfiggere il grande e terribile Signore Oscuro, e che è già \textit{finita}. Completata. Definitivamente terminata. È come essere Frodo Baggins e scoprire che i tuoi genitori ti hanno portato a Monte Fato e ti hanno fatto gettare l’Anello quando avevi solo un anno, e non ti ricordi niente.»

Il sorriso della professoressa McGonagall era diventato piuttosto fisso.

«Sa, se fossi chiunque altro, qualunque altra persona, probabilmente sarei preoccupato di essere all’altezza di quell’inizio. \textit{Mio dio, Harry, cos’hai fatto da quando hai sconfitto il Signore Oscuro? Hai aperto una tua libreria? Che bella idea! Ehi, lo sapevi che ho dato il tuo nome a mio figlio?} Ma conto sul fatto che questo non sarà un problema.» Harry sospirò. «Ad ogni modo… è quasi sufficiente a farmi sperare che vi siano alcune questioni in sospeso in questa impresa, giusto in modo che possa dire di avervi realmente, sa, \textit{partecipato} in qualche modo.»

«Oh?» disse la professoressa McGonagall in un tono strano. «Che cosa ha in mente?»

«Beh, per esempio, lei ha menzionato il fatto che i miei genitori sono stati traditi. Chi li ha traditi?»

«Sirius Black», disse la strega, quasi sibilando il nome. «È ad Azkaban. La prigione dei maghi.»

«Quanto è probabile che quel Sirius Black fugga dalla prigione e io debba inseguirlo e sconfiggerlo in una qualche sorta di duello spettacolare, o meglio ancora mettere una grossa taglia sulla sua testa e nascondermi in Australia mentre attendo i risultati?»

La professoressa McGonagall sbatté le palpebre. Due volte. «Improbabile. Nessuno è mai fuggito da Azkaban, e dubito che \textit{lui} sarà il primo.»

Harry era un po’ scettico riguardo quel «\textit{nessuno} è \textit{mai} fuggito da Azkaban». Eppure, forse con la magia era possibile ottenere una prigione perfetta fino a quasi il 100\%, specie se tu avevi una bacchetta e loro no. Il modo migliore per uscire sarebbe stato quello di non andarci, tanto per cominciare.

«Va bene, allora», disse Harry. «Sembra che tutto sia stato terminato alla perfezione.» Sospirò, strofinandosi la testa col palmo. «O forse il Signore Oscuro non morì \textit{davvero} quella notte. Non completamente. Il suo spirito si trattiene, sussurrando alle persone in incubi che sbiadiscono dopo il risveglio, cercando un modo di tornare alle terre dei vivi che ha giurato di distruggere, e ora, in accordo con un’antica profezia, lui e io siamo serrati in un duello mortale in cui il vincitore perderà e lo sconfitto vincerà –»

La testa della professoressa McGonagall ruotò, e i suoi occhi guizzarono intorno, come se scandagliasse la strada in cerca di ascoltatori.

«Sto \textit{scherzando}, professoressa», disse Harry con un certo fastidio. Cavolo, perché prendeva sempre tutto così seriamente –

Una sensazione di profonda ansia nacque alla bocca dello stomaco di Harry.

La professoressa McGonagall osservò Harry con un’espressione calma. Un’espressione molto, \textit{molto} calma. Poi mise su un sorriso. «Naturalmente, signor Potter.»

\textit{Oh cavolo.}

Se Harry avesse avuto bisogno di formalizzare la deduzione non verbale che era appena sorta nella sua mente, se ne sarebbe dovuto uscire con qualcosa tipo ‘se stimassi che la probabilità che la professoressa McGonagall faccia ciò che ho appena visto come risultato dell’attento controllo di sé stessa, contro la distribuzione di probabilità di tutte le cose che farebbe \textit{naturalmente} se facessi una pessima battuta, allora questo comportamento è una prova significativa del fatto che sta nascondendo qualcosa’.

Ma ciò che Harry pensò effettivamente fu, \textit{Oh cavolo.}

Harry girò la testa per scandagliare la strada. No, nessuno vicino. «\textit{Non è} morto, giusto?», sospirò Harry.

«Signor Potter –»

«Il Signore Oscuro è vivo. \textit{Ovviamente} è vivo. È stato un \textit{atto} di completo \textit{ottimismo} da parte mia aver persino sognato che non lo fosse. \textit{Devo} aver abbandonato il mio \textit{buon senso}, non riesco a \textit{immaginare} a cosa stavo \textit{pensando}. Solo perché \textit{qualcuno} ha detto che il suo corpo è stato trovato ridotto a un \textit{carbone}, non posso immaginare perché io abbia pensato che fosse \textit{morto. È chiaro} che ho ancora molto da imparare dell’arte del corretto \textit{pessimismo.}»

«Signor Potter –»

«Almeno mi dica che non c’è davvero una profezia…» La professoressa McGonagall gli stava ancora rivolgendo quel sorriso brillante e fisso. «Oh, lei mi sta \textit{certamente} prendendo in giro.»

«Signor Potter, non deve inventarsi cose di cui preoccuparsi –»

«Mi sta \textit{davvero} dicendo \textit{questo?} Immagini la mia reazione dopo, quando scoprissi che c’era davvero qualcosa di cui preoccuparsi.»

Il suo sorriso fisso vacillò.

Le spalle di Harry si ingobbirono. «Ho l’intero mondo della magia da studiare. \textit{Non ho tempo} per questo.»

Poi entrambi si zittirono, appena un uomo in fluenti vesti arancioni apparve sulla strada e lentamente li superò; gli occhi della professoressa McGonagall lo seguirono, discretamente. La bocca di Harry si muoveva mentre masticava con forza il proprio labbro, e qualcuno che avesse guardato da vicino avrebbe notato apparire una piccola macchia di sangue.

Quando l’uomo vestito di arancione si fu allontanato, Harry parlò di nuovo, in un basso mormorio. «Ha intenzione di dirmi la verità ora, professoressa McGonagall? E non si disturbi a minimizzare la questione, non sono stupido.»

«Lei ha solo \textit{undici anni}, signor Potter!» disse ella con un sussurro severo.

«E perciò subumano. Le chiedo scusa… per un momento l’avevo \textit{dimenticato}.»

«Si tratta di questioni terribili e importanti! Sono un \textit{segreto}, signor Potter! Si tratta di una \textit{catastrofe} che lei, che è ancora bambino, sappia anche solo questo! Non deve dirlo a \textit{nessuno}, ha capito? Assolutamente a nessuno!»

Come a volte accadeva quando Harry diventava \textit{sufficientemente} arrabbiato, il suo sangue si raffreddò, invece di scaldarsi, e una terribile e oscura chiarezza discese sulla sua mente, delineando le possibili tattiche e valutando le loro conseguenze con ferreo realismo.

\begin{itpars}
Fai notare che hai il diritto di sapere: fallimento. I bambini di undici anni non hanno diritto di sapere nulla, agli occhi della McGonagall.

Di’ che non sarete più amici: fallimento. Non tiene sufficientemente alla vostra amicizia.

Fai notare che sarai in pericolo se non sarai messo al corrente: fallimento. I piani sono già stati fatti sulla base della tua ignoranza. La scomodità certa di ripensarli sembrerà molto più sgradevole che la mera prospettiva incerta del tuo essere ferito.

Giustizia e ragione sono destinate entrambe a fallire. Devi trovare qualcosa che hai e che lei vuole, o qualcosa che tu puoi fare, e che lei teme…
\end{itpars}

Ah.

«Bene allora, Professoressa», disse Harry in una voce bassa e glaciale, «sembra che io abbia qualcosa che lei desidera. Lei può, se vuole, dirmi la verità, \textit{tutta} la verità, e in cambio manterrò i suoi segreti. O può cercare di tenermi all’oscuro in modo da usarmi come pedina, in tal caso non le dovrò nulla.»

McGonagall si fermò improvvisamente per strada. I suoi occhi divamparono e la sua voce eruppe in un sibilo. «Come osa!»

«\textit{Come osa lei!}» le sussurrò in risposta.

«Vorrebbe \textit{ricattarmi?}»

Le labbra di Harry si contorsero. «Le sto \textit{offrendo} un \textit{favore}. Le sto \textit{offrendo} la possibilità di proteggere il \textit{suo} prezioso segreto. Se rifiuta avrò \textit{ogni} legittimo motivo di domandare ad altri, non per ripicca contro di lei, ma perché \textit{devo sapere!} Superi la sua inutile rabbia contro un \textit{bambino} da cui si aspetta solo obbedienza, e comprenderà che ogni adulto sano di mente farebbe la stessa cosa! \textit{La guardi dalla mia prospettiva! Come si sentirebbe se \textsc{lei} fosse al mio posto?}»

Harry osservò McGonagall, notò il suo respiro pesante. Gli venne in mente che fosse il momento di allentare la pressione, di lasciarla cuocere a fuoco dolce per un po’. «Non deve decidere immediatamente», disse Harry in un tono più normale. «Capirei se volesse prendersi del tempo per pensare alla mia \textit{offerta}… ma l’avviso di una cosa», disse Harry con una voce che divenne più fredda. «Non provi a usare quell’incantesimo di Obliazione su di me. Qualche tempo fa ho elaborato una tecnica di segnalazione, e mi sono già mandato quel segnale. Se scoprissi quel segnale e non \textit{ricordassi} di essermelo mandato…» Harry lasciò che la propria voce si spegnesse in modo significativo.

Il volto di McGonagall lavorò, mentre le sue espressioni si scambiavano di posto. «Io… non stavo pensando di Obliarla, signor Potter… ma perché avrebbe dovuto \textit{inventare} tale segnale se non sapeva che –»

«L’ho pensato mentre leggevo un libro di fantascienza babbana, e ho detto a me stesso, \textit{bene, nel caso servisse}… E no, non le dirò che segnale è, non sono stupido.»

«Non era mia intenzione chiederlo», disse McGonagall. Sembrò ripiegare su sé stessa, e improvvisamente apparve molto anziana, molto stanca. «Si è trattato di una giornata molto faticosa, signor Potter. Possiamo prendere il suo baule e mandarla a casa? Confiderò che lei non parli di questa faccenda finché non avrò avuto il tempo di pensarci. Consideri che ci sono solo due altre persone in tutto il mondo che sanno di questa cosa, e sono il preside Albus Silente e il professor Severus Snape.»

Dunque. Una nuova informazione; quella era un’offerta di pace. Harry annuì segnalandone l’accoglimento, e girò la testa per guardare avanti, e iniziò a camminare nuovamente, mentre il suo sangue iniziò ancora una volta a riscaldarsi.

«Quindi ora devo trovare qualche maniera di uccidere un Mago Oscuro immortale», Harry disse sospirando per la frustrazione. «Vorrei davvero che me l’avesse detto \textit{prima} che iniziassi a fare le compere.»

\begin{figure}[h]
	\includegraphics[scale=0.4]{boccino.png}
	\centering
\end{figure}

Il negozio di bauli era più riccamente arredato di ogni altro negozio che Harry avesse visitato; le tende avevano decori lussuosi e delicati, il pavimento e le pareti erano di legno dipinto e lucidato, e i bauli occupavano posti d’onore su piattaforme in avorio lucidato. Il commesso era vestito con abiti eleganti, appena una spanna al di sotto di quelli di Lucius Malfoy, e parlò con una cortesia squisita e melliflua sia a Harry sia alla professoressa McGonagall.

Harry aveva fatto le sue domande, e aveva gravitato verso un baule di legno dall’aspetto pesante, non lucido ma caldo e solido, scolpito con il motivo di un drago guardiano i cui occhi si muovevano per guardare chiunque si avvicinasse. Un baule incantato per essere leggero, per ridursi a comando, per far spuntare piccoli tentacoli artigliati dal fondo e dimenarsi dietro al suo proprietario. Un baule con due cassetti su ognuno dei quattro lati, ciascuno che scivolava fuori per rivelare compartimenti profondi come l’intero baule. Un coperchio con quattro serrature, ognuna delle quali avrebbe rivelato uno spazio diverso all’interno. E — questa era la parte più importante — una maniglia sul fondo che faceva scivolare fuori un telaio contenente una scala che portava giù in una piccola stanza illuminata, la quale poteva contenere, Harry stimò, circa dodici librerie.

Se realizzavano bauli come quello, Harry non sapeva perché qualcuno si disturbasse a possedere una casa.

Cento e otto galeoni d’oro. Quello era il prezzo di un buon baule, appena usato. Con circa cinquanta sterline britanniche per un galeone, era abbastanza per comprare una macchina di seconda mano. Sarebbe stato più costoso di ogni altra cosa che Harry avesse comprato in tutta la sua vita messa insieme.

Novantasette galeoni. Tanto era rimasto nella borsa d’oro che Harry era stato autorizzato a prelevare da Gringotts.

La professoressa McGonagall aveva un’espressione di disappunto sul suo viso. Alla fine di una lunga giornata di acquisti, non aveva bisogno di chiedere a Harry quanto oro fosse rimasto nella borsa, dopo che il venditore aveva annunciato il suo prezzo, il che significava che la Professoressa era in grado di fare i conti mentalmente, senza carta e penna. Ancora una volta, Harry dovette ricordare a sé stesso che \textit{analfabeta scientifico} non era affatto la stessa cosa di \textit{stupido}.

«Mi dispiace, giovanotto», disse la professoressa McGonagall. «È tutta colpa mia. Le proporrei di riportarla da Gringotts, ma la banca sarà chiusa per tutti i servizi tranne quelli di emergenza, ora.»

Harry la osservò, chiedendosi…

«Bene», sospirò la professoressa McGonagall, mentre girò facendo perno su di un tallone, «allora possiamo andare, presumo.»

… \textit{non aveva} perso del tutto il controllo quando un bambino aveva osato sfidarla. Non ne era stata contenta, ma aveva \textit{pensato} invece di lasciare che la sua rabbia esplodesse. Poteva essere semplicemente perché c’era un Signore Oscuro immortale da combattere — perché aveva bisogno della buona volontà di Harry. Ma la maggior parte degli adulti non sarebbe stata in grado di pensare neppure quello; non avrebbero affatto considerato le \textit{conseguenze future}, se qualcuno di uno status inferiore si fosse rifiutato di obbedire loro…

«Professoressa?» disse Harry.

La strega si girò per guardarlo.

Harry fece un respiro profondo. Aveva bisogno di essere un po’ arrabbiato per fare ciò che voleva provare, era impossibile che trovasse il coraggio di farlo in caso contrario. \textit{Non mi ha ascoltato}, pensò, \textit{io avrei preso più oro, ma lei non mi ha ascoltato…} Concentrando tutto il suo mondo sulla McGonagall e sulla necessità di piegare quella conversazione alla propria volontà, parlò.

«Professoressa, lei pensava che cento galeoni sarebbero stati più che sufficienti per un baule. Ecco perché non si è presa il disturbo di avvertirmi prima che scendessero a novantasette. Il che è proprio il genere di cose che le ricerche dimostrano — questo è quello che succede quando le persone pensano di essersi lasciate un \textit{piccolo} margine di errore. Non sono abbastanza pessimiste. Se fosse stato per me, avrei preso \textit{duecento} galeoni giusto per essere sicuri. C’era abbondanza di soldi in quel deposito, e avrei potuto rimetterci quelli in eccesso successivamente. Ma ho pensato che non mi avrebbe permesso di farlo. Ho pensato che si sarebbe arrabbiata con me se solo l’avessi chiesto. Sbagliavo?»

«Suppongo di dover confessare che ha ragione», disse la professoressa McGonagall. «Ma, giovanotto –»

«Questo genere di cose è la ragione per cui ho difficoltà a fidarmi degli adulti.» In qualche modo Harry mantenne la voce salda. «Perché si arrabbiano se solo \textit{provi} a ragionare con loro. Per loro si tratta di un atteggiamento di ribellione e di insolenza, una sfida al loro status tribale superiore. Se tenti di parlare con loro si \textit{arrabbiano}. Quindi, se avessi qualcosa di \textit{veramente importante} da fare, non sarei in grado di fidarmi di lei. Anche se ascoltasse con profondo interesse ciò che le dicessi — perché anche questo fa parte del ruolo di qualcuno che interpreta la \textit{parte} dell’adulto interessato — non cambierebbe mai le sue azioni, non si comporterebbe mai realmente in modo differente, a causa di qualcosa che le avessi detto io.»

Il commesso li stava guardando entrambi con palese affascinazione.

«Posso capire il suo punto di vista», disse infine la professoressa McGonagall. «Se a volte sembro troppo severa, la prego di non scordare che ho servito come Preside di Casa Grifondoro per un tempo che mi sembra pari a diverse migliaia di anni.»

Harry annuì e continuò. «Quindi — supponga che io abbia il modo di prelevare altri galeoni dal mio deposito senza essere costretti a tornare alla Gringotts, ma che questo implichi una mia violazione del ruolo di bambino obbediente. Poteri fidarmi di lei per questa faccenda, anche se questo l’obbligasse ad abbandonare il suo ruolo di professoressa McGonagall per trarne vantaggio?»

«\textit{Cosa?}» esclamò la professoressa McGonagall.

«Per metterla in un’altra maniera, se potessi far sì che la giornata di oggi fosse andata in maniera differente, in modo che \textit{non avessimo} portato con noi denaro insufficiente, sarebbe questo accettabile, anche se implicherebbe che un bambino sia stato insolente verso un adulto, a posteriori?»

«Suppongo… di sì…» disse la strega, sembrando alquanto perplessa.

Harry prese la borsa mokeskin, e disse, «Undici galeoni provenienti dal mio deposito di famiglia.»

E ci fu dell’oro nella mano di Harry.

Per un momento la professoressa McGonagall rimase a bocca aperta, poi la sua mascella di chiuse di scatto e i suoi occhi formarono una fessura e la strega disse con una rabbia controllata, «\textit{Dove} ha preso quel –»

«Dal mio deposito di famiglia, come ho detto.»

«\textit{Come?}»

«Magia.»

«Questa non è una risposta!» scattò la professoressa McGonagall, e poi si fermò, battendo gli occhi.

«No, non lo è, vero? \textit{Dovrei} affermare di aver scoperto sperimentalmente il vero segreto del funzionamento della borsa e che può effettivamente recuperare gli oggetti da qualunque luogo, non solo dal proprio interno, se si formula la richiesta correttamente. Ma in realtà li ho presi prima, quando sono caduto in quel mucchio d’oro e ho spinto alcuni galeoni in tasca. Chiunque capisca il pessimismo sa che il denaro è qualcosa di cui potresti aver bisogno rapidamente e senza molto preavviso. Dunque ora è arrabbiata con me per aver sfidato la sua autorità? O felice perché abbiamo portato a termine la nostra importante missione?»

Gli occhi del commesso erano spalancati per lo stupore.

E l’alta strega restò ferma, in silenzio.

«A Hogwarts la disciplina \textit{deve} essere rispettata», disse dopo quasi un intero minuto. «Per il bene di \textit{tutti} gli studenti. E questo \textit{deve} includere cortesia e obbedienza da parte sua verso \textit{tutti} i professori.»

«Comprendo, professoressa McGonagall.»

«Bene, ora compriamo quel baule e andiamo a casa.»

Harry si sentì come se stesse per vomitare, o per esultare, o per svenire, o \textit{qualcosa}. Era la prima volta che il suo ragionamento attento aveva mai funzionato con \textit{chiunque}. Forse perché era la prima volta che aveva qualcosa di davvero importante di cui un adulto aveva bisogno, eppure –

Minerva McGonagall, +1 punto.

Harry fece un inchino, e mise la borsa d’oro e gli undici galeoni extra nelle mani di McGonagall. «La ringrazio di cuore, professoressa. Può completare l’acquisto per me? Devo far visita al bagno.»

Il commesso, nuovamente mellifluo, indicò in direzione di una porta inserita nel muro, con una maniglia dorata. Appena Harry iniziò ad allontanarsi, sentì il commesso chiedere nella sua voce melliflua, «Posso domandare chi sia quella persona, Madam McGonagall? Presumo sia un Serpeverde — del terzo anno, forse? — e di famiglia illustre, ma non ho riconosciuto –»

Il rumore della porta del bagno che si chiudeva interruppe le sue parole, e dopo che Harry ebbe individuato il nottolino di chiusura e l’ebbe azionato, afferrò l’asciugamano magico auto-pulente e, con le mani tremanti, si tolse il sudore dalla fronte. L’intero corpo di Harry era ricoperto di sudore, che aveva impregnato i suoi abiti babbani, sebbene, quanto meno, non si vedesse attraverso le vesti da mago.

\begin{figure}[h]
	\includegraphics[scale=0.4]{boccino.png}
	\centering
\end{figure}

Il sole stava tramontando ed era effettivamente molto tardi, quando si trovarono nuovamente nel cortile del Paiolo Magico, la silenziosa e fronzuta interfaccia tra la Diagon Alley della Gran Bretagna magica e l’intero mondo babbano. (Si trattava di un’economia disaccoppiata in maniera \textit{terribile}…) Una volta arrivato dall’altra parte, Harry raggiunse una cabina telefonica e chiamò suo padre. Non aveva bisogno di preoccuparsi che il suo bagaglio venisse rubato, apparentemente. Il suo baule godeva della condizione di oggetto magico maggiore, qualcosa che la maggior parte dei Babbani non notava neppure; quello era parte di ciò che potevi ottenere nel mondo magico, se eri disposto a pagare il prezzo di un’auto di seconda mano.

«Quindi le nostre strade si dividono qui, per ora», disse la professoressa McGonagall. Scosse la testa per la meraviglia. «Questo è stato il giorno più strano della mia vita da… diversi anni. Sin dal giorno in cui venni a sapere che un bambino aveva sconfitto Tu-Sai-Chi. Oggi mi chiedo, guardandomi indietro, se quello sia stato l’ultimo giorno ragionevole del mondo.»

Oh, come se \textit{lei} avesse qualcosa di cui lamentarsi. \textit{Pensi che la tua giornata sia stata surreale? Prova la mia.}

«Sono stato molto colpito da lei, oggi», le disse Harry. «Avrei dovuto ricordarmi di farle i complimenti ad alta voce, nella mia mente le stavo assegnando dei punti.»

«Grazie, signor Potter», disse la professoressa McGonagall. «Se fosse stato già assegnato a una Casa, le avrei sottratto talmente tanti punti che i suoi nipoti starebbero ancora perdendo la Coppa delle Case».

«Grazie a \textit{lei}, Professoressa». Era probabilmente ancora troppo presto per chiamarla Minnie.

Questa donna sarebbe potuta essere l’adulto più sano di mente che Harry avesse mai incontrato, malgrado la mancanza di un’educazione scientifica. Harry stava anche considerando di offrirle la posizione numero due in qualunque gruppo avesse formato per combattere il Signore Oscuro, sebbene non fosse abbastanza folle da dirlo ad alta voce. \textit{E quale sarebbe un nome appropriato per questo gruppo…? I Mangiamangiamorte?}

«La vedrò presto, quando inizierà la scuola», disse la professoressa McGonagall. «E, signor Potter, riguardo alla sua bacchetta –»

«So cosa sta per chiedermi», disse Harry. Tirò fuori la sua preziosa bacchetta e, con una profonda fitta di pena interiore, la fece ruotare nella mano, in modo da offrirle l’impugnatura. «La prenda. Non avevo intenzione di farci nulla, proprio niente, ma non voglio che abbia incubi in cui faccio saltare in aria la mia casa.»

La professoressa McGonagall scosse la testa rapidamente. «Oh no, signor Potter! Non si tratta di questo. Volevo solo avvisarla di non usare la sua bacchetta a casa, in quanto il Ministero è in grado di rilevare la pratica magica da parte di minori, ed è proibita in assenza di supervisori.»

«Ah», disse Harry. «Mi sembra una regola ragionevole. Sono contento di vedere che il mondo della magia prende sul serio questo genere di cose.»

La professoressa McGonagall lo fissò attentamente. «Sta dicendo sul serio.»

«Sì», rispose Harry. «Lo capisco. La magia è pericolosa e le regole esistono per buone ragioni. Anche altre faccende sono pericolose. Capisco anche quello. Si ricordi che non sono stupido.»

«È improbabile che me lo dimentichi. Grazie, Harry, questo mi fa sentire meglio a proposito delle altre cose che ti ho confidato. Arrivederci, per ora.»

Harry si girò per andarsene, nel Paiolo Magico e poi fuori verso il mondo dei Babbani.

Mentre la sua mano toccava la maniglia della porta sul retro, sentì un ultimo sussurro da dietro di sé.

«Hermione Granger.»

«Cosa?» disse Harry con la mano ancora sulla maniglia.

«Cerca una ragazza del primo anno chiamata Hermione Granger sul treno per Hogwarts.»

«Chi è?»

Non ci fu risposta, e quando Harry si girò, la professoressa McGonagall se n’era andata.

\begin{figure}[h]
	\includegraphics[scale=0.4]{boccino.png}
	\centering
\end{figure}

\subsubsection{Conseguenze}

Il preside Albus Silente si chinò in avanti sulla propria scrivania. I suoi occhi sfavillanti indagarono Minerva. «Dunque, mia cara, come hai trovato Harry?»

Minerva aprì la bocca. Poi chiuse la bocca. Poi aprì la bocca ancora una volta. Nessuna parola ne uscì.

«Capisco», disse Albus seriamente. «Grazie per la tua relazione, Minerva. Puoi andare.»
