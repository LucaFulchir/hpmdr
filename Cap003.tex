% !TeX root = Harry.tex

\chapter{Confrontare la realtà con le sue alternative}
\label{capitolo:3}

\emph{«Ma poi del resto la domanda è — chi?»}

~\\
~\\

«Buon Dio», disse il barista, scrutando Harry, «questo — può essere che sia –?»

Harry si avvicinò il più possibile al bancone del Paiolo Magico, sebbene gli arrivasse più o meno alla sommità delle sopracciglia. Una domanda come quella meritava il suo sforzo migliore.

«Sono forse io — potrei essere — forse — non lo sapremo mai — se \textit{non fossi} — ma poi del resto la domanda è — \textit{chi?}»

«Benedetta è la mia anima», bisbigliò l’anziano barista. «Harry Potter… quale onore.»

Harry rimase interdetto, poi si riprese. «Beh, sì, lei è molto perspicace; la maggior parte delle persone non lo capisce così rapidamente –»

«Basta così», disse la professoressa McGonagall. La sua mano si strinse sulla spalla di Harry. «Non infastidire il ragazzo, Tom, è nuovo da queste parti.»

«Ma è lui?» chiese tremando un’anziana signora. «È Harry Potter?» Con un suono raschiante, si alzò dalla sedia.

«Doris –» l’ammonì McGonagall. L’occhiata che lanciò a tutta la stanza sarebbe dovuta essere sufficiente a intimidire chiunque.

«Voglio solo stringergli la mano», sussurrò la donna. Si chinò e allungò una mano rugosa, che Harry, sentendosi confuso e più a disagio di quanto fosse mai stato in vita sua, strinse con attenzione. Le lacrime della signora caddero dai suoi occhi sulle loro mani intrecciate. «Mio nipote era un Auror», gli sussurrò. «Morto nel settantanove. Grazie, Harry Potter. Ringrazio il cielo per averci mandato te.»

«Prego», disse Harry automaticamente, e poi girò la testa e indirizzò alla professoressa McGonagall un’occhiata spaventata e implorante.

La professoressa McGonagall batté con forza il piede sul pavimento proprio mentre tutta la sala stava iniziando ad accorrere. Fece un rumore che diede a Harry un nuovo riferimento per la locuzione ‘rombo di tuono’, e tutti si bloccarono sul posto.

«Andiamo di fretta», disse la professoressa McGonagall con una voce che suonò perfettamente, assolutamente normale.

Lasciarono il bar senza alcuna difficoltà.

«Professoressa?» disse Harry, una volta che furono nel cortile. Aveva intenzione di chiedere che cosa stesse succedendo, ma stranamente si ritrovò a porre una domanda completamente diversa. «Chi era quell’uomo pallido, vicino l’angolo? L’uomo con il tic all’occhio?»

«Hm?» fece la professoressa McGonagall, sembrando un po’ sorpresa; forse non si aspettava neppure lei che la domanda fosse quella. «Era il professor Quirinus Quirrell. Quest’anno insegnerà Difesa Contro le Arti Oscure a Hogwarts.»

«Avevo la stranissima sensazione di conoscerlo…» Harry si strofinò la fronte. «E che non avrei dovuto stringergli la mano.» Come incontrare qualcuno che era stato un amico, una volta, prima che qualcosa fosse andato drasticamente male… non era esattamente così, ma Harry non riusciva a trovare le parole. «E che cos’è \textit{successo}… davvero là dentro?»

La professoressa McGonagall gli stava rivolgendo una strana espressione. «Signor Potter… lei sa… quanto le è stato detto… di come sono morti i suoi genitori?»

Harry le restituì uno sguardo saldo. «I miei genitori sono vivi e vegeti, e si sono sempre rifiutati di parlare di come i miei genitori \textit{genetici} sono morti. Dal che deduco che non sono morti bene.»

«Una lealtà ammirevole», disse la professoressa McGonagall. La sua voce si abbassò. «Per quanto mi faccia un po’ male sentirglielo dire in quel modo. Lily e James erano miei amici.»

Harry distolse lo sguardo, provando improvvisamente vergogna. «Mi dispiace», disse con un filo di voce. «Ma io \textit{ho} una mamma e un papà. E so che mi renderei infelice mettendo a confronto questa realtà con… qualcosa di perfetto che costruirei nella mia immaginazione.»

«Questo è sorprendentemente saggio da parte sua», disse sommessamente la professoressa McGonagall. «Ma i suoi genitori \textit{genetici} sono morti davvero molto bene, proteggendola.»

\textit{Proteggendo me?}

Qualcosa di strano strinse il cuore di Harry. «Che cosa… è accaduto \textit{realmente}?»

La professoressa McGonagall sospirò. La sua bacchetta toccò la fronte di Harry, e la vista del ragazzo si offuscò per un momento. «Una sorta di travestimento», disse, «in modo che non accada di nuovo, fino a quando lei non sarà pronto.» Allora la sua bacchetta si mosse nuovamente, e batté tre volte su un muro di mattoni…

… che rientrarono formando un buco, lo dilatarono e ampliarono, e si contrassero in un enorme arco, rivelando una lunga fila di negozi con insegne pubblicitarie di calderoni e fegati di drago.

Harry non batté ciglio. Non era come se qualcuno si fosse trasformato in un gatto.

E avanzarono, insieme, entrando nel mondo dei maghi.

C’erano venditori ambulanti di Stivali Rimbalzanti («Fatti con vero Flubber!») e «Coltelli +3! Forchette +2! Cucchiai con un bonus di +4!» C’erano visori che coloravano tutto ciò che osservavi di verde, e un’esposizione di confortevoli poltrone con seggiolini eiettabili per le emergenze.

La testa di Harry continuò a ruotare e ruotare come se stesse cercando di svitarsi dal suo collo. Era come camminare attraverso la sezione degli oggetti magici di un libro di regole di \textit{Advanced Dungeons \& Dragons} (non aveva mai giocato una partita, ma si divertiva a leggere i libri delle regole). Harry non voleva assolutamente perdere neppure un solo oggetto in vendita, nel caso in cui fosse stato uno dei tre necessari per completare il ciclo infinito degli incantesimi di \textit{desiderio}.

Poi Harry vide qualcosa che lo spinse, del tutto involontariamente, ad allontanarsi dalla Vicepreside e a dirigersi dritto nel negozio, l’ingresso di mattoni blu con finiture in metallo color bronzo. Fu riportato alla realtà solo quando la professoressa McGonagall si mise proprio di fronte a lui.

«Signor Potter?»

Harry sbatté le palpebre, poi comprese ciò che aveva appena fatto. «Mi dispiace! Ho dimenticato per un momento che ero con lei invece che con la mia famiglia.» Harry indicò la finestra del negozio, che mostrava lettere fiammeggianti che brillavano forti eppure remote, componendo \textit{Bigbam’s Brilliant Books}. «Quando passi davanti a una libreria che non hai visitato prima, devi entrare e dare un’occhiata. Questa è la regola di famiglia.»

«È la frase più Corvonero che abbia mai sentito.»

«Cosa?»

«Nulla. Signor Potter, il nostro primo passo è quello di fare visita a Gringotts, la banca del mondo magico. Il deposito della sua famiglia genetica è lì, con l’eredità che i suoi genitori \textit{genetici} le hanno lasciato, e avrà bisogno di denaro per il suo corredo scolastico.» Sospirò. «E, suppongo, un certo ammontare di denaro per i libri potrebbe essere comunque giustificato. Anche se potrebbe voler rimandare, per il momento. Hogwarts ha una biblioteca sulle materie magiche piuttosto vasta. E la torre in cui, sospetto fortemente, andrà a vivere, ha una sua biblioteca su argomenti più vari. Ogni libro che comprerebbe ora sarebbe probabilmente un duplicato.»

Harry annuì, e poi riprese a camminare.

«Non mi fraintenda, è un diversivo \textit{grandioso}», disse Harry mentre la sua testa continuava a girare, «probabilmente il miglior diversivo che qualcuno abbia mai provato con me, ma non pensi che abbia dimenticato la nostra discussione in sospeso.»

La professoressa McGonagall sospirò. «I suoi genitori — o sua madre, ad ogni modo — potrebbero essere stati molto saggi a non dirglielo.»

«Allora desidera che prosegua nella mia beata ignoranza? C’è un certo difetto in questo piano, professoressa McGonagall.»

«Suppongo che sarebbe piuttosto inutile», disse la strega fermamente, «quando chiunque per strada potrebbe raccontarle questa storia. Molto bene.»

E gli raccontò di Colui-Che-Non-Deve-Essere-Nominato, il Signore Oscuro, Voldemort.

«Voldemort?» sussurrò Harry. Sarebbe dovuto essere divertente, ma non lo era. Il nome bruciava con una sensazione di freddo, spietatezza, chiarezza cristallina, un martello in titanio puro che calava sopra un’incudine di carne cedevole. Un brivido invase Harry proprio mentre pronunciava la parola, e decise lì per lì di usare termini più sicuri come Tu-Sai-Chi.

Il Signore Oscuro aveva imperversato sulla Gran Bretagna magica come un lupo feroce, strappando e lacerando il tessuto della loro vita quotidiana. Altri Paesi si erano contorti le mani ma avevano esitato a intervenire, per apatico egoismo o semplice paura, poiché chiunque fosse stato il primo tra loro a opporsi al Signore Oscuro, avrebbe segnato la propria pace come l’obiettivo successivo del suo terrore.

(\textit{L’effetto dello spettatore}, pensò Harry, riferendosi all’esperimento di Latane e Darley che aveva mostrato che era più probabile ricevere aiuto se si aveva un attacco epilettico in presenza di una persona sola invece che di tre. \textit{Diffusione della responsabilità, ciascuno spera che qualcun altro intervenga per primo.})

I Mangiamorte erano arrivati nella scia del Signore Oscuro e nella sua avanguardia, avvoltoi per tormentare le ferite, o serpenti per mordere e indebolire. I Mangiamorte non erano così terribili come il Signore Oscuro, ma erano terribili, ed erano molti. E i Mangiamorte brandivano ben più delle bacchette; c’era ricchezza all’interno di quelle file mascherate, e potere politico, e segreti usati per ricattare, per paralizzare una società che cercava di proteggere sé stessa.

Un anziano e rispettato giornalista, Yermy Wibble, aveva richiesto l’aumento delle tasse e la coscrizione. Gridò che era assurdo che molti tremassero per paura di pochi. La sua pelle, solo la sua pelle, fu trovata inchiodata al muro della redazione la mattina successiva, accanto alle pelli di sua moglie e delle loro due figlie. Ognuno voleva che fosse fatto qualcosa di più, e nessuno osava prendere l’iniziativa per proporlo. Chiunque fosse risaltato maggiormente sarebbe diventato l’esempio successivo.

Fino a quando i nomi di James e Lily Potter salirono in cima a quella lista.

E quei due sarebbero potuti morire con le loro bacchette in mano e non essere pentiti delle loro scelte, perché erano eroi; ma avevano un bambino neonato, loro figlio, Harry Potter.

Le lacrime stavano salendo agli occhi di Harry. Le asciugò via per la rabbia o forse la disperazione, \textit{Non ho conosciuto queste persone, per niente, non sono i miei genitori adesso, sarebbe inutile sentirsi così triste per loro –}

Quando Harry ebbe finito di singhiozzare nelle vesti della strega, alzò gli occhi, e si sentì un po’ meglio nel vedere le lacrime anche negli occhi della professoressa McGonagall.

«Poi, cos’è successo?» chiese Harry con voce tremante.

«Il Signore Oscuro giunse a Godric’s Hollow», disse la professoressa McGonagall in un sussurro. «Doveva essere un nascondiglio sicuro, ma foste traditi. Il Signore Oscuro uccise James, e uccise Lily, ed arrivò infine a te, alla tua culla. Ti lanciò contro la Maledizione Mortale, e quello fu il momento in cui tutto finì. La Maledizione Mortale è formata da odio puro, e colpisce direttamente l’anima, separandola dal corpo. Non può essere bloccata, e chiunque ne sia colpito muore. Ma tu sei sopravvissuto. Tu sei l’unica persona al mondo a essere sopravvissuta. La Maledizione Mortale rimbalzò e colpì il Signore Oscuro, lasciando solo la carcassa bruciata del suo corpo e una cicatrice sulla tua fronte. Quella fu la fine del terrore, e fummo liberi. Questo, Harry Potter, è il motivo per cui la gente vuole vedere la cicatrice sulla tua fronte, e perché vuole stringerti la mano.»

La tempesta di pianto che si era riversata attraverso Harry aveva esaurito tutte le sue lacrime; non poteva più piangere, aveva finito.

(E da qualche parte nei recessi della sua mente c’era una piccola, piccola nota di confusione, la sensazione che ci fosse qualcosa di sbagliato in quella storia; e sarebbe dovuta essere una delle tecniche di Harry notare quella minuscola nota, ma era distratto. Poiché è una triste regola che ogni volta che si ha più bisogno delle proprie tecniche da razionalista, quello è il momento in cui è più probabile che siano dimenticate.)

Harry si staccò dal fianco della professoressa McGonagall. «Dovrò — pensarci su», disse cercando di mantenere la propria voce sotto controllo. Si guardò le scarpe. «Uhm. Può riferirsi a loro come ai miei genitori, se vuole, non deve dire ‘genitori genetici’ o cose così. Credo che non ci sia motivo per cui non possa avere due madri e due padri.»

Dalla professoressa McGonagall non provenne alcun suono.

E camminarono insieme in silenzio, finché giunsero davanti a un grande edificio bianco con enormi porte di bronzo, e le parole che vi erano scolpite sopra recitavano \textit{Gringotts Bank}.
