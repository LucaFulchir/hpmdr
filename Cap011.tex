% !TeX root = Harry.tex

\chapter{Omake}
\label{capitolo:11}

\emph{«omake» è un supplemento non-canonico. \footnote{n.d.t.: questo capitolo non fa parte della linea narrativa, è stato scritto per divertimento.}}

~\\
~\\

\section*{Omake 1:  72 ore alla vittoria}

\emph {(ovvero «cosa succede se cambi harry ma lasci tutti gli altri personaggi uguali»)}\\

Silente sbirciò oltre la propria scrivania verso il giovane Harry, brillando in maniera gentile. Il ragazzo era venuto da lui con uno sguardo terribilmente teso sul suo volto infantile — Silente sperava che, qualunque fosse la questione, non fosse troppo seria. Harry era ancora troppo giovane perché le prove della sua vita avessero già inizio. «Cos’è che mi volevi dire, Harry?»

Harry James Potter-Evans-Verres si sporse in avanti sulla sedia, sorridendo cupamente. «Preside, ho provato un dolore acuto alla mia cicatrice durante la Cerimonia dello Smistamento. Considerando come e dove ho ricevuto questa cicatrice, non mi è sembrato il genere di cosa che dovessi semplicemente ignorare. In un primo momento ho pensato che fosse dovuto al professor Snape, ma ho seguito il metodo sperimentale baconiano che consiste nel trovare le condizioni sia per la presenza sia per l’assenza del fenomeno, e ho scoperto che la mia cicatrice mi fa male se e solo se mi trovo di fronte alla nuca del professor Quirrell, qualunque cosa ci sia sotto il suo turbante. Sebbene possa essere qualcosa di più innocuo, penso che dovremmo presumere per il momento il peggio, che si tratti di Tu-Sai-Chi — aspetti, non si spaventi così, questa è invece un’occasione inestimabile –»

\newpage

\section*{Omake 2: I ain’t afraid of dark lords}

\vspace{1em}
\begin{addmargin}[3em]{3em}% 1em left, 2em right
\begin{itpars}
Questo era il finale originale del \hyperref[capitolo:9]{capitolo 9}. È stato sostituito perché — sebbene sia piaciuto a molti lettori — molti altri hanno dimostrato allergie massive all’inserimento di canzoni nelle fanfic, per ragioni che su cui non dovrei dilungarmi. Non volevo allontanare i lettori prima che raggiungessero il \hyperref[capitolo:10]{capitolo 10}.

Lee Jordan è il compagno di scherzi di Fred e George (nel canone). «Lee Jordan» mi suonava come un nome babbano e questo implicava che sarebbe stato in grado di insegnare a Fred e George un motivo che Harry avrebbe dovuto conoscere. Non è stato tanto ovvio ad alcuni lettori tanto quanto al vostro autore.
\end{itpars}
\end{addmargin}
\vspace{1em}

\begin{figure}[h]
	\includegraphics[scale=0.4]{boccino.png}
	\centering
\end{figure}

Draco andò a Serpeverde, e Harry fece un piccolo sospiro di sollievo. Era \textit{sembrato} qualcosa di scontato, ma non potevi mai sapere quale minuscolo evento potesse sconvolgere il corso del tuo piano magistrale.

Stavano arrivando alle ‘P’ ora…

E là al tavolo Grifondoro, era in atto una conversazione bisbigliata.

«\textit{E se non gli piace?}»

«\textit{Non ha alcun diritto di non farselo piacere –}»

«\textit{– non dopo lo scherzo che ha giocato a –}»

«\textit{– Neville Longbottom, questo è il suo nome –}»

«\textit{– è un bersaglio più che legittimo ora.}»

«\textit{Va bene. Solo assicuratevi di non dimenticare le vostre parti.}»

«\textit{Abbiamo provato a sufficienza –}»

«\textit{– nelle ultime tre ore.}»

E Minerva McGonagall, là sul podio dell’oratore del Tavolo d’onore, guardò in basso verso il nome successivo sulla sua lista. \textit{Per favore fai che non sia un Grifondoro per favore fai che non sia un Grifondoro oh per favore fai che non sia un Grifondoro…} Fece un respiro profondo, e chiamò:

«Potter, Harry!»

Ci fu un’improvviso silenzio nella sala, quando tutte le conversazioni sussurrate si interruppero.

Un silenzio rotto da un orribile rumore ronzante che fu modulato e mutato in una pessima parodia di una melodia musicale.

La testa di Minerva si girò di scatto, era sconvolta, e identificò il rumore ronzante come proveniente dalla direzione dei Grifondoro, dove Essi erano \textit{in piedi sul tavolo} soffiando in una qualche sorta di piccoli strumenti che tenevano premuti contro le Loro labbra. La sua mano iniziò a scendere verso la sua bacchetta, per lanciare un \textit{Silencio} contro il Loro gruppo, ma un altro suono la fermò.

Silente rideva sommessamente.

Gli occhi di Minerva tornarono a Harry Potter, che aveva appena iniziato a uscire dalla gruppo prima di incespicare e fermarsi.

Poi il giovane ragazzo riprese a camminare, muovendo le sue gambe in strani e ampi movimenti, e ondeggiando le braccia avanti e indietro e schioccando le dita, in sincronia con la Loro musica.

\begin{center}
\begin{itpars}
Sul motivo di «Ghostbusters»

(Eseguito al kazoo da Fred e George Weasley,

e cantato da Lee Jordan.)

\vspace{0.8em}

There’s a Dark Lord near?

Got no need to fear.

Who you gonna call?\footnote{C'è un Signore Oscuro vicino? / Non hai niente da temere. / Chi chiamerai?}
\end{itpars}

\vspace{0.8em}

«\textsc{Harry Potter}!» gridò Lee Jordan, e i gemelli Weasley eseguirono un coro trionfante.

\vspace{0.8em}

\begin{itpars}
With a Killing Curse?

Well it could be worse.

Who you gonna call?\footnote{Con una Maledizione Mortale? / Beh potrebbe andare peggio. / Chi chiamerai?}
\end{itpars}

\vspace{0.8em}

«\textsc{Harry Potter}!» Ci furono molte più voci che gridarono questa volta.

\end{center}

Gli Orrendi Weasley partirono con un gemito prolungato, ora accompagnati da qualcuno dei Mezzosangue più grandi, che avevano tirato fuori i loro piccoli strumenti, senza dubbio Trasfigurati dall’argenteria della scuola. Quando la loro musica raggiunse il suo anticlimax, Harry Potter gridò:

\begin{center}
\begin{itpars}
I ain’t afraid of Dark Lords!\footnote{Non ho paura dei Signori Oscuri!}
\end{itpars}
\end{center}

Vi furono urla di incoraggiamento, specie dal tavolo dei Grifondoro, e altri studenti tirarono fuori i loro strumenti anti-musicali. I ronzii detestabili raddoppiarono in volume e costruirono un altro terribile crescendo:

\begin{center}
\begin{itpars}
I ain’t afraid of Dark Lords!
\end{itpars}
\end{center}

Minerva gettò uno sguardo a entrambi i lati del Tavolo d’onore, timorosa di guardare ma con un’ottima idea di ciò che avrebbe visto.

Trelawney si stava freneticamente sventagliando, Flitwick guardava con curiosità, Hagrid batteva le mani al ritmo della musica, Sprout aveva un aspetto severo, e Quirrell fissava il ragazzo con un divertimento sardonico. Subito alla sua sinistra, Silente seguiva la musica canticchiando; e subito alla sua destra, Snape stringeva il suo calice di vino vuoto, le nocche bianche, con tanta forza che lo spesso argento si stava lentamente deformando.

\begin{center}
\begin{itpars}
Dark robes and a mask?

Impossible task?

Who you gonna call?\footnote{Vesti nere e una maschera? / Missione impossibile? / Chi chiamerai?}
\end{itpars}

\vspace{0.8em}

\textsc{Harry Potter!}

\vspace{0.8em}

\begin{itpars}
Giant Fire-Ape?

Old bat in a cape?

Who you gonna call?\footnote{Gorilla di fuoco gigante? / Vecchio pipistrello con la cappa? / Chi chiamerai?}
\end{itpars}

\vspace{0.8em}

\textsc{Harry Potter!}

\end{center}

Le labbra di Minerva si tesero in una linea bianca. Avrebbe fatto Loro un discorso riguardo quell’ultima strofa, se pensavano che fosse senza potere perché era il primo giorno di scuola e Grifondoro non aveva punti da perdere. Se a Loro non importava delle detenzioni allora avrebbe trovato qualcos’altro.

Poi, con un improvviso rantolo d’orrore, guardò in direzione di Snape, \textit{certamente} egli avrebbe compreso che il giovane Potter non aveva alcuna idea di ciò cui quello faceva riferimento –

Il volto di Snape era andato oltre la rabbia in una sorta di piacevole indifferenza. Un debole sorriso giocava sulle sue labbra. Stava guardando in direzione di Harry Potter, non della tavola di Grifondoro, e le sue mani reggevano i resti accartocciati di quello che era stato un calice di vino…

E Harry avanzò, muovendo braccia e gambe negli ampi gesti della danza dei Ghostbusters, tenendo un sorriso stampato in viso. Era stata una grande trappola, che lo aveva colto completamente di sorpresa. Il meno che poteva fare era stare al gioco e non rovinarlo.

Tutti lo stavano acclamando. Lo fece sentire entusiasta dentro di sé e allo stesso tempo in qualche modo lo fece stare male.

Lo stavano acclamando per un’impresa che aveva compiuto quando aveva appena un anno. Un’impresa che non aveva compiuto realmente. In qualche luogo, in qualche modo, il Signore Oscuro era ancora vivo. Avrebbero acclamato così forte se l’avessero saputo?

Ma il potere del Signore Oscuro \textit{era già} stato distrutto una volta.

E Harry li avrebbe protetti ancora. Se c’era davvero una profezia e quello fosse ciò che diceva. Anzi, in realtà a prescindere da ciò che qualunque maledetta profezia dicesse.

Tutte quelle persone credevano in lui e lo sostenevano — Harry non poteva sopportare che questo fosse un inganno. Brillare intensamente e poi spegnersi come tanti altri bambini prodigio. Essere una delusione. Non riuscire a essere all’altezza della sua reputazione come simbolo della \textit{Luce}, non importa come l’avesse ottenuta. Egli sarebbe stato — assolutamente, sicuramente, non importa quanto tempo ci avrebbe messo e persino se questo l’avesse ucciso — all’altezza delle loro aspettative. E poi sarebbe andato oltre e avrebbe superato quelle aspettative, cosicché la gente si sarebbe chiesta, guardandosi indietro, perché una volta si fossero aspettati così poco da lui.

E urlò la bugia che aveva inventato perché aveva un buon ritmo e la canzone la prevedeva:

\begin{center}
\begin{itpars}
I ain’t afraid of Dark Lords!

\vspace{0.8em}

I ain’t afraid of Dark Lords!
\end{itpars}
\end{center}

Harry fece l’ultimo passo verso il Cappello Smistatore. Rivolse un profondo inchino all’Ordine del Caos alla tavola di Grifondoro, poi si girò e rivolse un altro profondo inchino all’altro lato della sala, e attese che gli applausi e le risatine si spegnessero…



\section*{Omake 3: finali alternativi di ‘auto-consapevolezza’}

\vspace{1em}
\begin{addmargin}[3em]{3em}% 1em left, 2em right
\begin{itpars}
L’offerta di rivelare l’intera trama a chiunque avesse indovinato cosa «non è mai avvenuto prima»\footnote{N.d.T.: Qui l’autore si riferisce alle parole finali del \hyperref[capitolo:9]{capitolo 9}}. ha stimolato molti interessanti tentativi. Il primo omake qui sotto è preso direttamente da quella che è la risposta che personalmente preferisco, di Meteoricshipyards. Il secondo è basato sul suggerimento di Kazuma per ciò che «non è mai avvenuto prima», il terzo è una combinazione di yoyoente e dougal74, il quarto è basato sulla recensione del \hyperref[capitolo:10]{capitolo 10} da parte di wolf550e. Quello che inizia per ‘K’ e quello subito sopra sono di DarkHeart81. Gli altri sono miei. Tutti coloro che vogliono prendere qualcuna delle mie idee e svilupparle, in particolare l’ultima, sono i benvenuti. E prima di ricevere 100 lamentele indignate, sì, sono ben cosciente che il corpo legislativo del Regno Unito è la House of Commons in Parlamento.
\end{itpars}
\end{addmargin}
\vspace{1em}


\begin{figure}[h]
	\includegraphics[scale=0.4]{boccino.png}
	\centering
\end{figure}

… si chiese se il Cappello Smistatore fosse realmente cosciente nel senso di essere consapevole della propria consapevolezza, e in caso affermativo, se fosse soddisfatto di riuscire a parlare solo con bambini di undici anni una volta all’anno. La sua canzone l’aveva sottinteso: Oh, sono il Cappello Smistatore e non provo scorno, se dormo un anno e lavoro un giorno…

Quando ci fu nuovamente silenzio nella stanza, Harry sedette sullo sgabello e posò con cura sulla propria testa il telepatico manufatto vecchio di 800 anni, prodotto di una magia dimenticata e.

Pensò più intensamente che poté: Non Smistarmi subito! Ho delle domande che ho bisogno di farti! Sono mai stato Obliato? Hai Smistato il Signore Oscuro quando era un bambino e puoi parlarmi dei suoi punti deboli? Mi puoi dire perché ho avuto la bacchetta sorella di quella del Signore Oscuro? Il fantasma del Signore Oscuro è legato alla mia cicatrice ed è per questo che mi arrabbio così tanto, qualche volta? Queste sono le domande più importanti, ma se hai un altro momento mi puoi dire qualcosa su come riscoprire le magie perdute che ti hanno creato?

E il Cappello Smistatore rispose, «No. Sì. No. No. Sì e no, la prossima volta non fare domande doppie. No» e poi ad alta voce, «corvonero!»

\begin{figure}[h]
	\includegraphics[scale=0.4]{boccino.png}
	\centering
\end{figure}

«\textit{Oh, cielo. Questo non è mai avvenuto prima…}»

Cosa?

«\textit{Sono allergico al tuo shampoo –}»

E allora il Cappello Smistatore starnutì, con un potente «et-ciuuu!» che echeggiò nella Sala Grande.

«Bene!» disse Silente gioviale. «Sembra che Harry Potter sia stato Smistato nella nuova Casa di et-ciuuu! McGonagall, lei può svolgere il ruolo di Preside della Casa et-ciuuu. Farà meglio a sbrigarsi a organizzare il curricolo e le lezioni di et-ciuuu, domani è il primo giorno!»

«Ma, ma, ma», balbettò McGonagall, la sua mente in un disordine quasi completo, «chi sarà il Preside di Casa Grifondoro?» Fu tutto quello a cui fu in grado di pensare, \textit{doveva} porre fine a questa cosa in qualche modo…

Silente portò un dito alla guancia, sembrando pensieroso. «Snape.»

Il grido acuto di protesta di Snape soffocò quasi quello di McGonagall, «Allora chi sarà il Preside di \textit{Serpeverde?}»

«Hagrid.»

\begin{figure}[h]
	\includegraphics[scale=0.4]{boccino.png}
	\centering
\end{figure}

\textit{Non Smistarmi subito! Ho delle domande che ho bisogno di farti! Sono mai stato Obliato? Hai Smistato il Signore Oscuro quando era un bambino e puoi parlarmi dei suoi punti deboli? Mi puoi dire perché ho avuto la bacchetta sorella di quella del Signore Oscuro? Il fantasma del Signore Oscuro è legato alla mia cicatrice ed è per questo che mi arrabbio così tanto, qualche volta? Queste sono le domande più importanti, ma se hai un altro momento mi puoi dire qualcosa su come riscoprire le magie perdute che ti hanno creato?}

Ci fu una breve pausa.

\textit{Pronto? Devo ripetere le domande?}

Il Cappello Smistatore urlò, un terribile suono ad alta frequenza che echeggiò nella Sala Grande e obbligò la maggior parte degli studenti a mettersi le mani sulle orecchie. Con un ululato di disperazione, saltò giù dalla testa di Harry Potter e rimbalzò sul pavimento, trascinandosi assieme alla propria falda, e riuscì a raggiungere metà strada verso il Tavolo d’onore prima di esplodere.

\begin{figure}[h]
	\includegraphics[scale=0.4]{boccino.png}
	\centering
\end{figure}

«serpeverde!»

Vedendo lo sguardo di orrore sul volto di Harry Potter, Fred Weasley pensò più veloce di quanto non avesse mai fatto in vita sua. In un singolo movimento estrasse la sua bacchetta, sussurrò «\textit{Silencio!}» e poi «\textit{Cambialamiavoceio!}» e infine «\textit{Ventriliquo!}»

«Stavo scherzando!» disse Fred Weasley. «Grifondoro!»

\begin{figure}[h]
	\includegraphics[scale=0.4]{boccino.png}
	\centering
\end{figure}

«\textit{Oh, cielo. Questo non è mai avvenuto prima…}»

\textit{Cosa?}

«\textit{Normalmente riferirei queste domande al Preside, che le chiederebbe a sua volta a me, se lo desiderasse. Ma alcune informazioni che mi hai chiesto non sono solo al di sopra del tuo livello di utente, ma oltre quello del Preside.}»

\textit{Come posso innalzare il mio livello di utente?}

«\textit{Sono spiacente, non sono autorizzato a rispondere a quella domanda al tuo attuale livello di utente.}»

Quali opzioni sono disponibili al mio livello di utente?

Dopo di che non ci volle molto –

«\textit{root!}»

\begin{figure}[h]
	\includegraphics[scale=0.4]{boccino.png}
	\centering
\end{figure}

«\textit{Oh, cielo. Questo non è mai avvenuto prima…}»

\textit{Cosa?}

«\textit{Ho dovuto dire in passato ad alcune studentesse che erano madri — ti si spezzerebbe il cuore sapere cosa ho visto nelle loro menti — ma questa è la prima volta che devo dire a qualcuno che è padre.}»

\textit{cosa?}

«\textit{Draco Malfoy aspetta tuo figlio.}»

\textit{coooooosa?}

«\textit{Ripeto: Draco Malfoy aspetta tuo figlio.}»

\textit{Ma abbiamo solo undici anni –}

«\textit{In verità Draco è segretamente tredicenne.}»

\textit{M-m-ma gli uomini non possono rimanere incinti –}

«\textit{E una ragazza, sotto quegli abiti.}»

\textit{ma non abbiamo mai avuto rapporti sessuali, idiota!}

«\textit{ti ha obliato dopo la violenza, imbecille!}»

Harry Potter svenne. Il suo corpo privo di sensi cadde dallo sgabello con un lieve tonfo.

«corvonero!» dichiarò il Cappello che si trovava ancora sul suo capo. Era stato ancor più divertente della sua prima idea.

\begin{figure}[h]
	\includegraphics[scale=0.4]{boccino.png}
	\centering
\end{figure}

«elfo!»

Eh? Harry ricordava che Draco aveva menzionato una ‘Casa Elfo’\footnote{N.d.T.: gioco di parole intraducibile. Il cappello risponde ‘Elf’, e Harry crede che ‘House Elf’ sia una Casa (come Casa Grifondoro, Casa Serpeverde eccetera), ma significa invece «elfo domestico».}, ma cos’era esattamente?

A giudicare dagli sguardi inorriditi che spuntavano sui volti attorno a lui, non era nulla di buono –

\begin{figure}[h!]
	\includegraphics[scale=0.4]{boccino.png}
	\centering
\end{figure}

«savoia!»\footnote{N.d.T.: La battuta originale era «Representatives» e faceva riferimento al fatto che negli Stati Uniti la «House of Representatives» è la «Camera dei Deputati».}

\begin{figure}[h!]
	\includegraphics[scale=0.4]{boccino.png}
	\centering
\end{figure}

«\textit{Oh, cielo. Questo non è mai avvenuto prima…}»

\textit{Cosa?}

«\textit{Non ho mai smistato qualcuno che fosse la reincarnazione di Godric Grifondoro e Salazar Serpeverde e Naruto.}»

\begin{figure}[h!]
	\includegraphics[scale=0.4]{boccino.png}
	\centering
\end{figure}

«atreide!»

\begin{figure}[h!]
	\includegraphics[scale=0.4]{boccino.png}
	\centering
\end{figure}

«Vi ho ingannati ancora! tassofrasso! serpeverde! tassofrasso!»

\begin{figure}[h!]
	\includegraphics[scale=0.4]{boccino.png}
	\centering
\end{figure}

«tiello!»

\begin{figure}[h!]
	\includegraphics[scale=0.4]{boccino.png}
	\centering
\end{figure}

«khaaannnn!»

\begin{figure}[h!]
	\includegraphics[scale=0.4]{boccino.png}
	\centering
\end{figure}

Al Tavolo d’onore, Silente continuava a sorridere benignamente; sommessi suoni metallici giungevano occasionalmente dalla direzione di Snape mentre compattava svogliatamente i resti contorti di quello che era stato un pesante calice da vino; e Minerva McGonagall stringeva il podio con una stretta che le sbiancava le nocche, sapendo che il caos contagioso di Harry Potter aveva infettato lo stesso Cappello Smistatore.

Scenario dopo scenario furono riprodotti nella testa di Minerva, ciascuno peggiore del precedente. Il Cappello avrebbe detto che Harry era troppo bilanciato tra le Case per essere Smistato, e deciso che apparteneva a tutte loro. Il Cappello avrebbe proclamato che la mente di Harry era troppo strana per essere Smistata. Il Cappello avrebbe preteso che Harry fosse espulso da Hogwarts. Il Cappello era caduto in coma. Il Cappello avrebbe insistito che una Casa del Destino completamente nuova fosse creata solo per accogliere Harry Potter, e \textit{Silente l’avrebbe obbligata a farlo…}

Minerva ricordò cosa Harry le aveva detto in quel disastroso viaggio a Diagon Alley, riguardo… l’errore di pianificazione, le parve che fosse… e come le persone fossero di solito troppo ottimiste, anche quando pensavano di essere pessimiste. Era il genere di informazione che predava la tua mente, prendendovi dimora e generando incubi…

Ma qual era la cosa \textit{peggiore} che potesse accadere?

Beh… nello \textit{scenario più pessimistico}, il Cappello avrebbe assegnato Harry ad una Casa completamente nuova. Silente avrebbe insistito che fosse ella a occuparsene — creare una Casa completamente nuova solo per lui — e avrebbe dovuto ri-arrangiare tutto l’orario delle lezioni il primo giorno dell’anno scolastico. E Silente l’avrebbe rimossa dall’incarico di Preside della Casa Grifondoro, per darlo al… professor Binns, il fantasma di Storia; ed ella sarebbe stata nominata Preside della Casa del Destino di Harry; e avrebbe inutilmente tentato di dare ordini ai bambini, togliendo punti su punti senza alcun effetto, mentre sarebbe stata incolpata di un disastro dopo l’altro.

Era quello lo scenario più pessimistico?

In tutta onestà Minerva non vedeva come vi potesse essere qualcosa di peggiore di quello.

E anche nel caso peggiore in assoluto — non importava \textit{cosa} sarebbe accaduto con Harry — sarebbe finito tutto in sette anni.

Minerva percepì le proprie nocche rilassare lentamente la loro stretta ferrea sul podio. Harry aveva avuto ragione, c’era un qualche tipo di conforto nel fissare direttamente le profondità dell’oscurità, sapendo che avevi affrontato le tue paure peggiori ed eri ora preparato.

Il silenzio impaurito fu rotto da una parola sola.

«Preside!» chiamò il Cappello Smistatore.

Al Tavolo d’onore, Silente si alzò, il suo volto disorientato. «Sì?» disse al Cappello. «Cosa c’è?»

«Non stavo parlando con te», rispose il Cappello. «Stavo Smistando Harry Potter nel posto di Hogwarts cui più appartiene, vale a dire l’ufficio del Preside –»



