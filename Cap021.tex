% !TeX root = Harry.tex

\chapter{Razionalizzazione}
\label{capitolo:21}

Hermione Granger era preoccupata di stare diventando Cattiva.

La differenza tra Buono e Cattivo era generalmente facile da riconoscere, non aveva mai capito perché le altre persone avessero così tanti problemi. A Hogwarts, «Buono» erano il professor Flitwick, la professoressa McGonagall e la professoressa Sprout. «Cattivo» erano il professor Snape, il professor Quirrell e Draco Malfoy. Harry Potter… era uno dei quei casi particolari in cui \textit{non potevi} dirlo solo guardandolo. Stava ancora cercando di capire a quale schieramento appartenesse.

Ma per quanto riguardava \textit{sé stessa…}

Hermione si stava \textit{divertendo troppo} ad annientare Harry Potter.

Aveva fatto meglio di lui in ogni singolo corso che avevano seguito. (Fatta eccezione per volo con la scopa, che era come ginnastica, non contava.) Aveva guadagnato \textit{veri} punti-Casa quasi ogni giorno della loro prima settimana, non per bizzarre attività eroiche, ma per cose \textit{brillanti} come apprendere velocemente gli incantesimi e aiutare gli altri studenti. Sapeva che quel genere di punti-Casa era di un altro livello, e la cosa migliore era che lo sapeva anche Harry Potter. Lo poteva vedere nei suoi occhi ogni volta che lei guadagnava un altro \textit{vero} punto-Casa.

Se fosse stata buona, non avrebbe dovuto godere così tanto a vincere.

Era iniziato il giorno del viaggio in treno, anche se a quella tempesta c’era voluto un po’ per placarsi. Era stato solo quella notte sul tardi che Hermione aveva cominciato a rendersi conto di \textit{quanto} avesse permesso a quel ragazzo di trattarla male.

Prima di incontrare Harry Potter non c’era mai stato nessuno che avesse voluto annientare. Se qualcuno non stava facendo bene come lei in una materia, era suo dovere aiutarlo, non rinfacciarglielo. Era questo che significava essere Buono.

E adesso…

… adesso stava \textit{vincendo}, Harry Potter trasaliva ogni volta che lei otteneva un altro punto-Casa, ed era \textit{così} divertente, i suoi genitori l’avevano messa in guardia contro la droga e sospettava che quello fosse \textit{più divertente} della droga.

Le erano sempre piaciuti i sorrisi che gli insegnanti le rivolgevano quando faceva qualcosa correttamente. Aveva sempre amato osservare la lunga fila di spuntature su di un compito svolto in maniera perfetta. Ma ora, quando andava bene a lezione, dava casualmente un’occhiata in giro e osservava fugacemente Harry Potter che digrignava i denti, e la cosa le faceva venir voglia di scoppiare a cantare come in un film Disney.

E questo era Cattivo, no?

Hermione era preoccupata di stare diventando Cattiva.

E poi le sovvenne un pensiero che spazzò via tutte le sue paure.

Lei e Harry stavano per iniziare una Storia d’amore! Certo! Tutti sapevano cosa voleva dire quando un ragazzo e una ragazza iniziavano a bisticciare tutto il tempo. Si stavano \textit{corteggiando} l’un l’altra! Non c’era nulla di Cattivo in \textit{quello}.

Non era plausibile che stesse semplicemente \textit{godendo} nello stracciare il più famoso allievo nella scuola, qualcuno che era \textit{nei} libri e \textit{parlava} come un libro, il ragazzo che aveva in qualche modo sconfitto il Signore Oscuro e anche schiacciato il professor Snape come un piccolo e triste insetto, il ragazzo che era, come il professor Quirrell avrebbe detto, dominante, su tutti gli altri Corvonero del primo anno, \textit{a eccezione} di Hermione Granger, che stava completamente \textit{stracciando} il Ragazzo-Che-È-Sopravvissuto in tutti i corsi, tranne volo con la scopa.

Perché quello sarebbe stato Cattivo.

No. Era una Storia d’amore. \textit{Ecco} cos’era. \textit{Ecco} perché litigavano.

Hermione era contenta di averlo capito in tempo per quel giorno, quando Harry avrebbe perso la loro gara di lettura, di cui \textit{tutta la scuola} era al corrente, e desiderò iniziare a \textit{danzare} per la pura e travolgente gioia che provava.

Erano le 14:45 di sabato e Harry Potter aveva ancora metà di \textit{Storia della magia} di Bathilda Bagshot da leggere, e Hermione stava fissando il proprio orologio da tasca mentre ticchettava con terribile lentezza verso le 14:47.

E l’intera sala comune di Corvonero stava guardando.

Non erano solo studenti del primo anno, la notizia si era diffusa come un fulmine e una buona metà dei Corvonero avevano affollato la stanza, spremuti nei divani e appoggiati alle librerie e seduti sui braccioli delle sedie. Tutti e sei i prefetti erano lì, inclusa la Caposcuola di Hogwarts. Qualcuno aveva avuto bisogno di lanciare un Incantesimo Rinfresca-Aria solo per far sì che vi fosse sufficiente ossigeno. E il frastuono della conversazione si era affievolito in sussurri che ora si erano spenti nel silenzio più totale.

14:46.

La tensione era insopportabile. Se fosse stato qualcun altro, \textit{chiunque} altro, la sua sconfitta sarebbe stata una conclusione scontata.

Ma qui si trattava di Harry Potter, e non si poteva escludere la possibilità che avrebbe, in qualche momento nei prossimi secondi, alzato una mano e schioccato le dita.

Con un terrore improvviso si rese conto che Harry Potter avrebbe potuto fare esattamente quello. Sarebbe stato \textit{tipico di lui} aver già \textit{finito di leggere} in precedenza la seconda metà del libro…

La vista di Hermione iniziò a ondeggiare. Cercò di obbligarsi a respirare, e scoprì che semplicemente non poteva.

Dieci secondi al termine, e ancora non aveva alzato la mano.

Cinque secondi al termine.

14:47.

Harry Potter mise con cura un segnalibro nel suo libro, lo chiuse, e lo mise da parte.

«Vorrei sottolineare a beneficio dei posteri», disse il Ragazzo-Che-È-Sopravvissuto con voce limpida, «che mi era rimasta solo la metà di un libro, e che mi sono imbattuto in una serie di ritardi imprevisti –»

«\textit{Hai perso!}» gridò Hermione. «È \textit{così!} Hai \textit{perso la nostra gara!}»

Ci fu un’espirazione collettiva quando tutti ripresero nuovamente a respirare.

Harry Potter le lanciò uno Sguardo di Fuoco Fiammeggiante, ma ella stava fluttuando in un alone di felicità pura e bianca e nulla poteva toccarla.

«\textit{Ti rendi conto di che tipo di settimana ho avuto?}» chiese Harry Potter. «Chiunque con una tempra più debole avrebbe avuto difficoltà a leggere otto libri del Dr. Seuss!»

«\textit{Tu} hai scelto la scadenza.»

Lo Sguardo di Fuoco Fiammeggiante di Harry divenne ancor più caldo. «Non avevo alcun modo logico di sapere che avrei dovuto salvare l’intera scuola dal professor Snape, o essere picchiato alla lezione di Difesa, e se ti dicessi come ho perso tutto il tempo tra le 17 e la cena di giovedì penseresti che sono pazzo –»

«Ooooh, sembra che \textit{qualcuno} sia caduto vittima dell’\textit{errore di pianificazione.}»

Il colpo subito fu palese sul volto di Harry Potter.

«Oh, questo mi ricorda che ho finito di leggere il primo lotto di libri che mi hai prestato», disse Hermione con la sua migliore espressione innocente. Un paio di loro erano stati anche libri \textit{difficili}. Si chiese quanto tempo ci avesse messo \textit{lui} per finirli.

«Un giorno», disse il Ragazzo-Che-È-Sopravvissuto, «quando i lontani discendenti dell’\textit{Homo sapiens} ripercorreranno la storia della galassia e si chiederanno come possa essere andato tutto così male, concluderanno che l’errore originale fu quando qualcuno insegnò a Hermione Granger a leggere.»

«Ma tu hai perso comunque», disse Hermione. Portò una mano al mento e sembrò pensierosa. «Ora, esattamente, che pegno dovrai pagare, mi chiedo?»

«\textit{Cosa?}»

«Hai perso la scommessa», spiegò Hermione, «quindi devi pagare pegno.»

«Non ricordo di aver accettato questa condizione!»

«Davvero?» disse Hermione Granger. Il suo viso assunse un’espressione pensierosa. Poi, come se l’idea le fosse venuta in mente solo in quel momento, «Votiamo, allora. Tutti coloro che in Corvonero pensano che Harry Potter debba pagare pegno, alzino la mano!»

«\textit{Cosa?}» urlò di nuovo Harry Potter.

Si girò e vide che era circondato da un oceano di mani alzate.

E se Harry Potter avesse guardato \textit{con più attenzione}, avrebbe notato che una terrificante maggioranza degli spettatori sembravano essere ragazze e che praticamente ogni donna nella stanza aveva alzato la mano.

«Fermi!» gemette Harry Potter. «Non sapete quello che sta per chiedere! Non vi \textit{rendete conto} di che cosa sta facendo? Vi sta convincendo a prendere un impegno anticipato ora, e poi la spinta a restare coerenti vi farà accettare tutto quello che dirà dopo!»

«Non preoccuparti», disse il prefetto Penelope Clearwater. «Se ci chiedesse qualcosa di irragionevole, possiamo semplicemente cambiare idea. Giusto ragazzi?»

E ci furono cenni d’assenso impazienti da parte di tutte le ragazze alle quali Penelope Clearwater aveva parlato del piano di Hermione.

\begin{figure}[h!]
        \includegraphics[scale=0.4]{boccino.png}
        \centering
\end{figure}

Una figura silenziosa scivolò sommessamente attraverso i freddi corridoi dei sotterranei di Hogwarts. Doveva essere presente in una certa stanza alle 18:00 per incontrare una certa persona, e se possibile era meglio arrivare prima, per mostrare rispetto.

Ma quando la sua mano girò la maniglia e aprì la porta di quell’aula buia, silenziosa, e inutilizzata, c’era già una sagoma lì in piedi, in mezzo alle file di vecchi banchi polverosi. Una sagoma che reggeva una piccola verga verde luminescente, la cui pallida luce illuminava con difficoltà persino colui che la reggeva, per non dire della stanza circostante.

La luce del corridoio morì appena la porta si chiuse dietro di lui, e gli occhi di Draco iniziarono il processo di adeguamento alla fioca luce.

La sagoma si voltò lentamente per osservarlo, rivelando un viso ombroso solo parzialmente illuminato dall’inquietante luce verde.

A Draco questo incontro piaceva già. Doveva mantenere la fredda luce verde, rendere entrambi più alti, fornire loro cappucci e maschere, spostarli da un’aula a un cimitero, e sarebbe stato identico all’inizio della metà delle storie che gli amici di suo padre raccontavano sui Mangiamorte.

«Voglio che tu sappia, Draco Malfoy», disse la sagoma nei toni di una calma mortale, «che non biasimo te per la mia recente sconfitta.»

Senza pensarci, Draco aprì la bocca per protestare, non vi era alcuna ragione plausibile per cui egli sarebbe \textit{dovuto} essere biasimato –

«È stata dovuta, più che altro, alla mia stupidità», proseguì la figura indistinta. «Ci sono state molte altre cose che avrei potuto fare, a ciascun passo lungo la strada. Tu non mi hai chiesto di fare \textit{esattamente} quello che ho fatto. Mi hai solo chiesto di aiutarti. Sono stato io a scegliere incautamente quel metodo in particolare. Ma resta il fatto che ho perso la gara per metà libro. Le azioni del tuo idiota addomesticato, e il favore che mi hai chiesto, e, sì, la mia stupidità nel soddisfarlo, mi hanno fatto \textit{perdere tempo}. Più tempo di quanto tu non sappia. Tempo che, alla resa dei conti, si è rivelato decisivo. Resta il fatto, Draco Malfoy, che se non mi avessi chiesto quel favore, \textit{avrei vinto}. E non… invece… \textit{perso}.»

Draco aveva già sentito parlare della sconfitta di Harry e del pegno che Granger aveva riscosso. La notizia si era diffusa più velocemente di quanto i gufi avrebbero potuto fare.

«Comprendo», disse Draco. «Sono dispiaciuto.» Non c’era altro che \textit{potesse dire} se voleva che Harry Potter gli fosse amico.

«Non chiedo comprensione o compassione», disse la sagoma oscura, ancora con quella calma mortale. «Ma ho appena trascorso due intere ore alla presenza di Hermione Granger, vestito con tali indumenti quali mi sono stati forniti, visitando luoghi di Hogwarts tanto affascinanti come una piccola cascata gorgogliante quello che mi è sembrato muco, accompagnato da un gruppo di altre ragazze che si sono ostinate in attività tanto utili quanto cospargere il nostro cammino con petali di rosa Trasfigurati. Sono stato a un appuntamento, rampollo dei Malfoy. Il mio \textit{primo} appuntamento. \textit{E quando vorrò riscuotere quel favore, tu lo ripagherai.}»

Draco annuì solennemente. Prima di arrivare aveva preso la saggia precauzione di informarsi su ogni dettaglio disponibile sull’appuntamento di Harry, in modo da esaurire tutte le proprie risate isteriche prima dell’ora stabilita per l’incontro, e non commettere un passo falso ridacchiando ininterrottamente fino a perdere conoscenza.

«Credi», disse Draco, «che dovrebbe accadere qualcosa di triste alla ragazza Granger –»

«Passa parola all’interno di Serpeverde che la ragazza Granger è \textit{mia} e che \textit{chiunque} si immischi nei \textit{miei} affari avrà i suoi resti sparsi su una superficie sufficientemente vasta da coprire dodici diverse lingue parlate. E dal momento che io non sono in Grifondoro e uso \textit{l’astuzia} piuttosto che gli attacchi frontali immediati, non dovranno farsi prendere dal panico se mi vedranno sorriderle.»

«O se dovessi essere visto a un secondo appuntamento?» chiese Draco, consentendosi appena una piccola nota di incredulità nella voce.

«\textit{Non ci sarà nessun secondo appuntamento}», disse la sagoma illuminata di verde, con una voce così temibile che sembrò non solo simile a un Mangiamorte, ma ad Amycus Carrow quella volta, subito prima che suo Padre gli dicesse di smetterla, che non era il Signore Oscuro.

Naturalmente era ancora la voce acuta e bianca di un giovane ragazzo, e associata alle \textit{parole pronunciate}, beh, semplicemente non era la stessa cosa. Se un giorno Harry Potter \textit{fosse} diventato il prossimo Signore Oscuro, Draco avrebbe usato un Pensatoio per conservare una copia di questa memoria in un luogo sicuro, e Harry Potter non avrebbe mai osato tradirlo.

«Ma parliamo di cose più allegre», disse la figura ombreggiata di verde. «Parliamo di conoscenza e di potere. Draco Malfoy, parliamo di Scienza.»

«Sì», disse Draco. «Parliamo.»

Draco si chiese quanta parte del suo volto potesse essere vista, e quanta fosse in ombra, in quell’irreale luce verde.

E anche se Draco mantenne il suo volto serio, c’era un sorriso nel suo cuore.

Stava \textit{finalmente} avendo una reale conversazione da adulto.

«Ti offro il potere», disse la figura in ombra, «e ti parlerò di quel potere e del suo prezzo. Il potere viene dal conoscere la forma della realtà e così facendo ottenere il controllo su di essa. Ciò che comprendi, puoi comandare, e questo è un potere sufficiente a camminare sulla Luna. Il prezzo di tale potere è che devi imparare a fare domande alla Natura e, cosa di gran lunga più difficile, accettare le risposte della Natura. Condurrai esperimenti, eseguirai prove e osserverai quello che accade. E dovrai accettare il significato di quei risultati quando ti dicono che stai sbagliando. Dovrai \textit{imparare a perdere}, non contro di me, ma contro la Natura. Quando ti troverai a litigare con la realtà, dovrai lasciar vincere la realtà. Troverai tutto ciò doloroso, Draco Malfoy, e io non so se sei così forte. Conoscendone il prezzo, è ancora tuo desiderio imparare il potere umano?»

Draco fece un respiro profondo. Ci aveva riflettuto. Ed era difficile capire come potesse rispondere in altro modo. Aveva ricevuto l’istruzione di cogliere ogni opportunità di amicizia con Harry Potter. Si trattava semplicemente di \textit{imparare}, non stava promettendo di \textit{fare} nulla. Poteva sempre interrompere le lezioni in qualsiasi momento…

Certamente c’era un numero sufficiente di aspetti di questa situazione che la facevano sembrare una trappola, ma in tutta onestà, Draco non vedeva come tutto questo potesse andare storto.

Inoltre Draco aveva una certa voglia di conquistare il mondo.

«Sì», disse Draco.

«Eccellente», replicò la figura in ombra. «Ho avuto quel che si dice una \textit{settimana affollata}, e ci vorrà tempo per pianificare il tuo piano di studi –»

«Ci sono molte cose che io stesso devo fare per consolidare il mio potere in Serpeverde», disse Draco, «per non parlare dei compiti. Forse dovremmo iniziare a ottobre?»

«Sembra ragionevole», disse la figura ombrosa, «ma quello che intendevo dire è che per pianificare il tuo piano di studi, ho bisogno di sapere che cosa ti insegnerò. Ho pensato a tre possibilità. La prima è che ti insegni la mente umana e il cervello. La seconda opzione è che ti insegni l’universo fisico, quelle arti che si trovano lungo la strada che porta alla Luna. Questo comporta una grande quantità di numeri, ma per un certo tipo di mente quei numeri sono più belli di ogni altra cosa la Scienza abbia da insegnare. Ti piacciono i numeri, Draco?»

Draco scosse la testa.

«Allora questo è determinato. Imparerai la matematica, prima o poi, ma non subito, credo. La terza opzione è che ti insegni la genetica e l’evoluzione e l’ereditarietà, ciò che potresti chiamare il sangue –»

«Questo», disse Draco.

La figura annuì. «Pensavo che avresti potuto rispondere così. Ma credo che sarà la strada più dolorosa per te, Draco. Che succede se la tua famiglia e gli amici, i puristi del sangue, dicono una cosa, e scoprissi che la prova sperimentale ne dice un’altra?»

«Allora scoprirò come far dare alla prova sperimentale la risposta \textit{giusta!}»

Ci fu una pausa, come se la figura in ombra fosse rimasta lì, con la bocca aperta, per un breve periodo.

«Uhm», disse la figura ombrosa. «Non funziona proprio così. È da questo che stavo cercando di metterti in guardia, Draco. Non \textit{puoi} far sì che la risposta sia qualunque cosa tu voglia.»

«Si può \textit{sempre} far sì che la risposta sia quella che vuoi», disse Draco. Quella era stata praticamente la prima cosa che i suoi precettori gli avevano insegnato. «È solo una questione di trovare gli argomenti giusti.»

«No», disse la figura ammantata d’ombra, la voce che si alzò per la frustrazione, «no, no, no! Allora otterresti la \textit{risposta sbagliata} e non puoi andare sulla Luna in questo modo! La Natura non è una persona, non puoi ingannarla e farle credere qualcosa di diverso, se cerchi di dire che la Luna che è fatta di formaggio puoi discutere per giorni e questo non cambierà la Luna! Ciò di cui stai parlando è la \textit{razionalizzazione}, è come iniziare con un foglio di carta, andare direttamente all’ultima linea, scrivere ‘e \textit{quindi}, la Luna è fatta di formaggio’, e poi tornar in cima per scriverci sopra ogni genere di argomentazione scaltra. Ma o la Luna è fatta di formaggio o non lo è. Nel momento in cui hai scritto l’ultima linea, era già vero o già falso. Che l’intero foglio di carta finisca per contenere la conclusione giusta o quella sbagliata si decide nell’istante in cui annoti l’ultima riga. Se stai cercando di scegliere tra due bauli costosi, e ti piace quello lucido, non importa quali argomenti intelligenti riesci a trovare per acquistarlo, la \textit{vera} regola che hai usato per \textit{scegliere in favore di quale baule argomentare} è stata ‘scegli quello lucido’, e otterrai un baule tanto buono quanto quella regola è in grado di scegliere. La razionalità \textit{non può} essere usata per sostenere una posizione prefissata, il suo unico uso possibile è \textit{decidere da che parte schierarsi}. Lo scopo della scienza non è \textit{convincere} qualcuno che i puristi del sangue hanno ragione. Questa è \textit{politica}! Il potere della scienza deriva dallo \textit{scoprire il modo in cui la Natura è realmente e che non può essere cambiato discutendo!} Ciò che la scienza \textit{può} fare è dirci \textit{come il sangue funziona realmente}, come i maghi ereditano realmente i propri poteri dai loro genitori, e se i Nati babbani sono davvero più deboli o più forti –»

«\textit{Più forti!}» sbottò Draco. Aveva cercato di seguire il ragionamento, un’espressione perplessa sul volto, e poteva capire come avesse \textit{una specie} di senso ma certamente non era simile a nulla che avesse mai sentito prima. E poi Harry Potter aveva detto qualcosa che Draco non poteva lasciar passare. «Pensi che i Sanguemarcio siano \textit{più forti?}»

«Non penso niente», disse la figura ombrosa. «Non so niente. Non credo in niente. La mia ultima riga non è stata ancora scritta. Scoprirò come mettere alla prova il potere magico dei Nati babbani, e il potere magico dei purosangue. Se le mie prove mi diranno che i Nati babbani sono più deboli, crederò che sono più deboli. Se le mie prove mi diranno che i Nati babbani sono più forti, crederò che sono più forti. Con la conoscenza di questa e altre verità, guadagnerò una certa quantità di potere –»

«E ti aspetti che \textit{io} creda a qualunque cosa tu dica?» Draco domandò infervorato.

«Mi aspetto che tu esegua le prove \textit{personalmente}», disse tranquillamente la figura in ombra. «Hai paura di ciò che \textit{tu} scoprirai?»

Draco fissò la figura ammantata dall’ombra per un po’, con gli occhi socchiusi. «Bella trappola, Harry», disse. «Dovrò ricordarmela, è nuova.»

La figura ombrosa scosse la testa. «Non è una trappola, Draco. Ricorda — \textit{io non so} cosa scopriremo. Ma non si comprende l’universo discutendo con lui o dicendogli di tornare con una risposta diversa la prossima volta. Quando indossi le vesti dello scienziato devi dimenticare tutta la tua politica e le discussioni e le fazioni e i partiti, mettere a tacere le posizioni disperatamente arroccate della tua mente, e desiderare solo di ascoltare la risposta della Natura.» La figura ombrosa fece una pausa. «La maggior parte delle persone non può farlo. Ecco perché è difficile. Sei sicuro che non vorresti piuttosto studiare il cervello?»

«E se ti dicessi che preferirei studiare il cervello», disse Draco, con la voce ora dura, «andresti in giro a dire alla gente che ho avuto paura di quello che avrei scoperto.»

«No», disse la figura ombrosa. «Non farei nulla del genere.»

«Ma potresti fare lo stesso tipo di prove da solo, e se ottenessi la risposta sbagliata, non sarei lì a dirti qualcosa prima che la mostrassi a qualcun altro.» La voce di Draco era ancora dura.

«Lo chiederei comunque prima a te, Draco», disse con calma la figura ombrosa.

Draco fece una pausa. Non se l’era aspettato, aveva pensato di aver visto la trappola, ma… «Lo \textit{faresti?}»

«Certo. Come posso \textit{io} sapere chi ricattare o cosa potremmo chiedere loro? Draco, ti ripeto ancora una volta che questa non è una trappola che ho creato per te. Almeno non per te personalmente. Se la tua posizione politica fosse stata diversa, ti starei chiedendo che cosa succederebbe se la prova mostrasse che i purosangue sono più forti.»

«Seriamente.»

«\textit{Sì!} Questo è il prezzo che \textit{chiunque} deve pagare per diventare uno scienziato!»

Draco alzò una mano. Doveva pensare.

La figura, ombrosa e illuminata di verde, attese.

Non ci volle molto tempo per pensarci, però. Se scartavi tutte le parti disorientanti… allora Harry Potter stava progettando di trafficare con qualcosa che avrebbe potuto provocare un gigantesco sconvolgimento politico, e sarebbe stato folle andarsene via e basta, e lasciare che facesse tutto da solo. «Studieremo il sangue», disse Draco.

«\textit{Eccellente}», disse la figura, e sorrise. «Complimenti per essere disposto a fare la domanda.»

«Grazie», disse Draco, non riuscendo del tutto a escludere l’ironia dalla sua voce.

«Ehi, pensavi che andare sulla Luna fosse \textit{facile?} Accontentati del fatto che richieda solo di cambiare idea qualche volta, e non un sacrificio umano!»

«Il sacrificio umano sarebbe \textit{molto} più facile!»

Ci fu una breve pausa, e poi la figura annuì. «Giusta osservazione.»

«Ascolta, Harry», disse Draco senza molte speranze, «pensavo che l’idea fosse di prendere tutte le cose che i Babbani sanno, combinarle con le cose che i maghi sanno, e diventare padroni di entrambi i mondi. Non sarebbe molto più facile studiare tutte le cose che i Babbani \textit{hanno già} scoperto, come la roba della Luna, e usare \textit{quel} potere –»

«\textit{No}», disse la figura scuotendo decisamente la testa, ombre verdi che si mossero intorno al naso e agli occhi. La sua voce era diventata molto triste. «Se non puoi imparare l’arte dello scienziato di accettare la realtà, allora \textit{non devo} dirti cosa quell’accettazione ha portato a scoprire. Sarebbe come se un potente mago ti svelasse le porte che non devono essere aperte e i sigilli che non devono essere spezzati, prima che tu abbia dimostrato di possedere l’intelligenza e la disciplina necessarie per sopravvivere ai pericoli minori.»

Un brivido scese lungo la schiena di Draco, che tremò involontariamente. Sapeva che era stato visibile anche nella penombra. «Va bene», disse Draco. «Comprendo.» Suo Padre gliel’aveva detto molte volte. Quando un mago più potente ti diceva che non eri pronto a sapere, non dovevi curiosare oltre, se volevi vivere.

La figura chinò il capo. «Già. Ma c’è un’altra cosa che dovresti comprendere. I primi scienziati, essendo Babbani, non avevano le vostre tradizioni. In principio, semplicemente non compresero il concetto di conoscenza pericolosa, e pensarono che di tutte le cose conosciute si dovesse parlare liberamente. Quando le loro ricerche divennero pericolose, riferirono ai loro politici cose che sarebbero dovute restare segrete — non guardarmi così, Draco, non fu semplicemente stupidità. Dovevano essere abbastanza intelligenti da scoprire quei segreti, del resto. Ma erano Babbani, era la prima volta che avevano trovato qualcosa di \textit{veramente} pericoloso, e non erano stati \textit{iniziati} a una tradizione di segretezza. C’era una guerra in corso, e gli scienziati di una parte erano preoccupati che se loro \textit{non avessero} parlato, gli scienziati del paese \textit{nemico} l’avrebbero detto per primi ai \textit{loro} politici…» La frase rimase sospesa in modo significativo. «Non distrussero il mondo. Ma ci andarono vicino. E \textit{noi} non abbiamo intenzione di ripetere lo stesso errore.»

«Giusto», disse Draco, la voce ora molto ferma. «\textit{Noi} non lo faremo. Noi siamo maghi, e studiare la scienza non ci rende Babbani.»

«È come dici», riprese la sagoma illuminata di verde. «Creeremo la \textit{nostra} Scienza, una Scienza magica, e questa Scienza possiederà tradizioni più intelligenti fin dal principio.» La voce si fece dura. «La conoscenza che condivido con te sarà insegnata insieme alle discipline per accettare la verità, il livello di questa conoscenza sarà vincolato ai tuoi progressi in quelle discipline, e non condividerai tale conoscenza con nessun altro che non abbia imparato quelle discipline. Accetti queste condizioni?»

«Sì», disse Draco. Cosa ci si aspettava da lui, che dicesse no?

«Bene. E quello che scoprirai tu stesso, lo terrai per te a meno che non riterrai che altri scienziati siano pronti a conoscerlo. Ciò che condivideremo tra di noi, non lo riveleremo al mondo se non saremo d’accordo che sia sicuro per il mondo saperlo. E qualunque siano le nostre relazioni politiche e chiunque sia a godere della nostra lealtà, \textit{tutti} noi puniremo \textit{chiunque} nel nostro gruppo riveli magie o armi pericolose, non importa che tipo di guerra sia in corso. Da questo giorno in poi, questa sarà la tradizione e la legge della scienza tra i maghi. Siamo d’accordo su questo?»

«Sì», disse Draco. In verità tutto questo cominciava a sembrare piuttosto interessante. I Mangiamorte avevano tentato di prendere il potere facendo in modo di essere più terrificanti di chiunque altro, e in realtà non avevano ancora vinto. Forse era il momento di provare a governare usando i segreti, piuttosto. «E il nostro gruppo rimane nascosto il più a lungo possibile, e tutti al suo interno devono accettare le nostre regole.»

«Naturalmente. Assolutamente.»

Ci fu una breve pausa.

«Avremo bisogno di vesti migliori», disse la figura ombrosa, «con cappucci e cose simili –»

«Ci stavo \textit{proprio pensando}», disse Draco. «Non abbiamo bisogno di vesti completamente nuove, però, solo mantelli a cappuccio da indossarci sopra. Ho un’amica Serpeverde, prenderà le tue misure –»

«Non dirle \textit{a cosa} serve, però –»

«Non sono \textit{stupido!}»

«E niente maschere per ora, non quando siamo solo tu e io –» disse la figura ombrosa.

«Giusto! Ma in seguito dovremmo trovare una sorta di marchio speciale che tutti i nostri servitori avranno, il Marchio della Scienza, come un serpente che mangia la Luna sul braccio destro –»

«Si chiama dottorato di ricerca e non renderebbe troppo facile identificare i nostri?»

«Eh?»

«Voglio dire, che succederebbe se qualcuno dicesse ‘ok, ora sollevate tutti le vostre vesti sul braccio destro’ e il nostro uomo rispondesse ‘oops, scusate, sembra che io sia una spia’ –»

«\textit{Dimentica quello che ho detto}», disse Draco, il sudore diffusosi improvvisamente per tutto il corpo. Aveva bisogno di un diversivo, \textit{velocemente} — «E come ci chiameremo? I Mangiascienza?»

«No», disse lentamente la figura ombrosa. «Non suona bene…»

Draco si asciugò la fronte col braccio, rimuovendo le gocce di sudore. Ma che era andato a \textit{pensare} il Signore Oscuro? Suo Padre aveva detto che il Signore Oscuro era \textit{intelligente!}

«Ce l’ho!» disse improvvisamente la figura ombrosa. «Non lo capirai ora, ma fidati, è adatto.»

In quel momento Draco avrebbe accettato anche ‘Masticatori di Malfoy’, purché si cambiasse argomento. «Che cos’è?»

E in piedi in mezzo ai banchi polverosi in un’aula inutilizzata nei sotterranei di Hogwarts, la sagoma illuminata di verde di Harry Potter allargò platealmente le braccia e disse: «Questo giorno segnerà l’alba della… \textit{Cospirazione bayesiana}.»

\begin{figure}[h!]
        \includegraphics[scale=0.4]{boccino.png}
        \centering
\end{figure}

Una figura silenziosa si trascinò stancamente per i corridoi di Hogwarts in direzione di Corvonero.

Harry era andato direttamente dall’incontro con Draco a cena, e ci era rimasto per un tempo a malapena sufficiente a trangugiare un paio di bocconi veloci prima di andare a letto.

Non erano neppure le 19, ancora, ma era ben oltre l’ora di coricarsi per Harry. Si era reso conto \textit{ieri} sera che sabato non sarebbe stato in grado di utilizzare il Giratempo fino a dopo la fine della gara di lettura. Ma poteva ancora usare il Giratempo \textit{venerdì} sera, e guadagnare tempo in questo modo. Così Harry si era sforzato di restare sveglio venerdì fino alle 21, quando il guscio protettivo si era aperto, e poi aveva utilizzato le quattro ore restanti sul Giratempo per tornare alle 17 e collassare dal sonno. Si era svegliato intorno alle 2 di sabato mattina, come previsto, e aveva letto per le successive dodici ore di fila… e ancora non era stato sufficiente. E ora Harry sarebbe andato a dormire piuttosto presto per i prossimi giorni, fino a quando il suo ciclo di sonno non si fosse nuovamente riallineato.

Il ritratto sulla porta chiese a Harry qualche sciocco enigma pensato per bambini di undici anni, che risolse senza che le parole passassero attraverso la sua mente cosciente, e poi Harry barcollò su per le scale fino alla sua stanza del dormitorio, si mise il pigiama e crollò sul letto.

E trovò che il suo cuscino sembrava piuttosto bitorzoluto.

Harry gemette. Si sedette con riluttanza, si contorse nel letto, e sollevò il cuscino.

Questo rivelò una nota, due galeoni d’oro, e un libro intitolato \textit{Occlumanzia: l’Arte Nascosta.}

Harry prese la nota e lesse:

\vspace{1em}
\begin{addmargin}[3em]{3em}% 1em left, 2em right
\begin{itpars}		
Accidenti, ti metti davvero nei guai, e in fretta. Tuo padre non era al tuo livello.

Ti sei fatto un nemico potente. Snape gode della lealtà, dell’ammirazione e della paura di tutti a Casa Serpeverde. Non puoi fidarti di chiunque appartenga a quella Casa ora, che venga a te in veste amichevole o tremenda.

D’ora in poi non devi fissare gli occhi di Snape. È un Legilimens e sarebbe in grado di leggere la tua mente se lo facessi. Ho incluso un libro che potrebbe aiutarti a imparare a proteggerti, anche se senza un precettore c’è un limite a quanto puoi ottenere. Ad ogni modo puoi sperare quanto meno di accorgerti delle intrusioni.

Affinché tu possa trovare un po’ di tempo in più in cui studiare Occlumanzia, ho allegato due galeoni, che sono il prezzo delle soluzioni degli esami e dei compiti a casa per il primo anno del corso di Storia della Magia (da quando è morto il professor Binns ha dato gli stessi esami e gli stessi compiti ogni anno). I tuoi nuovi amici, i gemelli Weasley, dovrebbero essere in grado di venderti una copia. Va da sé che è necessario non farti prendere con essi in tuo possesso.

Del professor Quirrell so poco. È un Serpeverde e un professore di Difesa, e questi sono due indizi contro di lui. Considera attentamente ogni consiglio che ti dà, e non dirgli nulla che non desideri sia reso noto.

Silente fa solo finta di essere pazzo. È molto intelligente, e se continui a entrare nei ripostigli e a svanire, certamente dedurrà il tuo possesso di un mantello dell’invisibilità, se non l’ha già fatto. Evitalo quando possibile, nascondi il Mantello dell’Invisibilità in qualche posto sicuro (\textsl{\textsc{non}} la tua borsa) ogni volta che non puoi evitarlo, e agisci con gran circospezione in sua presenza.

Per favore, stai più attento in futuro, Harry Potter.

–- Babbo Natale
\end{itpars}
\end{addmargin}
\vspace{1em}

Harry fissò la nota.

E \textit{sembrava} essere un ottimo consiglio. Certo Harry non aveva intenzione di imbrogliare nel corso di Storia, anche se gli avessero dato una scimmia morta come professore. Ma la Legilimanzia di Severus… chiunque avesse inviato questa nota sapeva parecchi segreti importanti ed era disposto a raccontarli a Harry. La nota lo stava ancora mettendo in guardia contro la possibilità che Silente rubasse il mantello, ma a questo punto onestamente Harry non aveva idea se fosse un brutto segno, sarebbe potuto essere solo un errore comprensibile.

Pareva che ci fosse una sorta di intrigo in atto all’interno di Hogwarts. Forse se Harry avesse \textit{confrontato le storie} tra Silente e il mittente della nota, avrebbe potuto tirare fuori un quadro \textit{combinato} che sarebbe stato accurato? Del tipo se \textit{entrambi} fossero stati d’accordo su qualcosa, allora…

… evabbè…

Harry ficcò tutto nella borsa, alzò il Quietus, tirò la coperta sopra la testa e si spense.

\begin{figure}[h!]
        \includegraphics[scale=0.4]{boccino.png}
        \centering
\end{figure}

Era domenica mattina e Harry stava mangiando frittelle nella Sala Grande, con piccoli morsi frequenti, gettando occhiate nervose al suo orologio ogni pochi secondi.

Erano le 8:02, e tra due ore e un minuto precisi sarebbe stata \textit{esattamente una settimana} da quando aveva visto i Weasley e attraversato il varco verso il Binario Nove e Tre Quarti.

E gli venne un pensiero… Harry non sapeva se questo fosse un modo valido di pensare riguardo all’universo, non sapeva più nulla, ma \textit{sembrava possibile…}

Che…

\textit{Non gli fossero successe abbastanza cose interessanti durante la settimana passata.}

Per quando ebbe finito di fare colazione, Harry aveva deciso di andare dritto in camera sua e nascondersi nel livello inferiore del suo baule e non parlare con nessuno fino alle 10:03.

E fu allora che Harry vide i gemelli Weasley camminare verso di lui. Uno di loro portava qualcosa nascosto dietro la schiena.

Avrebbe dovuto scappare urlando.

Avrebbe dovuto scappare urlando.

Di qualunque cosa si trattasse… poteva benissimo essere…

\textit{… il gran finale…}

Avrebbe davvero dovuto scappare urlando.

Con la rassegnazione che l’universo l'avrebbe scovato \textit{comunque}, Harry continuò a tagliare la frittella con la forchetta e il coltello. Non riusciva a chiamare a raccolta le proprie forze. Quella era la triste verità. Harry sapeva ormai come le persone si sentivano quando erano stanche di correre, stanche di cercare di sfuggire al destino, e cadevano semplicemente a terra e lasciavano che i demoni orribilmente artigliati e tentacolati del più oscuro abisso le trascinassero via verso il loro indicibile destino.

I gemelli Weasley si avvicinarono.

E furono ancora più vicini.

Harry mangiò un altro boccone di frittella.

I gemelli Weasley arrivarono, sorridendo allegramente.

«Ciao, Fred», disse Harry debolmente. Uno dei gemelli annuì. «Ciao, George.» L’altro gemello annuì.

«Sembri stanco», disse George.

«Dovresti rallegrarti», disse Fred.

«Guarda che cosa ti abbiamo portato \textit{noi!}»

E George prese, da dietro le spalle di Fred –

Una torta con dodici candeline fiammeggianti.

Ci fu una pausa, mentre il tavolo di Corvonero li fissò.

«Vi sbagliate», disse qualcuno. «Harry Potter è nato il trentuno lugl-»

«\textsc{Egli sta arrivando}», disse un’alta voce cavernosa che passò attraverso tutte le conversazioni come una spada di ghiaccio. «\textsc{Colui che distruggerà persino} –»

Silente era saltato fuori dal suo trono ed era corso dritto oltre il Tavolo d’onore e aveva afferrato la donna che pronunciava quelle parole terribili, Fawkes era apparso in un lampo e tutti e tre erano scomparsi in uno schiocco di fuoco.

Ci fu una pausa traumatizzata…

… seguita dal movimento delle teste che si voltarono in direzione di Harry Potter.

«Non sono stato io», disse Harry con voce stanca.

«Quella era una \textit{profezia!}» sibilò qualcuno al tavolo. «E scommetto che riguarda \textit{te!}»

Harry sospirò.

Si alzò dal suo posto, schiarì la voce, e disse a voce alta al di sopra delle conversazioni che stavano nascendo, «\textit{Non si tratta di me! È ovvio! Io non sto venendo qui, io sono già qui!}»

Harry tornò nuovamente a sedersi.

Le persone che erano state a guardarlo si girarono di nuovo.

Qualcun altro al tavolo chiese «Allora \textit{di chi} si tratta?»

E con una sorda e plumbea sensazione, Harry si rese conto di chi \textit{non era} già a Hogwarts.

Chiamatelo tirare a indovinare, ma Harry aveva la sensazione che il non-morto Signore Oscuro sarebbe comparso uno di quei giorni.

La conversazione proseguì intorno a lui.

«E poi, distruggere \textit{cosa?}»

«Mi pare che Trelawney stesse per dire qualcosa che iniziava per ‘s’, poco prima che il Preside l’afferrasse».

«Come… spirito? Sole?»

«Se qualcuno sta per distruggere il Sole siamo \textit{davvero} nei guai!»

Sembrava piuttosto improbabile a Harry, a meno che il mondo non contenesse cose spaventose che avevano sentito parlare delle idee di David Criswell circa l’aspirazione delle stelle.

«E così», disse Harry in tono stanco, «cose come queste succedono ogni domenica a colazione, vero?»

«No», disse uno studente che avrebbe potuto essere al settimo anno, accigliato e torvo. «Non succedono.»

Harry scrollò le spalle. «Pazienza. Qualcuno vuole un po’ di torta di compleanno?»

«Ma \textit{non} è il tuo compleanno!» disse lo stesso studente che aveva obiettato in precedenza.

Quello fu lo spunto per Fred e George per iniziare a ridere, naturalmente.

Anche Harry riuscì a fare un sorriso stanco.

Mentre gli veniva servita la prima fetta, Harry disse: «Ho avuto una \textit{settimana molto lunga.}»

\begin{figure}[h!]
        \includegraphics[scale=0.4]{boccino.png}
        \centering
\end{figure}

E Harry era seduto nel livello sotterraneo del suo baule, chiuso a chiave in modo che nessuno potesse entrare, una coperta tirata sulla testa, in attesa che la settimana fosse finita.

10:01.

10:02.

10:03, ma giusto per essere sicuri…

10:04 e la prima settimana era andata.

Harry tirò un sospiro di sollievo, e cautamente tolse la coperta dalla testa.

Pochi istanti dopo, era emerso nell’aria del suo dormitorio illuminato vividamente dal sole.

Ancora poco, e fu nella sala comune di Corvonero. Alcune persone lo guardarono, ma nessuno disse niente o cercò di parlargli.

Harry trovò una bella e ampia scrivania, avvicinò una comoda sedia, e si sedette. Dalla sua borsa estrasse un foglio di carta e una matita.

Mamma e papà avevano detto a Harry senza mezzi termini che, sebbene avrebbero capito il suo entusiasmo di allontanarsi da casa e dai suoi genitori, avrebbe dovuto scrivere loro \textit{ogni settimana senza fallo}, giusto in modo che sapessero che era vivo, illeso, e ancora a piede libero.

Harry fissò il foglio di carta bianco. \textit{Vediamo…}

Dopo aver lasciato i suoi genitori alla stazione ferroviaria, aveva…

… fatto conoscenza con un ragazzo cresciuto da Darth Vader, fatto amicizia con i tre burloni più famigerati di Hogwarts, incontrato Hermione, e poi c’era stato l’incidente con il Cappello Smistatore… lunedì aveva ricevuto una macchina del tempo per curare il suo disturbo del sonno, ricevuto un leggendario mantello dell’invisibilità da un benefattore sconosciuto, salvato sette Tassofrasso fissando cinque spaventosi ragazzi più grandi uno dei quali aveva minacciato di rompergli il dito, compreso che possedeva un misterioso lato oscuro, imparato a lanciare \textit{Frigideiro} nella lezione di Incantesimi, e iniziato la sua rivalità con Hermione… martedì aveva fatto conoscenza con Astronomia insegnata dalla professoressa Aurora Sinistra, che era interessante, e con Storia della Magia insegnata da un fantasma che avrebbe dovuto essere esorcizzato e sostituito con un registratore… mercoledì, era stato dichiarato lo Studente Più Pericoloso della Classe… giovedì, meglio non pensarci nemmeno a giovedì… venerdì, l’incidente della lezione di Pozioni, seguito dal suo ricatto al Preside, seguito dal Professore di Difesa che l’aveva fatto picchiare a lezione, seguito dal Professore di Difesa che si era rivelato l’essere umano più fantastico che camminasse sulla faccia della Terra… sabato aveva perso una scommessa ed era andato al suo primo appuntamento e aveva iniziato la redenzione di Draco… e poi quella mattina la profezia incompleta della professoressa Trelawney che poteva o non poteva indicare che un Signore Oscuro immortale stava per attaccare Hogwarts.

Harry organizzò mentalmente il materiale, e iniziò a scrivere.

\vspace{1em}
\begin{addmargin}[3em]{3em}% 1em left, 2em right
\begin{itpars}

Cari Mamma e Papà,

Hogwarts è molto divertente. Ho imparato a violare la Seconda Legge della Termodinamica alla lezione di Incantesimi, e ho incontrato una ragazza di nome Hermione Granger che legge più velocemente di me.

Sarà meglio che mi fermi qui.

Vostro figlio che vi ama tanto,

Harry James Potter-Evans-Verres.

\end{itpars}
\end{addmargin}
\vspace{1em}

