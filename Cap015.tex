% !TeX root = Harry.tex

\chapter{Coscienziosità}
\label{capitolo:15}

\emph{«Sono certo che troverò il tempo da qualche parte.»}

~\\
~\\

«Frigideiro!»

Harry mise un dito nel bicchiere d’acqua appoggiato sul banco. Avrebbe dovuto essere fredda. Ma tiepida era inizialmente, e tiepida era rimasta. Ancora una volta.

Harry si sentiva veramente, veramente preso in giro.

C’erano centinaia di romanzi fantasy sparsi per casa Verres. Harry ne aveva letti un bel po’. E iniziava a essere evidente che avesse un misterioso lato oscuro. Così, dopo che il bicchiere d’acqua si era rifiutato di cooperare le prime volte, Harry aveva gettato uno sguardo in giro per l’aula di Incantesimi per assicurarsi che nessuno lo stesse osservando, e poi aveva fatto un respiro profondo, si era concentrato, e aveva fatto in modo di infuriarsi. Aveva pensato ai Serpeverde che perseguitavano Neville, e al gioco in cui qualcuno gettava a terra i tuoi libri ogni volta che tentavi di prenderli di nuovo. Aveva pensato a quello che Draco Malfoy aveva detto circa la decenne ragazza Lovegood e di come funzionava davvero il Wizengamot…

E la furia era entrata nel suo sangue, aveva tenuto la bacchetta in una mano che tremava per l’odio e aveva detto in un tono gelido «Frigideiro!» e non era successo assolutamente niente.

Harry era stato truffato. Voleva scrivere a qualcuno e chiedere un rimborso per il suo lato oscuro, che chiaramente doveva avere un irresistibile potere magico, ma che si era rivelato essere difettoso.

«Frigideiro!» disse di nuovo Hermione dal banco accanto a lui. La sua acqua era ghiaccio solido e c’erano cristalli bianchi che si stavano formando sul bordo del bicchiere. Sembrava essere totalmente concentrata sul proprio lavoro e niente affatto consapevole che gli altri studenti la fissavano con occhi carichi d’odio, cosa che era (a) una dimostrazione di pericolosa disattenzione da parte sua o (b) un’interpretazione perfetta elevata al livello di arte sopraffina.

«Oh, molto bene, signorina Granger!» squittì Filius Flitwick, il loro Professore di Incantesimi e il Preside di Corvonero, piccolo ometto senza segni visibili di essere stato in passato un campione di duelli. «Eccellente! Stupendo!»

Harry si era aspettato di essere, nel peggiore dei casi, secondo alle spalle di Hermione. Harry avrebbe preferito che lei fosse la sua rivale, naturalmente, ma avrebbe potuto accettare anche il contrario.

Giunti a lunedì, Harry si stava indirizzando verso il fondo della classe, una posizione per cui stava socievolmente rivaleggiando con tutti gli altri studenti educati da Babbani tranne Hermione. Che era tutta sola e senza rivali in testa, poverina.

Il professor Flitwick stava in piedi sopra il banco di una degli altri Nati babbani e stava sommessamente correggendo il modo in cui impugnava la bacchetta.

Harry guardò Hermione. Deglutì. Nello schema delle cose, era il ruolo scontato per lei… «Hermione?» Harry disse timidamente. «Hai una vaga idea di quello che potrei fare di sbagliato?»

Gli occhi di Hermione si illuminarono di una terribile luce di disponibilità e qualcosa nei recessi del cervello di Harry urlò per la disperata umiliazione.

Cinque minuti più tardi, l’acqua di Harry sembrava notevolmente più fredda rispetto alla temperatura ambiente ed Hermione gli aveva dato un paio di pacche sulle spalle verbali e detto di pronunciare con più attenzione la volta successiva, e se n’era andata per aiutare qualcun altro.

Il professor Flitwick le aveva concesso un punto per averlo aiutato.

Harry stava stringendo i denti così forte che la mascella gli faceva male e questo non stava aiutando la sua pronuncia.

Non mi importa se si tratta di concorrenza sleale. So esattamente quello che farò con due ore in più ogni giorno. Andrò a sedermi nel mio baule e studierò fino a quando sarò alla pari con Hermione Granger.

\begin{figure}[h!]
        \includegraphics[scale=0.4]{boccino.png}
        \centering
\end{figure}

«La Trasfigurazione è una delle magie più complesse e pericolose che imparerete a Hogwarts», disse la professoressa McGonagall. Non c’era alcuna traccia di levità sul volto della vecchia strega austera. «Chiunque combini pasticci nel mio corso se ne andrà per non tornare più. Siete stati avvertiti.»

La sua bacchetta scese a battere sulla cattedra, che dolcemente si rimodellò in un maiale. Un paio di studenti Nati babbani emisero piccoli guaiti. Il maiale si guardò intorno e sbuffò, sembrava confuso, e poi divenne nuovamente una cattedra.

La Professoressa di Trasfigurazione guardò la classe intorno a sé, e poi i suoi occhi si posarono su di uno studente.

«Signor Potter», disse la professoressa McGonagall. «Ha ricevuto i libri di scuola soltanto un paio di giorni fa. Ha già iniziato a leggere il suo libro di testo di Trasfigurazione?»

«No, mi dispiace professoressa», disse Harry.

«Non c’è bisogno di chiedere scusa, signor Potter, se foste stati tenuti a leggerlo in anticipo vi sarebbe stato detto di farlo.» McGonagall batté le dita sulla cattedra di fronte a lei. «Signor Potter, le andrebbe di indovinare se questa è una cattedra che ho trasfigurato in un maiale, o se inizialmente era un maiale e ho brevemente rimosso la Trasfigurazione? Se avesse letto il primo capitolo del libro di testo, lo saprebbe.»

Harry aggrottò leggermente la fronte. «Direi che sarebbe più semplice iniziare con un maiale, in quanto se si partisse da una cattedra, potrebbe non sapere come stare in piedi.»

La professoressa McGonagall scosse la testa. «Non glie ne faccio una colpa, signor Potter, ma la risposta corretta è che a Trasfigurazione non ci interessa tirare a indovinare. Le risposte sbagliate saranno valutate con estrema severità, le domande lasciate in bianco saranno valutate con grande indulgenza. Dovete imparare a riconoscere ciò che non conoscete. Se vi pongo una qualunque domanda, non importa quanto ovvia o elementare, e rispondete ‘Non ne sono sicuro’, non la valuterò a vostro sfavore e chiunque rida perderà punti per la sua Casa. Sa dirmi perché esiste questa regola, signor Potter?»

Poiché un singolo errore a Trasfigurazione può essere incredibilmente pericoloso. «No.»

«Corretto. Trasfigurazione è più pericolosa di Apparizione, che non viene insegnata prima del sesto anno. Purtroppo, Trasfigurazione deve essere appresa e praticata in giovane età per massimizzare la vostra capacità da adulti. Quindi è una materia pericolosa, e dovreste essere alquanto spaventati dall’idea di commettere degli errori, perché nessuno dei miei studenti si è mai fatto male seriamente e io sarò estremamente offesa se foste la prima classe a rovinare questo mio primato.»

Diversi studenti deglutirono.

La professoressa McGonagall si alzò e si avvicinò alla parete dietro la cattedra, che reggeva una tavola di legno levigata. «Ci sono molte ragioni per cui Trasfigurazione è pericolosa, ma una ragione domina su tutte le altre.» Tirò fuori una penna d’oca con una spessa estremità, e la usò per disegnare delle lettere in rosso; poi le sottolineò, usando la stessa penna, in blu:

trasfigurazione non è permanente!

«Trasfigurazione non è permanente!» disse la professoressa McGonagall. «Trasfigurazione non è permanente! Trasfigurazione non è permanente! Signor Potter, supponga che uno studente Trasfigurasse un blocco di legno in una coppa d’acqua, e lei la bevesse. Cosa pensa le accadrebbe quando la Trasfigurazione svanisse?» Ci fu una pausa. «Mi scusi, non avrei dovuto chiederlo a lei, signor Potter, ho dimenticato che ha ricevuto il dono di una fantasia insolitamente pessimista –»

«Non è un problema», disse Harry, deglutendo con difficoltà. «Dunque, la prima risposta è che non lo so», la professoressa annuì in approvazione, «ma immagino che ci possa essere… del legno nel mio stomaco, e nel mio flusso sanguigno e in tutta quell’acqua che è stata assorbita dai tessuti del mio corpo — che fosse polpa di legno o legno pieno o…» La sua comprensione della magia lo tradì. Non era in grado di capire in che modo il legno si mappasse nell’acqua, tanto per cominciare, quindi non riusciva a comprendere che cosa sarebbe accaduto dopo che le molecole d’acqua fossero state rimescolate in seguito ai normali moti termici e la magia fosse svanita e la mappatura invertita.

Il volto di McGonagall era severo. «Come il signor Potter ha correttamente dedotto, si sarebbe ammalato gravemente e avrebbe avuto bisogno di un immediato trasferimento via Metropolvere al St. Mungo’s Hospital, per avere una minima speranza di sopravvivenza. Per favore, aprite i vostri libri di testo a pagina 5.»

Anche se nessun suono proveniva dalla foto in movimento, si poteva capire che la donna con la pelle orribilmente scolorita stava urlando.

«Il criminale che inizialmente trasfigurò dell’oro in vino e lo diede da bere a questa donna, ‘a saldo del debito’ come disse lui, ricevette una condanna a dieci anni ad Azkaban. Per favore, andate a pagina 6. Questo è un Dissennatore. Sono i guardiani di Azkaban. Vi succhiano via la magia, la vita e qualunque pensiero felice abbiate. La figura a pagina 7 ritrae il criminale dieci anni dopo, al suo rilascio. Noterete che è morto — sì, signor Potter?»

«Professoressa», chiese Harry, «se dovesse accadere il peggio in un caso simile, c’è qualche modo di sostenere la Trasfigurazione?»

«No», disse la professoressa McGonagall con voce piatta. «Sostenere una Trasfigurazione è un drenaggio costante della vostra magia che scala con le dimensioni della forma dell’obbiettivo. E avreste bisogno di rimettervi in contatto col bersaglio ogni poche ore, cosa che è, in un caso come questo, impossibile. Disastri come questo sono irrecuperabili!»

La professoressa McGonagall si inclinò in avanti, l’espressione sul viso molto severa. «Assolutamente mai, in nessuna circostanza, Trasfigurerete qualunque cosa in un liquido o in un gas. Niente acqua, niente aria. Nulla di simile all’acqua, nulla di simile all’aria. Anche se non va bevuto. I liquidi evaporano, piccole parti di essi finiscono nell’aria. Non Trasfigurerete nulla che vada bruciato. Produrrebbe del fumo e qualcuno potrebbe respirare quel fumo! Non Trasfigurerete mai qualunque cosa che in linea di principio possa finire nel corpo di qualcuno in qualunque modo. Niente cibo. Nulla che sembri del cibo. Neppure per un divertente scherzetto in cui volete dire alle vittime della vostra torta di fango prima che la mangino realmente. Non lo farete mai. Punto. Dentro questa classe o fuori di essa o ovunque. Questo è ben chiaro a ogni singolo studente?»

«Sì» dissero Harry, Hermione, e pochi altri. Il resto sembrava essere senza parole.

«Questo è ben chiaro a ogni singolo studente?»

«Sì», dissero o mormorarono o sussurrarono.

«Se infrangerete una qualunque di queste regole non studierete più Trasfigurazione durante tutta la vostra permanenza a Hogwarts. Ripetete con me. Non Trasfigurerò mai nulla in un liquido o in un gas.»

«Non Trasfigurerò mai nulla in un liquido o in un gas», dissero gli studenti in un coro irregolare.

«Ancora! Più forte! Non Trasfigurerò mai nulla in un liquido o in un gas.»

«Non Trasfigurerò mai nulla in un liquido o in un gas.»

«Non Trasfigurerò mai nulla che sembri del cibo o qualunque altra cosa che finisce all’interno di un corpo umano.»

«Non Trasfigurerò mai nulla che debba essere bruciato perché potrebbe fare del fumo.»

«Non Trasfigurerete mai nulla che sembri del denaro, incluso il denaro babbano», disse la professoressa McGonagall. «I goblin hanno dei metodi per scoprire chi l’ha fatto. E per quanto riguarda la legge, la nazione goblin è in stato di guerra permanente con tutti i falsari magici. Non manderanno degli Auror. Manderanno un esercito.»

«Non Trasfigurerò mai nulla che sembri del denaro», ripeterono gli studenti.

«E soprattutto», disse la professoressa McGonagall, «non Trasfigurerete nessun essere vivente, specialmente voi stessi. Vi farebbe ammalare gravemente e persino morire, a seconda di come vi Trasfigurereste e per quanto tempo manterreste il cambiamento.» La professoressa McGonagall fece una pausa. «Il signor Potter sta alzando la mano perché ha visto una trasformazione Animagus — precisamente un umano trasformarsi in un gatto e poi riassumere la propria forma. Ma una trasformazione Animagus non è una Trasfigurazione libera.»

La professoressa McGonagall prese un piccolo blocchetto di legno dalla tasca. Con un tocco della sua bacchetta divenne una palla di vetro. Poi disse «Crystferrium!» e la palla di vetro divenne una sfera d’acciaio. La toccò con la bacchetta per l’ultima volta e la sfera d’acciaio divenne un pezzo di legno ancora una volta. «Crystferrium trasforma un soggetto di vetro solido in un bersaglio di acciaio solido di forma simile. Non può fare il contrario, né può trasformare una cattedra in un maiale. La forma più generale di Trasfigurazione — Trasfigurazione libera, che imparerete qui — è in grado di trasformare qualunque soggetto in qualunque bersaglio, per lo meno per quanto riguarda la forma fisica. Per questo motivo, Trasfigurazione libera deve essere eseguita senza parole. Usare degli Incantesimi richiederebbe parole differenti per ciascuna differente trasformazione tra soggetto e bersaglio.»

La professoressa McGonagall indirizzò ai suoi studenti uno sguardo penetrante. «Alcuni insegnanti iniziano con gli Incantesimi di Trasfigurazione per poi passare successivamente alla Trasfigurazione libera. Sì, all’inizio questo approccio sarebbe più facile. Ma potrebbe creare una forma mentis che più tardi comprometterebbe le vostre capacità. Qui imparerete la Trasfigurazione libera sin dal principio, cosa che richiede che lanciate l’incantesimo senza parole, tenendo la forma del soggetto, la forma del bersaglio, e la trasformazione all’interno della vostra mente.»

«E per rispondere alla domanda del signor Potter», continuò la professoressa McGonagall, «è la Trasfigurazione libera che non dovete mai lanciare su nessun soggetto vivo. Ci sono Incantesimi e pozioni che possono trasformare con sicurezza e reversibilmente bersagli vivi in modo limitato. Un Animagus con un arto mancante continuerà a non avere quell’arto dopo la trasformazione, per esempio. La Trasfigurazione libera non è sicura. Il vostro corpo muterà mentre è Trasfigurato — respirare, per esempio, implica una perdita costante della materia del corpo in favore dell’aria circostante. Quando la Trasfigurazione si estingue e il vostro corpo cerca di ritornare alla sua forma originale, non sarà più in grado di farlo. Se appoggiate la vostra bacchetta al vostro corpo e immaginate di avere capelli dorati, in seguito i vostri capelli cadranno. Se vi visualizzate come qualcuno con la pelle più chiara, sarete costretti a una lunga permanenza al St. Mungo’s. E se vi Trasfigurerete in un corpo adulto, allora, quando la Trasfigurazione cesserà, voi morirete.»

Questo spiegava perché aveva visto cose come ragazzi grassi, o ragazze meno che perfettamente carine. O persone anziane, se è per questo. Tutto ciò non sarebbe accaduto se fosse stato possibile Trasfigurare sé stessi ogni mattina… Harry alzò la mano e cercò di attirare l’attenzione della professoressa McGonagall con gli occhi.

«Sì, signor Potter?»

«È possibile trasfigurare un soggetto vivente in un bersaglio che è statico, come una moneta — no, mi scusi, mi dispiace, diciamo in una semplice sfera di acciaio?»

La professoressa McGonagall scosse la testa. «Signor Potter, anche gli oggetti inanimati subiscono piccoli cambiamenti interni nel corso del tempo. Non ci sarebbero cambiamenti visibili nel vostro corpo, e per il primo minuto non vi accorgereste di nulla. Ma in un’ora vi ammalereste, e in un giorno sareste morti.»

«Ehm, mi scusi, quindi se avessi letto il primo capitolo avrei potuto immaginare che la cattedra era in origine una cattedra e non un maiale», disse Harry, «ma solo se avessi fatto l’ipotesi ulteriore che non volesse uccidere il maiale, cosa che potrebbe sembrare molto probabile, ma –»

«Prevedo che valutare i suoi compiti sarà una fonte inesauribile di gioia per me, signor Potter. Ma se ha altre domande, posso chiederle di attendere fino a dopo la lezione?»

«Nessun’altra domanda, professoressa.»

«Ora ripetete dopo di me», disse la professoressa McGonagall. «Non proverò mai a Trasfigurare nessun soggetto vivo, specialmente me stesso, se non specificatamente istruito a farlo usando un Incantesimo specializzato o una pozione.»

«Se non sono certo che una Trasfigurazione sia sicura, non la proverò finché non l’avrò chiesto alla professoressa McGonagall o al professor Flitwick o al professor Snape o al Preside, che sono le uniche autorità riconosciute per la Trasfigurazione a Hogwarts. Chiedere a un altro studente non è accettabile, anche se affermassero di ricordare di aver fatto la stessa domanda.»

«Anche se l’attuale Professore di Difesa di Hogwarts mi dicesse che una Trasfigurazione è sicura, e anche se vedessi il Professore di Difesa effettuarla senza che nulla di male accada, non la proverò io stesso.»

«Ho il diritto assoluto di rifiutarmi di effettuare ogni Trasfigurazione che mi renda minimamente nervoso. Poiché neppure il Preside di Hogwarts può ordinarmi di fare diversamente, non accetterò certamente un tale ordine dal Professore di Difesa, anche se il Professore di Difesa minacciasse di sottrarre cento punti alla mia Casa e di farmi espellere.»

«Se violerò una qualunque di queste regole non studierò più Trasfigurazione durante la mia permanenza a Hogwarts.»

«Ripeteremo queste regole all’inizio di ogni lezione per il primo mese», disse la professoressa McGonagall. «E ora, iniziamo con dei fiammiferi come soggetti e degli aghi come bersagli… mettete via le vostre bacchette, con ‘iniziamo’ intendevo dire inizierete a prendere appunti.»

Mezz’ora prima della fine della lezione, la professoressa McGonagall distribuì i fiammiferi.

Alla fine della lezione, Hermione aveva un fiammifero argenteo e il resto della classe, Nati babbani o meno, aveva esattamente ciò con cui aveva iniziato.

La professoressa McGonagall le assegnò un altro punto per Corvonero.

\begin{figure}[h!]
        \includegraphics[scale=0.4]{boccino.png}
        \centering
\end{figure}

Dopo che la lezione di Trasfigurazione fu terminata, Hermione andò al banco di Harry, proprio mentre Harry stava rimettendo i libri nella borsa.

«Sai», disse Hermione con un’espressione innocente sul viso, «ho guadagnato due punti per Corvonero oggi.»

«Già», disse laconicamente Harry.

«Ma non è all’altezza dei tuoi sette punti», continuò lei. «Credo semplicemente di non essere intelligente come te.»

Harry finì di inserire i propri compiti nella borsa e si girò verso Hermione con gli occhi socchiusi. Si era davvero dimenticato di quello.

Lei gli sbatté le ciglia. «Abbiamo lezioni tutti i giorni, però. Mi chiedo quanto tempo ti ci vorrà per trovare qualche altro Tassofrasso da salvare? Oggi è lunedì. Quindi hai fino a giovedì.»

I due si fissarono l’un l’altro negli occhi, senza battere ciglio.

Harry parlò per primo. «Naturalmente capisci che questo significa guerra.»

«Non sapevo che fossimo in pace.»

Tutti gli altri studenti stavano ora guardando con occhi affascinati. Tutti gli altri studenti, oltre, purtroppo, alla professoressa McGonagall.

«Oh, signor Potter», cinguettò la professoressa McGonagall dall’altra parte della stanza, «ho una buona notizia per lei. Madam Pomfrey ha approvato il suo suggerimento per prevenire le rotture nei suoi Cancelletti ruotanti, e l’idea è quella di finire il lavoro entro la fine della prossima settimana. Direi che questo meriti… diciamo dieci punti per Corvonero.»

Hermione rimase a bocca aperta per la sensazione di tradimento e di stupore. Harry immaginò che il proprio volto non apparisse molto diverso.

«Professoressa…» sibilò Harry.

«Quei dieci punti sono indiscutibilmente meritati, signor Potter. Non assegnerei mai dei punti per un capriccio. Per lei può essere stato semplicemente notare qualcosa di fragile e suggerire un modo per proteggerlo, ma i Cancelletti ruotanti sono costosi, e il Preside non è stato contento l’ultima volta che se n’è rotto uno.» La professoressa McGonagall sembrò pensierosa. «Accidenti, mi chiedo se qualche altro studente abbia mai guadagnato diciassette punti nel suo primo giorno di lezioni. Devo controllare, ma sospetto che questo sia un nuovo primato. Forse dovremmo fare un annuncio a cena?»

«professoressa!» strillò Harry. «Questa è la nostra guerra! La smetta di intromettersi!»

«Ora ha fino a giovedì della prossima settimana, signor Potter. A meno che, naturalmente, lei non sia coinvolto in qualche genere di guaio e perda dei punti prima di allora. Rivolgendosi a un’insegnante in maniera irriguardosa, per esempio.» La professoressa McGonagall appoggiò un dito sulla guancia e sembrò riflettere. «Mi aspetto che lei raggiunga i numeri negativi prima della fine di venerdì.»

La bocca di Harry si chiuse di scatto. Lanciò il suo miglior Sguardo Mortale a McGonagall, ma ella sembrò trovarlo solo divertente.

«Sì, senz’altro un annuncio a pranzo», rifletté la professoressa McGonagall. «Ma non sarebbe utile offendere i Serpeverde, quindi l’annuncio sarà breve. Solo il numero di punti e la faccenda del primato… e se qualcuno dovesse venire da lei per essere aiutato e fosse deluso che non ha neppure iniziato a leggere i suoi libri di testo, potrebbe sempre mandarlo dalla signorina Granger.»

«Professoressa!» disse Hermione in un tono acuto.

La professoressa McGonagall la ignorò. «Accipicchia, mi chiedo quanto ci vorrà alla signorina Granger per fare qualcosa che meriti un annuncio a cena? Non vedo l’ora di scoprirlo, qualunque cosa sia.»

Harry ed Hermione, di tacito e comune accordo, si voltarono e si precipitarono fuori dalla classe. Furono seguiti da una scia di Corvonero ipnotizzati.

«Uhm», disse Harry. «Siamo ancora d’accordo per il dopo cena?»

«Naturalmente», disse Hermione. «Non vorrei che rimanessi ancora più indietro nello studio.»

«Certo, grazie. E lasciami dire che per quanto tu sia già brillante, non posso fare a meno di chiedermi come sarai una volta che avrai ricevuto un po’ di formazione di base in razionalità.»

«È davvero così utile? Non sembra averti aiutato in Incantesimi o Trasfigurazione.»

Ci fu una breve pausa.

«Beh, ho ricevuto i mie libri di scuola appena quattro giorni fa. È per questo che ho dovuto guadagnare quei diciassette punti senza usare la mia bacchetta.»

«Quattro giorni fa? Forse non puoi leggere otto libri in quattro giorni, ma avresti potuto leggerne almeno uno. Quanti giorni ti ci vorranno per finire con questo ritmo? Sai tutta quella matematica, quindi puoi dirmi quanto fa otto, per quattro, diviso zero?»

«Ho le lezioni ora, che tu non hai avuto, ma i fine settimana sono liberi, quindi… limite di otto per quattro diviso epsilon per epsilon che tende a zero più… 10:47 di domenica.»

«Io l’ho fatto in tre giorni, veramente.»

«E 14:47 di sabato sia, allora. Sono certo che troverò il tempo da qualche parte.»

E fu sera e fu mattina, il primo giorno.



