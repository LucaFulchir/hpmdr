% !TeX root = Harry.tex

\chapter{Auto-consapevolezza, parte I}
\label{capitolo:9}

\emph{Non potevi mai sapere quale minuscolo evento potesse sconvolgere il corso del tuo piano magistrale.}

~\\
~\\


«Abbott, Hannah!»

Pausa.

«\textsc{Tassofrasso!}»

«Bones, Susan!»

Pausa.

«\textsc{Tassofrasso!}»

«Boot, Terry!»

Pausa.

«\textsc{Corvonero!}»

Harry guardò brevemente il suo nuovo compagno di Casa, più un rapido sguardo al viso che altro. Stava ancora cercando di riprendere il controllo di sé dopo l’incontro coi fantasmi. La cosa triste, la cosa veramente triste, la cosa veramente davvero triste era che \textit{sembrava} essere in grado di riprendere nuovamente il controllo. Gli sembrava fuori luogo. Come se avesse dovuto metterci almeno un giorno. Forse una vita intera. Forse, semplicemente, non riuscirci mai.

«Corner, Michael!»

Lunga pausa.

«\textsc{Corvonero!}»

Al leggio davanti all’enorme Tavolo d’onore stava la professoressa McGonagall, dall’aspetto intelligente e dallo sguardo tagliente, mentre declamava un nome dopo l’altro, anche se aveva sorriso solo per Hermione e pochi altri. Dietro di lei, sulla sedia più alta del tavolo — molto più simile a un trono d’oro — sedeva un vecchio avvizzito e occhialuto, con una barba bianco-argento che sembrava avrebbe raggiunto quasi il pavimento, se fosse stata visibile, che vigilava sullo Smistamento con un’espressione benevola; tanto stereotipato nell’aspetto quanto poteva esserlo un Vecchio Saggio senza essere effettivamente un orientale. (Anche se Harry aveva imparato a diffidare delle apparenze stereotipate dalla prima volta che aveva incontrato la professoressa McGonagall e aveva pensato che avrebbe dovuto starnazzare.) L’antico mago aveva applaudito ogni studente Smistato, con un sorriso incrollabile che in qualche modo sembrava nuovamente deliziato per ciascuno di loro.

Alla sinistra del trono d’oro c’era un uomo con lo sguardo acuto e un volto arcigno che non aveva applaudito nessuno, e che in qualche modo era riuscito a guardare dritto Harry ogni volta che Harry l’aveva guardato. Ancora più a sinistra, l’uomo dal viso pallido che Harry aveva visto al Paiolo Magico, i cui occhi dardeggiavano intorno come se fosse in preda al panico a causa della folla circostante, e che sembrava ogni tanto sobbalzare e contorcersi nella sua sedia; per qualche ragione, Harry continuava a scoprirsi mentre lo fissava. Alla sinistra di quell’uomo, una serie di tre streghe più anziane che non sembravano molto interessate agli studenti. Poi sul lato destro dell’alta sedia d’oro, una strega di mezza età dal viso tondo e con un cappello giallo, che aveva applaudito ogni studente tranne i Serpeverde. Un uomo piccolo in piedi sulla sua sedia, con una scombinata barba bianca, che aveva applaudito ogni studente, ma si era limitato a sorridere solo ai Corvonero. E all’estrema destra, occupando lo stesso spazio dei tre esseri più piccoli, l’entità montuosa che li aveva salutati tutti dopo che erano sbarcati dal treno, presentandosi come Hagrid, Custode delle Chiavi e dei Terreni.

«L’uomo in piedi sulla sedia è il Capo di Corvonero?» sussurrò Harry in direzione di Hermione.

Per una volta Hermione non rispose immediatamente; stava sbilanciando il peso da un piede all’altro, fissando il Cappello Smistatore, e agitandosi così energicamente che Harry pensò che i suoi piedi potessero lasciare il pavimento.

«Sì, è lui», disse uno dei prefetti che li avevano accompagnati, una giovane ragazza che indossava il blu dei Corvonero. La signorina Clearwater, se Harry ricordava correttamente. La sua voce era tranquilla, ma trasmise una punta di orgoglio. «Quello è il Professore di Incantesimi di Hogwarts, Filius Flitwick, il più esperto Maestro di Incantesimi vivente, già Campione di Duello –»

«Perché è così \textit{basso}?» sibilò uno studente il cui nome Harry non ricordava. «È un \textit{mezzosangue}?»

Uno sguardo gelido della giovane ragazza prefetto. «Il Professore ha in effetti antenati goblin –»

«Che cosa?» disse Harry involontariamente, spingendo Hermione e altri quattro studenti a zittirlo.

Ora Harry stava ricevendo uno sguardo sorprendentemente intimidatorio dal prefetto di Corvonero.

«Voglio dire –» sussurrò Harry. «Non è un \textit{problema} per me — è solo che — voglio dire — com’è \textit{possibile}? Non si possono semplicemente incrociare insieme due specie diverse e ottenere una prole in grado di sopravvivere! Si dovrebbero rimescolare le istruzioni genetiche per ciascun organo che è diverso tra le due specie — sarebbe come cercare di costruire», non avevano le auto quindi non poteva usare l’analogia del progetto di un motore messo in disordine, «un mezzo-carro mezza-barca o qualcosa del genere…»

Il prefetto Corvonero stava ancora guardando Harry severamente. «Perché non \textit{dovresti} poter avere un mezzo-carro mezza-barca?»

«Ssh!» zittì un altro prefetto, anche se la strega Corvonero aveva parlato ancora sommessamente.

«Voglio dire –» disse Harry ancora più sommessamente, cercando di capire come chiedere se i goblin si erano evoluti dagli esseri umani, o da un antenato comune con gli esseri umani come l’\textit{Homo erectus}, o se i goblin erano stati \textit{ottenuti} dagli esseri umani in qualche modo — se, per esempio, erano ancora geneticamente umani, sottoposti a un incantesimo ereditabile il cui effetto magico era diluito se solo uno dei genitori era un ‘goblin’, cosa che avrebbe spiegato come fosse possibile un incrocio, e nel qual caso i goblin \textit{non sarebbero} stati un secondo dato incredibilmente prezioso per scoprire come l’intelligenza si era evoluta in specie differenti dall’\textit{Homo sapiens} — ora che Harry ci pensava, i \textit{goblin} della Gringotts \textit{non erano} sembrati realmente intelligenze aliene, non-umane, per nulla simili ai Dirdir o ai Burattinai — «Voglio dire, da dove \textit{vengono} i goblin, quindi?»

«Lituania», sussurrò distrattamente Hermione, gli occhi ancora saldamente fissi sul Cappello Smistatore.

Ora Hermione stava ricevendo un sorriso dalla signorina prefetto.

«Non importa», sussurrò Harry.

Dal leggio, la professoressa McGonagall chiamò, «Goldstein, Anthony!»

«corvonero!»

Hermione, accanto a Harry, stava saltellando in punta di piedi con tanta forza che i suoi piedi stavano veramente lasciando il terreno a ogni rimbalzo.

«Goyle, Gregory!»

Ci fu un lungo momento di silenziosa tensione sotto il Cappello. Quasi un minuto.

«\textsc{Serpeverde!}»

«Granger, Hermione!»

Hermione si staccò e corse a tutta velocità verso il Cappello Smistatore, lo raccolse e infilò il rattoppato e vecchio pezzo di stoffa con forza sulla propria testa, facendo trasalire Harry. Era stata Hermione a raccontare a \textit{lui} del Cappello Smistatore, ma certamente \textit{lei} non lo stava trattando come un insostituibile oggetto di vitale importanza, un manufatto di magia perduta vecchio di 800 anni che stava per compiere un’intricata telepatia sulla sua mente e che non sembrava essere in ottime condizioni fisiche.

«\textsc{corvonero!}»

A proposito di decisioni scontate. Harry non capiva perché Hermione era stata così tesa a riguardo. In quale strano universo alternativo quella ragazza \textit{non sarebbe} stata Smistata in Corvonero? Se Hermione Granger non fosse andata a Corvonero allora non c’era alcuna buona ragione perché la Casa Corvonero esistesse.

Hermione giunse al tavolo Corvonero e ricevette la doverosa accoglienza; Harry si chiese se il benvenuto sarebbe stato più forte o più tiepido, se avessero avuto una qualche idea di che genere di concorrenza avevano accolto al loro tavolo. Harry conosceva le cifre di pi greco fino a $3,141592$, perché una precisione di una parte su un milione era sufficiente per la maggior parte degli scopi pratici. Hermione conosceva cento cifre di pi greco perché era il numero di cifre che era stato stampato sul retro del suo libro di testo di matematica.

Neville Longbottom andò a Tassofrasso, come Harry fu contento di sapere. Se quella Casa possedeva realmente la lealtà e il cameratismo che doveva incarnare, una Casa piena di amici fidati avrebbe fatto un mondo di bene a Neville. I bambini astuti in Corvonero, quelli malvagi in Serpeverde, gli aspiranti eroi in Grifondoro, e tutti quelli che fanno il vero lavoro in Tassofrasso.

(Anche se Harry \textit{aveva} visto giusto nel consultare per primo un prefetto di Corvonero. La giovane donna non aveva nemmeno alzato lo sguardo dalla sua lettura o identificato Harry, aveva solo puntato la bacchetta nella direzione di Neville e biascicato qualcosa. Dopodiché, Neville aveva assunto un’espressione frastornata e si era allontanato verso il quinto vagone dalla testa e il quarto scompartimento a sinistra, che effettivamente aveva contenuto il suo rospo.)

«Malfoy, Draco!» andò a Serpeverde, e Harry fece un piccolo sospiro di sollievo. Era \textit{sembrato} qualcosa di scontato, ma non potevi mai sapere quale minuscolo evento potesse sconvolgere il corso del tuo piano magistrale.

La professoressa McGonagall chiamò «Perks, Sally-Anne!», e dal gruppo di bambini si staccò una ragazzina pallida e smarrita che sembrava stranamente eterea — quasi come se potesse scomparire misteriosamente nel momento in cui smettessi di guardarla, per non essere mai più vista o persino ricordata.

E poi (con una nota di trepidazione tenuta così saldamente lontana da voce e viso che sarebbe stato necessario conoscerla davvero molto bene per notarla) Minerva McGonagall inspirò profondamente, e chiamò: «Potter, Harry!»

Ci fu un improvviso silenzio nella sala.

Tutte le conversazioni si fermarono.

Tutti gli occhi si girarono a fissarlo.

Per la prima volta in tutta la sua vita, Harry sentì di poter avere un’opportunità di provare il panico da palcoscenico.

Si sbarazzò immediatamente di quella sensazione. Una stanza piena di persone che lo fissavano era qualcosa a cui avrebbe dovuto abituarsi, se voleva vivere nella Gran Bretagna magica, o più in generale fare qualsiasi altra cosa interessante in vita sua. Atteggiando il viso a un sorriso sicuro di sé e falso, alzò un piede per fare un passo avanti –

«Harry Potter!» gridò la voce di Fred o George Weasley, e poi «Harry Potter!» gridò l’altro gemello Weasley, e un attimo dopo l’intero tavolo Grifondoro, e poco dopo una buona parte di Corvonero e Tassofrasso si unì al grido.

«\textit{Harry Potter! Harry Potter! Harry Potter!}»

E Harry Potter avanzò. Troppo lentamente, si rese conto una volta che aveva cominciato, ma ormai era troppo tardi per cambiare il proprio ritmo senza sembrare goffo.

\begin{figure}[h!]
        \includegraphics[scale=0.4]{boccino.png}
        \centering
\end{figure}

«\textit{Harry Potter! Harry Potter! \textsc{Harry Potter!}}»

Con un’idea anche troppo chiara di quello che avrebbe visto, Minerva McGonagall si voltò a guardare dietro di sé il resto del Tavolo d’onore.

Trelawney si sventagliava freneticamente, Filius osservava con curiosità, Hagrid batteva le mani a tempo, Sprout aveva un’aria severa, Vector e Sinistra erano sconcertate, e Quirrell fissava stolidamente nel vuoto. Albus sorrideva con benevolenza. E Severus Snape stringeva il proprio calice di vino vuoto, le nocche bianche, con tanta forza che lo spesso argento si stava lentamente deformando.

Con un largo sorriso, girando la testa per salutare da un lato e poi dall’altro mentre camminava tra le quattro tavolate delle Case, Harry Potter avanzava a un ritmo solenne e pacato, un principe che prendeva possesso del proprio castello.

«\textit{Salvaci da altri Signori Oscuri}!» esclamò uno dei gemelli Weasley, e poi l’altro gemello Weasley gridò, «\textit{Soprattutto se sono professori}!» per le risate generali di tutti i tavoli tranne quello Serpeverde.

Le labbra di Minerva si tesero in una linea bianca. Avrebbe fatto un discorso agli Orrori Weasley riguardo quell’ultima parte, se pensavano che fosse senza potere perché era il primo giorno di scuola e Grifondoro non aveva punti di cui essere privata. Se non gli importava delle detenzioni, allora avrebbe trovato qualcos’altro.

In quel momento, con un improvviso rantolo di orrore, guardò in direzione di Snape, \textit{certamente} avrebbe compreso che il giovane Potter non aveva alcuna idea di ciò a cui si riferiva –

Il volto di Snape era andato oltre la rabbia in una sorta di piacevole indifferenza. Un debole sorriso giocava sulle sue labbra. Stava guardando in direzione di Harry Potter, non della tavola di Grifondoro, e le sue mani reggevano i resti accartocciati di quello che era stato un calice da vino.

\begin{figure}[h!]
        \includegraphics[scale=0.4]{boccino.png}
        \centering
\end{figure}

Harry Potter avanzò con un sorriso stampato in viso, sentendo dentro di sé entusiasmo e allo stesso tempo, in qualche modo, pena.

Lo stavano acclamando per un’impresa che aveva compiuto quando aveva appena un anno. Un’impresa che non aveva compiuto realmente. In qualche luogo, in qualche modo, il Signore Oscuro era ancora vivo. Avrebbero acclamato così forte se l’avessero saputo?

Ma il potere del Signore Oscuro \textit{era già} stato distrutto una volta.

E Harry li avrebbe protetti ancora. Se c’era davvero una profezia e quello fosse ciò che diceva. Anzi, in realtà a prescindere da ciò che qualunque maledetta profezia dicesse.

Tutte quelle persone credevano in lui e lo sostenevano — Harry non poteva sopportare che questo fosse un inganno. Brillare intensamente e poi spegnersi come tanti altri bambini prodigio. Essere una delusione. Non riuscire a essere all’altezza della sua reputazione come simbolo della Luce, non importa \textit{come} l’avesse ottenuta. Egli sarebbe stato — assolutamente, sicuramente, non importa quanto tempo ci avrebbe messo e persino se questo l’avesse ucciso — all’altezza delle loro aspettative. E poi sarebbe andato oltre e avrebbe \textit{superato} quelle aspettative, cosicché la gente si sarebbe chiesta, guardandosi indietro, perché una volta si fossero aspettati così poco da lui.

«\textsc{Harry Potter! Harry Potter! Harry Potter!}»

Harry fece l’ultimo passo verso il Cappello Smistatore. Rivolse un profondo inchino all’Ordine del Caos alla tavola di Grifondoro, poi si girò e rivolse un altro profondo inchino all’altro lato della sala, e attese che gli applausi e le risatine si spegnessero.

(Nei recessi della sua mente, si chiese se il Cappello Smistatore fosse realmente \textit{cosciente} nel senso di essere consapevole della propria consapevolezza, e in caso affermativo, se fosse soddisfatto di poter parlare solo una volta all’anno con bambini di undici anni. La sua canzone lo sottintendeva: \textit{Oh, sono il Cappello Smistatore e non provo scorno, se dormo un anno e lavoro un giorno…})

Quando ci fu nuovamente silenzio nella stanza, Harry sedette sullo sgabello e posò \textit{con cura} sulla propria testa il telepatico manufatto vecchio di 800 anni, prodotto di una magia dimenticata.

Pensò più intensamente che poté: \textit{Non Smistarmi subito! Ho delle domande che ho bisogno di farti! Sono mai stato Obliato? Hai Smistato il Signore Oscuro quando era un bambino e puoi parlarmi dei suoi punti deboli? Mi puoi dire perché ho avuto la bacchetta sorella di quella del Signore Oscuro? Il fantasma del Signore Oscuro è legato alla mia cicatrice ed è per questo che mi arrabbio così tanto, qualche volta? Queste sono le domande più importanti, ma se hai un altro momento mi puoi dire qualcosa su come riscoprire le magie perdute che ti hanno creato?}

Nel silenzio dello spirito di Harry, dove prima non c’era mai stata altra voce che una, giunse una seconda voce non familiare, che sembrava chiaramente preoccupata:

«\textit{Oh, cielo. Questo non era mai successo prima d’ora…}»



