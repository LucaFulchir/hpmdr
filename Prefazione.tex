% !TeX root = Harry.tex

\chapter*{PREFAZIONE}

Cos’è \emph{Harry Potter e i metodi della razionalità} e perché dovresti leggere questa storia?

Quella che hai tra le mani è la traduzione in lingua italiana di \emph{Harry Potter and the Methods of Rationality} -- frequentemente abbreviato in \texttt{HPMOR} --, una \emph{fanfic} di Harry Potter scritta dallo studioso di teoria della decisione Eliezer Yudkowsky.

Ovviamente hai sentito parlare di Harry Potter e probabilmente sai che a scrivere la saga del giovane mago è stata la scrittrice inglese J.K. Rowling, non un tizio dal nome decisamente ebraico, studioso di qualcosa che non hai mai sentito prima. In effetti il trucco sta in quella parola in corsivo, \emph{fanfic}, che è la contrazione dell’inglese \emph{fan fiction}, ovvero \foreignquote{german}{narrativa composta dagli appassionati}: si tratta di una pratica letteraria amatoriale, con la quale gli appassionati di una serie di libri -- come pure di un film o di una serie televisiva o di un fumetto -- scrivono racconti ambientati nel mondo delle storie e dei personaggi che amano. La maggior parte delle fanfic sono di qualità scandente (ricorda, sono scritte da appassionati della serie, non da scrittori veri), ma alcune sono decisamente ben scritte e pochissime sono qualitativamente indistinguibili da romanzi scritti da professionisti.\footnote{Ad esempio, la trilogia letteraria di Cinquanta sfumature di grigio di E.L. James era nata come una fanfic della serie di romanzi Twilight di Stephenie Meyer.}

\emph{Harry Potter e i metodi della razionalità} è una fanfic talmente ben scritta che, sebbene non avrà mai un’edizione commerciale per questioni di diritti d’autore sulle storie del famoso mago, ha portato alla nascita di un’ampia e vibrante comunità di appassionati e ha ricevuto recensioni positive persino da parte di scrittori di professione. È una delle fanfic più lette tra quelle pubblicate sul sito \href{fanfiction.net}{fanciction.net}, con più di $23.000$ recensioni; ha un sito dedicato in cui sono raccolte le opere derivate, come fanart (disegni di appassionati), brani musicali ispirati alla storia, e persino fanfic di questa fanfic; una sezione del sito \href{https://reddit.com}{reddit.com} dedicata alla discussione della trama; \footnote{\href{https://www.reddit.com/r/HPMOR/}{www.reddit.com/r/HPMOR/}} infine, esiste un podcast che pubblica la drammatizzazione, in lingua inglese, dell’intera storia, con diversi interpreti dilettanti che recitano i diversi personaggi.\footnote{Il podcast è curato da Eneasz Brodski, ed è accessibile all’indirizzo \href{https://www.hpmorpodcast.com}{www.hpmorpodcast.com}}

Il fatto che \emph{Harry Potter e i metodi della razionalità} sia ben scritto è solo uno dei motivi per i quali dovresti leggerlo. La migliore qualità di questa fanfic è quella di aver portato la storia di Harry Potter — la versione «canonica» così come scritta dall’autrice — al livello \emph{superiore}. Da una piccola e apparentemente insignificante differenza iniziale, ovvero che Harry venga cresciuto da due zii amorevoli invece di essere l’omologo maschile di Cenerentola, ha origine una storia in cui ogni personaggio principale si comporta al meglio delle sue capacità, senza che nessuno segua degli stereotipi narrativi, o faccia cose stupide o prevedibili. Ogni decisione è presa al meglio delle conoscenze, degli scopi e del carattere dei personaggi, e questo conferisce loro una maggiore profondità e alla storia una capacità di mettere alla prova il lettore che altre storie non hanno: osservare qualcuno che si trova in una situazione estremamente stressante in cui prendere la decisione giusta fa la differenza tra la vita e la morte, ma in cui quella decisione giusta è difficile da riconoscere, rende la narrazione estremamente eccitante, infatti. Questo vale per i personaggi principali (Silente, McGonagall, Snape e Quirrell), ma soprattutto per Harry, il quale bilancia la propria ignoranza del mondo della magia con la conoscenza di due strumenti che mancano ai maghi: la scienza e la teoria della decisione razionale. L’Harry di hpmor è infatti un seguace appassionato del metodo scientifico, e in particolare ha letto e compreso buona parte della letteratura scientifica che riguarda il modo di prendere le decisioni migliori in situazioni di conoscenza incompleta (la teoria della decisione, appunto, di cui non a caso l’autore è studioso). Vedere Harry che cerca di superare i propri limiti, di correggere i propri difetti, di scegliere sempre la soluzione giusta a prescindere da quanto sia difficile o contro-intuitiva, rientra in quello scontro tra l’essere umano e i propri limiti che rende così interessanti alcune storie letterarie.

C’è poi un’altra cosa che rende \emph{Harry Potter e i metodi della razionalità} alquanto insolito, il fatto che sia un meccanismo educativo. Gli esseri umani imparano attraverso la narrativa, immaginare qualcosa è analogo mentalmente a ricordare qualcosa che è realmente avvenuto. Yudkowsky usa intenzionalmente questo meccanismo per far rafforzare le abilità razionali di te che sei il lettore; quasi ogni capitolo, o gruppo di capitoli, serve a insegnare una di queste tecniche di rafforzamento, e la tecnica è spesso evidenziata nel titolo del capitolo (ad esempio il \hyperref[capitolo 3]{capitolo 3} si intitola \hyperref[capitolo:3]{Confrontare la realtà con le sue alternative}), e spesso nel corso del capitolo Harry o qualche altro personaggio spiegano esplicitamente la tecnica, e poi la narrazione presenta uno o più casi in cui quella tecnica viene usata e ha successo o fallisce. E la cosa buffa è che spesso tu lettore neppure te ne accorgi, dato che questo meccanismo educativo è perfettamente integrato nella narrazione avvincente e nella caratterizzazione dei personaggi.

Infine, se sei uno di quei lettori a cui piace sviscerare in profondità i romanzi che leggi, sappi che Harry Potter e i metodi della razionalità è molto adatto a te; Yudkowsky ha infatti affermato che hpmor è progettato come una serie di enigmi risolvibile, con tutti gli indizi chiaramente presentati e la possibilità per il lettore di risolverli prima che vengano svelati dall’autore (il \hyperref[capitolo:13]{capitolo 13} è un esempio lampante di questo meccanismo, dato che i lettori della storia con un po’ di infarinatura della saga canonica di Harry Potter sono in grado di risolvere il mistero là contenuto prima che venga svelato nel capitolo successivo). Alcuni misteri sono lasciati sospesi a lungo, o risolti in maniera celata; per aiutare i lettori curiosi, esistono diversi siti in cui gli appassionati di hpmor si riuniscono per discutere di questi enigmi.

Detto questo, va aggiunto che questa storia non è per tutti. Lo stesso autore afferma che se non ti dovesse interessare dopo il decimo capitolo, faresti probabilmente meglio a cambiare libro. A molti lettori (specie genitori) non piace il modo adulto in cui Harry parla con gli adulti; a molti la storia semplicemente non «risuona», cosa legittima. Alcuni non gradiscono l’umorismo che pervade molte pagine, ad altri non piacciono le parti più oscure e tese, e la storia diventa particolarmente oscura e tesa da un certo punto in poi. Infine, certe tematiche adulte possono interessare poco ai bambini e agli adolescenti.

A me questa storia è piaciuta parecchio, tanto che mi sono imbarcato in quella che mi sembrava un’impresa titanica per le mie forze: tradurre tutta la storia dall’inglese all’italiano. Ora, quasi duemila pagine di traduzione dopo, posso dire che ne è valsa la pena; ti suggerisco, mio lettore, di provare a darci un’occhiata, magari questa storia entusiasmerà e cambierà anche te come me.

Cato Philosophus
Roma, luglio 2015