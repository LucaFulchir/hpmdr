% !TeX root = Harry.tex

\chapter{Localizzare le ipotesi}
\label{capitolo:17}

\emph{«Inizi a vedere lo schema, a sentire il ritmo del mondo.»}

~\\
~\\

Giovedì.

A voler essere precisi, le 7:24 di giovedì mattina.

Harry era seduto sul suo letto, un libro di testo che giaceva floscio tra le sue mani immobili.

Aveva appena avuto un’idea per un esperimento \textit{davvero brillante.}

Avrebbe significato aspettare un’altra ora prima di fare colazione, ma quello era il motivo per cui aveva le barrette di cereali. No, questa idea andava assolutamente, decisamente verificata subito, immediatamente, ora.

Mise da parte il libro di testo, saltò giù dal letto, vi girò intorno, si fiondò nel livello inferiore del suo baule, corse giù per le scale, e iniziò a spostare scatole di libri. (Aveva assolutamente bisogno di toglierli dalle scatole e metterli nelle librerie, prima o poi, ma era nel pieno della sua gara di lettura con Hermione e stava perdendo terreno, così non ne aveva il tempo.)

Harry trovò il libro che cercava e tornò indietro correndo su per le scale.

Gli altri ragazzi si stavano preparando per scendere a colazione nella Sala Grande e iniziare la giornata.

«Scusate, potreste fare una cosa per me?» disse Harry. Mentre parlava sfogliò l’indice del libro, trovò la pagina con i primi diecimila numeri primi, andò a quella pagina e spinse il libro in mano ad Anthony Goldstein. «Scegli due numeri di tre cifre da questa lista. Non dirmi quali sono. Moltiplicali insieme e dimmi solo il prodotto. Oh, e potresti fare la moltiplicazione due volte, per sicurezza? Per favore, assicurati di avere la risposta esatta, non sono sicuro di cosa possa accadere all’universo se fai un errore di moltiplicazione.»

Rivelò molto di come era stata la vita in quel dormitorio nei pochi giorni precedenti il fatto che Anthony non si preoccupò neppure di dire qualcosa tipo «Perché sei improvvisamente uscito di testa?» oppure «Mi sembra una cosa molto strana, perché me la chiedi?» oppure «Che significa che non sei sicuro di cosa accadrà all’universo?»

Anthony accettò il libro senza proferir parola e tirò fuori una pergamena e una penna. Harry girò su sé stesso e chiuse gli occhi, per assicurarsi di non vedere nulla, oscillando avanti e indietro e saltellando su e giù. Prese un taccuino di carta e una portamina e si preparò a scrivere.

«Okay», disse Anthony, «Centottantunomila quattrocentoventinove.»

Harry appuntò $181.429$. Ripeté quanto aveva appena scritto, e Anthony lo confermò.

Poi Harry corse indietro nel livello inferiore del suo baule, diede un’occhiata all’orologio (segnava le 4:28 che significavano le 7:28) e chiuse gli occhi.

Circa trenta secondi dopo, udì il rumore dei passi, seguito dal rumore del livello inferiore del baule che veniva chiuso. (Harry non era preoccupato di soffocare. Un Incantesimo di Ricambio dell’Aria era parte di ciò che si otteneva se si era disposti ad acquistare un baule veramente buono. Non era meravigliosa la magia, non ci si doveva preoccupare delle bollette dell’elettricità?)

E quando riaprì gli occhi, vide ciò che aveva sperato di vedere, un pezzo di carta piegato lasciato sul pavimento, il dono del suo sé stesso futuro.

Si chiami quel pezzo di carta «Carta-2».

Harry strappò un pezzo di carta dal proprio blocco.

Lo si chiami «Carta-1». Si trattava, ovviamente, dello stesso pezzo di carta. Si poteva anche vedere, se si fosse guardato da vicino, che i lati strappati coincidevano.

Harry riesaminò mentalmente l’algoritmo che avrebbe seguito.

5Se Harry avesse aperto Carta-2  e questa fosse stata bianca, allora avrebbe scritto «$101 \times 101$» su Carta-1, l’avrebbe piegata, avrebbe studiato per un’ora, sarebbe tornato indietro nel tempo, avrebbe lasciato Carta-1 (che sarebbe in tal modo divenuta Carta-2) per terra e sarebbe uscito dal baule per raggiungere i suoi compagni a colazione.

Se Harry avesse aperto Carta-2 e questa avesse avuto due numeri scritti sopra, Harry avrebbe moltiplicato quei numeri.

Se il loro prodotto fosse stato pari a $181.429$, Harry avrebbe scritto quei due numeri su Carta-1 e l’avrebbe mandata indietro nel tempo.

Altrimenti avrebbe aggiunto 2 al numero di destra e avrebbe scritto la nuova coppia di numeri su Carta-1. A meno che ciò non rendesse il numero sulla destra più grande di $997$, in tal caso Harry avrebbe aggiunto $2$ al numero sulla sinistra e scritto $101$ sulla destra.

E se Carta-2 avesse riportato $997 \times 997$, Harry avrebbe lasciato bianca Carta-1.

Questo significava che l’unico ciclo temporale possibile e stabile sarebbe stato quello in cui Carta-2 conteneva i due fattori primi di $181.429$.

Se avesse funzionato, Harry avrebbe potuto utilizzare questo metodo per ottenere ogni tipo di risposta che fosse facile da verificare ma difficile da trovare. Non avrebbe \textit{semplicemente} dimostrato che \textsc{p = np} una volta che avevi un Giratempo, questo trucco era \textit{più generale}. Avrebbe potuto usarlo per trovare le combinazioni dei lucchetti a combinazione, o password di ogni tipo. Forse persino l’ingresso alla Camera dei Segreti di Serpeverde, se fosse riuscito a immaginare un modo sistematico di descrivere tutti i luoghi di Hogwarts. Sarebbe stato un trucco terrificante persino per i suoi criteri.

Harry prese Carta-2 nella propria mano tremante, e l’aprì.

Carta-2 diceva con una calligrafia appena incerta:

\textsc{Non si scherza col tempo}

Harry scrisse «\textsc{Non si scherza col tempo}» su Carta-1 con una calligrafia appena incerta, la piegò attentamente, e prese la decisione di non compiere più esperimenti davvero brillanti col Tempo fin quando non avesse avuto almeno quindici anni.

In base alle conoscenze di Harry, quello era stato il risultato sperimentale più terrificante dell’intera storia della scienza.

Fu alquanto difficile per lui concentrarsi sul leggere il libro di testo nell’ora successiva.

Fu così che cominciò il giovedì di Harry.

\begin{figure}[h!]
        \includegraphics[scale=0.4]{boccino.png}
        \centering
\end{figure}

Giovedì.

A voler essere precisi, le 15:32 di giovedì pomeriggio.

Harry e gli altri ragazzi del primo anno erano all’aperto su di un campo erboso, con Madam Hooch, in piedi di fianco al parco manici di scopa di Hogwarts. Le ragazze avrebbero imparato a volare separatamente. Apparentemente, per qualche ragione, le ragazze non volevano imparare a volare sui manici di scopa in presenza dei ragazzi.

Harry era stato un po’ incerto tutto il giorno. Semplicemente non poteva smettere di chiedersi come quel \textit{particolare} ciclo temporale stabile fosse stato selezionato all’interno di quello che, in retrospettiva, era un vasto spazio delle possibilità.

E poi: sul serio, \textit{manici di scopa}? Stava per volare, in pratica, su di un segmento di linea? Non era fondamentalmente la singola forma più instabile che fosse possibile trovare, a meno di non tentare di reggersi su di una biglia puntiforme? Chi aveva scelto \textit{quella} forma per un mezzo per volare, tra tutte le possibilità? Harry aveva sperato che si trattasse solo di un modo di dire, ma no, erano in piedi di fronte a quelli che sarebbero sembrati a chiunque degli ordinari manici di scopa da cucina in legno. Qualcuno si era semplicemente fissato sull’idea dei manici di scopa e non era riuscito a prendere in considerazione nient’altro? Doveva essere così. Non era proprio possibile che gli strumenti \textit{ottimali} per pulire una cucina e per andarsene in giro volando coincidessero per caso, se qualcuno ci avesse lavorato sopra ripartendo da zero.

Era una giornata limpida con un cielo blu luminoso e un sole brillante che stava semplicemente pregando di finirti negli occhi e renderti impossibile vedere, se stavi cercando di volartene per quel cielo. Il terreno era bello e asciutto, con un odore come se fosse stato cotto, e in qualche modo sembrava molto, molto duro sotto le scarpe di Harry.

Harry continuò a ricordare a sé stesso che ci si aspettava che il minimo comune denominatore degli undicenni imparasse quella lezione, e che non poteva essere così difficile.

«Allungate la vostra mano destra sul manico, o la sinistra se siete mancini», disse ad alta voce Madam Hooch. «E dite, \textsc{su!}»

«\textsc{Su!}» gridarono tutti.

Il manico di scopa saltò impaziente nella mano di Harry.

Cosa che lo rese il primo della classe, per una volta. Apparentemente dire «\textsc{su!}» era molto più difficile di quanto sembrasse, e la maggior parte dei manici stavano rotolandosi per terra o cercando di allontanarsi lentamente dal proprio aspirante pilota.

(Naturalmente Harry avrebbe scommesso dei soldi che Hermione aveva fatto almeno altrettanto bene, quando era stato il suo turno di provare, in precedenza quella stessa mattina. Non era possibile che esistesse qualcosa che egli potesse padroneggiare al primo tentativo e che avrebbe messo in difficoltà Hermione, e se \textit{fosse} esistito e fosse risultato essere il \textit{volo con i manici di scopa} invece di qualcosa di intellettuale, Harry ne sarebbe semplicemente morto.)

Ci volle un po’ prima che tutti riuscissero ad avere un manico di scopa di fronte a sé. Madam Hooch mostrò loro come montare e poi girò per il campo a piedi, correggendo impugnature e posizioni. A quanto pareva, anche ai pochi bambini che erano stati autorizzati a volare a casa non era stato insegnato a farlo correttamente.

Madam Hooch osservò il campo pieno di ragazzi e annuì. «Ora, quando soffierò nel fischietto, voi darete un calcio al terreno, forte.»

Harry deglutì, cercando di sedare la sensazione di nausea allo stomaco.

«Stringete fermamente i manici, alzatevi di pochi piedi, e poi tornate subito giù piegandovi leggermente in avanti. Al mio fischio — tre — due –»

Uno dei manici partì sparato verso il cielo, accompagnato dalle urla di un giovane ragazzo — urla di orrore, non di delizia. Il ragazzo stava ruotando a una velocità terribile mentre saliva, e poterono vedere solo di sfuggita il suo volto bianco –

Come fosse al rallentatore, Harry stava saltando giù dal proprio manico di scopa e cercando a tastoni la propria bacchetta, sebbene non sapesse realmente cosa volesse farci, aveva avuto esattamente due lezioni di Incantesimi, e l’ultima \textit{era} stata sull’Incantesimo di Levitazione, ma Harry era stato in grado di lanciare l’incantesimo solo una volta su tre e certamente non era in grado di far levitare intere persone –

\textit{Se esiste un qualunque potere nascosto in me, che si riveli} \textsl{\textsc{ora!}}

«Torna indietro, ragazzo!» urlò Madam Hooch (si doveva trattare dell’ordine più inutile che si potesse immaginare per domare un manico di scopa fuori controllo, e proveniva da un \textit{istruttore di volo}, e una sezione completamente automatizzata del cervello di Harry aggiunse Madam Hooch alla sua lista dei folli).

E il ragazzo fu disarcionato dal manico di scopa.

Sembrò muoversi per aria molto lentamente, all’inizio.

«\textit{Wingardium Leviosa!}» urlò Harry.

L’incantesimo fallì. Poté percepirlo mentre falliva.

Ci fu un \textsc{tunf} e il suono distante di qualcosa che si rompeva, e il povero ragazzo giacque rattrappito a faccia in giù nell’erba.

Harry rinfoderò la bacchetta e corse avanti alla massima velocità. Arrivò a fianco al ragazzo contemporaneamente a Madam Hooch, infilò la mano nella borsa e cercò di richiamare oh dio come si chiamava non fa niente avrebbe semplicemente provato «Pacchetto di Guarigione!» e gli comparve in mano e –

«Polso fratturato», disse Madam Hooch. «Calmati, ragazzo, ha solo un polso fratturato!»

Vi fu una sorta di sbandamento mentale quando la mente di Harry si riprese dalla Modalità Panico.

Il Pacchetto Extra di Guarigione di Emergenza giaceva aperto di fronte a lui, e in mano a Harry c’era una siringa di fuoco liquido, che sarebbe stato in grado di mantenere ossigenato il cervello del ragazzo nel caso in cui si fosse rotto il collo.

«Ah…» fece Harry con una voce piuttosto incerta. Il suo cuore stava battendo così forte che quasi non riusciva a sentirsi ansimare. «Ossa rotte… va bene… Filo da Sutura?»

«Quello è solo per le emergenze», reagì bruscamente Madam Hooch. «Mettilo via, sta bene.» Si piegò sul ragazzo, offrendogli una mano. «Forza, ragazzo, va tutto bene, alzati!»

«Non avrà seriamente intenzione di farlo salire ancora su quel manico?» chiese Harry inorridito.

Madam Hooch gli lanciò un’occhiataccia. «Naturalmente no!» Sollevò il ragazzo in piedi usando il suo braccio sano — Harry vide con turbamento che si trattava \textit{ancora una volta} di Neville Longbottom, ma cosa aveva quel ragazzo? — e tornò verso tutti gli altri bambini che stavano guardando. «Nessuno di voi deve muoversi mentre porto questo ragazzo all’ospedale! Lasciate i manici dove sono o sarete fuori da Hogwarts prima che possiate dire ‘Quidditch’. Forza, caro.»

E Madam Hooch si allontanò con Neville, che si stava stringendo il polso e cercava di controllare i singhiozzi.

Quando furono abbastanza lontani, uno dei Serpeverde iniziò a ridacchiare.

Questo diede il la agli altri.

Harry si voltò e li guardò. Sembrò un’ottima occasione per memorizzare alcune facce.

E Harry vide che Draco stava passeggiando verso di lui, accompagnato dal signor Crabbe e dal signor Goyle. Il signor Crabbe non stava sorridendo. Il signor Goyle decisamente sì. Draco stesso aveva un’espressione controllata che occasionalmente si contorceva, da cui Harry dedusse che Draco pensava che la situazione fosse spassosa ma non vedeva come guadagnare alcun vantaggio politico ridendone ora invece che dopo nel sotterraneo dei Serpeverde.

«Insomma, Potter», disse Draco in una voce bassa che non era udibile da lontano, ancora con quell’espressione molto controllata che occasionalmente si contorceva, «Volevo semplicemente dirti, quando sfrutti un’emergenza per fare sfoggio della tua capacità di capo, vuoi che sembri che la situazione sia sotto il tuo controllo, piuttosto che, diciamo, farti prendere dal panico più completo.» Il signor Goyle ridacchiò, e Draco gli lanciò uno sguardo di reprimenda. «Nonostante ciò probabilmente hai ottenuto qualche vantaggio, comunque. Hai bisogno di aiuto per mettere via la cassetta di pronto soccorso?»

Harry si voltò a guardare il Pacchetto di Guarigione, cosa che gli fece distogliere lo sguardo da Draco. «Penso di farcela», disse Harry. Mise la siringa al suo posto, riannodò i lacci, e si alzò.

Ernie Macmillan arrivò proprio mentre Harry stava inserendo il pacchetto nella sua borsa mokeskin.

«Grazie, Harry Potter, a nome dei Tassofrasso», disse formalmente Ernie Macmillan. «È stato un buon tentativo e una buona idea.»

«Una buona idea davvero», disse Draco strascicando le parole. «Perché i Tassofrasso non hanno tutti estratto le bacchette? Forse se \textit{tutti} voi aveste aiutato, invece del solo Potter, avreste potuto prenderlo. Pensavo che i Tassofrasso dovessero aiutarsi a vicenda.»

Ernie sembrò diviso tra l’arrabbiarsi e il voler morire di vergogna. «Non ci abbiamo pensato in tempo –»

«Ah», disse Draco, «non ci avete \textit{pensato}, credo che sia per questo che è meglio avere un Corvonero per amico che tutti i Tassofrasso.»

Oh, al diavolo, come avrebbe dovuto destreggiarsi in tutto ciò Harry… «Non sei d’aiuto», disse in tono mite. Sperando che Draco l’avrebbe interpretato come \textit{stai interferendo nei miei piani, per favore taci.}

«Ehi, cos’è questa?» chiese il signor Goyle. Si chinò sull’erba e raccolse qualcosa dalle dimensioni simili a una grossa biglia, una palla di vetro che sembrava piena di una vorticante nebbiolina bianca.

Ernie sbatté le palpebre. «La Ricordella di Neville!»

«Cos’è una Ricordella?» chiese Harry.

«Diventa rossa se ti sei dimenticato qualcosa», rispose Ernie. «Non ti dice cosa hai dimenticato, però. Dalla a me, per favore, la restituirò a Neville più tardi.» Ernie tese la mano.

Un’improvviso ghigno balenò sul volto del signor Goyle, che si girò e corse via.

Ernie rimase fermo un momento per la sorpresa, poi gridò «Ehi!» e corse dietro al signor Goyle.

E il signor Goyle afferrò un manico di scopa, vi saltò sopra con un unico movimento fluido e si alzò in aria.

Harry rimase sconvolto. Madam Hooch non aveva detto che quello avrebbe causato la sua \textit{espulsione?}

«\textit{Quell’idiota!}» sibilò Draco. Aprì la bocca per gridare –

«\textit{Ehi!}» urlò Ernie. «Quella è di Neville! \textit{Restituiscila!}»

I Serpeverde iniziarono ad acclamare e a fischiare.

La bocca di Draco si chiuse di colpo. Harry colse l’improvvisa espressione di indecisione sul suo volto.

«Draco», chiamò Harry a bassa voce, «se non ordini a quell’idiota di tornare a terra, l’insegnante tornerà e –»

«\textit{Venitevela a prendere, Tassofrasso!}» gridò il signor Goyle, e una grande acclamazione si alzò dai Serpeverde.

«Non \textit{posso!}» sussurrò Draco. «Tutti i Serpeverde penseranno che sono \textit{debole!}»

«E se il signor Goyle si fa espellere», sibilò Harry, «tuo \textit{padre} penserà che sei un \textit{imbecille!}»

Il volto di Draco si deformò per l’agonia.

In quel momento –

«Ehi, \textit{Serpemarcia}», gridò Ernie, «nessuno ti ha mai detto che i Tassofrasso restano uniti? \textit{Mano alle bacchette, Tassofrasso!}»

E ci fu improvvisamente un gran numero di bacchette puntate nella direzione del signor Goyle.

Tre secondi dopo –

«\textit{Mano alle bacchette, Serpeverde!}» dissero circa cinque diversi Serpeverde.

E ci fu un gran numero di bacchette puntate nella direzione dei Tassofrasso.

Due secondi dopo –

«\textit{Mano alle bacchette, Grifondoro!}»

«\textit{Fa’ qualcosa, Potter!}» bisbigliò Draco. «\textit{Non posso essere io a fermarli, devi essere tu! Ti dovrò un favore, pensa a qualcosa, non dovresti essere geniale?}»

In circa cinque secondi e mezzo, capì Harry, qualcuno avrebbe lanciato la Fattura d’Attacco Semplice Sumera e quando tutto fosse terminato e i professori avessero finito di espellere studenti, gli unici ragazzi rimasti del suo anno sarebbero stati i Corvonero.

«\textit{Mano alle bacchette, Corvonero!}» gridò Michael Corner, che apparentemente si sentiva lasciato fuori dal disastro.

«\textsc{Gregory Goyle!}» urlò Harry. «\textit{Ti sfido a una competizione per il possesso della Ricordella di Neville!}»

Ci fu un’improvvisa pausa.

«Oh, davvero?» disse Draco nella parlata annoiata più alta che Harry avesse mai sentito. «Sembra interessante. Che tipo di competizione, Potter?»

Ehm…

«Competizione» era il massimo che l’ispirazione di Harry avesse prodotto. Che tipo di competizione, non avrebbe potuto dire «scacchi» perché Draco non sarebbe stato in grado di accettare senza sembrare strano, non poteva dire «braccio di ferro» perché il signor Goyle l’avrebbe schiacciato –

«Che ne dite di questo?» Harry disse ad alta voce. «Gregory Goyle e io ce ne stiamo in piedi, distanti l’uno dall’altro, e a nessuno è permesso avvicinarsi a noi. Non usiamo le nostre bacchette e nessun altro lo fa. Non mi muovo da dove sono, e non lo fa neppure lui. E se riesco a mettere le mani sulla Ricordella di Neville, allora Gregory Goyle rinuncia a tutte le sue pretese sulla Ricordella che tiene in mano e la dà a me.»

Ci fu un’altra pausa mentre le espressioni di sollievo si tramutavano in confusione.

«Aha, Potter!» disse Draco ad alta voce. «Voglio proprio vedertelo \textit{fare}! Il signor Goyle accetta!»

«Andata!» disse Harry.

«Potter, \textit{cosa diavolo?}» bisbigliò Draco, riuscendoci, in qualche modo, senza muovere le labbra.

Harry non sapeva come rispondere senza muovere le sue.

Iniziarono tutti a mettere via le bacchette, e il signor Goyle discese repentinamente al suolo con grazia, l’espressione alquanto confusa. Alcuni Tassofrasso iniziarono a muoversi verso il signor Goyle, ma Harry lanciò loro uno sguardo di disperata preghiera, ed essi si fermarono.

Harry si diresse verso il signor Goyle e si fermò quando si trovarono a pochi passi di distanza, abbastanza lontani da non potersi raggiungere l’un l’altro.

Lentamente, deliberatamente, Harry ripose la bacchetta.

Tutti gli altri indietreggiarono.

Harry deglutì. Sapeva a grandi linee quello che \textit{voleva} fare, ma doveva essere realizzato in modo tale che nessuno capisse cosa aveva fatto –

«Va bene», disse Harry a voce alta. «E ora…» Fece un respiro profondo e alzò una mano, le dita pronte a schioccare. Ci furono rantoli emessi da chiunque avesse sentito parlare delle torte, ovvero praticamente tutti. «\textit{Invoco la follia di Hogwarts! Felice felice bum bum palude palude palude!}» E Harry schioccò le dita.

Parecchi trasalirono.

E nulla accadde.

Harry lasciò che il silenzio si allungasse per un po’, crescendo, finché…

«Uhm», fece qualcuno. «Tutto qui?»

Harry guardò il ragazzo che aveva parlato. «Guarda davanti a te. Vedi quella zona di terra che sembra spoglia, senza erba sopra?»

«Uhm, sì», disse il ragazzo, un Grifondoro (Dean qualcosa?).

«Scava lì.»

Adesso Harry ricevette parecchie occhiate strane.

«Ehm, perché?» disse Dean qualcosa.

«Fallo e basta», rispose Terry Boot, con la voce stanca. «È inutile chiedere perché, fidatevi.»

Dean qualcosa si inginocchiò e cominciò a scavare per terra.

Dopo un minuto o giù di lì, Dean si alzò di nuovo. «Non c’è niente qui.»

Ah. Harry aveva avuto intenzione di tornare indietro nel tempo e di seppellire una mappa del tesoro che avrebbe portato a un’altra mappa del tesoro che avrebbe portato alla Ricordella di Neville, che avrebbe messo lì dopo averla ottenuta indietro dal signor Goyle…

Poi Harry si rese conto che c’era un modo molto più semplice che non minacciava di rivelare il segreto del Giratempo allo stesso modo.

«Grazie, Dean!» disse Harry a voce alta. «Ernie, daresti un’occhiata al terreno dove Neville è caduto per vedere se è possibile ritrovare la sua Ricordella?»

Tutti sembrarono ancora più confusi.

«Fallo e basta» disse Terry Boot. «Continuerà così finché qualcosa non funzionerà, e la cosa peggiore è che –»

«\textit{Merlino!}» rantolò Ernie. Reggeva la Ricordella di Neville. «È \textit{qui}! Proprio dov’è caduto!»

«\textit{Cosa?}» gridò il signor Goyle. Guardò in basso e vide…

… che stava ancora reggendo la Ricordella di Neville.

Ci fu una pausa piuttosto lunga.

«Ehm», disse Dean qualcosa, «questo non è possibile, giusto?»

«È un’inconsistenza nella trama», disse Harry. «Mi sono reso abbastanza strano da distrarre l’universo per un momento, ed esso ha dimenticato che Goyle aveva già raccolto la Ricordella.»

«No, aspetta, voglio dire, questo non è \textit{assolutamente} possibile –»

«Scusami, siamo tutti quanti qui aspettando di iniziare a volare su manici di scopa? Sì, lo siamo. Quindi taci. Ad ogni modo, una volta che metto le mani sulla Ricordella di Neville, la competizione è terminata e Gregory Goyle deve rinunciare a ogni pretesa sulla Ricordella che sta tenendo e darla a me. Questi erano i termini, ricordate?» Harry allungò una mano e fece un cenno a Ernie. «Falla semplicemente rotolare qui, poiché nessuno deve avvicinarsi a me, va bene?»

«Fermi tutti!» gridò un Serpeverde — Blaise Zabini, Harry avrebbe difficilmente dimenticato quel nome. «Come facciamo a sapere che quella è la Ricordella di Neville? Potresti aver semplicemente lasciato cadere lì \textit{un’altra} Ricordella –»

«Serpeverde è forte in lui», disse Harry, sorridendo. «Ma hai la mia parola che quella che Ernie sta tenendo è di Neville. Nulla affermo a proposito di quella che è in possesso di Gregory Goyle.»

Zabini si girò verso Draco. «\textit{Malfoy!} Non avrai intenzione di permettergli di cavarsela così –»

«Chiudi quella bocca, tu» rimbombò il signor Crabbe, in piedi dietro Draco. «Il signor Malfoy non ha bisogno che \textit{tu} gli dica cosa fare!»

\textit{Bravo} servitore.

«La mia scommessa era con Draco, della Nobile e Antichissima Casa Malfoy», disse Harry. «Non con te, Zabini. Ho fatto quello che il signor Malfoy ha detto gli sarebbe piaciuto vedermi fare, e per quanto riguarda il verdetto sulla scommessa, lo lascio nelle mani del signor Malfoy.» Harry inclinò la testa verso Draco e inarcò leggermente le sopracciglia. Questo avrebbe dovuto consentire a Draco di salvare sufficientemente la faccia.

Ci fu una pausa.

«Prometti che quella è \textit{effettivamente} la Ricordella di Neville?» disse Draco.

«Sì», rispose Harry. «Quella è la Ricordella che ritornerà a Neville e che fu originariamente sua. E quella che Gregory Goyle sta tenendo va a me.»

Draco annuì, l’espressione risoluta. «Non metterò in dubbio la parola della Nobile Casa Potter, allora, non importa quanto strano sia tutto questo. E allo stesso modo la Nobile e Antichissima Casa Malfoy mantiene la propria parola. Signor Goyle, la dia al signor Potter –»

«Ehi!» disse Zabini. «Non ha \textit{ancora} vinto, non ha ancora messo le mani su –»

«Prendi, Harry!» disse Ernie, e lanciò la Ricordella.

Harry prese facilmente al volo la Ricordella, aveva sempre avuto buoni riflessi. «Ecco», disse, «ho vinto…»

La voce di Harry si affievolì. Tutte le conversazioni si fermarono.

Nella sua mano la Ricordella stava brillando di un rosso intenso, ardendo come un sole in miniatura che getta ombre sul terreno nella piena luce del giorno.

\begin{figure}[h!]
        \includegraphics[scale=0.4]{boccino.png}
        \centering
\end{figure}

Giovedì.

A voler essere precisi, le 17:09 di giovedì pomeriggio, nell’ufficio della professoressa McGonagall, dopo le lezioni di volo (con un’ulteriore ora inserita in mezzo per Harry).

La professoressa McGonagall sedeva sul suo sgabello. Harry nella sedia dell’imputato di fronte alla sua scrivania.

«Professoressa», disse Harry con fermezza, «i Serpeverde stavano puntando le loro bacchette contro i Tassofrasso, i Grifondoro stavano puntando le loro bacchette contro i Serpeverde, qualche \textit{idiota} ha ordinato ai Corvonero di estrarre le bacchette, e io ho avuto forse cinque secondi per impedire all’intera faccenda di esplodere! È stato tutto quello che sono riuscito a pensare!»

Il volto della professoressa McGonagall era emaciato e infuriato. «\textit{Non deve usare il Giratempo in quel modo, signor Potter!} Le è forse difficile comprendere il concetto di segretezza?»

«Non \textit{sanno} come ho fatto! Pensano solo che io possa fare cose veramente bizzarre schioccando le dita! Ho fatto anche altre cose bizzarre che non possono essere realizzate con i Giratempo, e farò \textit{altre} cose di questo genere, e \textit{questo} caso non sarà neppure ricordato! \textit{Dovevo farlo}, professoressa!»

«Lei \textit{non aveva} bisogno di farlo!» scattò la professoressa McGonagall. «Tutto ciò che doveva fare era era portare questo \textit{Serpeverde anonimo} a terra e deporre le bacchette! Avrebbe potuto sfidarlo a una partita di Spara Schiocco, ma no, lei doveva usare il Giratempo in maniera plateale e inutile!»

«È stato tutto quello che sono riuscito a pensare! Non so neppure cosa \textit{sia} Spara Schiocco, non avrebbero acconsentito a una partita a scacchi e se avessi scelto braccio di ferro avrei perso!»

«\textit{Allora avrebbe dovuto scegliere braccio di ferro!}»

Harry sbatté le palpebre. «Ma allora avrei \textit{perso} –»

Si fermò.

La professoressa McGonagall sembrava \textit{davvero} arrabbiata.

«Mi dispiace, professoressa McGonagall», disse Harry a bassa voce. «Onestamente non ci ho pensato, e lei ha ragione, avrei dovuto, sarebbe stato geniale se l’avessi fatto, ma non ci ho neppure pensato…»

La voce di Harry si spense. Gli fu improvvisamente chiaro che aveva avuto \textit{molte} altre opzioni. Avrebbe potuto chiedere a \textit{Draco} di suggerire qualcosa, avrebbe potuto chiedere alla folla… il suo uso del Giratempo era stato plateale e inutile. Aveva avuto a disposizione un gigantesco spazio delle possibilità, perché aveva scelto \textit{quella}?

Perché aveva visto un modo per \textit{vincere}. Vincere il possesso di un gingillo che gli insegnanti avrebbero comunque tolto al signor Goyle.

La volontà di vincere. Si era impadronita di lui.

«Mi dispiace», disse ancora Harry. «Per il mio orgoglio e per la mia stupidità.»

La professoressa McGonagall si passò una mano sulla fronte. Parte della sua rabbia sembrò svanire. Ma la voce venne fuori ancora molto dura. «Un’altra esibizione come questa, signor Potter, e restituirà il Giratempo. Sono stata chiara?»

«Sì», rispose Harry. «Comprendo e mi dispiace.»

«Allora, signor Potter, le sarà permesso tenere il Giratempo, per ora. E considerando le dimensioni del disastro che è riuscito, in conclusione, a evitare, non detrarrò alcun punto a Corvonero.»

\textit{Per di più non avrebbe potuto spiegare perché avesse detratto i punti}. Ma Harry non era sufficientemente sciocco da dirlo a voce alta.

«Soprattutto, perché la Ricordella si è accesa in quel modo?» chiese Harry. «Significa che sono stato Obliato?»

«Questo lascia perplessa anche me», disse lentamente la professoressa McGonagall. «Se fosse così semplice, credo che i tribunali utilizzerebbero le Ricordelle, e non lo fanno. Dovrò investigare la faccenda, signor Potter.» Sospirò. «Ora può andare.»

Harry iniziò ad alzarsi dalla sedia, poi si fermò. «Uhm, mi scusi, avevo qualcos’altro da chiederle –»

Poté notare a malapena il sussulto. «Di che si tratta, signor Potter?»

«Del professor Quirrell –»

«Sono certa, signor Potter, che non sia nulla di importante.» La professoressa McGonagall pronunciò le parole con molta fretta. «Sicuramente ha udito il Preside dire agli studenti che non dovete disturbarci con lamentele di scarsa importanza riguardo il Professore di Difesa.»

Harry era piuttosto confuso. «Ma questo \textit{potrebbe} essere importante, ieri ho percepito questa improvvisa sensazione di sventura quando –»

«Signor Potter! Anche io ho una sensazione di sventura! E la mia sensazione di sventura mi suggerisce che \textit{lei non debba terminare quella frase!}»

La bocca di Harry si spalancò. La professoressa McGonagall aveva avuto successo; Harry era senza parole.

«Signor Potter», disse la professoressa McGonagall, «se ha scoperto qualunque cosa che le sembra importante riguardo il professor Quirrell, la prego, si senta libero di non condividerla con me o con chiunque altro. Ora credo che abbia impegnato a sufficienza il mio tempo prezioso –»

«\textit{Questo non è da lei!}» scoppiò Harry. «Mi dispiace, ma questo mi sembra \textit{incredibilmente} irresponsabile! Da quanto ho sentito c’è una specie di maledizione sulla cattedra di Difesa, e se \textit{sapeste} già che qualcosa andrà male, credo che sareste tutti in allerta –»

«Andare \textit{male}, signor Potter? \textit{Io certamente spero di no}.» Il volto della professoressa McGonagall era senza espressione. «Dopo che la professoressa Blake è stata trovata in uno stanzino con non meno di cinque Serpeverde del quinto anno, lo scorso febbraio, e che un anno prima la professoressa Summers è stata un tale fallimento come educatrice che i suoi studenti credevano che un Molliccio fosse un tipo di mobile, sarebbe \textit{catastrofico} se qualche problema con lo straordinariamente competente professor Quirrell fosse portato alla mia attenzione ora, e oserei dire che la maggior parte dei nostri studenti fallirebbe i suoi \textsc{g.u.f.o}. in Difesa e i suoi \textsc{m.a.g.o.}»

«Capisco», disse Harry lentamente, meditando il tutto. «Quindi, in altre parole, qualunque cosa ci sia di sbagliato nel professor Quirrell, lei non ne vuole sapere nulla fino alla fine dell’anno scolastico. E poiché siamo a settembre, potrebbe assassinare il Primo Ministro in diretta televisiva e cavarsela, per quanto la riguarda.»

La professoressa McGonagall lo guardò senza battere ciglio. «Sono certa che non potrei mai essere sentita sostenere una posizione del genere, signor Potter. A Hogwarts ci sforziamo di prevenire \textit{qualunque cosa} minacci i traguardi educativi dei nostri studenti.»

\textit{Come i Corvonero del primo anno che non sanno tenere chiusa la bocca}. «Credo di capirla perfettamente, professoressa McGonagall.»

«Ne dubito, signor Potter. Ne dubito molto.» La professoressa McGonagall si sporse in avanti, il viso che si fece nuovamente severo. «Poiché lei e io abbiamo già discusso faccende molto più delicate di questa, le parlerò schiettamente. Lei, e solo lei, ha riferito questa sensazione di sventura. Lei, e solo lei, è un magnete per il caos del quale non ho mai visto il simile. Dopo il nostro giro di spese a Diagon Alley, e \textit{poi} il Cappello Smistatore, e poi il piccolo episodio di \textit{oggi}, posso ben presagire che sono destinata a stare seduta nell’ufficio del Preside ad ascoltare qualche spassosissima storia a proposito del professor Quirrell in cui lei e solo lei interpreta una parte da protagonista, dopo di che non vi sarà altra scelta che licenziarlo. Sono già rassegnata a questo, signor Potter. E se questo triste evento dovesse avere luogo prima delle Idi di maggio, la appenderò all’ingresso di Hogwarts con i suoi stessi intestini, e verserò delle coccinelle di fuoco nel suo naso. Mi ha capito bene, \textit{ora?}»

Harry annuì, gli occhi spalancati. Poi, dopo un secondo, «E se potessi farlo accadere l’ultimo giorno di scuola dell’anno?»

«\textit{Esca dal mio ufficio!}»

\begin{figure}[h!]
        \includegraphics[scale=0.4]{boccino.png}
        \centering
\end{figure}

Giovedì.

Doveva esserci qualcosa di strano a proposito dei giovedì a Hogwarts.

Erano le 17:32 di giovedì pomeriggio, e Harry era in piedi vicino al professor Flitwick, di fronte al grande gargoyle di pietra che faceva la guardia all’ingresso dell’ufficio del Preside.

Dopo essere tornato dall’ufficio della professoressa McGonagall alle sale studio di Corvonero, uno degli studenti gli aveva detto di presentarsi nell’ufficio del professor Flitwick, e lì Harry aveva scoperto che Silente voleva parlargli.

Sentendosi piuttosto ansioso, aveva chiesto al professor Flitwick se il Preside avesse detto di cosa si trattava.

Il professor Flitwick aveva alzato le spalle in maniera impotente.

Apparentemente Silente aveva affermato che Harry era troppo giovane per invocare le parole del potere e della follia.

\textit{Felice felice bum bum palude palude palude?} Harry aveva pensato ma non detto ad alta voce.

«La prego, non si preoccupi troppo, signor Potter», squittì il professor Flitwick da una qualche parte all’altezza delle spalle di Harry. (Harry era grato per la gigantesca e gonfia barba del professor Flitwick, era difficile abituarsi ad un Professore che non era solo più basso di lui, ma che parlava con un tono più acuto.) «Il preside Silente può sembrare un po’ strano, o molto strano, o anche estremamente strano, ma non ha mai fatto alcun male a uno studente, e non credo lo farà mai.» Il professor Flitwick rivolse a Harry un sorriso di incoraggiamento. «Cerchi di ricordarlo sempre e non si lascerà prendere dal panico!»

Non era d’aiuto.

«Buona fortuna!» squittì il professor Flitwick, si chinò verso il gargoyle e disse qualcosa che Harry, in qualche modo, non sentì neppure. (Ovviamente, la parola d’accesso non sarebbe stata un granché se l’avessi potuta sentire pronunciare.) E il gargoyle di pietra si mosse di lato con un movimento molto naturale e ordinario che Harry trovò alquanto sconvolgente, poiché il gargoyle continuò a sembrare pietra solida e immobile per tutto il tempo.

Dietro il gargoyle si trovava una scala a chiocciola che ruotava lentamente. C’era qualcosa di ipnotico e inquietante in tutto ciò, e ancora più inquietante era il fatto che una scala a chiocciola ruotante non avrebbe dovuto portarti da nessuna parte.

«Salga!» squittì Flitwick.

Harry mise piuttosto nervosamente il piede sulla scala, e si trovò, per qualche ragione che il suo cervello non sembrava affatto in grado di visualizzare, a muoversi verso l’alto.

Il gargoyle dietro di lui tornò al proprio posto con un tonfo, le scale a chiocciola continuarono a ruotare e Harry continuò a essere sempre più in alto, e dopo un periodo piuttosto vertiginoso, Harry si trovò di fronte a una porta di quercia con un grifone battente in ottone.

Harry allungò la mano e girò la maniglia.

La porta si aprì.

E Harry vide la stanza più interessante che avesse mai visto in vita sua.

C’erano minuscoli meccanismi metallici che ronzavano o ticchettavano o cambiavano lentamente forma o emettevano piccoli sbuffi di fumo. C’erano dozzine di fluidi misteriosi in dozzine di contenitori dalla forma strana, tutti gorgoglianti, ribollenti, trasudanti, mutanti colore, o formanti arrangiamenti interessanti che scomparivano mezzo secondo dopo che li avevi notati. C’erano oggetti che sembravano orologi con molte lancette, con numeri incisi o in lingue irriconoscibili. C’era un bracciale con incastonato un cristallo lenticolare che brillava di mille colori, e un uccello appollaiato in cima a una piattaforma d’oro, e una tazza di legno riempita con quello che sembrava sangue, e la statua di un falco incrostato in smalto nero. La parete era tutta tappezzata di immagini di persone dormienti, e il Cappello Smistatore era casualmente attaccato a una cappelliera che stava reggendo anche due ombrelli e tre pantofole rosse per piedi sinistri.

Nel mezzo di tutta quella confusione, stava una semplice scrivania di quercia nera. Davanti alla scrivania c’era uno sgabello di quercia. E dietro la scrivania c’era un trono ben imbottito contenente Albus Percival Wulfric Brian Silente, adorno di una lunga barba d’argento, un cappello simile a fungo gigante schiacciato, e ciò che a occhi babbani sembravano tre strati di pigiama rosa brillante.

Silente stava sorridendo, i suoi occhi luminosi che brillavano di una folle intensità.

Con un po’ di trepidazione, Harry si sedette di fronte alla scrivania. La porta si chiuse da sola dietro di lui con un forte \textit{tunc}.

«Ciao, Harry», disse Silente.

«Ciao, Preside», rispose Harry. Quindi si davano del tu? Silente gli avrebbe ora detto di chiamarlo –

«Per favore, Harry!» disse Silente. «Preside suona così formale. Chiamami semplicemente Pre, per semplicità.»

«Lo farò senz’altro, Pre», disse Harry.

Ci fu una breve pausa.

«Lo sai», disse Silente, «che sei la prima persona che mi abbia mai preso sul serio su questo punto?»

«Ah…» disse Harry. Cercò di controllare la propria voce malgrado l’improvvisa sensazione di sprofondare. «Mi scusi, io, ah, Preside, lei mi ha detto di farlo e quindi io –»

«Eh, ti prego!» disse Silente allegro. «E non c’è nessun bisogno di essere così preoccupato, non ti lancerò da una finestra solo perché hai fatto un errore. Ti avvertirò a sufficienza, prima, se stai facendo qualcosa di sbagliato! Inoltre, ciò che importa non è come le persone si rivolgono a te, è cosa pensano di te.»

\textit{Non ha mai fatto del male a uno studente, continua solo a ricordare questo e non ti farai prendere dal panico.}

Silente tirò fuori una scatolina di metallo e l’aprì, mostrando alcuni piccoli grumi gialli. «Frizlemon?», disse il Preside.

«Ehm, no, grazie, eh», disse Harry. \textit{Somministrare} \textsl{\textsc{lsd}} \textit{a uno studente conta come fargli del male, o ricade nella categoria innocuo divertimento?} «Lei ha, uhm, detto qualcosa a riguardo del fatto che sarei troppo giovane per invocare le parole del potere e della follia?»

«Che lo sei con assoluta certezza!» rispose Silente. «Fortunatamente le Parole del Potere e della Follia furono smarrite sette secoli fa e nessuno ha più la minima idea di quali siano. Era semplicemente una piccola osservazione.»

«Ah…» disse Harry. Era cosciente del fatto che la sua bocca fosse rimasta aperta. «Allora perché mi ha chiamato qui?»

«\textit{Perché?}» rispose Silente. «Ah, Harry, se andassi in giro tutto il giorno chiedendomi \textit{perché} faccio le cose, non avrei mai il tempo per fare una singola cosa! Sono una persona alquanto impegnata, sai.»

Harry annuì, sorridendo. «Sì, era una lista molto impressionante. Preside di Hogwarts, Stregone Capo del Wizengamot, e Supremo Pezzo Grosso della Confederazione Internazionale dei Maghi. Scusi la domanda, ma mi chiedevo, è possibile ricevere più di sei ore se si usano più Giratempo? Perché è abbastanza impressionante se sta facendo tutto con appena 30 ore in un giorno.»

Ci fu un’altra breve pausa, durante la quale Harry continuò a sorridere. Era un po’ ansioso, in realtà molto ansioso, ma una volta chiaro che Silente stava deliberatamente cercando di confonderlo, qualcosa dentro di lui si \textit{rifiutò assolutamente} di restare seduto a subire come una vittima predestinata.

«Ho paura che il Tempo non gradisca troppo essere stiracchiato», disse Silente dopo la breve pausa, «eppure noi stessi sembriamo un po’ troppo grossi per lui, e quindi è una lotta continua per far calzare le nostre vite all’interno del Tempo.»

«Infatti», disse Harry con profonda solennità. «Ecco perché è meglio arrivare direttamente alle nostre questioni.»

Per un momento Harry si chiese se avesse esagerato.

Poi Silente ridacchiò. «E dritti al punto sia.» Il Preside si chinò in avanti, inclinando il cappello a fungo schiacciato e spazzolando la barba contro la scrivania. «Harry, questo lunedì hai fatto qualcosa che avrebbe dovuto essere impossibile anche con un Giratempo. O meglio, impossibile con \textit{solo} un Giratempo. Da dove sono giunte quelle due torte, mi chiedo?»

Un’ondata di adrenalina attraversò Harry. L’aveva fatto usando il Mantello dell’Invisibilità, quello che gli era stato dato in un pacco natalizio insieme a una nota, e quella nota aveva detto: \textit{se Silente intravvedesse la possibilità di possedere uno dei Doni della Morte non se la farebbe scappare…}

«Un pensiero naturale», continuò Silente, «è che siccome nessuno degli studenti del primo anno presenti era in grado di lanciare un incantesimo di questo tipo, qualcun altro fosse presente, sebbene non visto. E se nessuno avesse potuto vederlo, ebbene, sarebbe stato facile per lui lanciare le torte. Si potrebbe ulteriormente sospettare che siccome tu avevi un Giratempo, fossi tu quello invisibile; e che poiché l’incantesimo di Disillusione è molto al di là delle tue attuali capacità, tu fossi in possesso di un mantello invisibile.» Silente gli rivolse un sorriso complice. «Sono sulla buona strada finora, Harry?»

Harry era paralizzato. Aveva la sensazione che una pura e semplice bugia non sarebbe stata saggia, e forse neppure utile, e non riuscì a pensare a nient’altro da dire.

Silente agitò la mano in maniera amichevole. «Non preoccuparti, Harry, non hai fatto niente di male. I mantelli invisibili non sono contro le regole — suppongo che siano talmente rari che nessuno si sia mai preoccupato di inserirli nell’elenco. Ma in realtà mi chiedevo qualcosa di completamente diverso.»

«Oh?» disse Harry con la voce più normale che gli riuscisse.

Gli occhi di Silente brillavano di entusiasmo. «Vedi, Harry, dopo che hai avuto una manciata di avventure, tendi a capire l’essenza di queste faccende. Inizi a vedere lo schema, a sentire il ritmo del mondo. Cominci ad albergare sospetti \textit{prima} del momento della rivelazione. Tu sei il Ragazzo-Che-È-Sopravvissuto, e in qualche modo un mantello invisibile si è fatto strada fino alle tue mani appena quattro giorni dopo che hai scoperto la nostra Gran Bretagna magica. Tali mantelli non sono in vendita a Diagon Alley, ma ce n’è \textit{uno} che potrebbe trovare la propria strada fino a un possessore predestinato. E così non posso fare a meno di chiedermi se per qualche strano caso hai trovato non solo \textit{un} mantello invisibile, ma \textit{il} Mantello dell’Invisibilità, uno dei tre Doni della Morte e ritenuto in grado di nascondere chi lo indossa dagli sguardi della Morte stessa.» Lo sguardo di Silente era luminoso e impaziente. «Posso vederlo, Harry?»

Harry deglutì. Ora c’era un’alta marea di adrenalina dentro di lui ed era del tutto inutile, quello era il mago più potente del mondo ed era impossibile che potesse raggiungere la porta e non c’era alcun luogo a Hogwarts dove potesse nascondersi se ci fosse riuscito, stava per perdere il Mantello che era stato tramandato dai Potter per chissà quanto tempo –

Lentamente Silente si sedette nuovamente nel suo trono. La luce era scomparsa dai suoi occhi, e sembrò perplesso e un po’ triste. «Harry», disse, «se vuoi, puoi semplicemente dire di no.»

«Posso?» disse Harry con voce rauca.

«Sì, Harry», disse Silente. La sua voce suonava triste, ora, e preoccupata. «Sembra che tu abbia paura di me, Harry. Posso chiedere che cosa ho fatto per guadagnare la tua diffidenza?»

Harry deglutì. «C’è qualche modo in cui può pronunciare un giuramento magico vincolante che non prenderà il mio mantello?»

Silente scosse lentamente la testa. «I Voti Infrangibili non devono essere presi così alla leggera. E poi, Harry, se non conosci già l’incantesimo, avresti solo la mia parola che l’incantesimo fosse vincolante. Ma sicuramente ti rendi conto che non ho \textit{bisogno} del tuo permesso per vedere il Mantello. Sono abbastanza potente per prenderlo io stesso, borsa mokeskin o meno.» Il volto di Silente era molto severo. «Ma non lo farò. Il Mantello è tuo, Harry. Non voglio sottrartelo. Nemmeno per guardarlo solo per un momento, a meno che non sia tu a decidere di mostrarmelo. Questa è una promessa e un giuramento. Dovessi avere il bisogno di proibirtene l’utilizzo all’interno della scuola, pretenderei che andassi alla tua camera blindata presso Gringotts e lo depositassi lì.»

«Ah…» disse Harry. Deglutì a fatica, cercando di calmare l’ondata di adrenalina e di pensare in maniera ragionevole. Prese la borsa mokeskin dalla cintura. «Se davvero \textit{non ha} bisogno del mio permesso… allora lo prenda.» Harry porse la borsa a Silente, e si morse con forza il labbro, inviando un segnale a sé stesso nel caso in cui fosse stato successivamente Obliato.

Il vecchio mago infilò la mano nel sacchetto, e senza pronunciare alcuna parola di recupero, estrasse il Mantello dell’Invisibilità.

«Ah», inspirò Silente. «Avevo ragione…» Fece scivolare sulla mano la cangiante trama vellutata e nera. «Vecchio di secoli, e ancora perfetto come il giorno in cui fu realizzato. Abbiamo perso gran parte della nostra arte nel corso degli anni, e ora io stesso non posso creare una cosa del genere, nessuno può. Riesco a sentirne il potere come un eco nella mia mente, come una canzone cantata per sempre senza nessuno che l’ascolti…» Il mago alzò lo sguardo dal Mantello. «Non venderlo», disse, «non trasferirne il possesso a nessuno. Pensaci due volte prima di mostrarlo a chicchessia, e pensaci tre volte prima di rivelare che si tratta di uno dei Doni della Morte. Trattalo con rispetto, perché questo è davvero un Oggetto di Potere.»

Per un attimo il volto di Silente divenne malinconico…

… e poi consegnò nuovamente il Mantello a Harry.

Harry lo rimise nella proprio borsa.

Il volto di Silente fu ancora una volta serio. «Posso chiedere di nuovo, Harry, come sei arrivato a diffidare in questo modo di me?»

All’improvviso Harry si sentì piuttosto imbarazzato.

«C’era una nota con il Mantello», disse Harry a bassa voce. «Diceva che lei avrebbe cercato di prendermi il Mantello, se avesse saputo. Non so chi abbia lasciato la nota, però, davvero non lo so».

«Io… capisco» disse Silente lentamente. «Bene, Harry, non voglio mettere in dubbio le motivazioni di chi ti ha lasciato questa nota. Chi lo sa, magari potrebbe aver avuto la migliore delle intenzioni, giusto? Ti ha dato il Mantello, dopo tutto.»

Harry annuì, colpito dall’indulgenza di Silente, e imbarazzato dal netto contrasto con il proprio atteggiamento.

Il vecchio mago continuò. «Ma tu e io siamo entrambi pezzi da gioco dello stesso colore, credo. Il ragazzo che finalmente sconfisse Voldemort, e il vecchio che lo tenne impegnato per un tempo sufficiente a permetterti di risolvere il problema. Non ce l’avrò con te per la tua cautela, Harry, noi tutti dobbiamo fare del nostro meglio per essere saggi. Mi limiterò a chiedere che tu pensi due volte e mediti di nuovo tre volte, la prossima volta che qualcuno ti dirà di diffidare di me.»

«Mi dispiace», disse Harry. In quel momento si sentiva miserabile, aveva appena rimproverato Gandalf, essenzialmente, e la gentilezza di Silente lo stava solo facendo sentire peggio. «Non avrei dovuto diffidare di lei.»

«Ahimè, Harry, in questo mondo…» L’anziano mago scosse la testa. «Non posso neppure dire che sei stato imprudente. Non mi conoscevi. E in verità vi sono alcuni a Hogwarts di cui faresti bene a non fidarti. Forse anche alcuni che chiami amici.»

Harry deglutì. Questo sembrava piuttosto sinistro. «Come chi?»

Silente si alzò dalla sedia, e cominciò a esaminare uno dei suoi strumenti, un quadrante con otto lancette di lunghezza differente.

Dopo pochi istanti, il vecchio mago parlò di nuovo. «Probabilmente ti sembra alquanto affascinante», disse Silente. «Garbato — con te, almeno. Ben educato, forse ti ammira, persino. Sempre pronto a dare una mano, un favore, una parola di consiglio –»

«Oh, \textit{Draco Malfoy!}» disse Harry, sentendosi piuttosto sollevato dal fatto che non fosse Hermione o qualcosa del genere. «Oh no, no no no, ha capito male, lui non sta convertendo me, io sto convertendo lui.»

Silente si fermò lì dov’era, mentre scrutava il quadrante. «Stai facendo \textit{cosa?}»

«Ho intenzione di allontanare Draco Malfoy dal Lato Oscuro», disse Harry. «Ha presente, farne un bravo ragazzo.»

Silente si raddrizzò e si voltò verso Harry. Aveva una delle espressioni più attonite che Harry avesse mai visto su chiunque, tanto più su qualcuno con una lunga barba d’argento. «Sei certo», disse il vecchio mago dopo un momento, «che è pronto per essere redento? Temo che qualunque bontà tu veda in lui sia solo un pio desiderio — o peggio, un richiamo, un’esca –»

«Ehm, improbabile», disse Harry. «Voglio dire, se sta cercando di farsi passare per bravo ragazzo è incredibilmente incompetente. Questa non è la situazione in cui Draco mi si avvicina e si comporta in maniera affabile e io decido che deve avere un nucleo di bontà nascosto in profondità. Ho scelto di redimere lui proprio perché è l’erede di Casa Malfoy e se dovessi scegliere una persona da redimere, sarebbe ovviamente lui.»

L’occhio sinistro di Silente ebbe una contrazione. «Hai intenzione di piantare i semi dell’amore e della gentilezza nel cuore di Draco Malfoy perché ti aspetti che l’erede di Malfoy si dimostri prezioso per te?»

«Non solo per \textit{me!}» disse Harry indignato. «Per l’intera Gran Bretagna magica, se questo funziona! \textit{E inoltre} lui stesso avrà una vita più felice e mentalmente sana! Guardi, non ho abbastanza tempo per allontanare \textit{tutti} dal Lato Oscuro e ho dovuto chiedermi dove la Luce può ottenere il massimo vantaggio nel modo più veloce –»

Silente cominciò a ridere. A ridere molto più forte di quanto Harry si sarebbe aspettato, quasi ululando. Sembrava decisamente \textit{indecoroso}. Un mago antico e potente avrebbe dovuto ridacchiare in toni profondi e rimbombanti, non ridere così forte da dover cercare di riprendere fiato. Una volta Harry era letteralmente caduto dalla sedia per le risate guardando il film \textit{La guerra lampo dei Fratelli Marx}, ed era così forte che Silente stava ridendo adesso.

«Non è \textit{così} divertente», Harry disse dopo un po’. Stava cominciando a preoccuparsi nuovamente per la sanità mentale di Silente.

Il quale riprese il controllo con uno sforzo visibile. «Ah, Harry, un sintomo della malattia chiamata saggezza è che si inizia a ridere di cose che nessun altro pensa siano divertenti, perché quando sei saggio, Harry, inizi a capire le battute!» Il vecchio mago si asciugò le lacrime dagli occhi. «Ohimé. Ohimé. Spesso il male guasta sé stesso, infatti».

Il cervello di Harry si riservò un momento per collocare quelle parole familiari… «Ehi, questa è una citazione di \textit{Tolkien!} Lo dice \textit{Gandalf!}»

«Théoden, in realtà», disse Silente.

«Lei è un \textit{Nato babbano?}» chiese Harry sconvolto.

«Temo di no», disse Silente, sorridendo di nuovo. «Sono nato settant’anni prima che il libro fosse pubblicato, caro ragazzo. Ma sembra che i miei studenti Nati babbani tendano a pensare tutti allo stesso modo, su certe cose. Ho accumulato non meno di una ventina di copie de \textit{Il Signore degli Anelli} e tre raccolte dell’opera completa di Tolkien, e tengo a ciascuna di loro». Estrasse la bacchetta, la sollevò e si mise in posa. «\textit{Tu non puoi passare!} Che te ne pare?»

«Ah», disse Harry in uno stato che si avvicinava al completo arresto cerebrale, «penso che le manchi un Balrog.» E il pigiama rosa e il cappello a fungo schiacciato non stavano minimamente aiutando.

«Capisco.» Silente sospirò e inguainò tristemente la bacchetta nella cintura. «Temo che ci siano stati pochi preziosi Balrog nella mia vita recente. Oggi è tutto un riunioni del Wizengamot, dove devo cercare disperatamente di impedire che si compia alcunché, e cene formali in cui i politici stranieri fanno a gara per vedere chi è più ostinatamente folle. E sembrare misterioso alle persone, sapere cose non ho modo di sapere, fare dichiarazioni criptiche che possano essere comprese solo col senno di poi, e tutti gli altri piccoli modi in cui i maghi potenti si divertono dopo che hanno lasciato la parte dello schema che consente loro di essere eroi. A proposito, Harry, ho una certa qual cosa da darti, qualcosa che è appartenuta a tuo padre.»

«Davvero?» disse Harry. «Accidenti, chi l’avrebbe mai pensato.»

«Sì, infatti» disse Silente. «Suppongo che sia un po’ prevedibile, non è vero?» Il suo volto divenne solenne. «Tuttavia…»

Silente tornò alla scrivania e si sedette, tirando fuori al contempo uno dei cassetti. Vi inserì entrambe le braccia, e, con un piccolo sforzo, tirò fuori dal cassetto un oggetto dall’aspetto piuttosto grande e pesante, che poi depositò sulla scrivania di quercia con un forte tonfo.

«Questa», disse Silente, «era la roccia di tuo padre.»

Harry la fissò. Era grigio chiaro, scolorita, di forma irregolare, con spigoli vivi, e molto chiaramente una semplice, vecchia, ordinaria e grossa roccia. Silente l’aveva depositata in modo che si appoggiasse sulla più ampia sezione trasversale disponibile, ma ancora traballava instabilmente sulla sua scrivania.

Harry alzò lo sguardo. «Questo è uno scherzo, vero?»

«Non lo è», rispose Silente, scuotendo la testa e sembrando molto serio. «L’ho presa dalle rovine della casa di James e Lily a Godric’s Hollow, dove ho trovato anche te; e l’ho conservata da allora fino a oggi, in attesa del giorno in cui l’avessi potuta dare a te.»

Nel miscuglio di ipotesi che fungevano da modello del mondo di Harry, la pazzia di Silente stava rapidamente crescendo in probabilità. Ma \textit{c’era} ancora una notevole quantità di probabilità assegnata ad altre alternative… «Uhm, è una pietra \textit{magica?}»

«Non per quanto ne so», rispose Silente. «Ma ti consiglio con la massima severità possibile di tenerla vicino sulla tua persona in ogni momento.»

D’accordo. Silente era \textit{probabilmente} folle, ma se non lo \textit{fosse stato…} beh, sarebbe stato troppo \textit{imbarazzante} finire nei guai ignorando il consiglio dell’imperscrutabile vecchio mago. Doveva essere al quarto posto nella lista dei Primi 100 Modi Evidenti di Fallire.

Harry fece un passo avanti e mise le mani sulla roccia, cercando di trovare qualche modo di sollevarla senza tagliarsi. «La metterò nella mia borsa, allora.»

Silente aggrottò la fronte. «Potrebbe non essere abbastanza vicino alla tua persona. E se la borsa mokeskin andasse perduta, o rubata?»

«Pensa che dovrei semplicemente portarmi dietro una grossa roccia ovunque vada?»

Silente osservò Harry con serietà. «Questo potrebbe rivelarsi saggio.»

«Ah…» reagì Harry. Sembrava piuttosto pesante. «Credo che gli altri studenti tenderebbero a farmi delle domande a riguardo».

«Di’ loro che te l’ho ordinato io», rispose Silente. «Nessuno ne dubiterà, dal momento che tutti pensano che io sia pazzo.» Il suo volto era ancora perfettamente serio.

«Ehm, a essere onesti, se va in giro ordinando ai suoi studenti di portarsi dietro grosse rocce, posso capire perché potrebbero pensarlo.»

«Ah, Harry», disse Silente. Il vecchio mago fece un gesto, un movimento circolare con una mano che sembrò coinvolgere tutti gli strumenti misteriosi intorno alla stanza. «Quando siamo giovani crediamo di sapere tutto, e così crediamo che se per qualcosa non vediamo alcuna spiegazione, allora non esista alcuna spiegazione. Quando siamo più vecchi ci rendiamo conto che tutto l’universo funziona secondo un ritmo e una ragione, anche se noi stessi non li conosciamo. È solo la nostra ignoranza che ci appare come follia.»

«La realtà è sempre rispettosa delle leggi», disse Harry, «anche se non conosciamo quelle leggi.»

«Precisamente, Harry», rispose Silente. «Comprendere ciò — e vedo che tu lo \textit{comprendi} — è l’essenza della saggezza.»

«Allora… \textit{perché} devo portare questa roccia, esattamente?»

«Non riesco a pensare ad alcun motivo, in realtà», disse Silente.

«… non riesce.»

Silente annuì. «Ma solo perché non riesco a pensare a un motivo non significa che non vi \textit{sia} alcun motivo.»

Gli strumenti continuarono a ticchettare.

«Va bene», disse Harry, «non sono nemmeno sicuro di doverlo dire, ma questo, semplicemente, non è il modo giusto di gestire la nostra ignoranza riguardo al modo in cui funziona l’universo.»

«Non lo è?» disse il vecchio mago, sembrando sorpreso e deluso.

Harry ebbe la sensazione che quella conversazione non si sarebbe risolta in suo favore, ma continuò ugualmente. «No. Non so nemmeno se questo errore abbia un nome ufficiale, ma se dovessi inventarne uno io stesso, sarebbe ‘privilegiare l’ipotesi’ o qualcosa di simile. Come posso spiegarlo in maniera formale… ehm… supponga di avere un milione di scatole, e che solo una delle scatole contenga un diamante. E che abbia anche un contenitore pieno di rivelatori di diamanti, e che ciascun rivelatore di diamanti reagisca sempre in presenza di un diamante, e reagisca metà delle volte in presenza di scatole che non contengono un diamante. Se usasse venti rilevatori su tutte le scatole, le rimarrebbero ancora, in media, un falso candidato e un vero candidato. E poi le resterebbero da usare ancora solo uno o due rivelatori prima restare con l’unico vero candidato. Il punto è che quando ci sono parecchie risposte possibili, \textit{la maggior parte} delle prove di cui si ha bisogno serve soltanto a \textit{localizzare} l’ipotesi vera tra milioni di possibilità — a portarla alla nostra attenzione, tanto per cominciare. A confronto, la quantità di prove di cui si ha bisogno per scegliere tra due o tre candidati plausibili è molto più piccola. Quindi, se si fa un salto in avanti senza prove e si mette una particolare possibilità al centro della nostra attenzione, si sta saltando la maggior parte del lavoro. È come quando si vive in una città dove ci sono un milione di persone, e c’è un omicidio, e un investigatore dice, bene, non abbiamo alcuna prova, quindi abbiamo considerato la possibilità che sia stato Mortimer Snodgrass.»

«È stato lui?» disse Silente.

«No», rispose Harry. «Ma più tardi si scopre che l’assassino aveva i capelli neri, e Mortimer ha i capelli neri, così tutti dicono, ah, sembra che sia stato Mortimer, tutto sommato. Quindi è ingiusto verso Mortimer che la polizia \textit{lo porti alla propria attenzione} senza avere già in mano buone ragioni per sospettare di lui. Quando ci sono parecchie possibilità, la maggior parte del lavoro serve semplicemente a \textit{localizzare} la vera risposta — a iniziare a prestarle la nostra attenzione. Non abbiamo bisogno di una \textit{prova}, o di quel tipo di prove ufficiali che richiedono gli scienziati o i tribunali, ma abbiamo bisogno di un qualche tipo di \textit{suggerimento}, e quel suggerimento deve essere in grado di discriminare quella particolare possibilità tra milioni di altre. In caso contrario, non possiamo semplicemente tirar fuori la risposta giusta dal nulla. Non è possibile neppure tirar fuori una possibilità che valga la pena di prendere in considerazione, dal nulla. E ci deve essere un milione di altre cose che potrei fare oltre a portare in giro la roccia di mio padre. Solo perché sono ignorante riguardo l’universo non significa che io sia incerto su come dovrei ragionare in presenza della mia incertezza. Le leggi per ragionare con le probabilità non sono meno ferree rispetto alle leggi che governano la vecchia logica, e quello che ha appena fatto \textit{non è permesso}.» Harry fece una pausa. «\textit{A meno che}, naturalmente, lei sia in possesso di un qualche \textit{suggerimento} che non sta menzionando.»

«Ah», disse Silente. Batté sulla guancia, pensieroso. «Un argomento interessante, certo, ma non va forse in frantumi nel punto in cui fai l’analogia tra un milione di potenziali assassini, solo uno dei quali ha commesso l’omicidio, e lo scegliere una delle tante possibili linee d’azione, quando molte di esse possono essere tutte sagge? Non dico che portare con te la roccia di tuo padre sia la singola migliore linea d’azione, solo che è più saggio seguirla che non farlo.»

Silente mise le mani ancora una volta nel cassetto della scrivania che aveva aperto precedentemente, questa volta apparentemente frugandoci dentro – o almeno il suo braccio sembrava muoversi. «Voglio sottolineare», disse Silente, mentre Harry stava ancora cercando di capire come rispondere a questa controreplica del tutto inaspettata, «che si tratta di un malinteso comune riguardo Corvonero che tutti i bambini intelligenti siano Smistati lì, non lasciandone nessuno per le altre Case. Non è così: essere Smistati in Corvonero significa che si è guidati dal proprio desiderio di conoscere le cose, che non è affatto la stessa qualità dell’essere intelligenti.» Il mago era sorridente mentre si chinava sopra il cassetto. «Tuttavia, sembri \textit{realmente} piuttosto intelligente. Meno simile a un normale giovane eroe e più simile a un giovane misterioso mago antico. Penso che forse ho scelto l’approccio sbagliato con te, Harry, e che potresti essere in grado di capire cose che pochi altri potrebbero afferrare. Quindi sarò audace, e ti offrirò un certo \textit{altro} cimelio.»

«Non vorrà dire…» ansimò Harry. «Mio padre… \textit{possedeva un’altra roccia?}»

«Chiedo scusa», disse Silente, «io \textit{sono} ancora più vecchio e più misterioso di te, e se ci fossero delle rivelazioni da fare, allora \textit{io} mi occuperò di farle, grazie… oh, dov’\textit{è} quella cosa!» Silente si infilò ulteriormente nel cassetto della scrivania, e poi ancora di più. La testa e le spalle e tutto il torso scomparvero all’interno, fino a quando solo i fianchi e le gambe rimasero fuori, come se il cassetto della scrivania lo stesse ingoiando.

Harry non poté fare a meno di chiedersi quanta roba ci fosse lì e che aspetto avrebbe avuto l’inventario completo.

Infine Silente uscì nuovamente fuori dal cassetto, reggendo l’obiettivo della sua ricerca, che posò sulla scrivania accanto alla roccia.

Era un libro di testo usato, dai bordi laceri e dal dorso consumato: \textit{Preparazione delle Pozioni Intermedie} di Libatius Borage. C’era una foto di una fiala fumante sulla copertina.

«Questo», intonò Silente, «era il libro di testo di Pozioni del quinto anno di tua madre.»

«Che devo portare con me in ogni momento», disse Harry.

«\textit{Che contiene un terribile segreto}. Un segreto il cui svelamento potrebbe rivelarsi così disastroso che devo chiederti di giurare — e ti chiedo di giurarlo seriamente, Harry, qualunque cosa tu possa pensare di tutto ciò — di non svelarlo mai a chicchessia o a qualunque cosa.»

Harry considerò il manuale di Pozioni del quinto anno di sua madre, il quale, apparentemente, conteneva un terribile segreto.

Il problema era che Harry \textit{prendeva} giuramenti simili molto seriamente. Ogni voto era un Voto Infrangibile se pronunciato dal giusto tipo di persona.

E…

«Ho sete», disse Harry, «e questo non è affatto un buon segno.»

Silente omise completamente qualunque domanda a proposito di questa dichiarazione criptica. «\textit{Giuri}, Harry?» I suoi occhi guardavano intensamente quelli di Harry. «Altrimenti non posso confidartelo.»

«Sì», disse Harry. «Lo giuro.» Era quello il problema di essere un Corvonero. Non potevi rifiutare un’offerta del genere o la tua curiosità ti avrebbe mangiato vivo, e tutti gli altri lo sapevano.

«E io giuro a mia volta», disse Silente, «che quella che sto per dirti è la verità.»

Silente aprì il libro, apparentemente a caso, e Harry si sporse per vedere.

«Vedi queste note», disse Silente a voce così bassa che era quasi un sussurro, «scritte sui margini del libro?»

Harry strizzò gli occhi leggermente. Le pagine ingiallite sembravano descrivere una cosa chiamata \textit{pozione dello splendore dell’aquila}, con molti ingredienti che Harry non riconobbe affatto e il cui nome non sembrava derivare dall’inglese. Scarabocchiata sul margine c’era un’annotazione scritta a mano che diceva, \textit{mi chiedo cosa succederebbe se qui si usasse il sangue di Thestral invece dei mirtilli?} e subito sotto c’era una risposta con una diversa calligrafia, \textit{Ti ammaleresti per settimane e forse moriresti.}

«Le vedo», disse Harry. «Cosa sono?»

Silente indicò il secondo scarabocchio. «Quelle in questa calligrafia», disse, ancora in quella a bassa voce, «furono scritte da tua madre. E quelle in questa calligrafia», spostando il dito per indicare il primo scarabocchio, «sono state scritte da me. Mi rendevo invisibile per intrufolarmi nella sua stanza del dormitorio mentre dormiva. Lily pensava che uno dei suoi amici le stesse scrivendo e avevano i litigi più incredibili.»

Fu quello il momento esatto in cui Harry comprese che il Preside di Hogwarts \textit{era}, infatti, folle.

Silente lo guardava con un’espressione seria. «Hai capito le implicazioni di ciò che ho appena detto, Harry?»

«Ehhh…» rispose Harry. La sua voce sembrava di essersi bloccata. «Scusi… io… non proprio…»

«Ah, bene», disse Silente, e sospirò. «Suppongo che la tua intelligenza abbia dei limiti, dopo tutto. Faremo finta che io non abbia detto niente?»

Harry si alzò dalla sedia, indossando un sorriso fisso. «Certo», disse Harry. «Sa, in realtà si sta facendo piuttosto tardi e io ho un po’ di fame, e dovrei andare a cena, in effetti», e Harry si mosse dritto verso la porta.

La maniglia non si mosse affatto.

«Mi ferisci, Harry», disse la voce di Silente in toni tranquilli che stavano arrivando da dietro di lui. «Non ti rendi conto almeno che avertelo rivelato è un segno di fiducia?»

Harry si voltò lentamente.

Di fronte a lui c’era un mago molto potente e molto folle con una lunga barba d’argento, un cappello simile a un fungo gigante schiacciato, e con indosso quelli che a occhi babbani sembravano tre strati di pigiama rosa brillante.

Dietro di lui c’era una porta che in quel momento non sembrava funzionare.

Silente appariva piuttosto rattristato e stanco, come se volesse appoggiarsi a un bastone da mago che non aveva. «Sinceramente», disse Silente, «tento qualcosa di nuovo invece di seguire lo stesso schema tutte le volte per cento e dieci anni, e la tutta gente inizia a scappare.» Il vecchio mago scosse la testa addolorato. «Mi aspettavo di meglio da te, Harry Potter. Avevo sentito che anche i tuoi amici pensano che tu sia folle. So che si sbagliano. Non vuoi credere lo stesso di me?»

«La prego di aprire la porta», disse Harry, la voce tremante. «Se vuole che mi fidi nuovamente di lei, apra la porta.»

Dietro di lui ci fu il rumore di una porta che si apriva.

«C’erano molte altre cose pensavo di dirti», disse Silente, «e se te ne vai ora, non saprai quali fossero.»

A volte Harry \textit{odiava} assolutamente essere un Corvonero.

\textit{Non ha mai fatto male a uno studente}, disse il lato Grifondoro di Harry. \textit{Basta tenerlo a mente e sarai sicuro di non farti prendere dal panico. Non hai intenzione di scappare solo perché le cose si stanno facendo interessanti, vero?}

\textit{Non puoi semplicemente prendere e andartene dall’ufficio del Preside!} disse la sua parte Tassofrasso. \textit{Che cosa succederebbe se iniziasse a sottrarti punti-Casa? Potrebbe rendere la tua vita scolastica molto difficile se decidesse che non gli piaci!}

E una parte di sé stesso che a Harry non piaceva molto, ma che non riusciva a zittire completamente, stava meditando i potenziali vantaggi di essere uno dei pochi amici di questo folle mago antico, che si dava il caso fosse anche Preside, Capo Stregone, e Supremo Pezzo Grosso. E purtroppo il suo Serpeverde interiore sembrava essere molto migliore di Draco a portare le persone verso il Lato Oscuro, perché diceva cose come \textit{poverino, sembra che abbia bisogno di parlare con qualcuno, non è vero? e non vorresti che un uomo così potente finisse per confidare in qualcuno meno virtuoso, vero? e mi chiedo che tipo di segreti incredibili Silente potrebbe rivelarti se, sai, diventassi suoi amico e anche scommetto che ha una collezione di libri davveeero interessante.}

\textit{Siete tutti un branco di folli}, Harry pensò rivolto all’intero assembramento, ma era stato messo in minoranza all’unanimità da ogni componente di sé stesso.

Harry si voltò, fece un passo verso la porta, allungò la mano, e deliberatamente la richiuse. Fu un sacrificio gratuito, dato che aveva intenzione di restare in ogni caso, Silente poteva comunque controllare i suoi movimenti, ma forse così l’avrebbe impressionato.

Quando Harry si voltò vide che il potente mago folle era ancora una volta sorridente e amichevole. Ciò era un bene, forse.

«La prego, non lo faccia di nuovo», disse Harry. «Non mi piace sentirmi in trappola.»

«Sono \textit{davvero} dispiaciuto per questo, Harry» disse Silente in quello che sembrava un tono di sincere scuse. «Ma sarebbe stato terribilmente imprudente lasciarti andare senza la roccia di tuo padre.»

«Certo», disse Harry. «Non è stato ragionevole aspettarmi che la porta si aprisse prima che mettessi gli oggetti della ricerca nel mio inventario.»

Silente sorrise e annuì.

Harry si avvicinò alla scrivania, girò la borsa mokeskin lungo la cintura fino alla parte anteriore, e, con un certo sforzo, riuscì a sollevare la roccia nelle sue braccia undicenni e a infilarla dentro.

Poté davvero sentire il peso diminuire lentamente, mentre l’incantesimo del Bordo Allargante mangiò la roccia, e il rutto che ne seguì fu piuttosto rumoroso ed ebbe un tono distintamente lamentoso.

Il libro di testo di Pozioni del quinto anno di sua madre (che conteneva un segreto in effetti piuttosto terribile), la seguì poco dopo.

E poi il Serpeverde dentro Harry avanzò un subdolo suggerimento per ingraziarsi il Preside, che, purtroppo, fu formulato in maniera perfetta per ottenere il sostegno del partito di maggioranza Corvonero.

«Bene», disse Harry. «Uhm. Visto che sono da queste parti, non credo che desideri farmi fare un giro del suo ufficio, vero? Sono un po’ curioso di sapere cosa siano alcune di queste cose», e quello fu l’eufemismo del mese per settembre.

Silente lo guardò, e poi annuì con un lieve sorriso. «Sono lusingato dal tuo interesse» disse Silente, «ma temo non ci sia molto da dire.» Silente fece un passo avvicinandosi alla parete e indicò un dipinto di un uomo addormentato. «Questi sono i ritratti dei precedenti Presidi di Hogwarts.» Si voltò e indicò la sua scrivania. «Questa è la mia scrivania.» Indicò la sedia. «Questa è la mia sedia –»

«Mi scusi», disse Harry, «in realtà ero curioso riguardo a quelli.» Harry indicò un piccolo cubo che stava sommessamente sussurrando «blorple… blorple… blorple.»

«Oh, i piccoli strumenti di precisione?» disse Silente. «Sono forniti con l’ufficio del Preside e non ho assolutamente idea di quello che la maggior parte di loro faccia. Anche se \textit{questo} quadrante con le otto lancette conta il numero di, chiamiamoli starnuti, da parte di streghe mancine entro i confini della Francia, e non ci crederesti quanto c’è voluto per capirlo. E questo con le oscigliatrici dorate è una mia invenzione e Minerva non riuscirà mai, mai a capire cosa fa.»

Silente fece un passo verso la cappelliera mentre Harry stava ancora elaborando quanto appena sentito. «Qui, naturalmente, abbiamo il Cappello Smistatore, credo che voi due vi siate conosciuti. Mi ha detto che non dovrà mai più essere messo sul tuo capo, per nessuna ragione. Tu sei solo il quattordicesimo studente nella storia di cui si dica lo stesso, Baba Yaga fu un’altra e ti parlerò degli altri dodici quando sarai più grande. Questo è un ombrello. Questo è un altro ombrello.» Silente fece un altro paio di passi e si voltò, ora con un sorriso piuttosto largo. «E, naturalmente, la maggior parte delle persone che vengono nel mio ufficio vogliono vedere Fawkes.»

Silente era in piedi accanto all’uccello sulla piattaforma dorata.

Harry si avvicinò, piuttosto perplesso. «Questo è Fawkes?»

«Fawkes è una fenice», disse Silente. «Creature magiche molto rare e molto potenti.»

«Ah…» disse Harry. Abbassò la testa e fissò i piccoli occhi neri lucenti, che non mostravano il minimo segno di forza o intelligenza.

«Ahhh…» disse Harry di nuovo.

Era abbastanza sicuro di riconoscere la forma dell’uccello. Era piuttosto difficile da confondere.

«Umm…»

\textit{Di’ qualcosa di intelligente!} Urlò a sé stessa la mente di Harry. \textit{Non startene lì come un idiota balbettante!}

\textit{Beh cosa diavolo dovrei dire?} Rispose la mente di Harry.

\textit{Qualunque cosa!}

\textit{Vuoi dire, qualunque cosa a parte «Fawkes è una gallina» –}

\textit{Sì! Qualunque cosa a parte quella!}

«Quindi, ah, che tipo di magie fanno le fenici?»

«Le loro lacrime hanno il potere di guarire», disse Silente. «Sono creature di fuoco, e si muovono facilmente tra tutti i luoghi così come il fuoco può spegnersi in un posto e accendersi in un altro. L’enorme peso della loro innata magia fa invecchiare i loro corpi in modo rapido, eppure sono più vicine all’immortalità di ogni altra creatura che esista in questo mondo, poiché ogni volta che i loro corpi smettono di funzionare, esse si immolano in una fiammata e si lasciano dietro un pulcino, o a volte un uovo.» Silente si avvicinò e ispezionò il pollo, accigliato. «Uhm… sembra un po’ malaticcio, direi.»

Prima che questa affermazione fosse interamente registrata nella mente di Harry, la gallina era già in fiamme.

Il becco della gallina si aprì, ma non ebbe il tempo di un singolo verso prima di iniziare ad appassire e carbonizzare. La fiammata fu breve, intensa e del tutto limitata; non ci fu odore di bruciato.

E poi il fuoco si spense solo pochi secondi dopo che era scaturito, lasciando dietro di sé un piccolo, patetico mucchio di ceneri sulla piattaforma dorata.

«Non fare quella faccia inorridita, Harry!» disse Silente. «Fawkes non si è fatto male.» La mano di Silente si immerse in una tasca, e poi la stessa mano setacciò tra le ceneri e tirò fuori un piccolo uovo giallastro. «Guarda, qui c’è un uovo!»

«Oh… uau… incredibile…»

«Ma ora dovremmo davvero procedere oltre», disse Silente. Lasciando l’uovo nelle ceneri della gallina, tornò al suo trono e si sedette. «È quasi ora di cena, dopo tutto, e noi non vorremmo dover usare i nostri Giratempo.»

Una violenta lotta per il potere si svolse nel Governo di Harry. Serpeverde e Tassofrasso avevano cambiato schieramento dopo aver visto il preside di Hogwarts dare alle fiamme una gallina.

«Sì, procedere oltre», dissero le labbra di Harry. «E poi la cena.»

\textit{Sembri nuovamente un idiota balbettante} osservò il Critico Interiore di Harry.

«Bene», disse Silente. «Ho paura di avere una confessione da fare, Harry. Una confessione e delle scuse.»

«Le scuse sono una buona cosa» \textit{ma non ha nessun senso! Di che sto parlando?}

Il vecchio mago sospirò profondamente. «Potresti non pensarla ancora così dopo aver compreso ciò che ho da dire. Ho paura, Harry, di aver manipolato tutta la tua vita. Sono stato io a consegnarti alle cure dei tuoi malvagi genitori adottivi –»

«I miei genitori adottivi non sono malvagi!» sbottò Harry. «I miei \textit{genitori}, voglio dire!»

«Non lo sono?» disse Silente, sembrando sorpreso e deluso. «Neanche un po’ cattivi? Questo non rientra nei piani…»

Il Serpeverde dentro Harry urlò mentalmente a squarciagola, \textsl{\textsc{Stai zitto idiota ti porterà via da loro!}}

«No, no», disse Harry, labbra congelate in una smorfia orribile, «stavo solo cercando di proteggere i suoi sentimenti, sono molto malvagi in realtà…»

«Lo sono?» Silente si chinò in avanti, fissandolo intensamente. «Che cosa fanno?»

\textit{Parla veloce} «Loro, ah, devo fare i piatti lavare il bucato e non mi lasciano leggere molti libri e –»

«Ah, bene, è bene sentirlo», disse Silente, reclinandosi nuovamente. Sorrise in modo alquanto triste. «Mi scuso per \textit{questo}, allora. Ora, dove ero rimasto? Ah, sì. Mi dispiace dire, Harry, che io sono responsabile di praticamente tutto il male che ti sia mai capitato. So che questo probabilmente ti renderà molto arrabbiato.»

«Sì, sono molto arrabbiato!» disse Harry. «Grrr!»

Prontamente il Critico Interiore di Harry gli assegnò il Premio per la Peggior Interpretazione di Tutta la Storia.

«E volevo solo che tu sapessi», disse Silente, «volevo dirti appena possibile, nel caso in cui poi succeda qualcosa a uno di noi, che sono veramente, veramente dispiaciuto. Per tutto ciò che è già accaduto, e tutto ciò che avverrà.»

Gli occhi umidi del vecchio mago brillavano.

«Sono molto arrabbiato!» disse Harry. «Così arrabbiato che voglio andarmene ora, se non ha nient’altro da dire!»

\textsl{\textsc{Vattene}} \textit{prima che ti dia fuoco!} strillarono Serpeverde, Tassofrasso, e Grifondoro.

«Capisco», disse Silente. «Un’ultima cosa allora, Harry. Tu \textit{non devi} provare a varcare la porta proibita nel corridoio del terzo piano. Non c’è modo che tu possa superare tutte le trappole, e non vorrei sentire che sei stato ferito provandoci. Chissà, dubito che saresti in grado di aprire anche solo la prima porta, dal momento che è bloccata e tu non conosci l’incantesimo \textit{Alohomora} –»

Harry si voltò di scatto e si precipitò verso l’uscita alla massima velocità, la maniglia girò piacevolmente nella sua mano e poi stava correndo giù per le scale a chiocciola, anche mentre ruotavano, i piedi che quasi inciampavano su sé stessi, in un momento appena fu sul fondo e il gargoyle si spostò da parte e Harry uscì sparato fuori dalla tromba delle scale come una palla di cannone.

\begin{figure}[h!]
        \includegraphics[scale=0.4]{boccino.png}
        \centering
\end{figure}

Harry Potter.

Doveva esserci qualcosa di strano a proposito di Harry Potter.

Era giovedì per tutti, in fin dei conti, e tuttavia questo genere di cose non sembravano accadere a nessun altro.

Erano le 18:21 di giovedì pomeriggio quando Harry Potter, uscendo sparato come una palla di cannone dalla tromba delle scale e accelerando al massimo della velocità, corse dritto contro Minerva McGonagall, che stava svoltando un angolo nel suo percorso verso l’ufficio del Preside.

Per fortuna nessuno si fece troppo male. Come era stato spiegato a Harry un po’ prima nel corso di quella stessa giornata — quando si stava rifiutando di avvicinarsi di nuovo minimamente a un manico di scopa — il Quidditch aveva bisogno di Bolidi di ferro pieno per avere una probabilità appena decente di ferire i giocatori, dal momento che i maghi tendevano a essere molto più resistenti agli urti dei Babbani.

Harry e la professoressa McGonagall finirono entrambi sul pavimento, e le pergamene che ella stava portando si sparsero per il corridoio.

Ci fu una terribile, terribile pausa.

«Harry Potter», prese fiato la professoressa McGonagall da là dove giaceva sul pavimento, proprio a fianco a Harry. La sua voce salì fino a divenire quasi un urlo. «\textit{Cosa stava facendo nell’ufficio del Preside?}»

«Nulla!» squittì Harry.

«\textit{Stavate parlando del Professore di Difesa?}»

«No! Silente mi ha convocato, e mi ha dato questa grossa roccia, e mi detto che era di mio padre e che dovrei portarmela sempre dietro!»

Ci fu un’altra terribile pausa.

«Capisco», disse la professoressa McGonagall, la sua voce un po’ più calma. Si alzò, si pulì, e fissò le pergamene sparse, che saltarono in una pila ordinata e corsero contro la parete del corridoio, come per nascondersi al suo sguardo. «Le mie simpatie, signor Potter, e mi scuso per aver dubitato di lei.»

«Professoressa McGonagall», Harry disse. La sua voce tremava. Si sollevò dal pavimento, si mise in piedi e guardò in su verso quel volto affidabile e da persona \textit{sana di mente}. «Professoressa McGonagall…»

«Sì, signor Potter?»

«Pensa che dovrei farlo?» Harry chiese a bassa voce. «Portare la roccia di mio padre ovunque con me?»

La professoressa McGonagall sospirò. «Questa è una faccenda tra lei e il Preside, temo.» Esitò. «Direi che ignorare completamente il Preside non è quasi mai saggio. Sono \textit{davvero} dispiaciuta per il suo dilemma, signor Potter, e se ci fosse un modo in cui \textit{possa} aiutarla con qualunque cosa lei decida di fare –»

«Uhm», Harry disse. «In realtà stavo pensando che una volta imparato come si fa, potrei Trasfigurare la roccia in un anello e infilarlo al dito. Se potesse insegnarmi come sostenere una Trasfigurazione –»

«È un bene che prima abbia chiesto a me», disse la professoressa McGonagall, il suo volto che stava diventando severo. «Se perdesse il controllo della Trasfigurazione, l’inversione le taglierebbe il dito e probabilmente le squarcerebbe la mano a metà. E alla sua età, anche un anello è un oggetto troppo grande per essere sostenuto indeterminatamente senza essere un severo drenaggio della sua magia. Ma posso farle forgiare un anello con un castone per un gioiello, un piccolo gioiello, a contatto con la sua pelle, e lei può far pratica sostenendo un oggetto sicuro, come una caramella morbida. Quando l’avrà sostenuto con successo, anche durante il sonno, per un mese intero, le permetterò di Trasfigurare, ah, la roccia di suo padre…» la voce della professoressa McGonagall si spense. «Il Preside le ha \textit{davvero} –»

«Sì. Ah… uhm…»

La professoressa McGonagall sospirò. «È un po’ strano anche per lui.» Si chinò e raccolse la pila di pergamene. «Sono dispiaciuta per tutto ciò, signor Potter. Mi scuso di nuovo per aver diffidato di lei. Ma ora è il mio turno di vedere il Preside.»

«Ah… buona fortuna, credo. Ehm…»

«Grazie, signor Potter.»

«Uhm…»

La professoressa McGonagall si avvicinò al gargoyle, pronunciò impercettibilmente la parola d’accesso, ed entrò nelle scale a chiocciola ruotanti. Aveva cominciato a uscire fuori dal campo visivo, e il gargoyle aveva iniziato a tornare al proprio posto, quando –

«\textit{Professoressa McGonagall, il Preside ha dato fuoco a una gallina!}»

«Che \textit{cos-}»



