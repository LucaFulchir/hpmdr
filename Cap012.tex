% !TeX root = Harry.tex

\chapter{Controllo degli impulsi}
\label{capitolo:12}

\emph{«Mi chiedo cosa ci sia di sbagliato in lui.»}

~\\
~\\

«Turpin, Lisa!»

Bisbiglio bisbiglio bisbiglio harry potter bisbiglio bisbiglio serpeverde bisbiglio bisbiglio no davvero che diamine bisbiglio bisbiglio

«corvonero!»

Harry si unì all’applauso che accolse la giovane ragazza, la quale si stava incamminando timidamente verso la tavola Corvonero, i bordi delle sue vesti ora mutati in blu scuro. Lisa Turpin apparve combattuta tra l’impulso di sedersi il più lontano possibile da Harry Potter e quello di correre verso di lui, infilarsi a forza al suo fianco e iniziare a strappargli delle risposte.

Trovarsi al centro di un evento straordinario e curioso, e poi essere Smistati nella Casa Corvonero, era parente stretto dell’essere intinti nella salsa barbecue e gettati nella fossa dei gattini affamati.

«Ho promesso al Cappello Smistatore di non parlarne», bisbigliò Harry per l’ennesima volta.

«Sì, sul serio.»

«No, ho davvero promesso al Cappello Smistatore di non parlarne.»

«Va bene, ho promesso al Cappello Smistatore di non riferirne gran parte e il resto è privato proprio come nel vostro caso quindi smettetela di chiedere.»

«Volete sapere cos’è successo? Va bene! Ecco una parte di ciò che è successo! Ho detto al Cappello che la professoressa McGonagall aveva minacciato di bruciarlo e lui mi ha detto di dire alla professoressa McGonagall che è una giovane impudente e che deve lasciarlo in pace!»

«Se non avete intenzione di credere a ciò che vi dico allora perché mi fate delle domande?»

«No, neppure io so come ho sconfitto il Signore Oscuro! Ditelo voi a me se lo capite!»

«Silenzio!», gridò la professoressa McGonagall dal podio del Tavolo d’onore. «È proibito parlare fino alla fine della Cerimonia di Smistamento!»

Ci fu un beve abbassamento del volume, quando tutti attesero di vedere se avrebbe proferito una qualche minaccia specifica e credibile, e poi i bisbigli ricominciarono.

Allora l’antico uomo dalla barba d’argento si alzò dalla sua grande sedia dorata, sorridendo allegramente. Silenzio immediato. Qualcuno diede freneticamente di gomito a Harry mentre tentava di continuare a bisbigliare, e Harry si interruppe a metà frase.

Il vecchio dall’aspetto allegro si sedette nuovamente.

Nota a sé stesso: non si scherza con Silente.

Harry stava ancora tentando di esaminare tutto ciò che era avvenuto durante l’Incidente col Cappello Smistatore. E non meno importante era stato ciò che era accaduto nell’istante in cui Harry aveva sollevato il Cappello dalla testa; in quel momento, aveva udito un sommesso bisbiglio proveniente dal nulla, qualcosa che suonava stranamente come inglese e un sibilo allo stesso tempo, qualcosa che aveva detto «Ssaluti da Sserpeverde a Sserpeverde: sse vuoi cercare miei ssegreti, parla con mio sserpente.»

Harry aveva indovinato che non fosse una parte ufficiale del processo di Smistamento. E che fosse un po’ di magia extra inserita da Salazar Serpeverde durante la creazione del Cappello. E che il Cappello stesso non ne fosse a conoscenza. E che entrasse in azione appena il Cappello avesse detto «serpeverde», più o meno altre condizioni. E che un Corvonero come lui non avrebbe dovuto assolutamente, assolutamente udirlo. E che se avesse trovato il modo di obbligare Draco a mantenere il segreto così da poterglielo chiedere, quello sarebbe stato il momento giusto per avere un po’ di SpiriTè sotto mano.

Ragazzi, decidi di non intraprendere la strada del Signore Oscuro e l’universo inizia a crearti problemi dall’istante in cui il Cappello ti viene tolto dalla testa. In alcuni giorni è proprio inutile combattere il destino. Forse attenderò fino a domani per iniziare a seguire il mio proposito di non essere un Signore Oscuro.

«grifondoro!»

Ron Weasley ricevette molti applausi, e non solo dai Grifondoro. Evidentemente la famiglia Weasley era molto popolare da quelle parti. Harry, dopo un attimo, sorrise e iniziò ad applaudire insieme agli altri.

Ma, del resto, non c’era momento migliore del presente per abbandonare il Lato Oscuro.

Al diavolo il destino e al diavolo l’universo. Gliel’avrebbe fatta vedere al Cappello.

«Zabini, Blaise!»

Pausa.

«serpeverde!» gridò il Cappello.

Harry applaudì anche Zabini, ignorando gli strani sguardi che stava ricevendo da tutti, Zabini incluso.

Non fu chiamato nessun altro nome, e Harry si rese conto che «Zabini, Blaise» sembrava davvero prossimo alla fine dell’alfabeto. Grande, così ora aveva applaudito solo Zabini… Oh, vabbé.

Silente si alzò di nuovo e cominciò a dirigersi verso il podio. A quanto pareva erano in procinto di ricevere l’onore di un discorso –

E Harry fu colpito dall’ispirazione per una brillante prova sperimentale.

Hermione aveva detto che Silente era il mago vivente più potente, giusto?

Harry infilò una mano nella borsa e sussurrò, «SpiriTè.»

Se lo SpiriTè avesse funzionato, avrebbe dovuto far dire a Silente qualcosa di così ridicolo durante il suo intervento che, anche nello stato di preparazione mentale in cui era Harry, gliel’avrebbe fatto comunque andare di traverso. Qualcosa come l’obbligo per tutti gli studenti di Hogwarts di non indossare abiti per l’intero anno scolastico, o che tutti sarebbero stati trasformati in gatti.

Ma del resto se una sola persona al mondo poteva resistere alla potenza dello SpiriTè, quella doveva essere Silente. Quindi, se avesse funzionato, lo SpiriTè sarebbe stato letteralmente invincibile.

Volendo agire in maniera discreta, Harry tolse la linguetta dello SpiriTè sotto il tavolo. La lattina fece un sommesso rumore sibilante. Qualche testa si voltò a guardarlo, ma presto tornò indietro quando –

«Benvenuti! Benvenuti a un nuovo anno a Hogwarts!» disse Silente raggiante agli studenti con le braccia spalancate, come se niente potesse fargli più piacere che vederli tutti lì.

Harry bevette un primo sorso di SpiriTè e abbassò la lattina di nuovo. Avrebbe deglutito la bevanda un po’ alla volta, cercando di non farsela andare di traverso qualunque cosa Silente avesse detto –

«Prima di iniziare il nostro banchetto, vorrei dire poche parole. Eccole: Felice felice bum bum palude palude palude! Grazie!»

Tutti applaudirono e acclamarono, e Silente si sedette di nuovo.

Harry rimase seduto, paralizzato, mentre la bevanda gli colava dagli angoli della bocca. Era, quantomeno, riuscito a soffocare in silenzio.

Davvero, davvero, davvero non avrebbe dovuto farlo. Incredibile quanto fosse molto più evidente appena un secondo dopo che era diventato troppo tardi.

Col senno di poi, avrebbe probabilmente dovuto accorgersi che c’era qualcosa di sbagliato, quando aveva pensato alla possibilità che tutti fossero trasformati in gatti… o anche prima, si ricordò della sua nota mentale di non ficcarsi nei guai con Silente… o il suo nuovo proposito di essere più rispettoso degli altri… o forse se avesse avuto solo un briciolo di buon senso…

Era senza speranza. Era marcio fino al midollo. Salutate il Signore Oscuro Harry. Non si poteva lottare contro il destino.

Qualcuno stava chiedendo se Harry stesse bene. (Altri avevano iniziato a servirsi da soli il cibo, che era magicamente apparso sul tavolo, evabbè.)

«Sto bene», disse Harry. «Scusatemi. Uhm. Quello era un discorso… normale per il Preside? Tutti voi… non sembravate… molto sorpresi…»

«Oh, Silente è folle, naturalmente», disse un Corvonero apparentemente più grande seduto accanto a lui, che si era presentato con un nome che Harry non riusciva nemmeno lontanamente a ricordare. «Molto divertente, mago incredibilmente potente, ma completamente fuori di testa.» Fece una pausa. «Vorrei anche chiederti come mai del fluido verde è fuoriuscito delle tue labbra ed è poi scomparso, anche se mi aspetto che tu abbia promesso al Cappello Smistatore di non parlare neppure di questo.»

Con un grande sforzo, Harry si impedì di guardare in basso verso l’incriminante lattina di SpiriTè nella sua mano.

Dopo tutto, lo SpiriTè non aveva semplicemente materializzato arbitrariamente un titolo de Il Cavillo su di lui e Draco. Draco l’aveva spiegato in un modo che aveva fatto sembrare che tutto fosse accaduto… naturalmente? Come se avesse alterato la storia per inserircisi dentro?

Harry stava mentalmente immaginando di colpire il tavolo con la fronte. Bam, bam, bam, faceva la sua testa nella sua immaginazione.

Un’altra studentessa abbassò la voce a un sussurro. «Ho sentito dire che Silente è segretamente il geniale orchestratore di un sacco di cose e che usa la follia come copertura in modo che nessuno sospetti di lui.»

«L’ho sentito anch’io», sussurrò un terzo studente, e vi furono cenni furtivi da tutto il tavolo.

Quello non poté non attirare l’attenzione di Harry.

«Capisco», mormorò Harry, abbassando la propria voce. «Così tutti sanno che Silente è segretamente un manipolatore.»

La maggior parte degli studenti annuì. Uno o due sembrarono improvvisamente pensierosi, compreso lo studente più anziano seduto accanto a Harry.

Siete sicuri che questo sia il tavolo di Corvonero? Harry riuscì a non chiederlo ad alta voce.

«Brillante!» sussurrò Harry. «Se tutti sanno, nessuno sospetterà che è un segreto!»

«Esattamente», sussurrò uno studente, e poi aggrottò la fronte. «Aspetta, non mi sembra del tutto giusto –»

Nota a sé stesso: il $75^o$ percentile degli studenti di Hogwarts, anche noto come Casa Corvonero, non è il programma più esclusivo del mondo per bambini prodigio.

Ma almeno aveva imparato un fatto importante oggi. Lo SpiriTè era onnipotente. E quello voleva dire che…

Harry sbatté le palpebre per la sorpresa quando la sua mente ebbe finalmente realizzato l’ovvia connessione.

… quello voleva dire che, non appena avesse imparato un incantesimo in grado di modificare temporaneamente il suo senso dell’umorismo, poteva far accadere qualsiasi cosa, facendo in modo che quella cosa fosse l’unica che avrebbe trovato abbastanza sorprendente da strozzarsi, e poi bere una lattina di SpiriTè.

Che dire, è stato un viaggio breve verso la divinità. Anch’io credevo che ci volesse più tempo che il mio primo giorno di scuola.

Ripensandoci, aveva anche completamente rovinato Hogwarts nel giro di dieci minuti netti da quando era stato Smistato.

Harry provava davvero un certo rammarico per questo — solo Merlino sapeva che cosa un Preside folle avrebbe fatto dei suoi successivi sette anni di scuola — ma non poteva fare a meno di provare una punta di orgoglio, anche.

Domani. Entro domani, al più tardi, avrebbe smesso di percorrere la strada che portava al Signore Oscuro Harry. Una prospettiva che suonava sempre più spaventosa di minuto in minuto.

Eppure persino, in qualche modo, sempre più attraente. Parte della sua mente stava già immaginando le uniformi dei suoi servitori.

«Mangia», ringhiò lo studente più anziano seduto accanto a lui, e gli diede una gomitata nelle costole. «Non pensare. Mangia.»

Harry iniziò automaticamente a caricare il proprio piatto con tutto ciò che era di fronte a lui, salsicce blu con minuscoli frammenti brillanti, evabbè.

«A cosa cosa stavi pensando, lo Smistamento –» cominciò a dire Padma Patil, una delle altre Corvonero del primo anno.

«Nessuna domanda importuna durante i pasti!» dissero in coro almeno tre persone. «Regola della Casa!» aggiunse un altro. «Altrimenti moriremmo tutti di fame da queste parti.»

Harry si stava scoprendo molto, molto speranzoso che la sua nuova geniale idea non funzionasse realmente. E che lo SpiriTè operasse in qualche altro modo e che non avesse realmente la capacità onnipotente di alterare la realtà. Non che non volesse essere onnipotente. Era solo che non riusciva a sopportare l’idea di vivere in un universo che funzionasse davvero in quel modo. C’era qualcosa di poco dignitoso nel salire al potere attraverso l’uso intelligente di bevande gassate.

Ma doveva verificarlo sperimentalmente.

«Sai», disse lo studente più anziano accanto a lui in tono abbastanza cordiale, «abbiamo un sistema per costringere le persone come te a mangiare, ti piacerebbe scoprire di cosa si tratta?»

Harry si arrese e cominciò a mangiare la sua salsiccia blu. Era abbastanza buona, soprattutto i pezzetti brillanti.

La cena passò con sorprendente rapidità. Harry cercò di assaggiare almeno un po’ di tutti gli strani e nuovi cibi che vide. La sua curiosità non poteva sopportare l’idea di non sapere quale sapore avesse un certo cibo. Grazie al cielo quello non era un ristorante dove si doveva ordinare solo una cosa e non avresti mai scoperto il sapore di tutte le altre pietanze presenti sul menu. Harry odiava quel fatto, era come una stanza delle torture per chiunque avesse una scintilla di curiosità: scoprite solo uno dei misteri di questa lista, ah ah ah!

Poi giunse il momento del dolce, per il quale Harry si era completamente dimenticato di lasciare spazio. Rinunciò dopo aver assaggiato un piccolo boccone di crostata di melassa. Sicuramente tutte quelle pietanze sarebbero state fatte passare di nuovo almeno una volta nel corso dell’anno scolastico.

Dunque, cosa c’era sulla sua lista delle cose da fare, oltre ai soliti impegni scolastici?

Da-fare 1: fai una ricerca sugli incantesimi di alterazione della mente così da poter esaminare lo SpiriTè e vedere se hai scovato una via verso l’onnipotenza. Anzi, fai una ricerca su ogni tipo di magia mentale che riesci a trovare. La mente è il fondamento del nostro potere come esseri umani, qualsiasi tipo di magia la influenzi è il tipo più importante di magia che ci sia.

Da-fare 2: in realtà questo è Da-fare 1 e l’altro è Da-fare 2. Passa in rassegna gli scaffali delle biblioteche di Hogwarts e Corvonero, familiarizza con il sistema e fai in modo di aver letto almeno tutti i titoli dei libri. Secondo passaggio: leggi tutti i sommari. Coordinati con Hermione che ha una memoria molto migliore della tua. Scopri se c’è un sistema di prestito inter-bibliotecario a Hogwarts e vedi se voi due, soprattutto Hermione, potete consultare anche quelle biblioteche. Se le altre Case hanno biblioteche private, scopri come accedervi legalmente o intrufolarti.

Opzione 3a: obbliga Hermione al segreto e prova a fare una ricerca riguardo ‘Da Serpeverde a Serpeverde: se vuoi cercare i miei segreti, parla col mio serpente’. Problema: Questo sembra molto riservato e potrebbe passare un po’ di tempo prima di trovare casualmente un libro che contenga un indizio.

Da-fare 0: controlla che tipo di incantesimi di ricerca-e-recupero-informazioni esistono, se esistono. La magia da biblioteca non ha la stessa fondamentale importanza della magia mentale ma ha una priorità molto più alta.

Opzione 3b: cerca un incantesimo per obbligare magicamente Draco Malfoy al segreto, o per verificare magicamente la sincerità della promessa di Draco di mantenere un segreto (Veritaserum?), e poi chiedi a lui del messaggio di Serpeverde…

In realtà… Harry aveva un presentimento piuttosto brutto riguardo l’opzione 3b.

Ora che Harry ci pensava, non si sentiva poi a suo agio neppure con l’opzione 3a.

I pensieri di Harry ritornarono a ciò che era stato forse il peggior momento della sua vita fino ad allora, quei lunghi secondi di orrore da raggelare il sangue sotto il Cappello, quando pensava di aver già fallito. In quel momento aveva desiderato di tornare indietro nel tempo di appena pochi minuti e cambiare qualcosa, qualsiasi cosa prima che fosse troppo tardi…

E poi si era scoperto che non era troppo tardi, dopo tutto.

Desiderio esaudito.

Non avevi il potere di cambiare la storia. Ma potevi indirizzarla nel modo giusto sin dal principio. Fare qualcosa in maniera differente sin dal primo tentativo.

Tutta quella faccenda della ricerca dei segreti di Serpeverde… sembrava tremendamente simile a quel genere di situazioni in cui, anni dopo, ti saresti guardato indietro e detto, ‘E lì è stato quando tutto ha iniziato ad andare a rotoli.

E avrebbe disperatamente desiderato di essere capace di tornare indietro nel tempo e compiere una scelta differente…

Desiderio esaudito. E adesso?

Harry sorrise lentamente.

Era un pensiero piuttosto contro-intuitivo… ma…

Ma poteva, non c’era nessuna regola che dicesse che non poteva, poteva semplicemente far finta di non aver mai sentito quel piccolo sussurro. Lasciare che l’universo andasse avanti esattamente nello stesso modo in cui avrebbe fatto se quel momento critico non fosse mai accaduto. Venti anni più tardi, questo sarebbe stato quello che avrebbe disperatamente desiderato fosse accaduto venti anni prima, e si dava il caso che vent’anni prima di vent’anni dopo fosse proprio quel momento. Alterare il passato remoto era facile, bastava solo pensarci al momento giusto.

Oppure… questo era ancora più contro-intuitivo… poteva anche informare, oh, diciamo, la professoressa McGonagall, invece che Draco o Hermione. E lei avrebbe potuto riunire le persone giuste e far sì che quel piccolo incantesimo aggiuntivo fosse tolto dal Cappello.

Eh, già. Sembrava un’idea eccezionalmente buona una volta che gli era realmente venuta in mente.

Così tanto ovvia, a posteriori, eppure in qualche modo, l’opzione 3c e l’opzione 3d non gli erano venute in mente.

Harry assegnò a sé stesso «più un punto» per il suo programma anti-Signore-Oscuro-Harry.

Era stato uno scherzo terribilmente crudele che il Cappello gli aveva giocato, ma non ci si poteva lamentare dei risultati in un’ottica consequenzialista. Certamente gli aveva dato un’idea migliore del punto di vista della vittima, però.

Da-fare 4: scusati con Neville Longbottom.

Ok, aveva una striscia vincente in corso, ora doveva solo continuarla. Ogni giorno, in ogni modo, sto diventando sempre più Luminoso…

Anche le persone attorno a Harry avevano per lo più smesso di mangiare, a quel punto, e i vassoi coi dolci cominciarono a svanire, come pure i piatti sporchi.

Quando tutti i piatti furono spariti, Silente ancora una volta si alzò dal suo posto.

Harry non poté impedirsi di sentire l’urgenza di bere altro SpiriTè.

Ma tu stai proprio scherzando, Harry pensò rivolto a quella parte di sé stesso.

Ma l’esperimento non avrebbe avuto valore se non fosse stato replicato, no? E il danno era stato già fatto, giusto? Non voleva vedere cosa sarebbe accaduto questa volta? Non era curioso? E se avesse ottenuto un risultato differente?

Ehi, scommetto che sei la stessa parte del mio cervello che ha spinto per fare lo scherzo a Neville Longbottom.

Ehm, forse?

E non è ovvio in maniera schiacciante che se lo faccio me ne pentirò un secondo dopo che sarà troppo tardi?

Uhm…

Già. Allora, no.

«Ehm», disse Silente dal podio, accarezzandosi la lunga barba d’argento. «Solo qualche altra parola, ora che siamo tutti nutriti e dissetati. Ho un paio di avvisi di inizio anno scolastico da darvi.»

«Gli studenti del primo anno dovrebbero prendere nota che la foresta circostante è vietata a tutti gli alunni. Questo è il motivo per cui è chiamata la Foresta Proibita. Se fosse consentita sarebbe chiamata la Foresta Consentita.»

Inequivocabile. Nota a sé stesso: la Foresta Proibita è proibita.

«Mi è stato anche chiesto dal signor Filch, il custode, di ricordare a tutti voi che nessuna magia dovrebbe essere utilizzata nei corridoi nel tempo che intercorre tra due lezioni. Ahimè, sappiamo tutti che quello che dovrebbe essere, e ciò che è, sono due cose diverse. Grazie di tenerlo a mente.»

Ehm…

«I provini di Quidditch si terranno nella seconda settimana dell’anno scolastico. Chiunque sia interessato a giocare per la squadra della propria Casa deve contattare Madam Hooch. Chiunque sia interessato a riformulare l’intero gioco del Quidditch deve contattare Harry Potter.»

Harry inspirò la propria saliva e fu colpito da un accesso di tosse proprio mentre tutti gli occhi si girarono verso di lui. Che diamine! Non aveva mai incrociato lo sguardo di Silente… non ci aveva neppure pensato. Di certo non aveva pensato al Quidditch in quel momento! Non aveva parlato con nessuno se non con Ron Weasley e non credeva che Ron l’avrebbe riferito a qualcun altro… o Ron era corso via da un professore per lamentarsi? Come diavolo…

«Inoltre, devo comunicarvi che quest’anno il corridoio del terzo piano sul lato destro è interdetto a tutti coloro che non vogliano morire di una morte molto dolorosa. È protetto da una serie complicata di trappole pericolose e potenzialmente letali, e non potete assolutamente superarle tutte, specialmente se siete solo al vostro primo anno.»

A quel punto Harry era ormai insensibile.

«E, infine, esprimo il mio più profondo ringraziamento a Quirinus Quirrell per aver eroicamente accettato di ricoprire la posizione di Professore di Difesa Contro le Arti Oscure a Hogwarts.» Lo sguardo di Silente si mosse indagatore tra tutti gli studenti. «Mi auguro che tutti gli studenti esprimano al professor Quirrell quella massima cortesia e tolleranza che sono dovute al suo straordinario servizio a voi e a questa scuola, e che non ci assillerete con alcuna fastidiosa lamentela su di lui, a meno che voi non vogliate provare a fare il suo lavoro.»

Di che sta parlando adesso?

«Cedo ora il podio al nostro nuovo membro di facoltà, il professor Quirrell, che vorrebbe dirvi alcune parole.»

Il giovane magro e nervoso che Harry aveva conosciuto al Paiolo Magico si fece lentamente strada fino al podio, lanciando occhiate timorose in ogni direzione. Harry intravide la parte posteriore della sua testa, e sembrava che il professor Quirrell stesse già diventando calvo, nonostante l’apparente giovane età.

«Mi chiedo cosa ci sia di sbagliato in lui», sussurrò lo studente apparentemente più grande seduto accanto a Harry. Simili osservazioni sommesse furono scambiate in altri punti della tavolata.

Il professor Quirrell si fece strada fin sul podio e rimase lì, sbattendo le palpebre. «Ah…», fece. «Ah…» Poi il coraggio sembrò mancargli completamente, e se ne stette lì, in silenzio, occasionalmente colpito da spasmi.

«Oh, eccellente», sussurrò lo studente più grande, «si prevede un altro lungo anno nel corso di Difesa –»

«Salve, miei giovani discenti», disse il professor Quirrell in un tono asciutto e sicuro. «Sappiamo tutti che Hogwarts tende a soffrire di una certa sfortuna nelle sue scelte per questa posizione, e senza dubbio molti di voi si staranno già chiedendo quale sventura si abbatterà su di me quest’anno. Ve lo garantisco, quella sventura non sarà la mia incompetenza.» Sorrise appena. «Che ci crediate o no, ho sempre desiderato cimentarmi un giorno come Professore di Difesa Contro le Arti Oscure qui alla Scuola di Magia e Stregoneria di Hogwarts. Il primo a tenere questo corso fu Salazar Serpeverde in persona, e ancora nel xiv secolo era tradizione che i più grandi maghi combattenti di ogni credo si cimentassero a tenere questo corso. I passati Professori di Difesa includono non solo il leggendario eroe errante Harold Shea, ma anche la virgolette immortale chiuse virgolette Baba Yaga, sì, vedo che alcuni di voi ancora rabbrividiscono al suono del suo nome anche se è morta da 600 anni. Deve essere stato un periodo interessante per frequentare Hogwarts, non credete?»

Harry deglutì con difficoltà, cercando di sopprimere l’improvvisa ondata di emozione che lo aveva sopraffatto quando il professor Quirrell aveva cominciato a parlare. I toni precisi gli ricordarono molto un docente di Oxford, e Harry stava cominciando a comprendere realmente che non avrebbe visto la sua casa o la sua mamma o il suo papà fino a Natale.

«Siete abituati al fatto che la cattedra di Difesa sia ricoperta da incompetenti, farabutti, e sventurati. Per chiunque abbia un minimo di conoscenza della storia, essa porta con sé tutt’altra reputazione. Non tutti coloro che hanno insegnato qui sono stati i migliori, ma i migliori hanno tutti insegnato a Hogwarts. In tale prestigiosa compagnia, e dopo aver atteso per tanto tempo questo giorno, mi vergognerei a pormi qualsiasi obiettivo inferiore alla perfezione. E così intendo realmente che ciascuno di voi ricordi per sempre quest’anno come il miglior corso di Difesa che abbiate mai avuto. Ciò che imparerete quest’anno vi sarà per sempre utile come solida base nelle arti della Difesa, non importa chi saranno stati i vostri insegnanti prima e dopo.»

L’espressione del professor Quirrell si fece seria. «Abbiamo una grande quantità di terreno perso da recuperare e non molto tempo per coprirlo. Perciò intendo discostarmi dalle convenzioni didattiche di Hogwarts in un certo numero di aspetti, come pure introdurre alcune attività opzionali nel dopo-scuola.» Fece una pausa. «Se questo non fosse sufficiente, forse posso trovare nuovi mezzi per spronarvi. Voi siete i miei tanto attesi studenti, e farete del vostro meglio nel mio tanto atteso corso di Difesa. Aggiungerei qualche sorta di orribile minaccia, come ‘Altrimenti soffrirete orribilmente’, ma sarebbe così banale, non credete? Mi vanto di essere più fantasioso di così. Grazie.»

Poi il vigore e la fiducia del professor Quirrell sembrarono esaurirsi. La sua bocca si spalancò come se si fosse improvvisamente trovato di fronte a un pubblico inatteso, si voltò con uno scatto convulso e si trascinò di nuovo al suo posto, ingobbito come se stesse per crollare su sé stesso e implodere.

«Sembra un po’ strano», bisbigliò Harry.

«Meh», disse lo studente più grande. «Non hai ancora visto niente.»

Silente riprese il podio.

«E ora», disse Silente, «prima di andare a letto, cantiamo l’inno della scuola! Ognuno scelga il proprio brano preferito e le parole preferite, e incominciamo!»



