% !TeX root = Harry.tex

\chapter*{Un messaggio dell’autore}

\emph{Avvertenza: J.K. Rowling detiene i diritti di Harry Potter, e nessuno detiene i diritti dei metodi della razionalità.}

È ampiamente ritenuto che questa narrazione inizi ad ingranare a partire dal capitolo 5 circa. Se non vi piace ancora dopo il capitolo 10, lasciate stare.

Questa storia non è una narrazione con un singolo punto di divergenza — esiste un punto primario di distacco, in qualche momento del passato, ma anche altre alterazioni. La migliore definizione che ho sentito per questa storia è «universo parallelo». Il ritmo della storia è quello di una narrazione seriale, ovvero quello di una serie televisiva che va non onda per un numero predeterminato di stagioni, i cui episodi sono scritti individualmente ma con un arco narrativo che conduce verso una conclusione definitiva.

Cambiamenti rispetto all’universo originale di Harry Potter possono avvenire per una qualsiasi delle seguenti ragioni:
cambiamenti derivanti dal punto di distacco — un evento avvenuto differentemente, in qualche punto nel passato. Ad esempio, pare che Petunia abbia sposato uno scienziato;
\begin{enumerate}
\item cambiamenti che livellano le cuciture dell’universo. È probabile che quando Rowling scrisse il primo libro, non avesse ancora deciso che ci fosse una sistematica e sicura maledizione sulla cattedra di Difesa. Ma ovviamente, se la maledizione fosse davvero operante da decenni, tutti l’avrebbero notata e asi sarebbero aspettati che qualcosa andasse male col professor Quirrell;
\item cambiamenti che dipendono dal prendere una certa ipotesi più seriamente. Per esempio, se Harry Potter avesse salvato l’intera nazione il 31 ottobre 1981, non l’avrebbero forse invitato alle cerimonie per il decimo anniversario?
\item cambiamenti alle leggi della magia in modo che, come dire, ce ne sia qualcuna. Ci sono storie che possono cavarsela pur avendo leggi della magia poco chiare, perché sono parte dell’ambientazione invece che la trama. Questa non è una di quelle storie;
\item cambiamenti dovuti al fatto che questo non è un libro per bambini. Scrivere un libro per bambini è molto più difficile che scrivere una storia per adulti. Non sono in grado di farlo. Perciò questa è una storia per adulti… e questo significa che è impossibile che il malvagio Peter Pettigrew si stia nascondendo come Scabbers, il topo domestico di una famiglia di maghi nemici. In un libro per bambini, questo può essere accettato incidentalmente. In una storia per adulti, significherebbe che Pettigrew si sta comportando in maniera molto idiota. È una regola generale dei Metodi della razionalità che nessuno si comporti in maniera molto idiota;
\item cambiamenti risultanti dalla filosofia generale di innalzare gli ostacoli affinché corrispondano al protagonista, molto più potente. Un esempio di questo è che nel Calice di Fuoco, c’è un punto in cui l’Harry del canone affronta sulla scopa volate un drago chiamato Ungaro Spinato… che nel capitolo 16 dei Metodi è detto soffiare fuoco così rapidamente da fondere un Boccino in volo, cosa che implica che l’Harry del canone sarebbe stato arrostito in un istante se avesse provato a fare la stessa cosa in questo universo. Ma del resto il Torneo Tre Maghi non potrebbe affatto aver luogo nello stesso modo in questo universo, perché questa non è una storia per bambini, e in una storia per adulti quel piano richiederebbe che Lord Voldemort si stesse comportando in maniera esponenzialmente idiota;
\item cambiamenti che sono necessari per renderla completa dal punto di vista artistica se vista sullo sfondo della fanfiction di Harry Potter, come includere almeno una relazione completamente sbagliata.
\end{enumerate}

Il testo contiene molti indizi: indizi ovvi, indizi non così ovvi, accenni realmente oscuri che sono stato scioccato quando alcuni lettori li hanno decodificati con successo, e una gran massa di prove lasciate in piena vista. Questa è una narrazione razionalista; i suoi misteri sono risolubili, e sono pensati per essere risolti.

Tutta la scienza menzionata è scienza vera. Ma per favore tenete a mente che, oltre il reame della scienza, i punti di vista dei personaggi possono non essere quelli dell’autore. Non tutto quello che il protagonista fa è una lezione di saggezza, e un suggerimento offerto da un personaggio più oscuro può essere inaffidabile o pericolosamente a doppia lama.