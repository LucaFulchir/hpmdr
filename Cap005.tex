% !TeX root = Harry.tex

\chapter{L’errore fondamentale di attribuzione}
\label{capitolo:5}

\emph{«Sarebbe stato necessario un intervento sovrannaturale affinché lui avesse la tua moralità dato il suo ambiente.»}

~\\
~\\

Il Moke Shop era un negozio piccolo e pittoresco (qualcuno avrebbe potuto dire persino carino) nascosto dietro una bancarella di ortaggi che era dietro un negozio di guanti magici che era in una viuzza che deviava da una traversa di Diagon Alley. Purtroppo, la negoziante non era una vecchia megera avvizzita; solo una giovane donna apparentemente nervosa, che indossava abiti di un giallo sbiadito. In quel momento stava porgendo una Moke Super Pouch qx31, i cui punti forti erano un Bordo Allargante e un Incantesimo dell’Estensione Impercettibile: potevi davvero infilarci dentro cose grandi, anche se il volume totale era ancora limitato.

Harry aveva \textit{insistito} per andare lì subito, come prima cosa — aveva insistito quanto più possibile pensasse di poter fare senza rendere sospettosa la professoressa McGonagall. Harry aveva qualcosa che aveva bisogno di mettere nella borsa non appena possibile. Non si trattava del sacchetto di galeoni che la professoressa McGonagall gli aveva consentito di prelevare da Gringotts. Erano tutti gli altri galeoni che Harry si era furtivamente infilato in tasca dopo essere caduto su uno dei mucchi di monete. Si era trattato di un vero incidente, ma Harry non era mai stato il tipo da sprecare un’occasione… anche se era stato più che altro un gesto impulsivo. Da quel momento Harry stava portando goffamente la borsa di galeoni autorizzata vicino alla tasca dei pantaloni, in modo che un eventuale tintinnio sembrasse provenire dal posto giusto.

Rimaneva la questione di come sarebbe stato in grado di mettere le \textit{altre} monete nella borsa senza essere scoperto. Le monete d’oro potevano essere sue, ma erano ugualmente rubate — auto-rubate? Auto-appropriate?

Harry alzò lo sguardo dalla Moke Super Pouch \textsc{qx}31 verso il bancone di fronte a lui. «Posso provarla per un po’? Per assicurarmi che funzioni, uhm, in maniera affidabile?» Spalancò gli occhi in un’espressione di innocenza giocosa e infantile.

Come previsto, dopo dieci ripetizioni della trafila di mettere il sacchetto di monete nella borsa, infilarci la mano, sussurrare «borsa d’oro» e riprenderla, la professoressa McGonagall si era allontanata per esaminare qualche altro oggetto del negozio, e la negoziante si era voltata a guardarla.

Harry lasciò cadere il sacchetto d’oro nella borsa mokeskin con la mano \textit{sinistra}; la \textit{destra} uscì dalla tasca stringendo con forza alcune delle monete d’oro, entrò nella borsa mokeskin, lasciò cadere i galeoni, e (con un «sacchetto d’oro» appena sussurrato) recuperò il sacchetto originale. Poi il sacchetto passò nella sua mano \textit{sinistra}, per essere riposto nuovamente, e la mano \textit{destra} di Harry tornò nella tasca…

Una sola volta la professoressa McGonagall si girò a guardarlo, ma Harry riuscì a evitare di irrigidirsi o di trasalire, ed ella non sembrò accorgersi di niente. Sebbene non fosse \textit{mai} possibile saperlo, con gli adulti che avevano il senso dell’umorismo. Ci vollero tre iterazioni per portare a termine il lavoro, e Harry stimò di essere riuscito a rubare a sé stesso circa trenta galeoni.

Harry portò la mano alla testa, si asciugò un po’ di sudore dalla fronte, ed espirò. «Vorrei questa, prego.»

Alleggerito di quindici galeoni (il doppio del prezzo di una bacchetta da mago, apparentemente) e appesantito da una Moke Super Pouch \textsc{qx}31 in più, Harry si diresse con la professoressa McGonagall verso la porta. Questa formò una mano che li salutò mentre se ne andavano, protendendo il braccio in una maniera che fece sentire Harry un po’ nauseato.

E poi, sfortunatamente…

«Sei \textit{davvero} Harry Potter?» sussurrò l’anziano signore, un’enorme lacrima che gli scendeva sulla guancia. «Non mentiresti su questo, vero? È solo che ho sentito delle voci secondo cui non eri \textit{realmente} sopravvissuto alla Maledizione Mortale ed è per questo che nessuno ha mai più sentito parlare di te.»

… sembrò che l’incantesimo di offuscamento della professoressa McGonagall fosse meno che perfettamente efficace con i praticanti esperti di magia.

La professoressa McGonagall aveva messo una mano sulla spalla di Harry e l’aveva tirato via verso il vicolo più vicino nel momento in cui aveva sentito «Harry Potter?». L’anziano signore li aveva seguiti, ma almeno sembrava che nessun altro l’avesse ascoltato.

Harry prese in considerazione la domanda. \textit{Era davvero} Harry Potter? «So solo ciò che altre persone mi hanno detto», rispose Harry. «Non è che mi ricordi di quando sono nato.» La sua mano gli sfiorò la fronte. «Ho questa cicatrice da quando sono in grado di ricordare, e mi hanno detto che il mio nome è Harry Potter da quando sono in grado di ricordare. Ma», disse pensieroso, «se ci sono già ragioni sufficienti per postulare una cospirazione, non c’è alcuna ragione per la quale non avrebbero dovuto scegliere un altro orfano e crescerlo facendogli credere che \textit{lui} fosse Harry Potter –»

La professoressa McGonagall si passò la mano sul volto esasperata. «Lei assomiglia a suo padre, James, l’anno in cui frequentò per la prima volta Hogwarts. E posso confermare sulla base della \textit{sola personalità} che lei è imparentato con il Flagello di Grifondoro.»

«Anche \textit{la Professoressa} potrebbe essere far parte della cospirazione», osservò Harry.

«No», disse l’anziano signore tremando. «Ha ragione. Hai gli stessi occhi di tua madre.»

«Uhm», Harry si accigliò. «Suppongo che anche \textit{lei} possa farne parte –»

«Ora basta, signor Potter.»

L’anziano uomo alzò la mano come per toccare Harry, ma poi la fece ricadere. «Sono semplicemente felice che tu sia vivo», mormorò. «Grazie, Harry Potter. Grazie per quello che hai fatto… Vi lascio soli, ora.»

E il suo bastone ticchettò via, fuori dal vicolo e giù per la strada principale di Diagon Alley.

La professoressa si guardò intorno, con l’espressione tesa e severa. Anche Harry si guardò automaticamente intorno. Ma il vicolo sembrava vuoto, se non per le foglie cadute, e dall’imbocco che conduceva a Diagon Alley si potevano vedere solo passanti che camminavano rapidamente.

Finalmente la professoressa McGonagall sembrò rilassarsi. «Non si è comportato bene», disse a bassa voce. «Lo so che non ci è abituato, signor Potter, ma le persone ci tengono realmente a lei. La prego di essere gentile con loro.»

Harry si guardò le scarpe. «Non dovrebbero», disse con una punta di amarezza. «Tenere a me, intendo dire.»

«Li ha salvati da Tu-Sai-Chi», disse la professoressa McGonagall. «Perché non dovrebbero tenere a lei?»

Harry alzò lo sguardo verso l’espressione severa sotto il cappello a punta della strega, e sospirò. «Suppongo che non ci sia alcuna possibilità che se dicessi \textit{errore fondamentale di attribuzione} avrebbe idea di cosa parlassi.»

«No», disse la professoressa nel suo preciso accento scozzese, «ma la prego di spiegarsi, signor Potter, se volesse essere così cortese.»

«Beh…» iniziò Harry, cercando di capire come descrivere quella particolare porzione di scienza babbana. «Supponga di arrivare a lavoro e vedere il suo collega che prende a calci la scrivania. Lei pensa, ‘che persona collerica che deve essere’. Il suo collega sta pensando a come qualcuno lo abbia tamponato mentre veniva a lavoro e poi gli abbia urlato contro. \textit{Chiunque} sarebbe arrabbiato per questo, pensa. Quando osserviamo gli altri, vediamo i tratti delle loro personalità che ne spiegano il comportamento, ma quando osserviamo noi stessi, vediamo le circostanze che spiegano il nostro comportamento. Le vicende personali hanno un significato intimo per le persone, che le muove dall’interno, ma noi non vediamo le vicende personali seguire da vicino le persone. Le vediamo solo in una certa situazione, e non vediamo come sarebbero in una situazione differente. Quindi, l’errore fondamentale di attribuzione consiste nello spiegare con tratti duraturi e permanenti ciò che sarebbe spiegato meglio dalle circostanze e dal contesto.» C’erano alcuni eleganti esperimenti che l’avevano confermato, ma Harry non aveva intenzione di parlarne.

Le sopracciglia della strega arrivarono fin sotto la tesa del suo cappello. «Credo di capire…» disse lentamente la professoressa McGonagall. «Ma che cosa ha a che fare con lei?»

Harry diede un calcio al muro di mattoni del vicolo con sufficiente forza da farsi male al piede. «La gente pensa che li abbia salvati da Tu-Sai-Chi perché sono una specie di grande guerriero della Luce.»

«Colui che ha il potere di sconfiggere il Signore Oscuro…» mormorò la strega, una strana ironia alleggerì sua voce.

«Sì», disse Harry, fastidio e frustrazione che lottavano in lui, «come se avessi distrutto il Signore Oscuro perché ho una sorta di qualità distruggi-il-Signore-Oscuro permanente e duratura. Avevo quindici mesi all’epoca! Non \textit{so} cosa sia successo, ma \textit{suppongo} che abbia avuto qualcosa a che fare con, come si dice, le circostanze ambientali contingenti. E certamente nulla a che fare con la mia personalità. Le persone non si preoccupano di \textit{me}, non stanno nemmeno prestando attenzione a me, vogliono stringere la mano a una \textit{pessima spiegazione}.» Harry fece una pausa, e guardò McGonagall. «\textit{Lei} lo sa cosa accadde veramente?»

«Mi \textit{sono} fatta un’idea…» disse la professoressa McGonagall. «Dopo averla conosciuta, voglio dire.»

«Sì?»

«Lei ha trionfato sul Signore Oscuro riuscendo a essere più tremendo di quanto non fosse \textit{lui}, ed è sopravvissuto alla Maledizione Mortale riuscendo a essere più terribile della Morte.»

«Ah. Ah. Ah.» Harry diede ancora un calcio al muro.

La professoressa McGonagall rise sommessamente. «La porterò da Madam Malkin, ora. Temo che i suoi vestiti babbani possano attirare l’attenzione.»

Incapparono in altri due ammiratori, lungo la strada.

\textit{Madam Malkin’s Robes} aveva un ingresso autenticamente noioso, mattoni di un rosso ordinario e vetrine che esponevano semplici vesti nere. Non vesti che splendessero o si trasformassero o girassero, o diffondessero strani raggi che attraversavano la tua camicia e ti solleticavano. Solo semplici vesti nere, questo era tutto quello che si poteva vedere attraverso la vetrina. La porta era tenuta aperta, come per far sapere che non c’erano segreti né qualcosa da nascondere.

«Ho intenzione di allontanarmi per un paio di minuti mentre le prendono le misure per i suoi abiti», disse la professoressa McGonagall. «Le sta bene, signor Potter?»

Harry annuì. Odiava con una passione feroce l’andare a comprare vestiti e non poteva biasimare l’anziana strega perché provava la stessa sensazione.

La bacchetta della professoressa McGonagall uscì dalla sua manica e toccò la testa di Harry con leggerezza. «E poiché avrà bisogno di apparire con chiarezza ai sensi di Madam Malkin, sto rimuovendo l’Offuscamento.»

«Uh…» fece Harry. Questo lo preoccupava un po’, non si era ancora abituato alla faccenda dell’‘Harry Potter’.

«Sono andata a Hogwarts con Madam Malkin», disse McGonagall. «Anche allora, era una delle persone più \textit{pacate} che conoscessi. Non le si scompiglierebbe un capello se Tu-Sai-Chi in persona entrasse nel suo negozio.» La voce di McGonagall era rievocativa e molto condiscendente. «Madam Malkin non la importunerà, e non permetterà che nessun altro la importuni.»

«Dove sta \textit{andando}?» chiese Harry. «Giusto nel caso, sa, in cui qualcosa accadesse davvero.»

McGonagall guardò Harry con severità. «Vado \textit{lì}», disse, indicando un edificio dall’altra parte della strada che esponeva l’insegna di un barilotto di legno, «a prendermi da bere, cosa di cui ho un disperato bisogno. Deve farsi prendere le misure, e \textit{nient’altro}. Tornerò a controllarla \textit{tra poco}, e \textit{pretendo} di trovare il negozio di Madam Malkin ancora in piedi e nient’affatto in fiamme.»

Madam Malkin era una vivace signora anziana che non proferì parola a proposito di Harry quando vide la cicatrice sulla sua fronte, e lanciò un’occhiata severa a un’assistente, quando la ragazza sembrò sul punto di dire qualcosa. Madam Malkin recuperò una serie di pezzi di stoffa animati che si contorcevano e che apparentemente le servivano per prendere le misure, e si mise al lavoro esaminando la materia prima della sua arte.

Accanto a Harry, un ragazzo pallido con un volto affilato e \textit{strafichissimi} capelli biondo-bianchi pareva essere nelle fasi finali di un processo simile. Una delle due assistenti di Madam Malkin stava esaminando il ragazzo dai capelli bianchi e la veste a quadratini che indossava; di tanto in tanto toccava un angolo della veste con la bacchetta, e la veste si allungava o stringeva.

«Ciao», disse il ragazzo. «Hogwarts anche tu?»

Harry poté prevedere dove quella conversazione sarebbe andata a parare, e decise in una frazione di secondo che quando era troppo era troppo.

«Santo cielo», sussurrò Harry, «non può essere.» Fece sì che i suoi occhi si spalancassero. «Il suo… nome, signore?»

«Draco Malfoy», disse Draco Malfoy, sembrando un po’ incuriosito.

«È proprio lei \textit{allora!} Draco Malfoy. Non — non avrei mai pensato di avere questo onore, signore.» Harry desiderò di potersi far venire le lacrime agli occhi. Di solito gli altri iniziavano a piangere, a quel punto.

«Oh», disse Draco, sembrando alquanto confuso. Poi le sue labbra si distesero in un sorriso compiaciuto. «È bello incontrare qualcuno che sa qual è il suo posto.»

Una delle assistenti, quella che era sembrata riconoscere Harry, emise un piccolo suono soffocato.

Harry farfugliò. «Sono lieto di incontrarla, signor Malfoy. Proprio felice in maniera inesprimibile. E frequentare Hogwarts nel suo stesso anno! Questo manda il mio cuore in estasi.»

Oops. Quell’ultima parte sarebbe potuta sembrare un po’ strana, come se stesse amoreggiando con Draco, o cose simili.

«E \textit{io} sono lieto di apprendere che sarò trattato con il rispetto dovuto alla famiglia dei Malfoy», rimpallò l’altro ragazzo, accompagnando le parole con un sorriso simile a quello che il più sommo dei sovrani avrebbe potuto concedere all’ultimo dei suoi sudditi, se questo soggetto fosse stato onesto, sebbene povero.

Eh… Dannazione, Harry stava avendo problemi a formulare la sua prossima battuta. Beh, tutti \textit{volevano} stringere la mano di Harry Potter, quindi — «Quando avranno finito di prendermi le misure, signore, si degnerebbe di stringermi la mano? Non potrei sperare in nulla di meglio come culmine della mia giornata, no, di questo mese, anzi, di tutta la mia vita.»

Il ragazzo dai capelli bianco-biondi lo gelò con lo sguardo. «E cosa hai fatto tu per i Malfoy che ti renda meritevole di un tale favore?»

\textit{Oh, devo assolutamente provare questa risposta con la prossima persona che vorrà stringermi la mano}. Harry chinò la testa. «No, no, signore, comprendo. Sono mortificato di averglielo chiesto. Dovrei essere onorato di pulire i suoi stivali, piuttosto.»

«Infatti», rispose bruscamente l’altro ragazzo. Il suo volto severo si ingentilì, in qualche modo. «Dimmi, in che Casa pensi essere mandato? Io sono destinato a Casa Serpeverde, naturalmente, come mio padre Lucius prima di me. E per quanto riguarda te, direi Casa Tassofrasso o Casa Elfo.»

Harry sorrise imbarazzato. «La professoressa McGonagall dice che sono la persona più Corvonero che abbia mai conosciuto o di cui abbia sentito raccontare nelle leggende, tanto che Rowena in persona mi direbbe di uscire più spesso di casa, qualunque cosa \textit{questo} significhi, e che senza dubbio finirò nella Casa Corvonero, se il cappello non starà urlando troppo forte da impedire a noi altri di capire ogni singola parola, chiuse virgolette.»

«Uau», disse Draco Malfoy, sembrando alquanto impressionato. Il ragazzo emise un sospiro malinconico. «La tua adulazione era ottima, o così mi è sembrato, ad ogni modo — farai bene anche a Casa Serpeverde. Di solito è solo mio padre che riceve questo genere di lusinghe. \textit{Spero} che gli altri Serpeverde mi adulino ora che andrò a Hogwarts… Credo che questo sia un buon presagio, quindi.»

Harry tossì. «In realtà, mi dispiace, ma non ho davvero idea di chi tu sia.»

«\textit{Oh, per favore!}» disse il ragazzo, tremendamente deluso. «Perché l’hai fatto, allora?» Gli occhi di Draco si spalancarono per l’improvviso sospetto. «E come fai a non conoscere i Malfoy? E cosa sono quei \textit{vestiti} che indossi? I tuoi genitori sono forse \textit{Babbani?}»

«Due dei miei genitori sono morti», disse Harry. Provò uno spasimo al cuore. Quando la metteva in quel modo — «Gli altri due genitori sono Babbani, e sono loro che mi hanno cresciuto.»

«\textit{Cosa?}» disse Draco. «Chi \textit{sei} tu?»

«Harry Potter, piacere di conoscerti.»

«\textit{Harry Potter?}» rantolò Draco. «\textit{Quel} Harry –» e il ragazzo si interruppe bruscamente.

Ci fu un breve silenzio.

Poi, con un entusiasmo scoppiettante, «Harry Potter? \textit{Quel} Harry Potter? Accidenti, ho sempre desiderato incontrarla!»

L’assistente di Draco emise un suono strozzato, ma continuò col suo lavoro, sollevando le braccia di Draco per rimuovere con attenzione la veste a scacchi.

«Sta’ zitto», intimò Harry.

«Posso avere il suo autografo? No, un momento, voglio fare una foto con lei, prima!»

«Stai\textit{zitto}stai\textit{zitto}stai\textit{zitto.}»

«Sono così \textit{felice} di averla incontrata!»

«Datti fuoco e muori.»

«Ma lei è Harry Potter, il glorioso salvatore del mondo magico! L’eroe di tutti, Harry Potter! Ho sempre voluto essere come lei da grande, così da –»

Draco interruppe la frase a metà, il suo volto congelato da un orrore assoluto.

Alto, capelli bianchi, freddamente elegante in abiti neri di altissima qualità. Una mano che stringeva un bastone dal manico d’argento, il quale assumeva il carattere di un’arma mortale solo per il fatto di essere impugnato da quella mano. I suoi occhi considerarono la stanza con l’aria spassionata di un boia, un uomo per il quale uccidere non era penoso, o addirittura deliziosamente proibito, ma solo un’attività consuetudinaria come respirare.

Quello era l’uomo che stava, proprio in quel momento, entrando disinvoltamente dalla porta aperta.

«Draco», disse l’uomo, in tono grave e molto arrabbiato, «\textit{che cosa} stai \textit{dicendo?}»

In una frazione di secondo di panico simpatico, Harry formulò un piano di salvataggio.

«Lucius Malfoy!» boccheggiò Harry Potter. «Quel \textit{Lucius} Malfoy?»

Una delle assistenti di Malkin dovette girarsi con la faccia al muro.

Occhi gelidamente sanguinari lo soppesarono. «Harry Potter.»

«Io sono così, così onorato di conoscerla!»

Gli occhi scuri si spalancarono, il trauma per la sorpresa che sostituì la minaccia letale.

«Suo figlio mi ha raccontato \textit{tutto} di lei», disse enfaticamente Harry, quasi senza sapere quello che stava uscendo dalla sua bocca, semplicemente parlando il più velocemente possibile. «Ma naturalmente sapevo già tutto di lei, tutti la conoscono, il grande Lucius Malfoy! La personalità più insigne di tutta la Casa di Serpeverde, ho pensato di cercare di entrare a Casa Serpeverde io stesso, solo perché ho sentito che c’è andato da bambino –»

«\textit{Cosa sta dicendo, signor Potter?}» giunse da fuori il negozio, quasi un urlo, e la professoressa McGonagall si precipitò dentro un secondo dopo.

C’era un orrore talmente puro sul suo viso, che la bocca di Harry si aprì automaticamente, e poi si bloccò senza sapere cosa dire.

«Professoressa McGonagall!» gridò Draco. «È davvero lei? Ho sentito parlare così tanto di lei da mio padre, stavo pensando di provare ad essere Smistato in Grifondoro in modo da poter –»

«\textit{Cosa?}» urlarono Lucius Malfoy e la professoressa McGonagall in perfetto unisono, l’uno di fianco all’altra. Le loro teste girarono su sé stesse per guardarsi vicendevolmente con movimenti simmetrici, e poi entrambi balzarono all’indietro allontanandosi l’uno dall’altra, come se stessero eseguendo una danza sincronizzata.

Ci fu un improvviso turbinio di azione mentre Lucius afferrò Draco e lo trascinò fuori dal negozio.

E poi ci fu silenzio.

Nella mano sinistra della professoressa McGonagall c’era un piccolo bicchiere, dimenticato inclinato da un lato nella frenesia, che ora sgocciolava lentamente dell’alcol nella piccola pozza di vino rosso che aveva fatto la sua comparsa sul pavimento.

La professoressa McGonagall avanzò a grandi passi fino a trovarsi Madam Malkin di fronte.

«Madam Malkin», disse la professoressa McGonagall con voce calma. «Cos’è accaduto qui?»

Madam Malkin la osservò silenziosamente per quattro secondi e poi scoppiò a ridere. Si appoggiò pesantemente contro il muro, ansimando per le risate, e questo fece scatenare entrambe le sue assistenti, una delle quali cadde a carponi sul pavimento, ridendo istericamente.

La professoressa McGonagall si girò lentamente per guardare Harry, la sua espressione gelida. «La lascio solo per sei minuti. Sei minuti d’orologio, signor Potter.»

«Stavo solo scherzando», protestò Harry, mentre i suoni delle risate isteriche proseguirono lì vicino.

«\textit{Draco Malfoy ha detto in presenza di suo padre di voler essere smistato a Grifondoro!} Scherzare \textit{non è sufficiente} per fare questo!» La professoressa McGonagall si interruppe, respirando visibilmente. «Quale parte di ‘si faccia prendere le misure’ le è sembrato significasse \textit{la prego di lanciare un Incantesimo Confundus sull’intero universo!}»

«Draco era in un contesto ambientale in cui quelle azioni avevano un senso interno –»

«No. Non si spieghi. Non voglio sapere cos’è successo qui, mai. Qualunque potere oscuro abiti in lei, è \textit{contagioso}, e non voglio finire come il povero Draco, la povera Madam Malkin e le sue due povere assistenti.»

Harry sospirò. Era chiaro che la professoressa McGonagall non fosse dell’umore adatto per ascoltare spiegazioni ragionevoli. Guardò Madam Malkin, che stava ancora ansimando contro il muro, e le due assistenti di Malkin, che erano ora \textit{entrambe} cadute in ginocchio, e infine il proprio corpo ricoperto di nastri da misura.

«Non ho ancora finito di farmi prendere le misure», disse gentilmente Harry. «Perché non torna a prendersi qualcos’altro da bere?»
