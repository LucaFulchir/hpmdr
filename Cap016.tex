% !TeX root = Harry.tex

\chapter{Pensiero laterale}
\label{capitolo:16}

\emph{«Non sono uno psicopatico, sono solo molto creativo.»}

~\\
~\\

Non appena mercoledì entrò nell’aula di Difesa, Harry seppe che quella materia sarebbe stata differente.

Si trattava, per cominciare, della più grande aula che avesse visto a Hogwarts, analoga all’aula di un’università di prestigio, con livelli sovrapposti di banchi che fronteggiavano un gigantesco palco piatto di marmo bianco. L’aula era situata in alto nel castello — al quinto piano — e Harry sapeva che quella era tutta la spiegazione che avrebbe ricevuto su come un’aula di quelle dimensioni potesse trovarvi posto. Stava diventando chiaro che Hogwarts semplicemente non aveva una geometria, euclidea o altra; aveva connessioni, non direzioni.

A differenza di un’aula universitaria, non c’erano file di sedili pieghevoli; invece c’erano dei banchi di legno e delle sedie di legno, allineati in una curva lungo ciascun livello dell’aula, piuttosto normali per Hogwarts. Tranne che ogni banco aveva un oggetto piatto, bianco, rettangolare e misterioso appoggiato sopra.

Al centro della gigantesca piattaforma, su di una piccola pedana di marmo scuro, c’era la solitaria scrivania dell’insegnante. Alla quale era seduto Quirrell, accasciato sulla sedia, la testa che ciondolava all’indietro, sbavando leggermente sopra le vesti.

Cosa mi ricorda…?

Harry era arrivato a quella lezione così presto che nessun altro studente era ancora lì. (Il vocabolario era imperfetto quando si trattava di descrivere il viaggio nel tempo; in particolare, mancava di qualsiasi parola capace di esprimere quanto fosse comodo.) Quirrell non sembrava essere… funzionale… in quel momento, e Harry non aveva particolarmente voglia di avvicinarglisi, comunque.

Scelse un posto, s’inerpicò fino ad esso, si sedette, e recuperò il libro di testo di Difesa. Era arrivato a circa sette ottavi — aveva avuto intenzione di finirlo prima di quella lezione, in realtà, ma stava rimanendo indietro col programma e aveva già usato il Giratempo due volte, quel giorno.

Presto vi furono dei suoni, mentre la classe cominciò a riempirsi. Harry li ignorò.

«Potter? Cosa ci fai tu qui?»

Quella voce era fuori luogo. Harry alzò lo sguardo. «Draco? Cosa ci fai tu in oh mio dio tu hai dei servitori.»

Uno dei ragazzi in piedi dietro Draco sembrava avere parecchi muscoli per un undicenne, e l’altro aveva assunto una posa bilanciata piuttosto sospetta.

Il ragazzo dai capelli biondo-bianchi sorrise con un certo compiacimento e fece un cenno dietro di sé. «Potter, ti presento il signor Crabbe», la sua mano si mosse da Muscolo a Bilanciato, «il signor Goyle. Vincent, Gregory, questo è Harry Potter.»

Il signor Goyle inclinò la testa e rivolse a Harry uno sguardo che doveva significare qualcosa ma che finì per sembrare semplicemente strabico. Il signor Crabbe disse «Piacere di conoscerti» in un tono che suonò come se stesse cercando di rendere la sua voce più grave possibile.

Una fugace espressione di sgomento attraversò il viso di Draco, ma fu rapidamente sostituita dal suo ghigno di superiorità.

«Tu hai dei servitori!» Harry ripeté. «Dove posso ottenere dei servitori anche io?»

Il ghigno di Draco si allargò. «Ho paura, Potter, che il primo passo sia quello di essere Smistati in Serpeverde –»

«Cosa? Non è giusto!»

«– e poi che le vostre famiglie abbiano stretto un accordo da ben prima che nasceste.»

Harry guardò il signor Crabbe e il signor Goyle. Entrambi stavano tentando disperatamente di sembrare minacciosi. Ovvero, piegandosi in avanti, curvando le spalle, tirando fuori il collo e fissandolo.

«Uhm… aspetta», disse Harry. «Tutto questo è stato organizzato anni fa?»

«Esattamente, Potter. Mi dispiace ma sei stato sfortunato.»

Il signor Goyle tirò fuori uno stuzzicadenti e iniziò a pulirsi i denti, ancora minaccioso.

«E», disse Harry, «Lucius ha insistito che non dovevi crescere conoscendo le tue guardie del corpo, e che dovevi incontrarle solo il tuo primo giorno di scuola.»

Questo cancellò il ghigno dal volto di Draco. «Sì, Potter, sappiamo tutti che sei brillante, tutta la scuola lo sa, ormai puoi smetterla di metterti in mostra –»

«Quindi gli è stato detto per tutta la loro vita che sarebbero stati i tuoi servitori e hanno speso anni immaginando come dovrebbero essere dei servitori –»

Draco fece una smorfia.

«– e cosa peggiore, loro si conoscono e hanno fatto pratica –»

«Il capo t’ha detto che ti devi stare muto», brontolò il signor Crabbe. Il signor Goyle addentò lo stuzzicadenti, tenendolo in bocca, e usò una mano per scrocchiare le nocche dell’altra.

«Vi ho detto di non fare così davanti a Harry Potter!»

I due sembrarono un po’ imbarazzati e il signor Goyle si mise rapidamente lo stuzzicadenti in una tasca posteriore della veste.

Ma nel momento in cui Draco diede loro le spalle per rivolgersi di nuovo a Harry, tornarono a farsi minacciosi.

«Ti chiedo scusa», disse Draco rigidamente, «per l’insulto che questi imbecilli ti hanno recato.»

Harry diede un’occhiata significativa al signor Crabbe e al signor Goyle. «Direi che sei un po’ duro con loro, Draco. Io penso si comportino esattamente come vorrei che i miei servitori agissero. Voglio dire, se avessi qualche servitore.»

Draco rimase a bocca aperta.

«Oh, Gregory, ma che questo sta provando a portarci via dal capo nostro, eh?»

«Sono sicuro che il signor Potter non sarebbe così stupido.»

«Oh, non me lo sognerei neppure», disse Harry tranquillamente. «È solo una cosa da tenere a mente se il vostro datore di lavoro attuale sembrasse ingrato. Inoltre, non fa mai male avere altre offerte, mentre si stanno negoziando le proprie condizioni di lavoro, giusto?»

«Che ci sta a fare questo a Corvonero?»

«Non me lo immagino, signor Crabbe.»

«Smettetela, entrambi», disse Draco a denti stretti. «Questo è un ordine.» Con uno sforzo visibile, trasferì l’attenzione nuovamente su Harry. «Ad ogni modo, che ci fai alla lezione di Difesa dei Serpeverde?»

Harry aggrottò la fronte. «Aspetta.» La sua mano andò alla borsa. «Orario.» Guardò sulla pergamena. «Difesa, 14:30, e ora sono le…» Harry guardò il proprio orologio meccanico, che faceva le 11:23. «14:23, a meno che non abbia perso traccia del tempo. È così?» Se sì, bene, Harry sapeva come raggiungere qualunque lezione si supponeva dovesse seguire in quel momento. Cielo, amava il suo Giratempo e un giorno, quando fosse stato abbastanza grande, si sarebbero sposati.

«No, sembra giusto», disse Draco, apparentemente confuso. Il suo sguardo si girò per guardare al resto dell’auditorium, che si stava riempiendo di vesti bordate di verde e…

«Grifondeficienti!» sputò Draco. «Cosa ci fanno loro qui?»

«Hm», fece Harry. «In effetti il professor Quirrell ha detto… ho dimenticato le sue parole esatte… che avrebbe ignorato alcune delle convenzioni didattiche di Hogwarts. Forse ha solo combinato tutti i corsi.»

«Huh», fece Draco. «Sei il primo Corvonero qui dentro.»

«Già. Sono arrivato presto.»

«Che ci fai all’ultima fila, allora?»

Harry batté le palpebre. «Non so, sembrava un buon posto dove sedersi?»

Draco emise un suono beffardo. «Non avresti potuto sederti più lontano dall’insegnante se ci avessi provato.» Il ragazzo dai capelli biondi si fece un po’ più vicino. «Comunque, è vero quello che hai detto a Derrick e alla sua cricca?»

«Chi è Derrick?»

«L’hai colpito con una torta.»

«Due torte, in verità. Cosa gli avrei detto?»

«Che non stava facendo nulla di scaltro o ambizioso e che era una vergogna per Salazar Serpeverde.» Draco stava fissando Harry intensamente.

«Questo… mi suona abbastanza corretto», disse Harry. «Penso fosse qualcosa di più simile a ‘è questa una specie di piano incredibilmente intelligente che ti garantirà un vantaggio futuro o è davvero una disgrazia per la memoria di Salazar Serpeverde come sembra’ o qualcosa del genere. Non ricordo le parole esatte.»

«Stai confondendo tutti, sai», disse il ragazzo dai capelli biondi.

«Eh?» fece Harry onestamente confuso.

«Warrington ha detto che passare molto tempo sotto il Cappello Smistatore è uno dei segni premonitori di un grande Signore Oscuro. Ne parlano tutti, chiedendosi se debbano iniziare a leccarti i piedi giusto nell’evenienza. Poi te ne sei uscito col proteggere un gruppo di Tassofrasso, per l’amor di Merlino. Poi hai detto a Derrick che è una disgrazia per la memoria di Salazar Serpeverde! Cosa dovrebbero pensare tutti?»

«Che il Cappello Smistatore ha deciso di mettermi nella Casa ‘Serpeverde! Sto scherzando! Corvonero!’ e che mi sto comportando di conseguenza.»

Sia il signor Crabbe sia il signor Goyle ridacchiarono, obbligando il signor Goyle a mettersi rapidamente una mano sulla bocca.

«Faremmo meglio a raggiungere i nostri posti», disse Draco. Esitò, si raddrizzò leggermente, parlò un po’ più formalmente. «Ma intendo sinceramente dare seguito alla nostra ultima conversazione e accolgo le tue condizioni.»

Harry annuì. «Ti dispiacerebbe terribilmente se attendessi fino a sabato pomeriggio? Al momento sono coinvolto in una piccola competizione.»

«Una competizione?»

«Vedere se sono in grado di leggere tutti i miei libri di testo tanto velocemente quanto Hermione Granger.»

«Granger», fece eco Draco. I suoi occhi si socchiusero. «La sanguemarcio che pensa di essere Merlino? Se stai cercando di mettere in imbarazzo lei allora tutti i Serpeverde ti augurano la miglior fortuna, Potter, e io non ti disturberò fino a sabato.» Draco inclinò la testa rispettosamente, e si allontanò, tallonato dai suoi servitori.

Oh, sarà molto divertente destreggiarsi tra tutto ciò, già lo sento.

L’aula si stava riempiendo rapidamente di tutti e quattro i colori delle rifiniture: verde, rosso, giallo e blu. Draco e i suoi due amici sembravano impegnati a cercare di ottenere tre posti contigui in prima fila — già occupati, ovviamente. Il signor Crabbe e il signor Goyle si erano fatti vigorosamente minacciosi, ma non sembrò che facesse molto effetto.

Harry si chinò sul suo libro di testo di Difesa e continuò a leggere.

\begin{figure}[h!]
        \includegraphics[scale=0.4]{boccino.png}
        \centering
\end{figure}

Alle 14:35, quando la maggior parte dei posti era occupata e nessun altro sembrava stesse arrivando, il professor Quirrell sobbalzò improvvisamente nella sua sedia e si sedette dritto, e il suo volto apparve su tutti gli oggetti bianchi piatti e rettangolari appoggiati sui banchi degli studenti.

Harry fu colto di sorpresa, sia dall’improvvisa apparizione del volto del professor Quirrell sia dalla somiglianza con la televisione babbana. C’era qualcosa di nostalgico e di triste in quello, sembrava tanto un pezzo di casa eppure non lo era veramente…

«Buon pomeriggio, miei giovani apprendisti», disse il professor Quirrell. La sua voce sembrava provenire dallo schermo sul banco e parlare direttamente a Harry. «Benvenuti alla vostra prima lezione di Magia da Battaglia, come avrebbero detto i fondatori di Hogwarts; oppure, come si dà il caso sia chiamata alla fine del ventesimo secolo, di Difesa contro le Arti Oscure.»

Ci fu un raspare frenetico mentre gli studenti, colti di sorpresa, cercarono le loro pergamene o i quaderni di appunti.

«No», disse il professor Quirrell. «Non perdete tempo a scrivere come questa materia era chiamata in passato. Nessuna domanda inutile come questa varrà per i vostri voti in nessuna delle mie lezioni. Questa è una promessa.»

Sentendo ciò, molti studenti si raddrizzarono, sembrando piuttosto scioccati.

Il professor Quirrell sorrideva blandamente. «Quelli di voi che hanno sprecato tempo leggendo i vostri inutili libri di testo di Difesa del primo anno –»

Qualcuno emise un suono soffocato. Harry si chiese se fosse Hermione.

«– potrebbero aver avuto l’impressione che, anche se questo corso è chiamato Difesa Contro le Arti Oscure, in realtà riguardi la difesa dalle Farfalle Incubo, che causano sogni moderatamente brutti, o dalle Lumache Acide, che possono dissolvere completamente una trave di legno di due pollici, se lasciate agire per quasi una giornata.»

Il professor Quirrell si alzò, spingendo indietro la sedia dalla scrivania. Lo schermo sul banco di Harry seguiva ogni sua mossa. Il professor Quirrell avanzò a grandi passi verso la parte anteriore della classe, e urlò:

«L’Ungaro Spinato è più alto di una dozzina di uomini! Sputa fuoco così rapidamente e in modo così preciso da poter sciogliere un Boccino in volo! Una sola Maledizione Mortale lo abbatterà!»

Vi furono sussulti da parte degli studenti.

«Il Troll di montagna è più pericoloso dell’Ungaro Spinato! È sufficientemente forte da masticare l’acciaio! La sua pelle è sufficientemente resistente da sopportare le Maledizioni Stordenti e gli Incantesimi di Taglio! Il suo olfatto è così acuto che può dire da lontano se la sua preda è parte di un branco, o sola e vulnerabile! Cosa più temibile di tutte, il Troll è l’unica tra le creature magiche che sostiene continuamente una sorta di Trasfigurazione su sé stessa — si sta sempre trasformando nel proprio corpo. Se riuscite in qualche modo a strappare il suo braccio ne farà crescere un altro in pochi secondi! Fuoco e acido producono un tessuto cicatriziale che può confondere temporaneamente i poteri rigenerativi di un Troll – per un’ora o due! Sono abbastanza intelligenti da usare i bastoni come strumenti! Il Troll di montagna è la terza macchina per uccidere più perfetta di tutta la Natura! Una sola Maledizione Mortale lo abbatterà.»

Gli studenti sembravano molto sconvolti.

Il professor Quirrell sorrideva piuttosto cupamente. «La vostra triste imitazione di un libro di Difesa del terzo anno vi suggerirà di esporre il Troll di montagna alla luce del sole, che lo bloccherà sul posto. Questo, miei giovani apprendisti, è il tipo di conoscenza inutile che non troverete mai nelle mie lezioni. Non si incontrano Troll di montagna in pieno giorno! L’idea di utilizzare la luce solare per fermarli è dovuta a sciocchi autori di libri di testo che cercano di far mostra della loro padronanza delle minuzie a scapito della praticità. Solo perché c’è un modo ridicolmente oscuro di affrontare i Troll di montagna non significa che dovreste effettivamente provare a usarlo! La Maledizione Mortale non è bloccabile, è inarrestabile e funziona ogni singola volta su qualsiasi cosa abbia un cervello. Se, da maghi adulti, vi trovaste impossibilitati a usare la Maledizione Mortale, allora potrete semplicemente Materializzarvi via! Stessa cosa se foste di fronte alla seconda macchina per uccidere più perfetta, un Dissennatore. Basta che vi Materializziate via!»

«A meno che, naturalmente», disse il professor Quirrell, la sua voce ora più bassa e più dura, «non siate sotto l’influenza di una fattura anti-Materializzazione. No, c’è esattamente un solo mostro che può minacciarvi una volta che siate diventati adulti. Il singolo mostro più pericoloso di tutto il mondo, così pericoloso che nient’altro gli è paragonabile. Il Mago Oscuro. Questa è l’unica cosa che sarà ancora in grado di minacciarvi.»

Le labbra del professor Quirrell erano fissate in una linea sottile. «Vi insegnerò a malincuore abbastanza banalità per ottenere un voto sufficiente ai vostri esami del primo anno sulle parti imposte dal Ministero. Dal momento che l’esatto voto su queste sezioni non farà alcuna differenza per la vostra vita futura, chiunque voglia un voto superiore alla sufficienza è invitato a sprecare il proprio tempo studiando la nostra patetica imitazione di un libro di testo. Il titolo di questo corso non è Difesa Contro i Flagelli Minori. Voi siete qui per imparare a difendervi contro le Arti Oscure. Il che significa, vediamo di essere molto chiari su questo, difendervi dai Maghi Oscuri. Persone dotate di bacchette che vogliono farvi del male e che probabilmente ci riusciranno, a meno che non facciate male loro per primi! Non c’è difesa senza attacco! Non c’è difesa senza combattimento! Questa realtà è ritenuta troppo dura per degli undicenni dai grassi e strapagati politici protetti da Auror che hanno imposto il vostro curriculum. All’abisso quegli sciocchi! Voi siete qui per la materia che è stata insegnata a Hogwarts per ottocento anni! Benvenuti al vostro primo anno di Magia da Battaglia!»

Harry iniziò ad applaudire. Non poteva farne a meno, era stato troppo stimolante.

Una volta che Harry ebbe iniziato ad applaudire ci fu qualche reazione sparpagliata da Grifondoro, e di più da Serpeverde, ma la maggior parte degli studenti sembravano semplicemente troppo sbalorditi per reagire.

Il professor Quirrell fece il gesto di troncare, e l’applauso morì istantaneamente. «Molte grazie», disse il professor Quirrell. «Ora gli avvisi pratici. Ho combinato tutti i miei corsi di Magia da Battaglia del primo anno in uno solo, cosa che mi permette di offrirvi il doppio del tempo in classe rispetto alle Sessioni doppie –»

Si udirono dei rantoli di orrore.

«– un aumento del carico per il quale vi compenserò non assegnandovi alcun compito a casa.»

I rantoli di orrore si interruppero bruscamente.

«Sì, mi avete sentito bene. Vi insegnerò a combattere, non a scrivere una relazione di dodici pollici sul combattimento per lunedì.»

Harry desiderò disperatamente di essere seduto accanto a Hermione in modo da poter vedere l’espressione del suo viso in quel momento, ma d’altra parte era abbastanza sicuro di immaginarsela con precisione.

E poi, Harry era innamorato. Sarebbe stato un matrimonio a tre: Harry, il Giratempo, e il professor Quirrell.

«Per quelli di voi che lo desiderassero, ho organizzato alcune attività di doposcuola che penso troverete molto interessanti ed educative. Volete mostrare al mondo le vostre capacità invece di guardare quattordici altre persone giocare a Quidditch? Più di sette persone possono combattere in un esercito.»

Santo cielo.

«Queste e altre attività di doposcuola vi permetteranno di guadagnare punti-Quirrell. Cosa sono i punti-Quirrell, vi chiederete? Il sistema dei punti-Casa non soddisfa le mie esigenze, perché rende i punti-Casa troppo rari. Preferisco che i miei studenti sappiano come stanno andando più frequentemente di così. E nelle rare occasioni in cui vi offrirò una prova scritta, si correggerà da sola man mano che la compilerete, e se sbaglierete troppe domande in relazione tra loro, il test vi mostrerà i nomi degli studenti che hanno risposto correttamente a quelle domande, e questi studenti saranno in grado di guadagnare punti-Quirrell aiutandovi.»

… uau. Perché gli altri professori non usavano un sistema simile?

«A che servono i punti-Quirrell, vi chiederete? Per cominciare, dieci punti-Quirrell varranno un punto-Casa. Ma vi permetteranno di guadagnare anche altre agevolazioni. Volete fare il vostro esame a un’ora inusuale? C’è una sessione particolare che preferireste davvero saltare? Scoprirete che posso essere molto flessibile per favorire gli studenti che hanno accumulato abbastanza punti-Quirrell. I punti-Quirrell determineranno il rango di generale degli eserciti. E per Natale — appena prima delle vacanze natalizie — accorderò a qualcuno un desiderio. Ogni impresa relativa alla scuola che rientri nei limiti del mio potere, della mia influenza o, sopratutto, della mia ingegnosità. Sì, ero in Serpeverde e mi sto offrendo per formulare un piano astuto a vostro nome, se questo è quello che ci vuole per raggiungere i vostri desideri. Questo desiderio andrà a chiunque abbia guadagnato più punti-Quirrell in tutti i sette anni.»

Quello sarebbe stato Harry.

«Ora lasciate i vostri libri e oggetti sui vostri banchi — saranno al sicuro, gli schermi veglieranno su di loro per voi — e scendete su questa piattaforma. È il momento di giocare a un gioco chiamato ‘Chi è lo studente più pericoloso della classe’.»

\begin{figure}[h!]
        \includegraphics[scale=0.4]{boccino.png}
        \centering
\end{figure}

Harry girò la sua bacchetta nella mano destra e disse «Ma-ha-su!»

Ci fu un altro «bing» acuto proveniente dalla sfera blu fluttuante assegnata a Harry dal professor Quirrell come bersaglio. Quel particolare suono indicava un colpo perfetto, che Harry aveva ottenuto in nove dei suoi ultimi dieci tentativi.

Da qualche parte il professor Quirrell aveva scovato un incantesimo che era incredibilmente facile da pronunciare, e aveva un movimento della bacchetta ridicolmente semplice, e aveva la tendenza a colpire ovunque tu stessi guardando in quel momento. Il professor Quirrell aveva sdegnosamente proclamato che la vera magia da battaglia era molto più difficile. Che quella fattura era del tutto inutile in un combattimento reale. Che si trattava solo di una scarica di magia a malapena organizzata la cui unica difficoltà reale era il puntamento, e che avrebbe prodotto, quando avesse colpito, un dolore equivalente a un pugno forte sul naso. Che l’unico scopo di questa prova era vedere chi fosse rapido a imparare, dal momento che il professor Quirrell era certo che nessuno avesse incontrato precedentemente quella fattura o qualcosa di simile.

A Harry non importava niente di tutto quello.

«Ma-ha-su!»

Un fulmine rosso di energia fu scagliato dalla sua bacchetta e colpì il bersaglio, e la sfera blu produsse ancora una volta il bing che significava che l’incantesimo gli era effettivamente riuscito.

Per la prima volta da quando era arrivato a Hogwarts, Harry si stava sentendo un vero mago. Avrebbe voluto che il bersaglio schivasse come le piccole sfere che Ben Kenobi aveva usato per l’addestramento di Luke, ma per qualche ragione il professor Quirrell aveva invece schierato tutti gli studenti e i bersagli in file ordinate, cosa che garantiva che non si sarebbero colpiti a vicenda.

Così Harry abbassò la bacchetta, balzò sulla destra, alzò di scatto la bacchetta e la rigirò gridando «Ma-ha-su!»

Ci fu un «dong» dal trono grave, che significava che l’aveva colpita quasi in pieno.

Harry si mise la bacchetta in tasca, balzò di nuovo a sinistra, estrasse e scagliò un altro fulmine rosso di energia.

L’acuto bing che ne risultò fu facilmente uno dei suoni più soddisfacenti che avesse mai sentito in vita sua. Harry avrebbe voluto urlare trionfante a squarciagola. posso fare magie! temetemi, leggi della fisica, vengo a violarvi!

«Ma-ha-su!» La voce di Harry era alta, ma appena percettibile sopra la cantilena costante delle grida simili provenienti da tutta la piattaforma.

«Basta», disse la voce amplificata del professor Quirrell. (Non sembrava alta. Aveva un volume normale, e aveva origine appena dietro la tua spalla sinistra, non importa dove fossi disposto rispetto al professor Quirrell.) «Vedo che tutti voi ci siete riusciti almeno una volta, ora.» Le sfere-bersaglio virarono sul rosso e iniziarono a salire verso il soffitto.

Il professor Quirrell era in piedi sulla pedana rialzata al centro della piattaforma, e si appoggiava appena alla sua cattedra con una mano.

«Vi ho detto», disse il professor Quirrell, «che avremmo fatto un gioco chiamato ‘Chi è lo studente più pericoloso della classe’. C’è uno studente in questa classe che ha padroneggiato la Fattura d’Attacco Semplice Sumera più rapidamente di ogni altro –»

Oh blah blah blah.

«– e poi è andato ad aiutare altri sette studenti. Ragion per cui ha guadagnato i primi sette punti-Quirrell assegnati al vostro anno. Venga avanti, Hermione Granger. È il momento della fase successiva del gioco.»

Hermione Granger iniziò a farsi avanti a grandi passi, uno sguardo misto di trionfo e di apprensione sul suo volto. I Corvonero osservavano con orgoglio, i Serpeverde con gelida intensità, e Harry con un chiaro fastidio. Harry aveva fatto bene questa volta. Probabilmente era anche nella metà superiore della classe, ora che tutti avevano dovuto affrontare una magia ugualmente sconosciuta e che Harry aveva letto tutto Teoria della Magia di Adalbert Waffling. Eppure Hermione stava ancora facendo meglio.

Da qualche parte nei recessi della sua mente c’era la paura che Hermione fosse semplicemente più intelligente di lui.

Ma per ora Harry aveva intenzione di basare le sue speranze sui fatti noti che (a) Hermione aveva letto molto più che i normali libri di testo e (b) Adalbert Waffling era un idiota privo di ispirazione che aveva scritto Teoria della Magia per assecondare una commissione scolastica che non aveva una grande opinione degli undicenni.

Hermione raggiunse la pedana centrale e vi salì sopra.

«Hermione Granger ha padroneggiato un incantesimo sconosciuto in appena due minuti, quasi un minuto intero più velocemente del secondo più bravo.» Il professor Quirrell si girò intorno lentamente per osservare tutti gli studenti che lo stavano fissando. «È in grado l’intelligenza della signorina Granger di renderla lo studente più pericoloso della classe? Allora? Che ne pensate?»

Nessuno parve pensare nulla in quel momento. Anche Harry non era sicuro di cosa dovesse dire.

«Allora scopriamolo, no?» disse il professor Quirrell. Si voltò verso Hermione, e fece un gesto a indicando genericamente la classe. «Scelga uno studente a piacere e gli lanci contro la Fattura d’Attacco Semplice Sumera.»

Hermione rimase paralizzata sul posto.

«Forza», disse il professor Quirrell con tranquillità. «Ha lanciato questo incantesimo in maniera perfetta per oltre cinquanta volte. Non causa danni permanenti né è così doloroso. Fa male quanto un pugno forte e dura appena pochi secondi.» La voce del professor Quirrell si fece più dura. «Questo è un ordine diretto del suo professore, signorina Granger. Scelga un bersaglio e lanci la Fattura d’Attacco Semplice.»

Il volto di Hermione era distorto dall’orrore e la bacchetta stava tremando nella sua mano. Le dita di Harry stavano stringendo con forza la sua bacchetta per simpatia. Anche se poteva capire ciò che il professor Quirrell stava cercando di fare. Anche se poteva capire cosa il professor Quirrell stava cercando di dimostrare.

«Se non solleva la sua bacchetta per fare fuoco, signorina Granger, perderà un punto-Quirrell.»

Harry fissò Hermione, volendo che guardasse nella sua direzione. La sua mano destra stava battendo con leggerezza sul suo petto. Scegli me, non ho paura…

La bacchetta di Hermione si contrasse nella sua mano; poi il suo volto si rilassò ed ella abbassò la bacchetta lungo il fianco.

«No», disse Hermione Granger.

La sua voce era calma, e anche se non era alta, ognuno poté udirla nel silenzio.

«Allora devo toglierle un punto», disse il professor Quirrell. «Questa era una prova, e lei l’ha fallita.»

Ne fu sconvolta. Harry poté vederlo. Ma continuò a tenere la schiena dritta.

La voce del professor Quirrell fu comprensiva e sembrò riempire l’intera aula. «La conoscenza non è sempre sufficiente, signorina Granger. Se non può infliggere e ricevere violenza di entità paragonabile allo sbattere un dito del piede, allora non può difendere sé stessa e non passerà Difesa. Prego, ritorni dai suoi compagni di classe.»

Hermione si diresse nuovamente verso il gruppo dei Corvonero. Il suo volto sembrava sereno e Harry, per una qualche strana ragione, voleva iniziare ad applaudire. Sebbene il professor Quirrell avesse avuto ragione.

«Dunque», disse il professor Quirrell. «È chiaro che Hermione Granger non è lo studente più pericoloso della classe. Chi pensate che possa essere, in realtà, la persona più pericolosa qui dentro? — A parte me, naturalmente.»

Senza nemmeno pensare, Harry si voltò a guardare il contingente Serpeverde.

«Draco, della Nobile e Antichissima Casa Malfoy», disse il professor Quirrell. «Pare che molti dei suoi compagni di scuola stiano guardando nella sua direzione. Si faccia avanti, se non le dispiace.»

Draco obbedì, avanzando con un certo orgoglio nel portamento. Salì sulla pedana e guardò il professor Quirrell sorridendo.

«Signor Malfoy», disse il professor Quirrell. «Fuoco.»

Harry avrebbe provato a fermarlo se ci fosse stato il tempo, ma con un singolo, fluido movimento Draco piroettò verso il contingente dei Corvonero e alzò la sua bacchetta e disse «Mahasu!» come se fosse stata un’unica sillaba e Hermione stava dicendo «Ahi!» e poi basta.

«Bel colpo», disse il professor Quirrell. «Due punti-Quirrell per lei. Ma mi dica, perché ha scelto la signorina Granger come bersaglio?»

Ci fu una pausa.

Alla fine Draco rispose, «Perché lei risaltava di più.»

Le labbra del professor Quirrell si sollevarono in un sorriso. «E quella è la vera ragione per la quale Draco Malfoy è pericoloso. Se avesse scelto chiunque altro, quel bambino avrebbe probabilmente provato del risentimento per essere stato discriminato, e il signor Malfoy si sarebbe fatto molto probabilmente un nemico. E sebbene il signor Malfoy avrebbe potuto dare qualche altra spiegazione per aver scelto lei, questo non gli sarebbe servito a nulla se non ad alienarsi alcuni di voi, mentre altri lo stanno già acclamando che dica qualcosa o no. Vale a dire che il signor Malfoy è pericoloso perché sa chi colpire e chi non colpire, come farsi degli alleati ed evitare di farsi dei nemici. Altri due punti-Quirrell per lei, signor Malfoy. E poiché ha dimostrato una virtù esemplare dei Serpeverde, credo che anche la Casa di Salazar abbia guadagnato un punto. Può riunirsi ai suoi amici.»

Draco accennò un inchino e tornò nel contingente dei Serpeverde. Alcuni applausi partirono dalle vesti bordate di verde, ma il professor Quirrell fece un gesto e cadde nuovamente il silenzio.

«Potrebbe sembrare che il nostro gioco sia terminato», disse il professor Quirrell. «Eppure c’è un singolo studente in quest’aula che è molto più pericoloso del rampollo dei Malfoy.»

E ora per qualche ragione sembrarono esserci un sacco di persone che guardavano verso…

«Harry Potter. Venga avanti.»

Non prometteva nulla di buono.

Harry si diresse a malincuore verso il luogo in cui il professor Quirrell se ne stava in piedi, ancora leggermente appoggiato alla cattedra.

Il nervosismo di essere stato messo sotto la luce dei riflettori sembrò affilare l’ingegno di Harry, e mentre egli si avvicinava alla pedana la sua mente cercò tra le varie possibilità quella che il professor Quirrell avrebbe potuto considerare una prova della pericolosità di Harry. Gli sarebbe stato chiesto di lanciare un incantesimo? Di sconfiggere un Signore Oscuro?

Dimostrare la sua supposta immunità alla Maledizione Mortale? Di certo il professor Quirrell era troppo intelligente per quello…

Harry si fermò abbondantemente prima della pedana, e il professor Quirrell non gli chiese di avvicinarsi oltre.

«La cosa ironica è», disse il professor Quirrell, «che tutti voi avete guardato la persona giusta per le ragioni completamente sbagliate. Voi state pensando», le labbra del professor Quirrell si contorsero, «che Harry Potter abbia sconfitto il Signore Oscuro, e che quindi debba essere molto pericoloso. Bah. Aveva un anno. Qualunque capriccio del fato abbia ucciso il Signore Oscuro aveva probabilmente poco a che fare con le capacità marziali del signor Potter. Ma dopo aver sentito alcune voci riguardo un Corvonero che avrebbe fronteggiato cinque Serpeverde più grandi, ho interrogato diversi testimoni e sono giunto alla conclusione che Harry Potter sarebbe stato il mio studente più pericoloso.»

Un’ondata di adrenalina si riversò attraverso Harry, facendolo stare in piedi ancora più dritto. Non sapeva quale conclusione avesse raggiunto il professor Quirrell, ma non poteva essere corretta.

«Ah, professor Quirrell –» Harry iniziò a dire.

Il professor Quirrell sembrò divertito. «Sta pensando che sono giunto alla conclusione sbagliata, non è vero, signor Potter? Imparerà ad aspettarsi di meglio da me.» Il professor Quirrell si raddrizzò, smettendo di appoggiarsi alla scrivania. «Signor Potter, tutte le cose hanno degli usi ordinari. Mi dia dieci usi inconsueti e utili per il combattimento di oggetti presenti in questa stanza!»

Per un momento, Harry fu reso senza parole dal puro e semplice sconcerto per essere stato compreso.

E poi le idee iniziarono a sgorgare.

«Ci sono banchi abbastanza pesanti da essere mortali se lasciati cadere da una grande altezza. Ci sono sedie con gambe di metallo che potrebbero impalare qualcuno se spinte con sufficiente forza. L’aria nell’aula sarebbe mortale se assente, poiché la gente muore nel vuoto, e può servire come vettore di gas letali.»

Harry dovette interrompersi brevemente per respirare, e in quel momento di pausa il professor Quirrell disse:

«Sono tre. Gliene servono dieci. Il resto della classe pensa che lei abbia già usato tutto il contenuto dell’aula.»

«Aha! Il pavimento può essere rimosso per creare una buca con degli spuntoni, il soffitto può essere fatto crollare su qualcuno, le mura possono servire come materia prima per una Trasfigurazione in un gran numero di oggetti letali — coltelli, per esempio.»

«Sono sei. Ma certamente starà grattando il fondo del barile, ora.»

«Non ho neppure iniziato! Guardi tutte le persone! Far sì che un Grifondoro attacchi il nemico è un uso ordinario, naturalmente –»

«Non lo aggiungerò alla conta.»

«– ma il loro sangue può anche essere usato per annegare qualcuno. I Corvonero sono noti per i loro cervelli, ma i loro organi interni potrebbero essere venduti al mercato nero per una somma sufficiente ad assoldare un sicario. I Serpeverde non sono utili solo come assassini, possono anche essere lanciati a una velocità sufficiente per schiacciare un nemico. E i Tassofrasso, oltre a essere grandi lavoratori, contengono anche delle ossa che possono essere rimosse, appuntite e usate per trafiggere qualcuno.»

Ormai il resto della classe fissava Harry con un certo orrore. Anche i Serpeverde sembravano turbati.

«Sono dieci, anche se sono generoso a contare quello dei Corvonero. Ora, per avere dei crediti extra, un punto-Quirrell per ciascun uso degli oggetti di questa stanza che non ha ancora nominato.» Il professor Quirrell rivolse a Harry un sorriso amichevole. «Il resto della sua classe crede che lei sia ora nei guai, poiché ha nominato tutto tranne i bersagli e non ha idea di cosa possa essere fatto con quelli.»

«Bah! Ho nominato tutte le persone, ma non le mie vesti, che possono essere usate per soffocare un nemico se avvolte attorno alla sua testa un numero sufficiente di volte, o le vesti di Hermione Granger, che possono essere strappate in strisce e annodate in una corda da usare per impiccare qualcuno, o le vesti di Draco Malfoy, che possono essere usate per appiccare un incendio –»

«Tre punti», disse il professor Quirrell, «niente più vestiti, ora.»

«La mia bacchetta può essere spinta nel cervello del nemico attraverso le sue orbite» e qualcuno emise un suono atterrito e strozzato.

«Quattro punti, niente più bacchette.»

«Il mio orologio da polso potrebbe soffocare qualcuno se spinto nella sua gola –»

«Cinque punti, e basta così.»

«Mah», disse Harry. «Dieci punti-Quirrell per un punto-Casa, giusto? Avrebbe potuto permettermi di andare avanti finché non avessi vinto la Coppa delle Case, non ho neppure iniziato con gli usi inconsueti di tutto ciò che ho nelle mie tasche» o la stessa mokeskin, e non poteva parlare del Giratempo o del mantello dell’invisibilità, ma ci doveva essere qualcosa che avrebbe potuto dire riguardo quelle sfere rosse…

«Basta, signor Potter. Bene, pensate di aver tutti capito cosa rende il signor Potter lo studente più pericoloso della classe?»

Ci fu un basso sussurro di assenso.

«Ditelo ad alta voce, prego. Terry Boot, cosa rende il suo compagno di dormitorio pericoloso?»

«Ah… uhm… è creativo?»

«Sbagliato!» gridò il professor Quirrell, e il suo pugno picchiò bruscamente sulla cattedra con un suono amplificato che fece sobbalzare tutti. «Tutte le idee del signor Potter erano peggio che inutili!»

Harry sussultò per la sorpresa.

«Rimuovere il pavimento per creare un trappola con degli spuntoni? Ridicolo! In combattimento non avrete tutto quel tempo per prepararvi, e se l’aveste ci sarebbero cento usi migliori! Trasfigurare materiale dalle pareti? Il signor Potter non è in grado di eseguire una Trasfigurazione! Il signor Potter ha avuto esattamente una idea che avrebbe potuto usare immediatamente, ora, senza una lunga preparazione o un nemico collaborativo o una magia che non conosce. Quell’idea era di spingere la sua bacchetta attraverso l’orbita oculare del suo nemico. Cosa che avrebbe molto più probabilmente rotto la sua bacchetta, che ucciso il suo nemico! In breve, signor Potter, temo che le sue proposte siano state uniformemente terribili.»

«Cosa?» Harry disse indignato. «Lei ha chiesto idee inusuali, non pratiche! Stavo pensando fuori dagli schemi! Come userebbe lei un oggetto in questa classe per uccidere qualcuno?»

L’espressione del professor Quirrell era di disapprovazione, ma c’erano le pieghe tipiche di un sorriso attorno ai suoi occhi. «Signor Potter, non ho mai detto che lei avrebbe dovuto uccidere. C’è un tempo e un luogo per prendere il suo nemico vivo, e l’interno di un’aula di Hogwarts è normalmente uno di quei posti. Ma per rispondere alla sua domanda, colpendoli sul collo con lo spigolo di una sedia.»

Ci furono alcune risate da parte dei Serpeverde, ma stavano ridendo con Harry, non di lui.

Tutti gli altri sembravano alquanto terrorizzati.

«Ma il signor Potter ha ora dimostrato perché egli è lo studente più pericoloso della classe. Avevo chiesto usi inconsueti di oggetti in questa stanza utili ai fini di un combattimento. Il signor Potter avrebbe potuto suggerire di usare un banco per bloccare una maledizione, o di usare una sedia per far inciampare un nemico che avanza, o di avvolgere una veste intorno al suo braccio per creare uno scudo improvvisato. Invece, ogni singolo uso che il signor Potter ha nominato era offensivo invece che difensivo, e o letale o potenzialmente letale.»

Cosa? Un attimo, non poteva essere vero… Harry provò un’improvvisa sensazione di vertigini mentre tentò di ricordare cosa esattamente avesse suggerito, certamente doveva esserci un controesempio…

«E questo», disse il professor Quirrell, «è il motivo per il quale le idee del signor Potter erano così strane e inutili — perché ha dovuto spingersi in profondità nell’inattuabile per soddisfare il suo obiettivo di uccidere il nemico. Per lui, ogni idea che non fosse in grado di farlo non era degna di considerazione. Questo riflette una qualità che potremmo chiamare volontà di uccidere. Io ce l’ho. Harry Potter ce l’ha, ed è per questo che ha potuto spuntarla su cinque Serpeverde più grandi. Draco Malfoy non ce l’ha, non ancora. Difficilmente il signor Malfoy si tirerebbe indietro dal discutere di un ordinario assassinio, ma persino lui era scioccato — sì, lo era, signor Malfoy, stavo guardando il suo volto — quando il signor Potter ha descritto come usare i corpi dei suoi compagni di classe come materia prima. Ci sono meccanismi censori dentro la vostra mente che vi fanno trasalire a un pensiero simile. Il signor Potter pensa meramente a uccidere il nemico, approfitterà di ogni mezzo per farlo, egli non trasalisce, i suoi censori sono spenti. Sebbene il suo giovane genio sia così indisciplinato e mancante di senso pratico da essere inutile, la sua volontà di uccidere fa di Harry Potter lo ‘Studente più pericoloso della classe’. Un ultimo punto per lui — no, facciamo un punto per Corvonero — per questo requisito indispensabile per un vero mago combattente.»

La bocca di Harry si spalancò in un turbamento senza parole, mentre cercava freneticamente qualcosa con cui rispondere. Io non sono affatto così!

Ma poté capire che gli altri studenti iniziavano a crederci. La mente di Harry stava scorrendo tutte le possibili smentite e non trovava nulla che potesse reggere al confronto con la voce autorevole del professor Quirrell. Il meglio che Harry aveva trovato era «non sono uno psicopatico, sono solo molto creativo» e quello suonava un po’ sinistro. Aveva bisogno di dire qualcosa di inaspettato, qualcosa che obbligasse le persone a fermarsi e a ripensarci –

«E ora», disse il professor Quirrell. «Signor Potter. Fuoco.»

Nulla accadde, naturalmente.

«Ah, beh», disse il professor Quirrell. Sospirò. «Credo che tutti dobbiamo iniziare da qualche parte. Signor Potter, scelga uno studente a piacere per la Fattura d’Attacco Semplice. Lei lo farà prima che io dichiari terminata la lezione di oggi. In caso contrario, inizierò a sottrarre punti-Casa, e continuerò a sottrarli finché non lo farà.»

Prudentemente Harry sollevò la propria bacchetta. Almeno quello doveva farlo, o il professor Quirrell avrebbe potuto iniziare a sottrarre punti immediatamente.

Lentamente, come se fosse stato su di uno spiedo, Harry si voltò a fronteggiare i Serpeverde.

E gli occhi di Harry incontrarono quelli di Draco.

Draco Malfoy non sembrò minimamente preoccupato. Il ragazzo dai capelli biondi non stava facendo alcun segno visibile di assenso come quello che Harry aveva rivolto a Hermione, ma del resto era difficile chiedergli il contrario. Gli altri Serpeverde l’avrebbero considerato piuttosto strano.

«Perché esita?» disse il professor Quirrell. «Certamente c’è una sola scelta ovvia.»

«Sì», disse Harry. «Solo una scelta ovvia.»

Harry girò la bacchetta e disse «Ma-ha-su!»

Nella classe ci fu un silenzio assoluto.

Harry agitò il braccio sinistro, cercando di liberarsi dalla fitta persistente.

Ci fu altro silenzio.

Alla fine il professor Quirrell sospirò. «Sì, molto originale, ma c’era una lezione da impartire e lei l’ha scansata. Un punto in meno a Corvonero per aver messo in mostra la sua astuzia a spese del vero obiettivo. La lezione è finita.»

E prima che chiunque altro potesse dire qualunque cosa, Harry disse ad alta voce:

«Stavo scherzando! corvonero!»

Dopo di che per un breve momento ci fu silenzio, il rumore di persone che pensavano, e poi i brusii iniziarono e crebbero rapidamente fino al rombo di una conversazione.

Harry si girò verso il professor Quirrell, dovevano parlare –

Quirrell si era ingobbito e si stava trascinando verso la sua sedia.

No. Non era accettabile. Avevano veramente bisogno di parlare. Al diavolo la scenata dello zombi, il professor Quirrell si sarebbe probabilmente svegliato se Harry l’avesse pungolato un paio di volte. Harry si fece avanti –

sbagliato

non farlo

pessima idea

Harry vacillò e si fermò improvvisamente, con una sensazione di vertigini.

E poi uno stormo di Corvonero discese su di lui e le discussioni iniziarono.



