\% !TeX root = Harry.tex



\chapter{Gratificazione differita}

\label{capitolo:19}



\emph{Draco aveva un’espressione severa sul volto, e le sue vesti bordate di verde sembravano in qualche modo molto più formali, serie, e di miglior qualità delle stesse esatte vesti indossate dai due ragazzi dietro di lui.}



~\\

~\\



«Parla», disse Draco.

«Sì! Parla!»

«Hai sentito al capo! Parla!»

«Voi due, d’altro canto, \textit{tacete}.»

L’ultima sessione di lezioni di venerdì stava per iniziare, in quel vasto auditorium dove tutte e quattro le Case imparavano Difesa, cioè, Magia da Battaglia.

L’ultima sessione di lezioni di venerdì.

Harry stava sperando che quella lezione sarebbe stata tranquilla, e che il brillante professor Quirrell avrebbe compreso che quello non era forse il momento migliore per scegliere Harry tra tanti per qualche dimostrazione. Harry si era ripreso un po’, ma…

… ma giusto per sicurezza, era probabilmente meglio alleviare un po’ la tensione, prima.

Harry si appoggiò allo schienale della sedia e accordò uno sguardo di grande solennità a Draco e ai suoi servitori.

«Voi chiedete: qual è il nostro obiettivo?» declamò Harry. «Posso rispondere con una parola. È la vittoria. Vittoria a tutti i costi — Vittoria malgrado qualunque terrore — Vittoria, per quanto lunga e dura possa essere la strada, perché senza vittoria non c’è –»

«\textit{Parla di Snape}», sibilò Draco. «\textit{Che cosa hai fatto?}»

Harry abbandonò la finta solennità e rivolse a Draco un’espressione più seria.

«L’hai visto. Tutti l’hanno visto. Ho schioccato le dita.»

«\textit{Harry! Smettila di prendermi in giro!}»

Quindi era stato promosso a \textit{Harry} ora. Interessante. E infatti, Harry era abbastanza sicuro che fosse previsto che se ne accorgesse, e si sentisse ingrato se non avesse contraccambiato in qualche modo…

Harry si toccò le orecchie e rivolse un’occhiata significativa ai servitori.

«Non parleranno», disse Draco.

«Draco», Harry disse, «sarò onesto al cento per cento e ti dirò che ieri non sono stato particolarmente impressionato dall’astuzia del signor Goyle.»

Il signor Goyle trasalì.

«Neppure io», disse Draco. «Gli ho spiegato che ho finito per doverti un favore, a causa di ciò.» (Il signor Goyle trasalì ancora.) «Ma \textit{esiste} una grande differenza tra quel genere di errore e l’essere indiscreto. Questo è davvero qualcosa che sono stati addestrati a capire sin dall’infanzia.»

«Allora va bene», disse Harry. Abbassò la voce, anche se i rumori di sottofondo erano divenuti confusi in presenza di Draco. «Ho dedotto uno dei segreti di Severus e ho esercitato un piccolo ricatto.»

L’espressione di Draco si indurì. «Bene, ora dimmi qualcosa che non hai detto nella più stretta confidenza a quegli idioti in Grifondoro, il che significa che quella era la storia che \textit{volevi} che fosse diffusa in tutta la scuola.»

Harry sorrise involontariamente, e seppe che Draco l’aveva notato.

«Cosa dice Severus?», chiese Harry.

«Che non si era reso conto di quanto siano sensibili i bambini piccoli», rispose Draco. «Anche in Serpeverde! Anche a \textit{me!}»

«Sei sicuro», disse Harry, «che vuoi sapere qualcosa che il responsabile della tua Casa preferirebbe che non sapessi?»

«Sì», Draco disse senza esitazione.

\textit{Interessante}. «Allora dovrai davvero mandare via i tuoi servitori, prima, perché non sono sicuro di poter credere a tutto ciò che tu credi riguardo a loro.»

Draco annuì. «Va bene.»

Il signor Crabbe e il signor Goyle sembrarono \textit{molto} scontenti. «Capo –» disse il signor Crabbe.

«Non avete dato al signor Potter nessuna ragione per fidarsi di voi», disse Draco. «Andate!»

Se ne andarono.

«In particolare», Harry disse abbassando ancora di più la voce, «non sono \textit{completamente} sicuro che non andrebbero a riferire quanto dicessi a Lucius.»

«Mio Padre non farebbe mai \textit{una cosa simile!}» disse Draco sembrando genuinamente inorridito. «Sono \textit{miei!}»

«Mi dispiace, Draco», Harry disse. «È solo che non sono sicuro di poter credere a tutto ciò che tu credi riguardo tuo padre. Immagina se fosse il tuo segreto e io ti dicessi che mio padre non lo farebbe.»

Draco annuì lentamente. «Hai ragione. Sono \textit{io} a essere dispiaciuto. È stato un mio errore chiedertelo.»

\textit{Come ho fatto a crescere} così tanto \textit{nella sua considerazione? Non dovrebbe odiarmi, ora?} Harry ebbe la sensazione di stare osservando qualcosa di sfruttabile… desiderò che il suo cervello non fosse così stanco. Normalmente avrebbe provato qualche macchinazione complicata.

«A ogni modo», disse Harry. «Facciamo uno scambio. Io ti racconto un fatto che non è tra i pettegolezzi, e non \textit{finisce} tra i pettegolezzi, e \textit{in particolare} non arriva a tuo padre, e in cambio mi dici cosa tu e i Serpeverde pensate di tutta questa faccenda.»

«Aggiudicato!»

Ora, per metterla nella maniera più vaga possibile… in qualche modo che non avrebbe causato un danno anche se si fosse saputo… «Ciò che ho detto è vero. Ho realmente scoperto uno dei segreti di Severus e ho esercitato un piccolo ricatto. Ma Severus non era l’unica persona coinvolta.»

«\textit{Lo sapevo!}» disse un Draco esultante.

Lo stomaco di Harry sprofondò. Apparentemente aveva detto qualcosa di veramente significativo e non sapeva perché. Non era un buon segno.

«Va bene», disse Draco. Ora aveva un sorriso largo. «Eccoti le reazioni in Serpeverde, allora. All’inizio, tutti gli idioti hanno detto ‘Odiamo Harry Potter! Andiamo a pestarlo!’»

Harry soffocò. «Ma cosa c’è di \textit{sbagliato} nel Cappello Smistatore? Questo non è Serpeverde, è \textit{Grifondoro} –»

«Non tutti i sono bambini prodigio» disse Draco, sebbene stesse sorridendo in un modo quasi complice, come a suggerire che privatamente concordava con l’opinione di Harry. «E a qualcuno ci sono voluti circa quindici secondi per spiegare loro perché questo sarebbe potuto non andare a favore di Snape, quindi sei al sicuro. Ad ogni modo, dopo c’è stata la seconda ondata di idioti, quelli che dicevano ‘Pare che Harry Potter sia un altro che vuole salvare il mondo, dopo tutto’».

«E poi?» disse Harry, sorridendo anche se non aveva idea del perché \textit{quello} fosse stupido.

«E poi le persone realmente intelligenti hanno iniziato a parlare. È ovvio che hai trovato un modo per mettere \textit{parecchia} pressione su Snape. E sebbene possa esserci più di un modo per farlo… l’ovvio pensiero \textit{successivo} è stato che abbia qualcosa a che fare con la misteriosa influenza di Snape su Silente. Ho ragione?»

«Nessun commento», disse Harry. Almeno il suo cervello stava elaborando questa parte correttamente. In Casa Serpeverde si erano \textit{realmente} chiesti perché Severus non fosse stato licenziato. E avevano concluso che Severus stava ricattando Silente. Poteva essere vero…? Ma Silente non era sembrato agire come se fosse stato così…

Draco continuò a parlare. «E la cosa \textit{successiva} che le persone intelligenti hanno sottolineato è stata che se hai potuto mettere abbastanza pressione su Snape da fargli lasciare in pace metà Hogwarts, questo significava che probabilmente avevi abbastanza potere da sbarazzarti di lui completamente, se avessi voluto. Quello che gli hai fatto è stato umiliarlo, proprio come lui ha cercato di umiliare te — ma ci hai lasciato il Preside della nostra Casa».

Harry rese il proprio sorriso più ampio.

«E poi le persone \textit{veramente} intelligenti», disse Draco, il suo volto ora serio, «si sono appartate e hanno avuto una piccola discussione tra di loro, e qualcuno ha sottolineato che sarebbe stata una cosa molto stupida lasciare in giro un nemico così. Se avessi potuto spezzare la sua influenza su Silente, la cosa più ovvia sarebbe stata quella di farlo e basta. Silente avrebbe cacciato Severus via da Hogwarts e forse l’avrebbe anche fatto uccidere, ti sarebbe stato \textit{molto} riconoscente, e tu non avresti dovuto preoccuparti che Severus entri furtivamente di notte nella tua stanza del dormitorio con delle pozioni interessanti.»

Il viso di Harry era ora neutro. Non ci aveva pensato, e avrebbe davvero, \textit{davvero} dovuto farlo. «E da ciò avete concluso…?»

«Che l’influenza di Severus fosse un segreto di Silente e che \textit{tu conosci quel segreto!}» Draco sembrava esultante. «Non può essere abbastanza potente da distruggere Silente completamente, o Snape l’avrebbe già usata. Snape rifiuta di usare la sua influenza per qualsiasi cosa eccetto restare signore incontrastato della Casa di Serpeverde di Hogwarts, e neppure in questo caso riesce sempre a ottenere ciò che vuole, quindi deve avere dei limiti. Ma \textit{deve} essere veramente buona! Mio Padre sta cercando di convincere Severus a rivelargliela da anni!»

«E», disse Harry, «ora Lucius pensa che forse \textit{io} potrò rivelargliela. Per caso, hai già ricevuto un gufo –»

«Lo riceverò questa sera», Draco disse, e rise. «Dirà», la sua voce assunse una cadenza differente, più formale, «\textit{Mio amato figlio: ti ho già detto della potenziale importanza di Harry Potter. Come avrai già capito, la sua importanza è diventata ora più grande e più urgente. Se notassi un possibile modo per stringere un’amicizia o per scoprire un suo punto debole, devi perseguirlo, e tutte le risorse di Malfoy sono a tua disposizione in caso di necessità.}»

Accidenti. «Bene», Harry disse, «evitando ogni commento sul fatto che tutto il tuo complicato edificio teorico sia vero o meno, lasciami dire che non siamo ancora così buoni amici.»

«Lo so», disse Draco. Poi il suo viso divenne \textit{molto} serio, e la sua voce si fece più bassa, malgrado i rumori esterni fossero già attutiti. «Harry, ti è mai venuto in mente che se tu sapessi qualcosa che Silente non vuole che sia risaputa, potrebbe semplicemente farti uccidere? E il Ragazzo-Che-È-Sopravvissuto passerebbe da potenziale concorrente capo-fazione a prezioso martire, persino.»

«Nessun commento», Harry disse di nuovo. Non aveva pensato neppure a quest’ultima cosa. Non \textit{sembrava} essere nello stile di Silente… ma…

«Harry», riprese Draco, «è evidente che tu possegga un talento \textit{impressionante}, ma non hai alcun addestramento né un mentore, e commetti davvero delle stupidaggini, talvolta, e \textit{hai davvero bisogno di un consigliere che sappia come si fa, o ti farai male!}» Il volto di Draco era feroce.

«Ah», disse Harry. «Un consigliere come Lucius?»

«Come \textit{me!}» rispose Draco. «Ti prometto di mantenere i tuoi segreti con mio Padre, con \textit{chiunque}, mi limiterò ad aiutarti a capire quello che vuoi fare!»

Uau.

Harry vide zombi-Quirrell che attraversava barcollando le porte.

«La lezione sta per iniziare», disse Harry. «Penserò a quello che mi hai detto, parecchie volte mi capita di desiderare di aver ricevuto tutto il tuo addestramento, è solo che non so come posso fidarmi di te così in fretta –»

«Non dovresti», disse Draco, «è troppo presto. Vedi? Ti darò buoni consigli anche se mi danneggiano. Ma forse dovremmo \textit{sbrigrci} e diventare amici più intimi.»

«Sono aperto a questa proposta», disse Harry, che stava già cercando di capire come sfruttarla.

«Un altro piccolo consiglio», disse Draco frettolosamente mentre Quirrell si trascinava verso la cattedra, «in questo momento in Serpeverde tutti si interrogano su di te, quindi se ci stai corteggiando, cosa che penso tu stia facendo, dovresti compiere qualche gesto che sia un segnale di amicizia verso Serpeverde. \textit{Presto}, tipo oggi o domani.»

«Permettere che Severus continui ad assegnare punti-Casa supplementari a Serpeverde non è stato sufficiente?» Non c’era ragione per cui Harry non si prendesse il merito di quello.

Le palpebre di Draco batterono per la comprensione, poi egli disse rapidamente, «Non è la stessa cosa, credimi, deve essere qualcosa di ovvio. Spingi contro un muro la tua rivale sanguemarcio Granger o qualcosa del genere, tutti in Serpeverde capiranno cosa significa –»

«Non è \textit{così} che funziona in Corvonero, Draco! Se devi spingere qualcuno contro un muro significa che il tuo cervello è troppo \textit{debole} per batterlo nel modo giusto e dentro Corvonero lo sanno \textit{tutti} –»

Lo schermo sulla scrivania di Harry si accese con uno sfarfallio, causando un’improvvisa ondata di nostalgia per la televisione e i computer.

«Ehm», disse la voce del professor Quirrell, che sembrò parlare personalmente a Harry dallo schermo. «Prego, raggiungete i vostri posti.»

\begin{figure}[h!]
        \includegraphics[scale=0.4]{boccino.png}
        \centering
\end{figure}

E i bambini erano tutti seduti a guardare gli schermi ripetitori sui loro banchi, o a guardare in basso direttamente al grande palcoscenico di marmo bianco, su cui il professor Quirrell stava in piedi, appoggiato alla sua cattedra in cima alla piccola pedana di marmo scuro.

«Oggi», disse il professor Quirrell, «avevo previsto di insegnarvi il vostro primo incantesimo difensivo, un piccolo scudo che fu l’antenato del \textit{Protego} odierno. Ma riflettendoci, ho cambiato il programma della lezione di oggi alla luce dei recenti avvenimenti.»

Lo sguardo del professor Quirrell cercò tra le file di sedili. Là dove era seduto, in ultima fila, Harry fece una smorfia. Aveva la sensazione di sapere chi stava per essere chiamato.

«Draco, della Nobile e Antichissima Casa Malfoy», disse il professor Quirrell.

Scampata.

«Sì, professore?» disse Draco. La sua voce era amplificata, e proveniva apparentemente dallo schermo ripetitore sul banco di Harry, che mostrava il volto di Draco mentre parlava. Poi lo schermo tornò al professor Quirrell, che disse:

«È sua ambizione diventare il prossimo Signore Oscuro?»

«Questa è una domanda strana, professore», disse Draco. «Voglio dire, chi sarebbe così stupido da ammetterlo?»

Alcuni studenti risero, ma non molti.

«Infatti», disse il professor Quirrell. «Ma sebbene sia inutile chiederlo a qualcuno di voi, non mi sorprenderebbe minimamente se ci fossero uno studente o due nelle mie classi che nutrissero l’ambizione di essere il prossimo Signore Oscuro. Dopo tutto, \textit{io} volevo essere il prossimo Signore Oscuro, quando ero un giovane Serpeverde.»

Questa volta la risata fu molto più diffusa.

«Beh, dopo tutto \textit{è} la Casa degli ambiziosi», disse il professor Quirrell, sorridendo. «Mi resi conto solo più tardi che quello che mi piaceva davvero era la Magia da Battaglia, e che la mia vera ambizione era di diventare un grande mago da combattimento e un giorno insegnare a Hogwarts. In ogni caso, quando avevo tredici anni, lessi attentamente le sezioni storiche della biblioteca di Hogwarts, vagliando le vite e i destini dei passati Signori Oscuri, e feci una lista di tutti gli errori che \textit{io} non avrei mai fatto quando fossi diventato un Signore Oscuro –»

Harry ridacchiò prima di riuscire a trattenersi.

«Sì, signor Potter, molto divertente. Allora, signor Potter, può indovinare quale fosse il primo elemento in assoluto su quella lista?»

\textit{Grande}. «Uhm… non usare mai un modo complicato di affrontare un nemico quando puoi semplicemente lanciargli contro un Abracadabra?»

«Le \textit{parole}, signor Potter, sono \textit{Avada Kedavra}», la voce del professor Quirrell suonò un po’ brusca, per qualche motivo, «e no, quello \textit{non era} sulla lista che stilai all’età di tredici anni. Le andrebbe di provare di nuovo?»

«Ah… mai vantarsi con nessuno del tuo piano malvagio?»

Il professor Quirrell rise. «Ah, beh \textit{quello} era al numero due. Accidenti, signor Potter, abbiamo forse letto gli stessi libri?»

Ci furono più risate, con una venatura di nervosismo. Harry serrò la mascella e non disse niente. Una smentita non sarebbe servita a nulla.

«Ma no. Il \textit{primo} elemento era, ‘Non me ne andrò in giro a provocare nemici forti e feroci’. La storia del mondo sarebbe molto diversa se Mornelithe Falconsbane o Hitler avessero compreso questo punto elementare. Ora \textit{se}, signor Potter — solo \textit{se} per caso lei nutrisse un’ambizione simile a quella che io avevo da giovane Serpeverde — anche in quel caso, spero non sia sua ambizione diventare un Signore Oscuro \textit{stupido.}»

«Professor Quirrell», disse Harry digrignando i denti, «io sono un \textit{Corvonero} e non è mia ambizione essere stupido, punto. So che ciò che ho fatto oggi era sciocco. Ma non era \textit{Oscuro!} Non sono stato \textit{io} a dare il primo pugno in quello scontro!»

«Lei, signor Potter, è un idiota. Ma del resto lo ero anch’io alla sua età. Così ho anticipato la sua risposta e mutato di conseguenza il piano della lezione di oggi. Signor Gregory Goyle, vorrebbe farsi avanti, per cortesia?»

Ci fu una pausa sorpresa nell’aula. Harry non se l’aspettava.

Né, a quanto pare, se l’aspettava il signor Goyle, che sembrò piuttosto incerto e preoccupato mentre salì sul palco di marmo e si avvicinò alla pedana.

Il professor Quirrell si raddrizzò allontanandosi dalla scrivania su cui si era appoggiato. Sembrò improvvisamente più forte, le sue mani si strinsero a pugno e assunse una posa da arti marziali chiaramente riconoscibile.

A quella vista, gli occhi di Harry si spalancarono, e comprese perché il signor Goyle era stato chiamato.

«La maggior parte dei maghi», disse il professor Quirrell, «non sprecano molto tempo con quelle che un Babbano chiamerebbe arti marziali. Non è forse una bacchetta più forte di un pugno? Questo atteggiamento è stupido. Le bacchette sono strette nei pugni. Se volete essere un grande mago da combattimento \textit{dovete} imparare le arti marziali a un livello tale da impressionare anche un Babbano. Passo ora a dare prova di una certa tecnica di vitale importanza, che ho appreso in un \textit{dojo}, una scuola babbana di arti marziali, di cui parlerò tra breve. Per ora…» Il professor Quirrell fece alcuni passi in avanti, ancora in posizione, avanzando verso il luogo dove era il signor Goyle. «Signor Goyle, le chiedo di attaccarmi.»

«Professor Quirrell», disse il signor Goyle, la sua voce ora amplificata come quella del professore, «posso chiederle a che livello –»

«Sesto \textit{dan}. Non si farà del male e neppure me ne farò io. E se vede un varco, la prego di sfruttarlo.»

Il signor Goyle annuì, sembrando molto sollevato.

«Notate», disse il professor Quirrell, «che il signor Goyle aveva paura di attaccare qualcuno che non conoscesse le arti marziali a un livello accettabile, per timore che io, o lui, ci saremmo fatti male. L’atteggiamento del signor Goyle è perfettamente corretto e perciò si è guadagnato tre punti-Quirrell. Ora, attacchi!»

Il giovane ragazzo si gettò avanti, pugni in aria, e il professore bloccò ogni colpo, danzando all’indietro, Quirrell calciò e Goyle bloccò e si girò, e cercò far inciampare Quirrell spazzando con la gamba e Quirrell la scavalcò e tutto stava accadendo troppo in fretta perché Harry potesse dare un senso a quello che succedeva e poi Goyle fu sulla propria schiena con le gambe che spingevano e Quirrell letteralmente \textit{volò per aria} e poi colpì terra con la spalla e rotolò via.

«Fermo!» gridò il professor Quirrell da terra, suonando un po’ impaurito. «Ha vinto!»

Il signor Goyle si fermò così bruscamente che barcollò, quasi inciampando e cadendo a causa del momento abortito della sua carica a testa bassa verso il professor Quirrell. Il suo volto mostrava un completo disorientamento.

Il professor Quirrell inarcò la schiena e si rialzò in piedi usando un peculiare movimento a molla che non faceva uso di mani.

Ci fu silenzio in aula, un silenzio nato dalla completa confusione.

«Signor Goyle», disse il professor Quirrell, «quale tecnica di importanza vitale ho dimostrato?»

«Come cadere in maniera corretta quando si è proiettati da qualcuno», disse il signor Goyle. «È una delle prime lezioni che si imparano –»

«Anche quella», disse il professor Quirrell.

Ci fu una pausa.

«La tecnica di importanza vitale che ho dimostrato», disse il professor Quirrell, «è stata come perdere. Può andare, signor Goyle, grazie.»

Il signor Goyle abbandonò la piattaforma, sembrando alquanto frastornato. Harry si sentiva nello stesso modo.

Il professor Quirrell tornò alla cattedra e riprese ad appoggiarcisi. «A volte dimentichiamo le cose più elementari, dal momento che è passato troppo tempo da quando le abbiamo imparate. Mi sono accorto che ho fatto lo stesso con il mio programma di lezioni. Non insegni agli studenti a lanciare fino a quando non hai insegnato loro a cadere. E non devo insegnarvi a combattere se non sapete come perdere.»

Il volto del professor Quirrell si indurì, e Harry credette di vedere un accenno di dolore, un tocco di tristezza, in quegli occhi. «Ho imparato a perdere in un \textit{dojo} in Asia, che, come ogni Babbano sa, è dove vivono tutti i bravi praticanti di arti marziali. Questo \textit{dojo} insegnava uno stile che tra i maghi da combattimento aveva la reputazione di adattarsi bene al duello di magia. Il Maestro di quel \textit{dojo} — un uomo anziano secondo i criteri babbani — era il più grande maestro vivente di tale stile. Non aveva idea dell’esistenza della magia, ovviamente. Feci richiesta di studiare lì, e fui uno dei pochi allievi ammessi per quell’anno, tra i tanti candidati. In quella decisione potrebbe aver giocato un ruolo un po’ di condizionamento speciale.»

Ci furono alcune risate dalla classe. Harry non le condivise. Quell’atto non era stato affatto corretto.

«Ad ogni modo. Durante uno dei miei primi combattimenti, dopo che ero stato battuto in un modo particolarmente umiliante, persi il controllo e attaccai il mio avversario –»

\textit{Accidenti.}

«– per fortuna con i pugni, piuttosto che con la magia. Il Maestro, sorprendentemente, non mi espulse immediatamente. Ma mi disse che c’era un difetto nel mio temperamento. Me lo spiegò, e io seppi che aveva ragione. E poi mi disse che avrei dovuto imparare a perdere.»

Il volto del professor Quirrell era inespressivo.

«Dietro suoi precisi ordini, tutti gli allievi del \textit{dojo} si misero in fila. Uno per uno, si avvicinarono a me. Il mio ordine era di \textit{non} difendermi. Dovevo solo chiedere pietà. Uno per uno, mi schiaffeggiarono, o mi diedero un pugno, e mi spinsero a terra. Alcuni di loro mi sputarono addosso. Mi chiamarono con nomi terribili nella loro lingua. E a ciascuno di loro, dovevo dire ‘ho perso!’ e cose simili, come ad esempio ‘ti prego di smetterla!’ e ‘riconosco che sei migliore di me!’»

Harry stava cercando di immaginare la scena e semplicemente non ci riusciva. Non c’era modo che una cosa del genere fosse potuta accadere al fiero professor Quirrell.

«Già allora ero un prodigio in Magia da Battaglia. Solo con la magia senza bacchetta avrei potuto uccidere tutti, in quel \textit{dojo}. Non lo feci. Imparai a perdere. Fino a oggi me la ricordo come una delle ore più sgradevoli della mia vita. E quando lasciai quel \textit{dojo} otto mesi dopo — che non furono neppure lontanamente sufficienti, ma erano tutto il tempo che potevo permettermi di passare lì — il Maestro mi disse che sperava avessi compreso perché era stato necessario. E gli risposi che era una delle lezioni più preziose che avessi mai imparato. Cosa che era, ed è, vera.»

L’espressione del professor Quirrell divenne più amara. «Vi starete chiedendo dove sia questo meraviglioso \textit{dojo}, e se possiate studiare lì. Non potete. Perché non molto tempo dopo, un altro potenziale allievo giunse in quel luogo nascosto, in quella remota montagna. Colui-Che-Non-Deve-Essere-Nominato.»

Ci fu il suono di molte inspirazioni simultanee. Harry sentì una stretta allo stomaco. Sapeva cosa stava per accadere.

«Il Signore Oscuro giunse in quella scuola apertamente, senza travestimento, con gli occhi che brillavano di rosso e tutto il resto. Gli allievi tentarono di ostacolare il suo ingresso ed egli, semplicemente, li superò Materializzandosi. Ci fu terrore, ma anche disciplina, e il Maestro si fece avanti. E il Signore Oscuro pretese — non chiese, ma pretese — che gli insegnasse.»

L’espressione del professor Quirrell era molto dura. «Forse il Maestro aveva letto troppi libri che raccontavano la bugia che un vero praticante di arti marziali può sconfiggere anche i demoni. Qualunque fosse la ragione, il Maestro rifiutò. Il Signore Oscuro chiese perché non potesse essere un allievo. Il Maestro rispose che non aveva pazienza, e fu allora che il Signore Oscuro gli strappò la lingua.»

Ci fu un sussulto collettivo.

«Potete indovinare che cosa è successo dopo. Gli allievi cercarono di avventarsi sul Signore Oscuro e caddero, storditi sul posto. E poi…»

La voce del professor Quirrell esitò per un momento, poi riprese.

«C’è una Maledizione Senza Perdono, la Maledizione Cruciatus, che produce un dolore insopportabile. Se la Cruciatus è mantenuta per più di qualche minuto produce una pazzia permanente. Uno a uno, il Signore Oscuro sottopose gli allievi del Maestro al Cruciatus fino alla follia, e poi li finì con la Maledizione Mortale, mentre il Maestro fu costretto a guardare. Quando tutti i suoi allievi furono morti in questo modo, il Maestro li seguì. Ho imparato tutto ciò dall’unico allievo sopravvissuto, che il Signore Oscuro lasciò in vita per raccontare la storia, e che era stato un mio amico…»

Il professor Quirrell si voltò, e quando si rigirò un momento dopo, sembrò ancora una volta calmo e composto.

«I Maghi Oscuri non riescono a controllare il proprio temperamento», disse con calma il professor Quirrell. «È un difetto quasi universale della loro specie, e chiunque faccia l’abitudine a combatterli impara presto a farvi affidamento. Dovete comprendere che quel giorno il Signore Oscuro \textit{non vinse}. Il suo scopo era imparare le arti marziali, eppure se ne andò senza una sola lezione. Il Signore Oscuro fu folle a volere che quella storia fosse raccontata. Non mostrò la sua forza, ma piuttosto una debolezza sfruttabile.»

Lo sguardo del professor Quirrell si focalizzò su di un singolo bambino nell’aula.

«Harry Potter», disse il professor Quirrell.

«Sì», disse Harry, la sua voce roca.

«In cosa \textit{esattamente} ha sbagliato oggi, signor Potter?»

Harry si sentiva come se stesse per vomitare. «Ho perso la calma.»

«\textit{Non è} preciso», disse il professor Quirrell. «Lo descriverò più precisamente. Ci sono molte specie animali in cui si praticano i cosiddetti giochi di dominanza. Si scontrano impattando con le corna — cercando di buttarsi giù a vicenda, non di trafiggersi l’un l’altro. Combattono con le zampe — con gli artigli retratti. Ma perché con gli artigli retratti? Sicuramente, se usassero gli artigli, avrebbero una probabilità maggiore di vincere, no? Ma in quel caso il loro avversario potrebbe ugualmente sfoderare gli artigli, e invece di risolvere il gioco di dominanza con un vincitore e un perdente, entrambi potrebbe essere gravemente feriti.»

Lo sguardo del professor Quirrell sembrò puntare dallo schermo ripetitore direttamente a Harry. «Quello che ha dimostrato oggi, signor Potter, è che — a differenza di quegli animali che tengono i loro artigli retratti e accettano i verdetti — lei non sa come perdere un gioco di dominanza. Quando un \textit{professore di Hogwarts} l’ha sfidata, lei non si è tirato indietro. Quando è sembrato che potesse perdere, lei ha tirato fuori gli artigli noncurante del pericolo. Lei ha \textit{alzato il livello dello scontro}, e poi l’ha alzato ancora. È iniziato con uno schiaffo datole dal professor Snape, che era ovviamente dominante su di lei. Invece di perdere, lei ha restituito lo schiaffo e ha perso dieci punti per Corvonero. Poco dopo stava parlando di lasciare Hogwarts. Il fatto che lei abbia intensificato il conflitto ancor di più in una direzione sconosciuta, e in qualche modo alla fine abbia vinto, non cambia il fatto che lei sia un idiota.»

«Comprendo», disse Harry. La sua gola era secca. Era stato \textit{preciso. Spaventosamente} preciso. Ora che il professor Quirrell l’aveva esposta, Harry poteva vedere col senno di poi che quella era una descrizione \textit{accurata} di ciò che era successo. Quando il modello che qualcuno ha di te è così buono, devi chiederti se abbia ragione anche a proposito di altro, come la tua volontà di uccidere.

«La \textit{prossima} volta, signor Potter, che sceglierà di alzare il livello di uno scontro piuttosto che ammettere la sconfitta, lei potrebbe perdere \textit{l’intera} posta in gioco. Non so quale fosse quella di oggi. So che era molto, molto più importante della perdita di dieci punti-Casa.»

Come il destino della Gran Bretagna magica. Quello era ciò che aveva fatto.

«Protesterà dicendo che stava cercando di aiutare l’intera Hogwarts, uno scopo ben più importante e degno di maggiori rischi. Questa è una \textit{bugia}. Se fosse stato questo –»

«Avrei incassato lo schiaffo, atteso, e scelto il miglior momento possibile per fare la mia mossa», disse Harry, la sua voce roca. «Ma questo avrebbe significato \textit{perdere}. Permettergli di dominarmi. Questo è quello che il Signore Oscuro non è stato in grado di fare col Maestro da cui voleva imparare.»

Il professor Quirrell annuì. «Vedo che ha compreso perfettamente. E quindi, signor Potter, oggi lei imparerà come perdere.»

«Io –»

«Non voglio sentire obiezioni, signor Potter. È evidente che lei ne abbia bisogno, e che è abbastanza forte da sopportarlo. Le assicuro che la sua esperienza non sarà così severa come quella che ho dovuto affrontare io, sebbene potrebbe ricordarseli come i peggiori quindici minuti della sua giovane vita.»

Harry deglutì. «Professor Quirrell», disse a voce bassa, «potremmo farlo un’altra volta?»

«No», rispose semplicemente il professor Quirrell. «Lei è al quinto giorno dei suoi studi a Hogwarts ed è già accaduto tutto questo. Oggi è venerdì. La nostra \textit{prossima} lezione di difesa è mercoledì. Sabato, domenica, lunedì, martedì, mercoledì… No, \textit{non abbiamo} il tempo per aspettare.»

Ci furono poche risate in risposta, ma davvero poche.

«La prego di considerarlo un ordine del suo professore, signor Potter. Quello che vorrei dirle è che in caso contrario non le insegnerò alcun incantesimo d’attacco, perché verrei poi a sapere che ha gravemente ferito o persino ucciso qualcuno. Sfortunatamente mi dicono che le sue dita sono già armi potenti. Non le schiocchi mai durante questa lezione.»

Altre risate sparpagliate, dal tono alquanto nervoso.

Harry sentì di potersi mettere a piangere. «Professor Quirrell, se farà qualcosa come quello che ha descritto, mi arrabbierò, e davvero preferirei non arrabbiarmi ancora, oggi –»

«Il punto \textit{non} è evitare di arrabbiarsi», disse il professor Quirrell, il suo volto severo. «La rabbia è naturale. Deve imparare come perdere anche quando è arrabbiato. O almeno \textit{fingere} di perdere così da poter \textit{pianificare} la sua vendetta. Come ho fatto io oggi col signor Goyle, a meno che qualcuno di voi non creda che sia \textit{davvero} migliore di me –»

«Io no!» gridò il signor Goyle dal suo banco, sembrando un po’ agitato. «Io so che non ha realmente perso! La prego, non mediti nessuna vendetta!»

Harry sentì una stretta allo stomaco. Il professor Quirrell non era a conoscenza del suo misterioso lato oscuro. «Professore, è realmente necessario che ne discutiamo dopo la lezione –»

«Lo faremo», promise il professor Quirrell. «Dopo che avrà imparato come si perde.» Il suo volto era serio. «Inutile dire che eviterò qualunque cosa che potrebbe ferirla o persino causarle un forte dolore. Il dolore sarà causato dalla difficoltà di perdere, invece di combattere a sua volta e intensificare lo scontro fino alla vittoria.»

Il respiro di Harry si ruppe in brevi e terrorizzati ansimi. Era più spaventato ora di quanto fosse stato lasciando la lezione di Pozioni. «Professor Quirrell», riuscì a dire, «non voglio che lei sia licenziato per questo –»

«Non lo sarò», disse il professor Quirrell, «se dopo \textit{lei} dirà che è stato necessario. E sono fiducioso che lo farà.» Per un momento la voce del professor Quirrell divenne secca. «Mi creda, hanno tollerato ben peggio nei loro corridoi. Questo caso sarà eccezionale solo perché avrà luogo in aula.»

«Professor Quirrell», sussurrò Harry, ma pensò che la sua voce fosse ancora trasmessa ovunque, «ritiene davvero che se non lo facessi, potrei far del male a qualcuno?»

«Sì», disse semplicemente il professor Quirrell.

«Allora», Harry sentiva la nausea crescere, «lo farò.»

Il professor Quirrell si girò a considerare i Serpeverde. «Dunque… con la piena approvazione del vostro professore, e in maniera tale che Snape non possa essere biasimato per le vostre azioni… qualcuno di voi desidera mostrare la propria supremazia sul Ragazzo-Che-È-Sopravvissuto? Spintonarlo, buttarlo a terra, ascoltarlo mentre implora la vostra pietà?»

Cinque mani si alzarono.

«Tutti quelli con la mano alzata sono dei completi idioti. Quale parte di \textit{fingere} di perdere non avete compreso? Se Harry Potter diventasse realmente il prossimo Signore Oscuro, vi darà la caccia e vi ucciderà dopo che si sarà diplomato.»

Le cinque mani scesero repentinamente sui banchi.

«Non lo farò», disse Harry, la sua voce uscì piuttosto debole. «Giuro che non mi vendicherò mai di coloro che mi aiuteranno a imparare a perdere. Professor Quirrell… \textit{la prego… smetterebbe} di fare così?»

Il professor Quirrell sospirò. «Sono \textit{davvero} dispiaciuto, signor Potter. Comprendo che debba trovare tutto questo molto seccante, che intenda diventare un Signore Oscuro o meno. Ma \textit{anche} quei bambini hanno un’importante lezione di vita da imparare. Sarebbe accettabile se le concedessi un punto-Quirrell per scusarmi?»

«Facciamo due», Harry disse.

Ci fu un’onda di risate sorprese, che sdrammatizzarono un po’ della tensione.

«Fatto», disse il professor Quirrell.

«E dopo che mi sarò diplomato vi darò la caccia e vi farò il \textit{solletico.}»

Ci furono altre risate, sebbene il professor Quirrell non sorridesse.

Harry si sentì come se stesse combattendo contro un anaconda, cercando di obbligare la conversazione all’interno di uno stretto percorso che avrebbe fatto comprendere alle persone che non era un Signore Oscuro, dopo tutto… \textit{perché} il professor Quirrell sospettava di lui?

«Professore», disse la voce non amplificata di Draco. «Diventare un Signore Oscuro stupido non è neppure una mia ambizione.»

Ci fu un silenzio sconcertato nell’aula.

\textit{Non sei obbligato a farlo!} scappò quasi dalla bocca di Harry, ma si controllò in tempo; Draco avrebbe potuto desiderare che non fosse noto che lo stava facendo per la sua amicizia con Harry… o per il suo desiderio di sembrare amico…

Aver chiamato \textit{quel gesto} un \textit{desiderio di sembrare amico} fece sentire Harry piccolo, e meschino. Se Draco aveva inteso impressionarlo, ci stava riuscendo perfettamente.

Il professor Quirrell stava considerando Draco con serietà. «\textit{Lei} dubita di non essere in grado di fingere, signor Malfoy? Che questo difetto che descrive il signor Potter descriva anche lei? \textit{Certamente} suo padre sarà stato in grado di insegnarglielo.»

«Quando si tratta di discutere, forse», disse Draco, ora sullo schermo ripetitore. «Non quando si tratta di essere spintonato e gettato a terra. Voglio essere forte proprio come lei, professor Quirrell.»

Le sopracciglia del professor Quirrell si alzarono e rimasero su. «Sono spiacente, signor Malfoy», disse dopo un po’ di tempo, «gli accordi che ho preso per il signor Potter, e che coinvolgono alcuni Serpeverde più grandi a cui sarà detto solo \textit{dopo} quanto siano stati stupidi, non si possono applicare a lei. Ma è mia opinione professionale che lei sia già molto forte. Dovessi venire a sapere che lei ha fallito, come ha fallito oggi il signor Potter, prenderò gli accordi appropriati e chiederò scusa a lei e a chiunque avesse ferito. Non penso che questo sarà necessario, ad ogni modo.»

«Capisco, Professore», disse Draco.

Il professor Quirrell diede un’occhiata alla classe. «Qualcun altro desidera diventare forte?»

Alcuni studenti si guardarono in giro nervosamente. Alcuni, pensò Harry dall’ultima fila, sembrarono aprire la bocca, ma non dire nulla. Alla fine, nessuno parlò.

«Draco Malfoy sarà uno dei generali degli eserciti del vostro anno», disse il professor Quirrell, «se dovesse compiacersi di prendere parte a quell’attività dopo-scuola. E ora, signor Potter, la prego di venire avanti.»

\begin{figure}[h!]
        \includegraphics[scale=0.4]{boccino.png}
        \centering
\end{figure}

\textit{Sì}, aveva detto il professor Quirrell, \textit{deve essere di fronte a tutti, di fronte ai suoi amici, perché è lì che Snape l’ha affrontata ed è lì che deve imparare a perdere.}

Così ora l’intero primo anno stava guardando. In un silenzio imposto magicamente, e con la richiesta sia di Harry sia del professore di non intervenire. Hermione aveva girato il viso per non vedere, ma non aveva parlato o anche solo indirizzato a Harry un’occhiata significativa, forse perché anche lei era stata presente a Pozioni.

Harry era in piedi su di un morbido tappeto blu, come se ne trovavano in un \textit{dojo} babbano, che il professor Quirrell aveva disposto sul pavimento per quando Harry fosse stato gettato a terra.

Harry aveva paura di quello che avrebbe potuto fare. Se il professor Quirrell aveva ragione circa la sua volontà di uccidere…

La bacchetta di Harry era poggiata sulla cattedra del professor Quirrell, non perché Harry conoscesse qualche incantesimo con cui difendersi, ma perché altrimenti (pensò Harry) avrebbe potuto cercare di spingerla attraverso l’orbita oculare di qualcuno. Anche la sua borsa giaceva lì, con all’interno il suo Giratempo, ora protetto ma ancora potenzialmente fragile.

Harry aveva implorato il professor Quirrell di Trasfigurare dei guantoni e fissarglieli alle mani. Il professor Quirrell gli aveva rivolto uno sguardo di muta comprensione, e aveva rifiutato.

\textit{Non colpirò i loro occhi, non colpirò i loro occhi, non colpirò i loro occhi, sarebbe la fine della mia vita a Hogwarts, verrei arrestato}, Harry recitò a sé stesso, cercando di imprimere quel pensiero nella propria mente, sperando che rimanesse lì se la sua volontà di uccidere avesse preso il sopravvento.

Il professor Quirrell tornò, scortando tredici Serpeverde più grandi, di vari anni. Harry riconobbe uno di loro come quello che aveva colpito con una torta. C’erano altri due presenti a quel confronto. Quello che aveva detto di fermarsi, che davvero non dovevano farlo, mancava.

«Ripeto», disse il professor Quirrell con un tono molto severo, «\textit{non dovete} fare realmente del male a Potter. Qualunque \textit{incidente} sarà considerato intenzionale. Avete capito?»

I Serpeverde più grandi annuirono, sogghignando.

«Allora sentitevi liberi di fargli abbassare un po’ la cresta», disse il professor Quirrell, con un sorriso contorto che solo quelli del primo anno compresero.

Per una qualche sorta di accordo, la vittima del lancio della torta era in testa al gruppo.

«Potter», disse il professor Quirrell, «le presento il signor Peregrine Derrick. È migliore di lei e sta per dimostrarglielo.»

Derrick avanzò a grandi passi e il cervello di Harry urlò in maniera dissonante, non doveva scappare, non doveva reagire –

Derrick si fermò a un passo di distanza da Harry.

Harry non era ancora arrabbiato, solo spaventato. E questo significò che vide un ragazzo un mezzo metro abbondante più alto di lui, con muscoli ben definiti, peluria sul viso, e un sorriso di terrificante trepidazione.

«Gli chieda di non farle del male», disse il professor Quirrell. «Forse se la trovasse abbastanza commovente, potrebbe decidere che lei non è interessante e andarsene.»

Dai Serpeverde più grandi che stavano guardando giunse una risata.

«Ti prego», disse Harry, la sua voce tremante, «non, farmi, del, male,…»

«Non sembrava molto sincero», disse il professor Quirrell.

Il sorriso di Derrick si allargò. Quel goffo imbecille sembrava molto borioso e…

… la temperatura del sangue di Harry stava scendendo…

«Ti prego, non farmi del male», Harry tentò di nuovo.

Il professor Quirrell scosse la testa. «In nome di Merlino, come ha fatto a farlo sembrare un insulto, Potter? C’è soltanto una risposta possibile che si può attendere dal signor Derrick.»

Derrick si fece avanti, e urtò deliberatamente Harry.

Harry barcollò all’indietro per qualche metro e, prima di riuscire a impedirselo, si raddrizzò gelidamente.

«Sbagliato», disse il professor Quirrell, «sbagliato, sbagliato, sbagliato.»

«Mi hai urtato, Potter», lo accusò Derrick. «Scusati.»

«Mi dispiace!»

«Non \textit{sembri} dispiaciuto», rispose Derrick.

Gli occhi di Harry si spalancarono per l’indignazione, \textit{era} riuscito a farla sembrare una supplica –

Derrick lo spinse, duramente, e Harry cadde al tappeto sulle mani e sulle ginocchia.

Il tessuto blu sembrò ondeggiare davanti agli occhi di Harry, non troppo lontano.

Stava cominciando a dubitare delle reali motivazioni del professor Quirrell nell’insegnare quella cosiddetta \textit{lezione}.

Un piede poggiò sulle natiche di Harry, che un attimo dopo fu spinto di lato con forza, finendo disteso sulla schiena.

Derrick rise. «\textit{È divertente}», disse.

Tutto quello che doveva fare era dire che era finita lì. E segnalare l’intera faccenda all’ufficio del Preside. Sarebbe stata la fine di questo \textit{Professore di Difesa} e del suo sventurato soggiorno a Hogwarts e… la professoressa McGonagall si sarebbe arrabbiata per questo, ma…

(Un’immagine del volto della professoressa McGonagall balenò davanti ai suoi occhi, non sembrava arrabbiata, solo triste –)

«Ora gli dica che è meglio di lei, Potter», disse la voce del professor Quirrell.

«Tu sei, meglio, di, me.»

Harry iniziò a sollevarsi e Derrick gli messe un piede sul petto e lo spinse di nuovo giù sul tappeto.

Il mondo stava diventando trasparente come un cristallo. Le possibili linee di azione e le loro conseguenze si dipanarono davanti ai suoi occhi in assoluta chiarezza. Lo stolto non si aspettava che lo attaccasse a sua volta, un rapido colpo all’inguine l’avrebbe stordito abbastanza a lungo per –

«Provi ancora», disse il professor Quirrell e con un improvviso movimento deciso Harry rotolò e balzò in piedi e si girò di scatto verso la direzione in cui si trovava il suo vero nemico, il Professore di Difesa –

Il professor Quirrell disse, «Lei non ha pazienza».

Harry vacillò. La sua mente, addestrata al pessimismo, disegnò l’immagine di un vecchio raggrinzito col sangue che sgorgava dalla bocca dopo che Harry gli aveva strappato via la lingua –

Un momento dopo, Derrick spinse Harry nuovamente al tappeto e poi si sedette su di lui, facendo espellere il fiato a Harry con un sibilo.

«Smettila!» Harry urlò. «Ti prego, smettila!»

«Meglio», disse il professor Quirrell. «Questa volta sembrava persino sincero.»

\textit{Era} stato sincero. Quella era la cosa orribile, la cosa nauseante, che \textit{era} stato sincero. Harry ansimava freneticamente, la paura e la rabbia fredda che scorrevano entrambe attraverso di lui –

«Perdi», disse il professor Quirrell.

«Io, ho perso», Harry si costrinse a dire.

«Mi piace», disse Derrick sopra di lui. «Perdi ancora.»

\begin{figure}[h!]
        \includegraphics[scale=0.4]{boccino.png}
        \centering
\end{figure}

Delle mani spintonarono Harry, mandandolo a incespicare dall’altra parte del cerchio dei Serpeverde più grandi verso altre mani che lo spintonarono nuovamente. Harry aveva già smesso da parecchio di provare a non piangere, e ora stava semplicemente provando a non cadere a terra.

«Cosa sei tu, Potter?» chiese Derrick.

«Un, p-perdente, ho p-perso, mi arrendo, avete vinto, siete migliori, d-di me, vi prego, smettetela –»

Harry inciampò su di un piede e andò a schiantarsi al suolo, le mani incapaci di proteggerlo completamente. Rimase stordito per un attimo, poi cominciò ad alzarsi faticosamente in piedi –

«\textit{Basta!}» disse la voce del professor Quirrell, un ordine così affilato da tagliare il ferro. «Allontanatevi dal signor Potter!»

Harry vide le espressioni sorprese dei loro volti. Il gelo nel suo sangue, che era montato e defluito, sorrise in fredda soddisfazione.

Poi Harry crollò sul tappeto.

Il professor Quirrell parlò. Si udirono sussulti provenire dai Serpeverde più grandi.

«E credo che anche il rampollo dei Malfoy abbia qualcosa che vuole spiegarvi», terminò il professor Quirrell.

La voce di Draco cominciò a parlare. Risuonava tagliente quasi quanto quella del professor Quirrell, aveva acquisito la stessa cadenza che Draco aveva usato per imitare suo padre, e stava dicendo cose come \textit{avrebbe potuto mettere in pericolo la Casa di Serpeverde e chissà quanti alleati solamente in questa scuola e totale mancanza di consapevolezza, per non dire di astuzia e ottusi delinquenti, buoni solo come lacchè} e qualcosa nei recessi del cervello di Harry, nonostante tutto ciò che sapeva, stava designando Draco come un alleato.

A Harry faceva male tutto, era stato probabilmente ferito, il suo corpo sentiva freddo, la sua mente era completamente esausta. Cercò di pensare alla canzone di Fawkes, ma senza la fenice presente non riusciva a ricordare la melodia e quando provò a immaginarla non fu in grado di pensare ad altro che al cinguettio di un uccello.

Poi Draco smise di parlare e il professor Quirrell disse ai Serpeverde più grandi che potevano andare, e Harry aprì gli occhi e si sforzò di mettersi a sedere, «Aspetti», disse forzando le parole, «c’è qualcosa che… voglio… dire… loro –»

«Aspettate il signor Potter», il professor Quirrell ordinò freddamente ai Serpeverde che si stavano allontanando.

Harry si alzò in piedi barcollando. Fece attenzione a non guardare in direzione dei suoi compagni di classe. Non voleva vedere come lo stavano guardando ora. Non voleva vedere la loro compassione.

Quindi, invece, Harry guardò i Serpeverde più grandi, che sembravano essere ancora sconvolti. Lo fissavano. Sui loro volti il timore.

Il suo lato oscuro, mentre aveva avuto il controllo, si era sorretto sulla fantasia di quel momento, ed era riuscito a fingere di perdere.

Harry parlò, «Nessuno –»

«Fermo», disse il professor Quirrell. «Se si tratta di quello che penso, la prego di attendere che se ne siano andati. Lo sentiranno dopo. Tutti noi abbiamo le nostre lezioni da imparare, signor Potter.»

«Va bene», concesse Harry.

«Voi. Andate.»

I Serpeverde più grandi fuggirono via e la porta si chiuse dietro di loro.

«Nessuno deve compiere qualunque tipo di vendetta contro di loro», Harry disse con voce roca. «Questa è una richiesta a tutti coloro che si considerano miei amici. Avevo la mia lezione da imparare, loro mi hanno aiutato a impararla, anche loro avevano la loro lezione di imparare, ora è finita. Se raccontate questa storia, assicuratevi di riferire anche questa parte.»

Harry si voltò a guardare il professor Quirrell.

«Ha perso», disse il professor Quirrell, la sua voce gentile per la prima volta. Suonava strana, provenendo dal Professore, come se la sua voce non dovesse essere neppure in grado di intonare quel registro gentile.

Harry \textit{aveva perso}. C’erano stati momenti in cui la fredda rabbia era svanita del tutto, sostituita dalla paura, e in quei momenti aveva supplicato i Serpeverde più grandi e l’aveva fatto sul serio…

«Eppure è ancora vivo?» chiese il professor Quirrell, ancora con quella strana gentilezza.

Harry riuscì ad annuire.

«Perdere non è sempre così», disse il professor Quirrell. «Ci sono compromessi e rese negoziate. Ci sono altri modi per placare i bulli. Vi è un’intera arte per manipolare gli altri consentendo loro di essere dominanti su di noi. Ma in primo luogo, la sconfitta deve essere \textit{concepibile}. Si ricorderà di come ha perso?»

«Sì.»

«Sarà capace di perdere?»

«Io… penso di sì…»

«Lo penso anch’io». Il professor Quirrell fece un inchino così profondo che i suoi capelli sottili quasi toccarono il pavimento. «Congratulazioni, Harry Potter, lei ha vinto.»

Non ci fu una singola origine, nessuno iniziò per primo, gli applausi cominciarono tutti contemporaneamente come il rombo fragoroso di un tuono.

Harry non riuscì a impedire alla sorpresa di manifestarsi sul suo volto. Diede una fugace occhiata ai suoi compagni di classe, e vide sui loro visi non la compassione, ma lo stupore. L’applauso veniva da Corvonero e Grifondoro e Tassofrasso, e persino da Serpeverde, probabilmente perché anche Draco Malfoy stava applaudendo. Alcuni studenti si erano alzati dalle sedie e la metà di Grifondoro era in piedi sui propri banchi.

Così Harry rimase lì, ondeggiando, lasciando che il loro rispetto lo avvolgesse, sentendosi più forte, e forse anche un po’ guarito.

Il professor Quirrell attese che l’applauso si spegnesse. Ci volle diverso tempo.

«Sorpreso, signor Potter?» chiese il professor Quirrell. La sua voce sembrava divertita. «Ha appena scoperto che il mondo reale non funziona \textit{sempre} come i nostri peggiori incubi. Certamente, se fosse stato un povero e anonimo ragazzo molestato, allora probabilmente dopo l’avrebbero rispettata di meno, l’avrebbero compatita anche mentre la consolavano dall’alto dei loro nobili piedistalli. Questa è la natura umana, temo. Ma \textit{lei}, la conoscono già come una figura di potere. E l’hanno vista affrontare la sua paura e continuare ad affrontarla, anche se avrebbe potuto andarsene via in qualsiasi momento. Ha pensato male di \textit{me} quando le ho detto che avevo deliberatamente sopportato che mi sputassero addosso?»

Harry provò una sensazione di bruciore alla gola e la soppresse con gran preoccupazione. Non si fidava abbastanza di quel miracoloso rispetto per iniziare a piangere di nuovo davanti a tutti.

«La sua impresa \textit{straordinaria} nella mia materia merita una ricompensa straordinaria, Harry Potter. La prego di accettarla con i miei complimenti a nome della mia Casa, e di ricordare da oggi in poi che non tutti i Serpeverde sono uguali. Ci sono Serpeverde e Serpeverde.» Il professor Quirrell sorrise piuttosto largamente mentre lo disse. «Cinquantuno punti a Corvonero.»

Ci fu una pausa di sbalordimento e poi scoppiò il pandemonio tra gli studenti Corvonero, con urla e fischi e applausi.

(E nello stesso momento Harry sentì che c’era qualcosa di \textit{sbagliato} in tutto ciò, la professoressa McGonagall aveva ragione, avrebbero \textit{dovuto} esserci delle conseguenze, ci sarebbe dovuto essere un costo, un prezzo da pagare, non si poteva semplicemente rimettere tutto al suo posto così –)

Ma Harry vide i volti euforici dei Corvonero e comprese che non poteva dire di no.

Il suo cervello gli diede un suggerimento. Fu un buon suggerimento. Harry non riusciva nemmeno a credere che il suo cervello fosse ancora in funzione, figuriamoci che formulasse buoni suggerimenti.

«Professor Quirrell», Harry disse il più chiaramente che poté con la gola in fiamme. «Lei è tutto quello che un membro della sua Casa dovrebbe essere, e penso che lei sia proprio quello che Salazar Serpeverde aveva in mente quando contribuì a fondare Hogwarts. Ringrazio lei e la sua Casa», Draco stava annuendo molto leggermente e ruotando impercettibilmente il dito, \textit{vai avanti}, «e penso che questo richieda tre urrà per Serpeverde. Siete tutti con me?» Harry fece una pausa. «Urrà!» Solo poche persone riuscirono a partecipare al primo tentativo. «Urrà!» Questa volta la maggior parte di Corvonero ci riuscì. «Urrà!» Questi erano quasi tutto Corvonero, una manciata di Tassofrasso e circa un quarto di Grifondoro.

La mano di Draco mosse il pollice in su in piccolo e breve gesto.

La maggior parte dei Serpeverde avevano espressioni di completo stupore. Alcuni stavano fissando meravigliati il professor Quirrell. Blaise Zabini stava guardando Harry con un’espressione calcolatrice e incuriosita.

Il professor Quirrell fece un inchino. «Grazie a lei, Harry Potter», disse, ancora con quel largo sorriso. Si girò verso la classe. «Ora, che ci crediate o meno, abbiamo ancora mezz’ora in questa sessione, ed è sufficiente per introdurre lo Scudo Semplice. Il signor Potter, ovviamente, sta uscendo a prendersi un meritato riposo.»

«Posso –»

«Idiota», il professor Quirrell disse affettuosamente. La classe stava già ridendo. «I suoi compagni di classe potranno insegnarglielo dopo, o le farò lezione privatamente, se necessario. Ma \textit{in questo momento}, lei sta per prendere la terza porta a sinistra dietro il palco, dove troverà un letto, un assortimento di spuntini eccezionalmente gustosi, e qualche lettura estremamente leggera dalla biblioteca di Hogwarts. Non può portare altro con sé, in particolare nessun libro di testo. Ora vada.»

Harry andò.



