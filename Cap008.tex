% !TeX root = Harry.tex

\chapter{Errore di conferma}
\label{capitolo:8}

\emph{«Permettimi di avvertirti che sfidare il mio ingegno è un genere di progetto pericoloso, e potrebbe tendere a rendere la tua vita molto più surreale.»}

~\\
~\\



Nessuno aveva chiesto il suo aiuto, quello era il problema. Se n’erano semplicemente andati in giro parlando, mangiando o guardando per aria mentre i loro genitori chiacchieravano. Per una strana ragione qualunque, nessuno era rimasto seduto a leggere un libro, e questo aveva significato che non aveva potuto semplicemente sedersi vicino a loro e tirar fuori il proprio libro. E anche quando aveva coraggiosamente preso l’iniziativa di sedersi e continuare la sua terza lettura di Storia di Hogwarts, nessuno era sembrato incline a sedersi a fianco a lei.

Oltre che aiutandole nei compiti, o in qualsiasi altra cosa di cui avessero bisogno, davvero non sapeva come conoscere le persone. Non sentiva di essere una persona timida. Pensava a sé stessa come al tipo di ragazza che si faceva carico dei problemi. Eppure, in qualche modo, se non c’era qualche richiesta del tipo «non mi ricordo come si fanno le divisioni» allora era semplicemente troppo imbarazzante andare da qualcuno e dire… cosa? Non era mai stata in grado di capire cosa. E non sembrava esserci un normale foglietto delle istruzioni, il che era ridicolo. L’intera faccenda di conoscere le persone non le era mai sembrata ragionevole. Perché doveva assumersi lei tutte le responsabilità quando erano coinvolte due persone? Perché gli adulti non erano mai d’aiuto? Desiderava solo che qualche altra ragazza venisse da lei e le dicesse «Hermione, la maestra mi ha detto di esserti amica».

Ma sia ben chiaro che Hermione Granger, seduta da sola il primo giorno di scuola in uno dei pochi scompartimenti rimasti vuoti, nell’ultima carrozza del treno, con la porta lasciata aperta nel caso in cui qualcuno per un motivo qualsiasi avesse voluto parlarle, non era triste, sola, avvilita, depressa, disperata, od ossessionata dai suoi problemi. Stava, piuttosto, rileggendo Storia di Hogwarts per la terza volta e le stava anche piacendo, con solo una lieve sfumatura di fastidio nei recessi della sua mente per l’irragionevolezza generale del mondo.

Ci fu il suono di una porta di comunicazione del treno che si apriva, e poi passi e uno strano suono strascicato che scesero lungo il corridoio della carrozza. Hermione mise da parte Storia di Hogwarts, si alzò e infilò la testa fuori — giusto nel caso in cui qualcuno avesse avuto bisogno di aiuto — e vide un giovane ragazzo in abiti da mago, forse del primo o secondo anno a giudicare dalla sua altezza, e che sembrava piuttosto sciocco con una sciarpa avvolta intorno alla testa. Un piccolo baule si trovava sul pavimento accanto a lui. Mentre lo osservava, bussò alla porta di un altro scompartimento chiuso, e disse con voce leggermente attutita dalla sciarpa, «Scusatemi, posso farvi una domanda veloce?»

Non sentì la risposta dall’interno dello scompartimento, ma dopo che il ragazzo ebbe aperto la porta, pensò davvero di averlo sentito chiedere — a meno che non avesse in qualche modo capito male — «Qualcuno qui conosce i sei quark o dove posso trovare una ragazza del primo anno di nome Hermione Granger?»

Dopo che il ragazzo ebbe chiuso la porta di quello scompartimento, Hermione gli chiese «Posso aiutarti in qualche modo?»

Il volto coperto dalla sciarpa si girò a guardarla, e la voce rispose «No, a meno che tu non possa dirmi i nomi dei sei quark o dove trovare Hermione Granger.»

«Up, down, strange, charm, truth, beauty, e perché la stai cercando?»

Era difficile dirlo da quella distanza, ma le parve di vedere il ragazzo fare un largo sorriso sotto la sciarpa. «Ah, quindi sei tu la ragazza del primo anno di nome Hermione Granger», disse quella voce giovane e attutita. «Sul treno per Hogwarts, per giunta.» Il ragazzo iniziò a camminare verso di lei e il suo scompartimento, e il suo baule scivolò dietro di lui. «Tecnicamente, tutto ciò che dovevo fare era cercarti, ma è probabile che intendesse che ti parli, o che ti inviti a entrare nel mio gruppo, o che ottenga da te un oggetto magico fondamentale, o che scopra che Hogwarts è stata costruita sopra la rovine di un antico tempio o qualcosa del genere. pg o png, questo è il problema.»

Hermione aprì la bocca per rispondere, ma poi non riuscì a formulare una qualsiasi risposta possibile… qualunque cosa fosse ciò che aveva appena sentito, anche mentre il ragazzo si avvicinò a lei, guardò all’interno dello scompartimento, annuì con soddisfazione, e sedette sul sedile di fronte al suo. Il suo baule si affrettò dietro di lui, crebbe di tre volte il suo diametro iniziale e si rannicchiò accanto a quello di lei in un modo stranamente inquietante.

«Per favore, siediti», disse il ragazzo, «e per favore, chiudi la porta dietro di te, se non ti dispiace. Non ti preoccupare, non mordo nessuno che non mi morda per primo.» Stava già srotolando la sciarpa dalla testa.

Il pensiero che questo ragazzo ritenesse che lei avesse paura di lui fece sì che la sua mano mandasse la porta scorrevole a chiudersi sbattendo contro la parete con forza eccessiva. Girò su sé stessa e vide un volto giovane con luminosi e ridenti occhi verdi, e una furiosa cicatrice rosso scuro incastonata sulla fronte che le ricordò qualcosa nei recessi della sua mente, ma in quel momento aveva cose più importanti a cui pensare. «Non ho detto di essere Hermione Granger!»

«E io non ho detto che tu hai detto di essere Hermione Granger, ho detto solo che tu sei Hermione Granger. Se stai per chiedermi come lo so, è perché so tutto. Buona sera signore e signori, il mio nome è Harry James Potter-Evans-Verres o Harry Potter in breve, so che probabilmente non significa niente per te tanto per cambiare –»

La mente di Hermione fece finalmente il collegamento. La cicatrice sulla fronte, a forma di fulmine. «Harry Potter! Sei in Storia moderna della Magia e Ascesa e declino delle Arti Oscure e Grandi eventi magici del Ventesimo secolo.» In realtà era la prima volta in tutta la sua vita che aveva incontrato qualcuno proveniente da dentro un libro, ed era una sensazione piuttosto strana.

Il ragazzo sbatté le palpebre tre volte. «Sono nei libri? Aspetta, ovviamente sono nei libri… che strano pensiero.»

«Santo cielo, non lo sapevi?» chiese Hermione. «Avrei scoperto tutto quello che potevo, se fossi stata io.»

Il ragazzo rispose piuttosto seccamente. «Signorina Granger, sono meno di 72 ore da quando sono andato a Diagon Alley e ho scoperto il motivo per cui sono famoso. Ho passato gli ultimi due giorni ad acquistare libri scientifici. Mi creda, ho intenzione di scoprire tutto ciò che posso.» Il ragazzo esitò. «Che cosa dicono i libri di me?»

La mente di Hermione Granger risentì del colpo, non si era resa conto che sarebbe stata messa alla prova su quei libri così li aveva letti solo una volta, ma era stato solo un mese fa e il materiale era ancora fresco nella sua mente. «Sei l’unico che sia sopravvissuto alla Maledizione Mortale quindi sei chiamato il Ragazzo-Che-È-Sopravvissuto. Sei nato da James Potter e Lily Potter, precedentemente Lily Evans, il 31 luglio 1980. Il 31 ottobre 1981 il Signore Oscuro, Colui-Che-Non-Deve-Essere-Nominato anche se non so perché, ha attaccato casa vostra. Sei stato trovato vivo con la cicatrice sulla fronte tra le rovine della casa dei tuoi genitori nei pressi dei resti carbonizzati del corpo di Tu-Sai-Chi. Il Capo Stregone Albus Percival Wulfric Brian Silente ti ha mandato via da qualche parte, nessuno sa dove. Ascesa e declino delle Arti Oscure sostiene che sei sopravvissuto grazie all’amore di tua madre e che la tua cicatrice contiene tutto il potere magico del Signore Oscuro e che i centauri ti temono, ma Grandi eventi magici del Ventesimo secolo non menziona nulla di simile e Storia moderna della Magia avverte che ci sono molte teorie pazzoidi su di te.»

La bocca del ragazzo era rimasta spalancata. «Ti è stato detto di aspettare Harry Potter sul treno per Hogwarts, o qualcosa del genere?»

«No», disse Hermione. «Chi ti ha parlato di me?»

«La professoressa McGonagall, e credo di capire perché. Hai una memoria eidetica, Hermione?»

Hermione scosse la testa. «Non è fotografica, ho sempre desiderato che lo fosse, ma ho dovuto leggere i miei libri di scuola per cinque volte per memorizzarli tutti.»

«Addirittura», disse il ragazzo con voce leggermente strozzata. «Spero che non ti dispiaccia se ti metto alla prova — non è che non ti creda, ma come dice il proverbio, ‘Fidati, ma verifica’. Inutile starselo a domandare quando posso sperimentalo.»

Hermione sorrise, piuttosto compiaciuta. Amava tanto gli esami. «Fai pure.»

Il ragazzo mise una mano in una borsa al suo fianco e disse «Infusi magici e pozioni di Arsenius Jigger.» Quando la ritirò, la mano reggeva il libro che aveva nominato.

Immediatamente Hermione volle una di quelle borse più di quanto avesse mai desiderato qualunque altra cosa.

Il ragazzo aprì il libro da qualche parte nel mezzo e guardò in basso. «Se stessi preparando l’olio dell’affilatezza –»

«Posso vedere la pagina da qui, sai!»

Il ragazzo inclinò il libro in modo che non potesse vederlo più, e sfogliò di nuovo le pagine. «Se stessi producendo una pozione dei movimenti del ragno, quale sarebbe l’ingrediente successivo che aggiungeresti dopo la seta di Acromantula?»

«Dopo averci lasciato cadere la seta, attendi che la pozione sia virata esattamente alla sfumatura del cielo senza nuvole all’alba, 8 gradi dall’orizzonte e 8 minuti prima che il disco solare diventi visibile. Mescola otto volte in direzione opposta al moto solare e una volta secondo il moto solare, e poi aggiungi otto dramme di caccole di unicorno.»

Il ragazzo chiuse il libro con un colpo secco e lo mise di nuovo nella borsa, che lo inghiottì con un ruttino. «Bene bene bene bene bene bene. Vorrei farle una proposta, signorina Granger.»

«Una proposta?» disse Hermione con sospetto. Le ragazze non avrebbero dovuto ascoltarle.

Fu a questo punto che Hermione si rese conto anche dell’altra cosa — beh, una delle cose — che era strana in quel ragazzo. A quanto pareva, le persone che erano nei libri effettivamente sembravano un libro quando parlavano. Questa fu una scoperta abbastanza sorprendente.

Il ragazzo mise una mano nella borsa e disse: «lattina di soda», recuperando un cilindro verde brillante. Glielo porse e chiese, «Posso offrirti qualcosa da bere?»

Hermione accettò educatamente la bevanda frizzante. In effetti si sentiva un po’ assetata, ora. «Mille grazie», disse Hermione mentre apriva la lattina. «Qual era la tua proposta?»

Il ragazzo tossì. «Niente», rispose. Proprio mentre Hermione iniziò a bere, disse «Mi piacerebbe che mi aiutassi a conquistare l’universo.»

Hermione finì di bere e abbassò la lattina. «No, grazie, io non sono malvagia.»

Il ragazzo la guardò sorpreso, come se si fosse aspettato qualche altra risposta. «Beh, stavo parlando un po’ retoricamente», disse. «Nel senso del progetto baconiano, sai, non del potere politico. ‘Il compimento di tutte le cose possibili’ e tutto il resto. Voglio condurre studi sperimentali sugli incantesimi, capirne le leggi sottostanti, portare la magia nel dominio della scienza, unire i mondi dei maghi e dei Babbani, innalzare la qualità della vita di tutto il pianeta, portare l’umanità avanti di secoli, scoprire il segreto dell’immortalità, colonizzare il Sistema solare, esplorare la galassia, e, soprattutto, capire cosa diavolo sta realmente accadendo qui, perché tutto questo è palesemente impossibile.»

Questo sembrava un po’ più interessante. «E?»

Il ragazzo la guardò incredulo. «E? Non è abbastanza?»

«E che cosa vuoi da me?» disse Hermione.

«Voglio che mi aiuti a condurre le ricerche, naturalmente. Con la tua memoria enciclopedica aggiunta alla mia intelligenza e razionalità, porteremo a compimento il progetto baconiano in poco tempo, dove con ‘poco tempo’ intendo dire probabilmente almeno trentacinque anni.»

Hermione iniziava a trovare quel ragazzo fastidioso. «Non ti ho visto fare nulla di intelligente. Forse permetterò a te di aiutare me con la mia ricerca.»

Ci fu un certo silenzio nello scompartimento.

«Così mi stai chiedendo di dimostrare la mia intelligenza, allora», disse il ragazzo dopo una lunga pausa.

Hermione annuì.

«Ti avverto che sfidare il mio ingegno è un progetto pericoloso, e tende a rendere la tua vita molto più surreale.»

«Non sono ancora impressionata», disse Hermione. Inosservata, la lattina verde le salì ancora una volta alle labbra.

«Beh, forse questo ti stupirà», disse il ragazzo. Si sporse in avanti e la guardò intensamente. «Ho già fatto un po’ di esperimenti e ho scoperto che non ho bisogno della bacchetta, posso far accadere tutto ciò che voglio semplicemente schioccando le dita.»

Fu detto proprio mentre Hermione era nel mezzo della deglutizione, e soffocò e tossì ed sputacchiò il liquido verde brillante.

Sulle sue vesti nuove di zecca, mai indossate, proprio il primo giorno di scuola.

Hermione gridò davvero. Fu un suono acuto che sembrò una sirena anti-aerea nello scompartimento chiuso. «Iiih! I miei vestiti!»

«Niente paura!» disse il ragazzo. «Posso risolvere il problema per te. Guarda!» Alzò una mano e fece schioccare le dita.

«Tu –» Poi guardò in basso, su di sé.

Il liquido verde era ancora lì, ma mentre guardava, cominciò a svanire e sbiadire, e nel giro di pochi istanti era come se non si fosse mai versata nulla addosso.

Hermione fissò il ragazzo, che aveva una specie di sorrisetto piuttosto compiaciuto.

Magia senza parole e senza bacchetta! Alla sua età? Quando aveva ricevuto i libri di scuola appena tre giorni fa?

Poi si ricordò di quello che aveva letto, e rimase a bocca aperta e si ritrasse lontano da lui. Tutto il potere magico del Signore Oscuro! Nella sua cicatrice!

Si alzò frettolosamente in piedi. «Io, io, io ho bisogno di andare in bagno, aspettami qui ok –» doveva andare a trovare un adulto doveva dirgli –

Il sorriso del ragazzo svanì. «Era solo un trucco, Hermione. Mi dispiace, non volevo spaventarti.»

La sua mano si fermò sulla maniglia della porta. «Un trucco?»

«Sì», disse il ragazzo. «Mi hai chiesto di dimostrare la mia intelligenza. Così ho fatto qualcosa di apparentemente impossibile, che è sempre un buon modo per mettersi in mostra. Non posso realmente fare di tutto semplicemente schioccando le dita.» Il ragazzo fece una pausa. «Almeno non credo di poterlo fare, non l’ho mai davvero provato sperimentalmente.» Il ragazzo alzò la mano e schioccò di nuovo le dita. «Nisba, niente banana.»

Hermione era confusa quanto non era mai stata in vita sua.

Il ragazzo stava ora nuovamente sorridendo per l’espressione sul suo volto. «Ti avevo avvertita che sfidare il mio ingegno tende a rendere la tua vita surreale. Ricordalo la prossima volta che ti avverto di qualcosa.»

«Ma, ma», balbettò Hermione. «Come hai fatto, allora?»

Lo sguardo del ragazzo assunse un’aria indagatoria, che non aveva mai visto prima a qualcuno della propria età. «Credi di avere già ciò che serve per essere uno scienziato, con o senza il mio aiuto? Allora vediamo come tu indagheresti un fenomeno sconcertante.»

«Io…» La mente di Hermione si spense per un attimo. Amava gli esami ma non aveva mai dato un esame come quello prima. Freneticamente, cercò di riesaminare il passato alla ricerca di qualcosa che avesse letto riguardo quello che gli scienziati dovevano fare. La sua mente cambiò marcia, riprese trazione, e risputò le istruzioni per fare un progetto scientifico:

Passo 1: formulare un’ipotesi.

Passo 2: fare un esperimento per verificare l’ipotesi.

Passo 3: misurare i risultati.

Passo 4: farne un poster cartonato.

Il Passo 1 era formulare un’ipotesi. Che significava provare a pensare a qualcosa che sarebbe potuto essere appena accaduto. «D’accordo. La mia ipotesi è che tu abbia lanciato un Incantesimo sulle mie vesti per far svanire qualunque cosa vi fosse stata versata.»

«Va bene», disse il ragazzo, «è questa la tua risposta?»

Il trauma stava passando, e la mente di Hermione cominciava a funzionare correttamente. «Aspetta, non può essere giusto. Non ti ho visto toccare la bacchetta o pronunciare nessuna magia, quindi come avresti potuto lanciare un Incantesimo?»

Il ragazzo aspettò, l’espressione del viso neutra.

«Ma supponiamo che tutte le vesti vengano dal negozio già con un Incantesimo sopra per tenerle pulite, un tipo di Incantesimo che sarebbe utile che avessero. Tu l’hai scoperto versando qualcosa addosso a te stesso in precedenza.»

Ora le sopracciglia del ragazzo si inarcarono. «È questa la tua risposta?»

«No, non ho fatto il Passo 2, ‘fare un esperimento per verificare l’ipotesi’.»

Il ragazzo richiuse la bocca, e cominciò a sorridere.

Hermione guardò la lattina di soda, che aveva messo automaticamente nel porta-bevande sotto il finestrino. La prese e ci guardò dentro, e scoprì che era circa un terzo piena.

«Bene», disse Hermione, «l’esperimento che voglio fare è quello di versarla sulle mie vesti e vedere cosa succede, e la mia previsione è che la macchia scomparirà. Solo che se non funziona, le mie vesti saranno macchiate, e non voglio che si sporchino.»

«Fallo sulle mie», disse il ragazzo, «in questo modo non dovrai preoccuparti che le tue vesti si macchino.»

«Ma –» disse Hermione. C’era qualcosa di sbagliato in quell’idea, ma non sapeva come dirlo esattamente.

«Ho delle vesti di ricambio nel mio baule», disse il ragazzo.

«Ma non c’è nessun posto per cambiarti», obiettò Hermione. Poi ci ripensò. «Anche se suppongo che potrei uscire e chiudere la porta –»

«Ho anche un posto per cambiarmi nel mio baule.»

Hermione guardò il baule del ragazzo, che, iniziava a sospettare, era alquanto più speciale del suo.

«Va bene», disse Hermione, «se lo dici tu», e versò piuttosto cautamente un po’ di soda verde su di un angolo delle vesti del ragazzo. Poi le fissò, cercando di ricordare quanto tempo il liquido aveva impiegato precedentemente per svanire…

E la macchia verde scomparve!

Hermione si lasciò sfuggire un sospiro di sollievo, anche perché questo significava che non aveva a che fare con tutto il potere magico del Signore Oscuro.

Beh, il Passo 3 era misurare i risultati, ma in questo caso consisteva semplicemente nell’osservare la scomparsa della macchia. E suppose di poter probabilmente saltare il Punto 4, a proposito del poster cartonato. «La mia risposta è che le vesti sono state Incantate per restare pulite.»

«Non proprio», disse il ragazzo.

Hermione sentì una fitta di delusione. Avrebbe davvero voluto non sentirsi in quel modo, il ragazzo non era un insegnante, ma si trattava comunque di un esame e aveva dato una risposta sbagliata e questo lo sentiva sempre come un piccolo pugno allo stomaco.

(Diceva quasi tutto ciò che c’era da sapere su Hermione Granger il fatto che non aveva mai lasciato che questo la fermasse, o addirittura che interferisse con il suo amore per l’essere messa alla prova.)

«La cosa triste è che», disse il ragazzo, «probabilmente hai fatto tutto ciò che il libro ti aveva detto di fare. Hai formulato una previsione che potesse distinguere tra le vesti incantate e quelle non incantate, l’hai messa alla prova e hai respinto l’ipotesi nulla che le vesti non fossero incantate. Ma a meno che tu non legga il genere di libri davvero, davvero migliori, non imparerai veramente come fare scienza in modo corretto. Abbastanza bene da ottenere davvero la risposta giusta, voglio dire, e non solo sfornare un’altra pubblicazione come Papà lamenta sempre. Quindi permettimi di spiegare — senza rivelare la risposta — cosa hai sbagliato questa volta, e ti darò un’altra possibilità.»

Stava cominciando a risentirsi per il tono di superiorità del ragazzo, quando si trattava solo di un altro undicenne come lei, ma questo era secondario rispetto a scoprire cosa aveva sbagliato. «D’accordo.»

L’espressione del ragazzo si fece più intenta. «Questo è un gioco basato su un famoso esperimento chiamato il compito 2-4-6, e funziona così. Io ho una regola — nota a me, ma non a te — che è rispettata da alcune triplette di numeri, ma non da altre. 2-4-6 è un esempio di una tripletta che rispetta la regola. In effetti… fammi mettere per iscritto la regola, in modo che tu sia sicura che è una regola fissa, poi piegherò il foglio e te lo darò. Ti prego di non guardare, dato che da ciò che è avvenuto prima deduco che puoi leggere capovolto.»

Il ragazzo disse «carta» e «portamina» alla sua borsa, ed ella chiuse gli occhi ermeticamente, mentre egli scriveva.

«Ecco», disse il ragazzo, e aveva in mano un pezzo di carta piegato strettamente. «Mettilo in tasca», ed ella lo fece.

«Ora, il modo in cui questo gioco funziona», disse il ragazzo, «è che tu mi dai una tripletta di numeri, e io ti dico ‘Sì’ se i tre numeri sono un’istanza della regola, e ‘No’ se non lo sono. Io sono la Natura, la regola è una delle mie leggi, e tu mi stai studiando. Sai già che 2-4-6 ottiene un ‘Sì’. Quando hai eseguito tutte le prove sperimentali che vuoi — mi hai chiesto tutte le triplette che ritieni necessarie — ti fermi e provi a indovinare la regola, e poi puoi aprire il foglio di carta e vedere come sei andata. Hai capito il gioco?»

«Naturalmente sì», disse Hermione.

«Procedi.»

«4-6-8» disse Hermione.

«Sì», disse il ragazzo.

«10-12-14», disse Hermione.

«Sì», disse il ragazzo.

Hermione cercò di far andare un po’ oltre la sua mente, dal momento che le sembrava di aver già fatto tutte le prove necessarie, e tuttavia non poteva essere così facile, no?

«1-3-5.»

«Sì.»

«Meno 3, meno 1, più 1.»

«Sì.»

Hermione non riusciva a pensare a nient’altro da chiedere. «La regola è che i numeri devono incrementarsi di due ogni volta.»

«Ora supponi che io ti riveli», disse il ragazzo, «che questo test è più difficile di quanto sembri, e che solo il 20\% degli adulti l’ha superato.»

Hermione si accigliò. Cosa aveva dimenticato? Poi, all’improvviso, pensò a una prova che doveva ancora fare.

«2-5-8!» disse trionfante.

«Sì.»

«10-20-30!»

«Sì.»

«La vera risposta è che i numeri devono incrementarsi della stessa quantità ogni volta. Non deve essere necessariamente due.»

«Molto bene», disse il ragazzo, «prendi il foglio e controlla come sei andata.»

Hermione prese il foglio dalla tasca e lo aprì.

Tre numeri reali in ordine crescente, dal più basso al più alto.

Hermione rimase a bocca aperta. Aveva la netta sensazione di aver subito una terribile ingiustizia, che il ragazzo fosse sporco, marcio, bugiardo e baro, ma quando ci ripensò, non riuscì a ricordare nessuna risposta che le avesse dato sbagliata.

«Ciò che hai appena scoperto è chiamato ‘errore di conferma’», disse il ragazzo. «Avevi una regola in mente, e hai continuato a pensare a triplette che avrebbero fatto dire ‘Sì’ alla regola. Ma non hai provato a testare nessuna tripletta che dovesse far rispondere ‘No’ a quella regola. In realtà non hai ottenuto neppure un singolo ‘No’, quindi ‘tre numeri qualsiasi’ avrebbe potuto essere la regola altrettanto facilmente. È un po’ come quando la gente immagina esperimenti che confermino le loro ipotesi invece di cercare di immaginare esperimenti che potrebbero falsificarle — non è esattamente lo stesso errore, ma ci si avvicina. Devi imparare a guardare il lato negativo delle cose, a fissare l’oscurità. Quando viene eseguito questo esperimento, solo il 20\% degli adulti ottiene la risposta giusta. E molti degli altri inventano ipotesi incredibilmente complicate e sono estremamente fiduciosi nelle loro risposte sbagliate in quanto hanno fatto tanti esperimenti e tutto è risultato essere come se l’aspettavano. Ora, vuoi fare un altro tentativo col problema originale?»

Il suo sguardo era piuttosto deciso ora, come se quello fosse il vero esame.

Hermione chiuse gli occhi e cercò di concentrarsi. Stava sudando sotto le vesti. Aveva la strana sensazione che questa fosse la volta in cui le era stato richiesto di pensare più intensamente che mai durante un esame, o forse anche la prima volta che le fosse mai stato richiesto di pensare a un esame.

Quale altro esperimento poteva fare? Aveva una Rana di Cioccolato, poteva provare a strofinarne un po’ sulle vesti e vedere se quella svaniva? Ma ancora non sembrava il genere di contorto pensiero negativo che il ragazzo si aspettava. Del resto lei stava ancora attendendosi un ‘Sì’ se la macchia della Rana di Cioccolato fosse scomparsa, invece di aspettarsi un ‘No’.

Quindi… secondo la sua ipotesi… quando la soda avrebbe dovuto… non svanire?

«Ho un esperimento da fare», disse Hermione. «Voglio versare un po’ di soda sul pavimento, e vedere se non scompare. Hai dei tovaglioli di carta nella tua borsa, così posso pulire la perdita se questo non dovesse funzionare?»

«Ho dei tovaglioli», disse il ragazzo. Il suo viso era ancora neutro.

Hermione prese la lattina e versò una piccola quantità di soda sul pavimento.

Pochi secondi dopo, scomparve.

Poi la realizzazione la colpì e provò l’impulso di prendersi a calci. «Ma certo! Tu mi hai dato la lattina! Non era la veste a essere incantata, è sempre stata la soda!»

Il ragazzo si alzò e le fece un solenne inchino. Sorrideva ampiamente, ora. «Allora… posso aiutarti con la tua ricerca, Hermione Granger?»

«Io, ah…» Hermione sentiva ancora la scarica di euforia, ma non era del tutto sicura di come rispondere a quella domanda.

Furono interrotti da un debole, incerto, fievole e piuttosto riluttante bussare alla porta.

Il ragazzo si girò a guardare fuori dalla finestra e disse: «Non sto indossando la mia sciarpa, puoi occupartene tu?»

Fu a questo punto che Hermione si rese conto del perché il ragazzo — no, il Ragazzo-Che-È-Sopravvissuto, Harry Potter — aveva indossato la sciarpa sul volto, e si sentì un po’ sciocca per non essersene accorta prima. Era in realtà un po’ strano, dato che avrebbe pensato che Harry Potter si sarebbe orgogliosamente mostrato al mondo; e le venne in mente il pensiero che in realtà potesse essere più timido di quello che sembrava.

Quando Hermione aprì la porta, fu salutata da un ragazzino tremante che era perfettamente descritto dal modo in cui aveva bussato.

«Scusa», disse il ragazzo in un filo di voce, «sono Neville Longbottom. Sto cercando il mio rospo domestico, io, io non riesco proprio a trovarlo da nessuna parte in questa carrozza… hai visto il mio rospo?»

«No», disse Hermione, e poi la sua disponibilità entrò in funzione a pieno regime. «Hai controllato in tutti gli altri scompartimenti?»

«Sì», sussurrò il ragazzo.

«Allora non ci resta che controllare tutte le altre carrozze», disse Hermione bruscamente. «Ti aiuterò io. Il mio nome è Hermione Granger, a proposito.»

Il ragazzo sembrò in procinto di svenire per la gratitudine.

«Aspetta», disse la voce dell’altro ragazzo — Harry Potter. «Non sono sicuro che questo sia il modo migliore di farlo.»

Ora Neville sembrò in procinto di piangere, e Hermione si girò di scatto, arrabbiata. Se Harry Potter era il tipo di persona che avrebbe abbandonato un bambino solo perché non voleva essere interrotto… «Che cosa? Perché no?»

«Beh», rispose Harry Potter, «ci vorrà un po’ per controllare personalmente l’intero treno, e potremmo non riuscire a trovare il rospo comunque, e se non lo avessimo trovato al nostro arrivo a Hogwarts, sarebbe nei guai. Quindi, sarebbe molto più sensato se andassimo direttamente alla carrozza di testa, dove sono i prefetti, e chiedessimo aiuto a uno di loro. Questa è stata la prima cosa che ho fatto quando cercavo te, Hermione, anche se in realtà non lo sapevano. Ma potrebbero avere incantesimi o oggetti magici che rendano molto più facile trovare un rospo. Siamo solo del primo anno.»

Quello… era molto sensato.

«Pensi di poter raggiungere la carrozza dei prefetti per conto tuo?» chiese Harry Potter. «Ho qualche ragione per non voler mostrare troppo la mia faccia.»

Improvvisamente Neville rimase senza fiato e fece un passo indietro. «Mi ricordo questa voce! Tu sei uno dei Signori del Caos! Tu sei quello che mi ha dato la cioccolata!»

Cosa? Cosa cosa cosa?

Harry Potter girò la testa dalla finestra e si alzò teatralmente. «Mai!», disse, la voce piena di indignazione. «Ti sembro il tipo di cattivo che darebbe caramelle a un bambino?»

Gli occhi di Neville si spalancarono. «Tu sei Harry Potter? Quel Harry Potter? Tu?»

«No, sono solo un Harry Potter, ce ne sono tre di me su questo treno –»

Neville emise un breve grido e corse via. Ci fu un rapido scalpiccio di passi frenetici e quindi il rumore della porta della carrozza che si aprì e si richiuse.

Hermione sedette pesantemente sul sedile. Harry Potter chiuse la porta e poi si sedette accanto a lei.

«Puoi per favore spiegarmi cosa sta succedendo?» disse Hermione con voce flebile. Si chiese se frequentare Harry Potter significasse sempre essere così confusi.

«Oh, beh, quello che è successo è che Fred e George e io abbiamo visto questo povero ragazzino alla stazione ferroviaria — la donna accanto a lui si era allontanata per un po’, e lui sembrava davvero spaventato, come se fosse sicuro di stare per essere attaccato dai Mangiamorte o cose del genere. Ora, si dice che la paura di qualcosa sia spesso peggiore della cosa in sé, così mi è venuto in mente che questo era un ragazzo che avrebbe potuto davvero trarre beneficio dal vedere il suo peggior incubo avverarsi e scoprire che non era così terribile come temeva –»

Hermione rimase seduta con la bocca spalancata.

«– e Fred e George hanno tirato fuori questo incantesimo per rendere più scure e indistinte le nostre sciarpe, come se fossimo re non-morti e quelli fossero i nostri sudari –»

Non le piaceva affatto dove stava andando a parare.

«– e dopo che avevamo finito di dargli tutti i dolci che avevo comprato, dicevamo ‘Diamogli un po’ di soldi! Ah ah ah! Prendi qualche zellino, ragazzo! Prendi qualche siclo d’argento!’ e ballavamo intorno a lui e ridevamo malignamente e così via. Penso che ci siano state alcune persone tra la folla che sarebbero volute intervenire, in un primo momento, ma l’apatia dell’astante le ha tenute lontane almeno fino a quando non hanno visto quello che stavamo facendo, e poi penso che tutti fossero troppo confusi per fare qualsiasi cosa. Alla fine con quel suo piccolo minuscolo sussurro ha detto ‘andate via’ così tutti e tre abbiamo urlato e siamo corsi via, gridando qualcosa a proposito della luce che ci bruciava. Speriamo che in futuro non sia così spaventato di essere molestato. Questa si chiama terapia di desensibilizzazione, tra parentesi.»

Va bene, non aveva indovinato dove volesse andare a parare.

L’ardente fuoco dell’indignazione che era uno dei motori principali di Hermione entrò scoppiettando in azione, anche se una parte di lei capiva in qualche modo quello che avevano cercato di fare. «È terribile! Tu sei terribile! Quel povero ragazzo! Quello che hai fatto è stato meschino!»

«Credo che la parola che stai cercando sia divertente, e in ogni caso stai ponendo la domanda sbagliata. La domanda è, ha causato più bene che male, o più male che bene? Se hai argomenti per contribuire a questa domanda mi farebbe piacere ascoltarli, ma non prenderò in considerazione eventuali altre critiche finché questa non sia risolta. Sono certamente d’accordo che quello che ho fatto sembri completamente terribile e prepotente e meschino, poiché riguarda un ragazzino spaventato e così via, ma non è certo la questione fondamentale, no? Si chiama consequenzialismo, per inciso, significa che se un atto sia giusto o sbagliato non è determinato dal fatto che sembri cattivo, o meschino, o qualcosa di simile, l’unica domanda è che risultati dia alla fine — quali ne siano le conseguenze.»

Hermione aprì la bocca per dire qualcosa di assolutamente feroce ma purtroppo sembrò aver trascurato la parte in cui pensava a qualcosa da dire prima di aprire bocca. Tutto ciò che poté dire fu, «E se avesse degli incubi?»

«Onestamente, non credo che avesse bisogno del nostro aiuto per avere incubi, e se avesse incubi su questo, invece, allora sarebbero incubi riguardanti orribili mostri che ti regalano cioccolata e questo era il punto della questione, più o meno.»

Il cervello di Hermione continuava a singhiozzare per la confusione ogni volta che cercava di diventare giustamente arrabbiata. «La tua vita è sempre così particolare?» chiese infine.

Il viso di Harry Potter brillò per l’orgoglio. «La rendo così particolare. Stai guardando il prodotto di molto duro lavoro e olio di gomito.»

«Quindi…» disse Hermione, per poi affievolirsi imbarazzata.

«Quindi», disse Harry Potter, «quanta scienza conosci esattamente? Io conosco il calcolo infinitesimale e un po’ di teoria della probabilità bayesiana e di teoria delle decisioni e molta scienza cognitiva, e ho letto La fisica di Feynman (o il volume 1, ad ogni modo) e Judgment Under Uncertainty: Heuristics and Biases, e Language in Thought and Action e Le armi della persuasione e Rational Choice in an Uncertain World e Gödel, Escher, Bach e A Step Farther Out e –»

L’interrogatorio e il contro-interrogatorio successivi proseguirono per diversi minuti prima di essere interrotti da un altro timido bussare alla porta. «Avanti», ella e Harry Potter dissero quasi allo stesso tempo, e la porta scorse di lato svelando Neville Longbottom.

Ora Neville stava piangendo davvero. «Sono andato al vagone anteriore e ho trovato un p-prefetto, ma mi ha detto che i prefetti non devono essere disturbati per cose insignificanti come rospi s-scomparsi.»

Il viso del Ragazzo-Che-È-Sopravvissuto mutò. Le labbra formarono una linea sottile. La voce, quando parlò, fu fredda e cupa. «Quali erano i suoi colori? Verde e argento?»

«N-no, il suo distintivo era r-rosso e oro.»

«Rosso e oro!» scattò Hermione. «Ma quelli sono i colori di Grifondoro!»

In risposta Harry Potter sibilò, un suono spaventoso che sarebbe potuto venire da un serpente e fece trasalire sia lei sia Neville. «Suppongo», esclamò Harry Potter, «che ritrovare il rospo di un alunno del primo anno non sia abbastanza eroico da essere degno di un prefetto Grifondoro. Forza, Neville, verrò io con te stavolta, vedremo se il Ragazzo-Che-È-Sopravvissuto riceverà più attenzione. Prima cercheremo un prefetto che sappia qualche incantesimo, e se questo non funziona, cercheremo un prefetto che non abbia paura di sporcarsi le mani, e se questo non funziona, inizierò a reclutare i miei sostenitori e se fosse necessario, smonteremo l’intero treno vite per vite.»

Il Ragazzo-Che-È-Sopravvissuto si alzò e afferrò la mano di Neville nella propria, e Hermione si rese conto con un improvviso sussulto mentale che erano quasi della stessa altezza, anche se una parte di lei aveva insistito sul fatto che Harry Potter fosse trenta centimetri più alto di lei, e Neville almeno quindici più basso.

«Resta!» sbottò Harry Potter verso di lei — no, un momento, verso il proprio baule — e uscendo si chiuse la porta dietro con decisione.

Probabilmente sarebbe dovuta andare con loro, ma in un solo breve momento Harry Potter era diventato così spaventoso che in realtà era piuttosto contenta di non aver pensato di suggerirlo.

La mente di Hermione era ormai così confusa che non ritenne neppure di poter leggere adeguatamente «Hogwarts di Storia». Si sentiva come se fosse stata appena investita da un rullo compressore e trasformata in una frittella. Non era sicura di cosa stesse pensando o cosa stesse provando o perché. Si sedette semplicemente accanto alla finestra e guardò il paesaggio in movimento.

Beh, almeno sapeva perché si sentiva un po’ triste dentro.

Forse Grifondoro non era così meraviglioso come aveva pensato.



