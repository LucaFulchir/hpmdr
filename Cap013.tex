% !TeX root = Harry.tex

\chapter{Porre le domande sbagliate}
\label{capitolo:13}

\emph{«Questo è uno degli indovinelli più facili che abbia mai sentito.»}

~\\
~\\

Non appena Harry aprì gli occhi all’interno del dormitorio Corvonero per i ragazzi del primo anno, la mattina del suo primo giorno completo a Hogwarts, seppe che qualcosa non andava.

Era tranquillo.

Troppo tranquillo.

Ah, già… C’era un Incantesimo Quietus sulla testiera del suo letto, controllato da una piccola barra di scorrimento, ed era l’unica ragione per la quale fosse possibile, per chiunque in Corvonero, andare a dormire.

Harry si mise a sedere e si guardò intorno, aspettandosi di vedere gli altri alzarsi per la giornata –

Il dormitorio, vuoto.

I letti, sgualciti e disfatti.

Il sole, che entrava con un angolo piuttosto alto.

Il suo Quietus portato fino al massimo.

E il suo orologio meccanico ancora in funzione, ma con la sveglia disattivata.

Era stato lasciato a dormire fino alle 9:52, evidentemente. Nonostante i suoi migliori sforzi per sincronizzare il proprio ciclo di sonno di 26 ore al suo arrivo a Hogwarts, la notte precedente non era riuscito ad addormentarsi prima dell’una. Aveva previsto di svegliarsi alle 7:00 con gli altri studenti, poteva sopportare gli effetti un po’ di privazione del sonno il suo primo giorno a patto che avesse ottenuto una qualche sorta di aiuto magico prima del giorno successivo. Ma ora si era perso la colazione. E la sua prima lezione a Hogwarts, in Erbologia, era iniziata un’ora e 22 minuti prima.

La rabbia stava lentamente, lentamente risvegliandosi in lui. Oh, che bello scherzetto. Disattivare la sveglia. Alzare il Quietus. E lasciare che il signor Pezzogrosso Harry Potter perdesse la sua prima lezione, e fosse accusato di essere un dormiglione.

Quando Harry avesse scoperto il colpevole…

No, non poteva essere stato fatto che con la collaborazione di tutti gli altri dodici ragazzi del dormitorio di Corvonero. Tutti loro avrebbero visto la sua sagoma addormentata. Tutti loro l’avevano lasciato dormire per l’intero periodo della colazione.

La rabbia scemò, sostituita dalla confusione e da un’orribile sensazione di essere stato ferito. Gli era \textit{piaciuto}. Così aveva pensato. La notte scorsa, aveva pensato che di essere piaciuto a tutti loro. \textit{Perché…}

Appena Harry si alzò dal letto, vide un pezzo di carta attaccato alla testata e rivolto verso l’esterno.

\vspace{1em}
\begin{addmargin}[3em]{3em}% 1em left, 2em right
\begin{itpars}
Compagni di Corvonero,

è stata una giornata estremamente lunga. Per favore lasciatemi dormire e non vi preoccupate se salto la colazione. Non ho dimenticato la mia prima lezione.

Vostro,

Harry Potter.
\end{itpars}
\end{addmargin}
\vspace{1em}

E Harry rimase lì, paralizzato, acqua ghiacciata che iniziò a scorrere nelle sue vene.

Il testo era di suo pugno, scritto con la sua portamina.

E non si ricordava di averlo scritto.

E… Harry strizzò gli occhi guardando il pezzo di carta. A meno che non lo stesse immaginando, le parole «Non ho dimenticato» erano state scritte in uno stile diverso, come se stesse cercando di dire qualcosa a sé stesso…?

Aveva \textit{saputo} che stava per essere Obliato? Era rimasto alzato fino a tardi, aveva commesso qualche crimine o attività segreta, e poi… ma non \textit{conosceva} l’Incantesimo di Obliazione… forse qualcun altro aveva… che cosa…

A Harry venne in mente un pensiero. Se \textit{avesse} saputo che stava per essere Obliato…

Ancora in pigiama, Harry girò intorno al letto verso il baule, premette il pollice contro la serratura, tirò fuori la borsa, ci infilò la mano e disse: «Nota per me stesso.»

E un altro pezzo di carta gli si materializzò in mano.

Harry lo tirò fuori, fissandolo. Anch’esso era di suo pugno.

Il biglietto diceva:

\vspace{1em}
\begin{addmargin}[3em]{3em}% 1em left, 2em right
\begin{itpars}
Caro Me,

per favore gioca la partita. Puoi giocare la partita solo una volta nella vita. Questa è un’opportunità unica.

Codice di riconoscimento 927, io sono una patata.

Tuo,

Tu.
\end{itpars}
\end{addmargin}
\vspace{1em}

Harry annuì lentamente. «Codice di riconoscimento 927, io sono una patata» era davvero il messaggio che aveva scelto in passato — qualche anno prima, mentre guardava la tv — e che solo \textit{egli} avrebbe conosciuto. Nel caso in cui avesse dovuto decidere se un duplicato di sé stesso fosse proprio lui, o qualcosa del genere. Giusto per sicurezza. Sii Preparato.

Harry non poteva \textit{fidarsi} del messaggio, potevano esserci altri incantesimi in azione. Ma qualsiasi scherzo semplice era escluso. Sicuramente era stato lui a scrivere quel messaggio e sicuramente non ricordava di averlo fatto.

Fissando il foglio, Harry notò che dell’inchiostro l’attraversava dall’altro lato.

Lo girò.

Il rovescio recitava:

\vspace{1em}
\begin{addmargin}[3em]{3em}% 1em left, 2em right
\begin{itpars}
\begin{center}
\textsc{istruzioni per il gioco:}

non conosci le regole del gioco

non conosci la posta in gioco

non conosci l’obiettivo del gioco

non sai chi controlla il gioco

non sai come porre fine al gioco

Inizi con 100 punti.

Procedi.
\end{center}
\end{itpars}
\end{addmargin}
\vspace{1em}

Harry fissò le «istruzioni». Quella parte non era stata scritta a mano: la scrittura era perfettamente regolare, quindi artificiale. Sembrava scritta da un Penna Prendiappunti, come quella che aveva comprato per scrivere sotto dettatura.

Non aveva \textit{assolutamente alcuna idea} di cosa stesse accadendo.

Beh… Il primo passo sarebbe stato vestirsi e mangiare. Forse nell’ordine inverso. Il suo stomaco sembrava piuttosto vuoto.

Aveva saltato la colazione, naturalmente, ma era Preparato a questa eventualità, avendola prevista in anticipo. Harry mise la mano nella borsa e disse «barretta», aspettandosi di ricevere la scatola di barrette di cereali che aveva comprato prima di partire per Hogwarts.

Ciò che comparve non aveva la forma di una scatola di barrette di cereali.

Quando Harry portò la mano all’interno del suo campo visivo vide due piccole caramelle — non sufficienti per un pasto — con allegata una nota, e la nota era stata scritta nella stessa calligrafia delle istruzioni del gioco.

Il biglietto diceva:

\begin{addmargin}[3em]{3em}% 1em left, 2em right
~

\textsc{tentativo fallito: -1 punto}

\textsc{punti correnti: 99}

\textsc{stato fisico: ancora affamato}

\textsc{stato mentale: confuso}\\
\end{addmargin}

«Gleehhhhh» fece la bocca di Harry senza alcun tipo di intervento consapevole o di decisione da parte sua.

Rimase fermo per circa un minuto.

Un minuto più tardi, la situazione non aveva \textit{ancora} alcun senso ed egli non aveva \textit{ancora} la minima idea di quello che stava succedendo e il suo cervello non aveva neppure \textit{cominciato} a cercare di afferrare un’ipotesi qualunque, come se le sue mani mentali fossero racchiuse da sfere di gomma e non potessero stringere nulla.

Il suo stomaco, che aveva le sue priorità, suggerì una possibile prova sperimentale.

«Ah…» disse Harry alla stanza vuota. «Non credo che potrei spendere un punto e riavere indietro la mia scatola di barrette di cereali, vero?»

Ci fu solo silenzio.

Harry mise la mano nella borsa e disse «Scatola di barrette di cereali.»

Una scatola che sembrò della forma giusta gli si materializzò in mano… ma era troppo leggera, ed era stata aperta, ed era vuota, e la nota allegata diceva:

\begin{addmargin}[3em]{3em}% 1em left, 2em right
~

\textsc{punti spesi: 1}

\textsc{punti correnti: 98}

\textsc{hai guadagnato: una scatola di barrette di cereali}\\
\end{addmargin}

«Mi piacerebbe spendere un punto e avere indietro le \textit{vere barrette di cereali}», disse Harry.

Anche in questo caso, silenzio.

Harry mise la mano nella borsa e disse «barrette di cereali.»

Non uscì nulla.

Persa ogni speranza, Harry scrollò le spalle e si diresse verso l’armadio che gli era stato dato vicino al suo letto, per prendere le sue vesti da mago per la giornata.

Sul fondo dell’armadio, sotto le sue vesti, c’erano le barrette di cereali, e una nota:

\begin{addmargin}[3em]{3em}% 1em left, 2em right
~

\textsc{punti spesi: 1}

\textsc{punti correnti: 97}

\textsc{hai guadagnato: 6 barrette di cereali}

\textsc{stai ancora indossando: pigiama}

\textsc{non mangiare mentre indossi il pigiama}

\textsc{riceverai una penalità per il pigiama}\\
\end{addmargin}

\textit{E adesso so che chiunque controlli il gioco è folle.}

«La mia ipotesi è che il gioco sia controllato da Silente», disse Harry ad alta voce. Forse questa volta avrebbe potuto stabilire un nuovo record di velocità su pista per essere stato veloce a capire.

Silenzio.

Ma Harry stava cominciando a cogliere lo schema, la nota sarebbe stata nel posto successivo in cui avrebbe guardato. Così Harry guardò sotto il letto.

\begin{addmargin}[3em]{3em}% 1em left, 2em right
~

\textsc{ah! ah ah ah ah ah!}

\textsc{ah ah ah ah ah ah!}

\textsc{ah! ah! ah! ah! ah! ah!}

\textsc{silente non controlla il gioco}

\textsc{tentativo pessimo}

\textsc{tentativo davvero pessimo}

\textsc{-20 punti}

\textsc{e indossi ancora il pigiama}

\textsc{è la tua quarta mossa}

\textsc{e indossi ancora il pigiama}

\textsc{penalità pigiama: -2 punti}

\textsc{punti correnti: 75}\\
\end{addmargin}

Beh, quello era un problema, chiaramente. Era solo il suo primo giorno di scuola e, una volta escluso Silente, non sapeva il nome di nessun altro là che fosse così folle.

Con il corpo che andava più o meno con il pilota automatico, Harry raccolse un completo di vestiti e biancheria intima, tirò fuori il livello sotterraneo del baule (era una persona molto riservata e qualcuno sarebbe potuto entrare nel dormitorio), si vestì, e poi tornò al piano di sopra a mettere a posto il pigiama.

Harry fece una pausa prima di estrarre il cassetto dell’armadio che conteneva il pigiama. Se lo schema continuava a valere…

«Come posso guadagnare punti?» chiese ad alta voce.

Poi aprì il cassetto.

\begin{addmargin}[3em]{3em}% 1em left, 2em right
~

\textsc{occasioni per fare del bene sono ovunque}

\textsc{ma è dove c’è tenebra che deve esserci luce}

\textsc{costo della domanda: 1 punto}

\textsc{punti correnti: 74}

\textsc{carina la tua biancheria intima}

\textsc{l’ha scelta tua madre?}\\
\end{addmargin}

Harry accartocciò il biglietto nella mano, il volto paonazzo. Gli tornò alla mente l’imprecazione di Draco. \textit{Figlio di un sanguemarcio –}

A quel punto era sufficientemente smaliziato da non dirlo ad alta voce. Avrebbe probabilmente ricevuto una Penalità per Volgarità.

Harry cinse la borsa mokeskin e la bacchetta. Tolse l’incarto a una delle barrette di cereali e lo gettò nel cestino della stanza, dove atterrò sopra una Rana di Cioccolato per lo più integra, una busta stropicciata e un po’ di carta da imballaggio verde e rossa. Mise le restanti barrette di cereali nella borsa mokeskin.

Si guardò intorno in un’ultima, disperata, e in definitiva inutile ricerca di indizi.

E poi Harry lasciò il dormitorio, mangiando mentre camminava, alla ricerca del sotterraneo Serpeverde. O quantomeno \textit{credeva} che fosse così.

Cercare di orientarsi tra i corridoi di Hogwarts era come… Probabilmente non così disorientante come passeggiare all’interno di un quadro di Escher, che era il tipo di cosa che avresti detto per l’effetto retorico, piuttosto che perché fosse vera.

Poco tempo dopo, Harry stava pensando che in effetti un quadro di Escher avrebbe avuto sia vantaggi sia svantaggi rispetto a Hogwarts. Lati negativi: nessun orientamento gravitazionale coerente. Vantaggi: almeno le scale non si sarebbe spostate \textsc{mentre ci stavi ancora sopra.}

In precedenza, Harry aveva salito quattro rampe di scale per arrivare al suo dormitorio. Dopo aver disceso non meno di dodici rampe di scale, senza arrivare neppure lontanamente vicino al sotterraneo, Harry aveva concluso che (1) un quadro di Escher sarebbe stato una \textit{passeggiata} a confronto, (2) era in qualche modo \textit{più in alto} nel castello di quando aveva cominciato, e (3) si era così \textit{scrupolosamente} perduto che non si sarebbe sorpreso a guardare fuori della finestra successiva e vedere due lune nel cielo.

Il piano di riserva a era stato di fermarsi a chiedere informazioni, ma sembrava esserci un’estrema penuria di persone che passassero di lì, come se tutti gli scocciatori stessero frequentando i corsi come si supponeva facessero o giù di lì.

Piano di riserva b…

«Mi sono perso», disse ad alta voce. «Può lo, uhm, spirito del castello di Hogwarts aiutarmi, o qualcosa del genere?»

«Non credo che il castello abbia uno spirito», osservò una rugosa e anziana signora in uno dei dipinti sulle pareti. «Vita, forse, ma non spirito.»

Ci fu una breve pausa.

«Sei –» disse Harry, e poi chiuse la bocca. A pensarci bene no, \textsc{non} aveva intenzione di chiedere al dipinto se fosse pienamente cosciente nel senso di essere consapevole della propria consapevolezza.

«Io sono Harry Potter», disse la sua bocca, più o meno in automatico. Sempre più o meno automaticamente, Harry tese la mano verso la pittura.

La donna nel dipinto abbassò lo sguardo sulla mano di Harry e alzò le sopracciglia.

Lentamente, la mano cadde di nuovo al fianco di Harry.

«Mi dispiace», disse Harry, «sono nuovo di queste parti.»

«Me ne rendo conto, giovane corvo. Dove stai cercando di andare?»

Harry esitò. «Non ne sono proprio sicuro», disse.

«Allora forse sei già lì.»

«Beh, ovunque io \textit{stia} cercando di andare, non credo che sia \textit{qui}…» Harry chiuse la bocca, conscio di quanto stesse sembrando idiota. «Mi permetta di ricominciare daccapo. Sto giocando questo gioco, solo non ne conosco le regole –» Neppure questo andava bene, no? «Va bene, terzo tentativo. Sto cercando delle occasioni per fare del bene così da poter ottenere dei punti, e tutto ciò che ho è questo suggerimento criptico riguardo l’oscurità come il luogo dove la luce deve essere, così stavo cercando di andare giù, ma mi pare di continuare ad andare su, invece…»

L’anziana signora nel dipinto lo stava guardando piuttosto incredula.

Harry sospirò. «La mia vita tende a essere alquanto particolare.»

«Sarebbe corretto dire che non sai dove stai andando o perché stai cercando di arrivarci?»

«\textit{Assolutamente} corretto.»

L’anziana signora annuì. «Non sono sicura che esserti perso sia il tuo maggior problema, giovanotto.»

«Vero, ma a differenza dei problemi più importanti, questo è tale che posso comprendere come risolverlo e \textit{uao} questa conversazione si sta trasformando in una metafora dell’esistenza umana, non mi ero neppure accorto che stesse succedendo fino a ora.»

La signora guardò Harry soppesandolo. «Tu \textit{sei} un gran bel giovane corvo, no? Per un momento avevo iniziato a dubitarne. Bene allora, in linea di principio, se continui a girare a sinistra, sei destinato a continuare a scendere.»

La cosa sembrava stranamente familiare, ma Harry non poté ricordare dove l’avesse già sentita. «Uhm… lei sembra una persona molto intelligente. O il dipinto di una persona molto intelligente… a ogni modo, ha mai sentito di un gioco misterioso che si può giocare solo una volta, e di cui non sono svelate le regole?»

«La vita», disse immediatamente la signora. «Questo è uno degli indovinelli più facili che abbia mai sentito.»

Harry rimase interdetto. «No», disse lentamente. «Voglio dire che ho ricevuto una vera nota che diceva che dovevo giocare questa partita ma che non mi sarebbero state svelate le regole, e qualcuno mi sta lasciando dei foglietti dicendomi quanti punti ho perso per aver violato le regole, come meno due punti per aver indossato un pigiama. Conosce qualcuno qui a Hogwarts che sia folle abbastanza e potente abbastanza per fare qualcosa del genere? A parte Silente, intendo?»

Il ritratto di una signora sospirò. «Sono solo un dipinto, giovanotto. Ricordo Hogwarts come era — non Hogwarts come è. Tutto quello che posso dirti è che se questo fosse un indovinello, la risposta sarebbe che il gioco è la vita, e che sebbene non siamo noi a scegliere tutte le regole, colui che concede o toglie punti sei sempre tu. Se non è un indovinello ma la realtà — allora non so.»

Harry s’inchinò molto profondamente al dipinto. «La ringrazio, mia signora.»

La signora gli fece la riverenza. «Vorrei poter dire che ti ricorderò con affetto», disse, «ma probabilmente non ti ricorderò affatto. Addio, Harry Potter.»

Si inchinò ancora in risposta, e prese a scendere la rampa di scale più vicina.

Quattro svolte a sinistra dopo si trovò a fissare un corridoio che si concludeva, bruscamente, in un cumulo di grosse rocce cadute — come se ci fosse stata una frana, solo che le pareti circostanti e il soffitto erano intatti e fatti delle normali pietre del castello.

«Va bene», disse Harry all’aria vuota, «mi arrendo. Sto chiedendo un altro suggerimento. Come faccio ad arrivare dove devo andare?»

«Un suggerimento! Un suggerimento, dici?»

La voce eccitata era giunta da un dipinto sul muro non lontano, questa volta il ritratto di un uomo di mezza età nei più sgargianti abiti rosa che Harry avesse mai visto o immaginato. Nel ritratto indossava un vecchio e cadente cappello a punta con un pesce sopra (non un disegno di un pesce, attenzione, ma un pesce).

«Sì!» disse Harry. «Un suggerimento! Un suggerimento, dico! Solo non un suggerimento \textit{qualsiasi}, sto cercando un suggerimento \textit{specifico}, si tratta di un gioco che sto giocando –»

«Sì, sì! Un suggerimento per il gioco! Tu sei Harry Potter, non è vero? Sono Cornelion Flubberwalt! Mi è stato detto da Erin il Consorte a cui è stato detto dal Lord Nasone a cui è stato detto da, l’ho proprio dimenticato. Ma era un messaggio che \textit{io} devo dare a te! \textit{Io}! Nessuno si è interessato di me da, non so da quanto tempo, forse da sempre, sono rimasto bloccato qua sotto, in questo dannato e inutile vecchio corridoio — un suggerimento! Ho il tuo suggerimento! Ti costerà solo tre punti! Lo vuoi?»

«Sì lo voglio!» Harry era cosciente che forse avrebbe dovuto tenere il suo sarcasmo sotto controllo, ma non ci riusciva.

«L’oscurità può essere trovata tra le sale da studio verdi e l’aula di Trasfigurazione di McGonagall! Questo è il suggerimento! E datti una mossa, sei più lento di un sacco di lumache! Meno dieci punti per essere lento! Ora hai 61 punti! Questo era il resto del messaggio!»

«Grazie», disse Harry. Stava davvero rimanendo indietro in questo gioco… «Uhm… non penso che sappia da dove il messaggio provenga \textit{originariamente}, vero?»

«È stato pronunciato da una voce cavernosa che è risuonata da un vuoto nell’aria stessa, un vuoto che si era aperto su di un feroce abisso! Questo è quello che mi hanno detto.»

Harry non era più sicuro, a quel punto, se quello fosse il tipo di cose di cui avrebbe dovuto essere scettico, o il genere di cose che avrebbe dovuto semplicemente accettare con tranquillità. «E come posso trovare la linea di separazione tra le sale studio verdi e l’aula di Trasfigurazione?»

«Basta girare su te stesso e andare a sinistra, destra, giù, giù, destra, sinistra, destra, su, e di nuovo a sinistra, sarai presso la sala studio verde e se ci entri e cammini dritto fuori dal lato opposto sarai in un grande corridoio sinuoso che arriva a un incrocio e sul lato destro di questo incrocio ci sarà un lungo corridoio rettilineo che va all’aula di Trasfigurazione!» La figura di un uomo di mezza età fece una pausa. «Almeno così era quando io ero a Hogwarts. Questo \textit{è} un lunedì di un anno dispari, giusto?»

«Mine e portacarta», disse Harry alla sua borsa. «Ehm, annulla, carta e portamine.» Alzò lo sguardo. «Potrebbe ripetere?»

Dopo essersi perso altre due volte, Harry sentì che stava cominciando a capire la regola fondamentale per orientarsi in quel labirinto in continua evoluzione che era Hogwarts, vale a dire \textit{chiedere informazioni a un dipinto}. Se quello rifletteva una sorta di lezione di vita incredibilmente profonda, non riusciva a capire quale fosse.

La sala da studio verde era uno spazio sorprendentemente piacevole, con la luce solare che sgorgava dalle finestre di vetro colorato di verde, raffiguranti draghi in tranquille scene pastorali. Aveva sedie che sembravano estremamente confortevoli, e tavoli che parevano molto adatti a studiare in compagnia di un numero di amici da uno a tre.

Harry non poté \textit{realmente} attraversarla senza deviare e uscire dalla porta sul lato opposto. C’erano delle \textit{librerie} fissate al muro, e dovette andare a leggere alcuni dei titoli, in modo da non perdere il suo diritto al nome della famiglia Verres. Ma lo fece in fretta, memore dell’accusa di essere lento, e poi uscì dall’altro lato.

Stava camminando lungo il «grande corridoio sinuoso» quando sentì la voce di un giovane ragazzo che gridava.

In momenti come quello, Harry aveva una scusa per correre a perdifiato, senza pensare di risparmiare energie o di fare esercizi di riscaldamento adeguati o preoccuparsi di sbattere contro gli oggetti, un improvviso volo frenetico che giunse a una quasi altrettanto immediata interruzione, quando fu lì lì per investire un gruppo di sei studenti del primo anno di Tassofrasso…

… che erano stretti l’uno all’altro, con l’aria piuttosto spaventata e come se volessero disperatamente fare qualcosa, ma senza riuscire a capire cosa, fatto che probabilmente era legato al gruppo di cinque Serpeverde più grandi che sembravano circondare un altro giovane.

Tutto d’un tratto Harry fu piuttosto arrabbiato.

«\textit{Permesso!}» gridò Harry a squarciagola.

Forse non sarebbe stato necessario. Lo stavano già guardando. Ma fu certamente utile a congelare l’azione in corso.

Harry passò davanti al gruppo di Tassofrasso diretto verso i Serpeverde.

I quali lo guardarono con espressioni che andavano dalla rabbia al divertimento alla delizia.

Parte del cervello di Harry stava urlando in preda al panico che si trattava di ragazzi molto più grandi e più grossi che potevano calpestarlo fino ad appiattirlo.

Un’altra parte rispose seccamente che tutti coloro che fossero stati sorpresi nel serio tentativo di calpestare il Ragazzo-Che-È-Sopravvissuto si sarebbero ficcati in un mondo di guai, soprattutto se fossero stati un branco di Serpeverde più grandi e ci fossero stati sette Tassofrasso a testimoniarlo, e che la probabilità che gli procurassero danni permanenti in presenza di testimoni era pari quasi a zero. L’unica vera arma che i ragazzi più grandi avevano contro di lui era la sua paura, se gliel’avesse permesso.

Poi Harry vide che il ragazzo che avevano intrappolato era Neville Longbottom.

Ovviamente.

Quello risolveva la questione. Harry aveva deciso di chiedere umilmente scusa a Neville e questo significava che Neville era \textit{suo}, come \textit{osavano}?

Harry allungò la mano, afferrò Neville per il polso e lo \textit{strattonò} fuori dal cerchio dei Serpeverde, e il ragazzo traumatizzato inciampò mentre Harry lo allontanava e quasi col medesimo movimento si faceva strada infilandosi attraverso lo stesso varco.

E Harry rimase in mezzo ai Serpeverde, là dove era stato Neville, guardando in su verso quei ragazzi molto più grandi, molto più grossi e molto più forti.

«Ciao», disse Harry. «Io sono il Ragazzo-Che-È-Sopravvissuto.»

Ci fu una pausa piuttosto imbarazzata. Nessuno sembrava sapere in che direzione si sarebbe dovuta muovere la conversazione.

Harry abbassò gli occhi e vide alcuni libri e fogli sparsi sul pavimento. Oh, il vecchio gioco in cui si lascia che il bambino tenti di raccogliere i libri e poi glieli si toglie di mano nuovamente. Harry non riusciva a ricordare di essere mai stato la vittima di quel gioco, ma aveva una buona immaginazione e la sua immaginazione lo stava rendendo furioso. Beh, una volta che la situazione più importante fosse stata risolta, sarebbe stato abbastanza facile per Neville tornare a prendere le sue cose, a condizione che i Serpeverde fossero rimasti troppo intenti su di lui per pensare di fare qualcosa ai libri.

Purtroppo la direzione del suo sguardo fu notata. «Oh», disse il più grande dei ragazzi, «vuoi il libro, piccolino –»

«Chiudi quella bocca», disse Harry freddamente. \textit{Continua a confonderli. Non fare quello che si aspettano. Non ricadere nello schema che prevede che ti intimidiscano.} «Tutto questo fa parte di un qualche piano incredibilmente intelligente che vi farà ottenere un vantaggio futuro, o è un’inutile vergogna per il nome di Salazar Serpeverde come –»

Il ragazzo più grosso spinse con forza Harry Potter, che finì lungo disteso sul pavimento di pietra dura di Hogwarts, fuori dal circolo dei Serpeverde.

E i Serpeverde risero.

Harry si alzò in quello che gli sembrò un movimento terribilmente lento. Non sapeva ancora come usare la bacchetta, ma non c’era alcun motivo di lasciare che questo lo fermasse, date le circostanze.

«Mi piacerebbe pagare \textit{tutti i punti necessari} per sbarazzarmi di questa persona», disse Harry, indicando con il dito il più grosso dei Serpeverde.

Poi Harry sollevò l’altra mano, disse «Abracadabra», e schioccò le dita.

Alla parola \textit{Abracadabra} due dei Tassofrasso, tra cui Neville, urlarono, tre Serpeverde si gettarono disperatamente lontano dalla direzione in cui puntava il dito di Harry, e il Serpeverde più grande barcollò all’indietro con un’espressione di stupore, un’improvviso schizzo rosso che gli decorava viso e collo e petto.

Harry \textit{non} si era aspettato \textit{quello}.

Lentamente, il Serpeverde più grosso portò la mano alla testa e staccò la tortiera con la crostata di ciliegie che gli si era appena panneggiata addosso. Tenne la tortiera in mano per un attimo, fissandola, poi la fece cadere sul pavimento.

Probabilmente non fu il momento migliore del mondo affinché un Tassofrasso iniziasse a ridere, ma ciò fu esattamente quello che uno dei Tassofrasso fece.

Poi Harry si accorse della nota sul fondo della tortiera.

«Aspetta», disse Harry, e si lanciò in avanti per raccogliere la nota. «Questa nota è per me, credo –»

«\textit{Tu}», ringhiò il Serpeverde più grosso, «\textit{tu, stai, per} –»

«\textit{Guarda qui}!» gridò Harry, brandendo la nota contro di lui. «Cioè, \textit{guarda}! Riesci a credere che mi sono stati addebitati trenta punti per la spedizione e la movimentazione di una pidocchiosa torta? Trenta punti! Si sta rivelando un affare in perdita anche salvare un ragazzo innocente in pericolo! E le spese di stoccaggio? Gli oneri di cessione? Gli \textit{ammortamenti}? Come fai ad avere degli \textit{ammortamenti} su di una torta?»

Ci fu un’altra di quelle pause imbarazzate. Harry rivolse pensieri mortali a chiunque fosse il Tassofrasso che non sembrava in grado di smettere di ridacchiare, quell’idiota stava per fargi fare una brutta fine.

Harry fece un passo indietro e rivolse ai Serpeverde il suo miglior sguardo letale. «Ora andate via o dovrò solo continuare a rendere la vostra esistenza sempre più surreale fino a che non lo farete. Lasciate che vi avverta… incasinare la \textit{mia} vita tende a rendere la \textit{vostra} vita… un po’ scabrosa. Avete inteso?»

Con un unico terribile movimento, il Serpeverde più grosso estrasse la bacchetta per puntarla contro Harry e nello stesso istante fu colpito dall’altro lato della testa da un’altra torta, questa volta color mirtillo acceso.

La nota su questa torta era piuttosto grande e chiaramente leggibile. «Potresti voler leggere la nota di quella torta», osservò Harry. «Penso che questa volta sia per te.»

Lentamente il Serpeverde allungò la mano, prese la tortiera, la rigirò con un glop umido che fece cadere altro mirtillo sul pavimento, e lesse una nota che diceva:


\begin{addmargin}[3em]{3em}% 1em left, 2em right
~
\begin{center}
\textsc{\underline{attenzione}}

\textsc{\underline{nessuna} magia può essere usata sul concorrente}

\textsc{mentre la partita è in corso}

\textsc{ulteriori interferenze con la partita}

\textsc{\underline{saranno} segnalate alle autorità del gioco}\\
\end{center}
\end{addmargin}

L’espressione di assoluto sconcerto sul volto del Serpeverde era un’opera d’arte. Harry pensò che questo Direttore del Gioco avrebbe potuto iniziare a piacergli.

«Ascoltate», disse Harry, «che ne direste di finirla qui? Credo che le cose stiano andando fuori controllo. Che ne pensate se voi ve ne tornate a Serpeverde e io me ne torno a Corvonero e ci calmiamo tutti un po’, va bene?»

«Ho un’idea migliore», sibilò il Serpeverde più grosso. «Che ne diresti se tu ti rompessi accidentalmente tutte le dita?»

«Come fai in nome di Merlino a mettere in scena un incidente credibile dopo avermi minacciato di fronte a una dozzina di testimoni, \textit{idiota} –»

Il Serpeverde più grosso allungò le mani lentamente, deliberatamente verso quelle di Harry, e Harry si bloccò sul posto, la parte del suo cervello che stava notando l’età e la forza dell’altro ragazzo che finalmente riusciva a farsi sentire, urlando \textsc{cosa diavolo sto facendo}?

«Aspetta!» disse uno degli altri Serpeverde, improvvisamente in preda al panico. «Basta, non devi farlo veramente!»

Il Serpeverde più grosso lo ignorò, prendendo la mano destra di Harry saldamente nella propria sinistra, e tenendo il dito indice di Harry nella propria mano destra.

Harry fissò il Serpeverde dritto negli occhi. Una parte di Harry stava urlando, non era previsto che questo accadesse, non era \textit{permesso} che accadesse, gli adulti non avrebbero mai lasciato che una cosa del genere accadesse \textit{realmente} –

Lentamente, il Serpeverde cominciò a piegargli l’indice all’indietro.

\textit{Non mi ha realmente rotto il dito e non è degno di me anche solo sussultare fino a che non lo farà. Fino ad allora, questo è solo un altro tentativo di farmi paura.}

«Fermo!», disse il Serpeverde che si era opposto in precedenza. «Basta, questa è una pessima idea!»

«Sono piuttosto d’accordo», disse una voce gelida. Una voce di donna anziana.

Il Serpeverde più grosso lasciò andare la mano di Harry e saltò all’indietro come se si fosse scottato.

«Professoressa Sprout!» gridò uno dei Tassofrasso, sembrando più felice di chiunque altro Harry avesse mai sentito in vita sua.

Mentre Harry si girava, nel suo campo visivo entrò una piccola donna tarchiata con i capelli ricci grigi e disordinati e le vesti ricoperte di sporcizia. Puntò il dito contro i Serpeverde. «Giustificatevi», disse. «Che cosa state facendo con i miei Tassofrasso e…» lo guardò. «Il mio bravo allievo, Harry Potter.»

\textit{Uh oh. È vero, era \textsc{sua} la lezione che ho saltato questa mattina.}

«Ha minacciato di ucciderci!» disse d’impulso uno dei Serpeverde, lo stesso che aveva chiesto all’altro di fermarsi.

«Cosa?» disse Harry senza capire. «\textit{Non è vero}! Se avessi voluto uccidervi non vi avrei minacciato prima pubblicamente!»

Un terzo Serpeverde rise senza riuscire a smettere e poi si fermò di colpo, quando gli altri gli rivolsero sguardi assassini.

La professoressa Sprout aveva assunto un’espressione piuttosto scettica. «Quale sarebbe stata questa minaccia di morte, esattamente?»

«La Maledizione Mortale! Ha finto di utilizzare la Maledizione Mortale!»

La professoressa Sprout si voltò a guardare Harry. «Sì, una minaccia davvero terribile da parte di un bambino di undici anni. Anche se si tratta di qualcosa che non dovrebbe \textit{mai} sognarsi di simulare, Harry Potter.»

«Non conosco neppure le \textit{parole} per la Maledizione Mortale», disse Harry immediatamente. «E non ho mai estratto la mia bacchetta in nessun momento.»

Ora la professoressa Sprout stava rivolgendo a Harry uno sguardo incredulo. «Suppongo che questo ragazzo si sia colpito con due torte \textit{da solo}, allora.»

«\textit{Non ha} usato la bacchetta!» disse tutto d’un fiato uno dei Tassofrasso più piccoli. «Non so come abbia fatto, ha semplicemente schioccato le dita e la torta è comparsa!»

«Naturalmente», disse la professoressa Sprout dopo una pausa. Estrasse la propria bacchetta. «Non glielo ordino, dal momento che sembra che qui lei sia la vittima, ma le dispiacerebbe se controllassi la sua bacchetta per verificarlo?»

Harry tirò fuori la propria bacchetta. «Cosa devo –»

«\textit{Prior Incantato}», recitò Sprout. Aggrottò la fronte. «Strano, la bacchetta non sembra essere mai stata utilizzata.»

Harry scrollò le spalle. «Non lo è stata, infatti, ho ricevuto la mia bacchetta e i miei libri di scuola solo un paio di giorni fa.»

Sprout annuì. «Allora abbiamo un chiaro caso di magia accidentale da parte un ragazzo che si sentiva minacciato. E le regole affermano chiaramente che lei non deve essere considerato responsabile. Quanto a \textit{voi…}» si girò verso i Serpeverde. I suoi occhi si abbassarono deliberatamente sui libri di Neville che giacevano sul pavimento.

Ci fu un lungo silenzio durante il quale guardò i cinque Serpeverde.

«Tre punti da Serpeverde, \textit{per ciascuno}», disse alla fine. «E sei per \textit{lui}», indicando il ragazzo coperto di torta. «Non importunate mai più i miei Tassofrasso, o il mio allievo Harry Potter. Adesso \textit{andate}.»

Non ebbe bisogno di ripetersi, i Serpeverde si voltarono e se ne andarono molto rapidamente.

Neville si avvicinò e iniziò a raccogliere i suoi libri. Sembrava che stesse piangendo, ma solo un poco. Forse era a causa di un trauma ritardato, o perché gli altri ragazzi lo stavano aiutando.

«Grazie \textit{mille}, Harry Potter», disse la professoressa Sprout. «Sette punti a Corvonero, uno per ogni Tassofrasso che hai contribuito a proteggere. E non dirò nulla di più.»

Harry sbatté le palpebre. Si era aspettato qualcosa di più simile a una lezione su come tenersi fuori dai guai, e una lavata di capo piuttosto severa per l’assenza dalla sua lezione.

Forse sarebbe \textit{dovuto} andare a Tassofrasso. Sprout era forte.

«\textit{Scourgify}», recitò Sprout al pasticcio di torta sul pavimento, che subito scomparve.

E se ne andò, camminando lungo il corridoio che portava alla sala da studio verde.

«Come hai \textit{fatto}?» sibilò uno dei ragazzi di Tassofrasso, non appena se ne fu andata.

Harry sorrise compiaciuto. «Posso far accadere tutto quello che voglio semplicemente schioccando le dita.»

Gli occhi del ragazzo si spalancarono. «\textit{Davvero}?»

«No», disse Harry. «Ma quando racconterete a tutti questa storia, assicuratevi di condividerla con Hermione Granger del primo anno di Corvonero, ha un aneddoto che potreste trovare divertente.» Non aveva assolutamente idea di quello che stava succedendo, ma non aveva intenzione di lasciarsi sfuggire l’opportunità di nutrire la propria crescente leggenda. «Oh, e che cos’era quella storia della Maledizione Mortale?»

Il ragazzo gli rivolse uno strano sguardo. «Davvero non lo sai?»

«Se lo sapessi, non te lo chiederei.»

«Le parole della Maledizione Mortale sono», il ragazzo deglutì e la sua voce si ridusse a un sussurro, e tenne le mani lontano dai fianchi come per mettere bene in chiaro che non reggeva una bacchetta, «\textit{Avada Kedavra}.»

\textit{Beh, ovviamente.}

Harry la mise sulla sua crescente lista di cose da non dire mai a suo padre, il professor Michael Verres-Evans. Era già abbastanza brutto spiegare come tu fossi l’unico a essere sopravvissuto alla terribile Maledizione Mortale, senza dover ammettere che la Maledizione Mortale era «Abracadabra».

«Capisco», disse Harry, dopo una pausa. «Beh, questa è l’ultima volta che avrò detto \textit{quella cosa} prima di schioccare le dita.» Anche se \textit{aveva} prodotto un effetto che sarebbe potuto essere tatticamente utile.

«Ma perché hai –»

«Sono figlio di Babbani, i Babbani credono che sia uno scherzo e che sia divertente. Davvero, è questo quello che è successo. Scusami, puoi ricordarmi il tuo nome?»

«Sono Ernie Macmillan», disse il Tassofrasso. Tese la mano, e Harry la strinse. «Onorato di conoscerti.»

Harry eseguì un leggero inchino. «Piacere di conoscerti, lascia stare l’onorato.»

Poi gli altri ragazzi si affollarono intorno a lui e ci fu un diluvio improvviso di presentazioni.

Quando ebbero finito, Harry deglutì. Quello sarebbe stato molto difficile. «Uhm… se mi scusate… ho qualcosa da dire a Neville –»

Tutti gli occhi si rivolsero a Neville, che fece un passo indietro, il volto preoccupato.

«Suppongo», Neville disse con un filo di voce, «che stai per dire che avrei dovuto essere più coraggioso –»

«Oh, no, niente del genere!» Harry si affrettò a dire. «Niente a che vedere con \textit{quello}. È solo, ehm, qualcosa che il Cappello Smistatore mi ha detto –»

Improvvisamente gli altri ragazzi sembrarono molto interessati, fatta eccezione per Neville, che sembrò preoccupato \textit{ancora di più.}

Apparentemente c’era qualcosa che bloccava la gola di Harry. Sapeva che avrebbe dovuto semplicemente dirlo tutto d’un fiato, e si sentiva come se avesse ingoiato un grosso mattone che si era bloccato di traverso.

Fu come se Harry dovesse prendere manualmente il controllo delle proprie labbra e produrre ogni sillaba singolarmente, ma ci riuscì. «Sono, dis, piaciuto.» Espirò e fece un respiro profondo. «Per quello che ho fatto, uhm, l’altro giorno. Tu… non c’è bisogno che tu sia generoso a riguardo o cose simili, ti capirei se mi odiassi. Non sto cercando di sembrare figo scusandomi o né devi ritenerti obbligato ad accettare le mie scuse. Quello che ho fatto era sbagliato.»

Ci fu una pausa.

Neville strinse i suoi libri più strettamente al petto. «Perché l’hai fatto?» disse con un filo di voce esitante. Sbatté le palpebre, come se cercasse di trattenere le lacrime. «Perché \textit{tutti} mi fanno questo, anche il Ragazzo-Che-È-Sopravvissuto?»

Harry si sentì improvvisamente più piccolo di quanto non fosse mai stato in vita sua. «Mi dispiace», disse Harry nuovamente, la sua voce ora roca. «È solo… sembravi così spaventato, era come un segno sopra la testa con scritto ‘vittima’, e volevo dimostrarti che le cose \textit{non sono} sempre brutte, che a volte i mostri ti regalano la cioccolata… Ho pensato che se te l’avessi dimostrato, avresti capito che non c’è poi così tanto di cui avere paura –»

«Ma \textit{c’è}», sussurrò Neville. «L’hai visto oggi, \textit{c’è} da aver paura!»

«Non ti avrebbero fatto niente di veramente brutto di fronte a dei testimoni. La loro arma principale è la paura. Ecco perché prendono di mira te, perché possono vedere che hai paura. Volevo renderti meno spaventato… mostrarti che la paura è peggiore della stessa realtà… o questo è quello che mi dicevo, ma il Cappello Smistatore mi ha detto che stavo mentendo a me stesso e che in realtà l’ho fatto perché era divertente. Ed è per questo che ti sto chiedendo scusa –».

«Mi hai fatto male», disse Neville. «Proprio ora. Quando mi hai afferrato e mi hai tirato lontano da loro.» Neville stese il braccio e indicò dove Harry l’aveva afferrato. «Potrei avere un livido qui, più tardi, da quanto duramente mi hai strattonato. Quando mi hai urtato mi hai persino fatto più male di qualunque cosa abbiano fatto i Serpeverde.»

«\textit{Neville}!» sibilò Ernie. «Stava cercando di \textit{salvarti}!»

«Mi dispiace», sussurrò Harry. «Quando ho visto la situazione mi sono… veramente infuriato…»

Neville lo guardò fisso. «Allora mi hai tirato fuori molto bruscamente e ti sei messo dove mi trovavo e te ne sei uscito con ‘Ciao, io sono il Ragazzo-Che-È-Sopravvissuto’.»

Harry annuì.

«Penso che sarai molto figo, un giorno», disse Neville. «Ma in questo momento, non lo sei.»

Harry inghiottì l’improvviso nodo alla gola e se ne andò. Proseguì lungo il corridoio fino all’incrocio, poi si girò a sinistra in un altro corridoio e continuò a camminare, alla cieca.

Che cosa si \textit{aspettavano} che facesse? Che non si arrabbiasse? Non era sicuro che avrebbe potuto fare qualcosa senza essere arrabbiato e chissà cosa sarebbe successo a Neville e ai suoi libri allora. Inoltre, Harry aveva letto abbastanza libri fantasy per sapere come sarebbe andata a finire \textit{quella} faccenda. Avrebbe cercato di sopprimere la rabbia e avrebbe fallito e sarebbe continuata a venire ancora fuori. E dopo tutto quel lungo viaggio alla scoperta di sé, alla fine avrebbe imparato che la sua rabbia era una parte di sé stesso e che solo accettandola poteva imparare a usarla con saggezza. \textit{Guerre stellari} era l’unico universo in cui la risposta in realtà era \textit{davvero} che ci si doveva tagliare fuori completamente dalle emozioni negative, e qualcosa di Yoda aveva sempre fatto odiare a Harry il piccolo imbecille verde.

Così l’ovvio piano per risparmiare tempo era quello di saltare il viaggio alla scoperta di sé stesso e andare dritto alla parte in cui si rendeva conto che solo accettando la sua rabbia come una parte di sé poteva mantenerne il controllo.

Il problema era che non \textit{si sentiva} fuori controllo quando era arrabbiato. La fredda rabbia lo faceva sentire come se fosse \textit{in} controllo della situazione. Era solo quando si guardava indietro che \textit{gli eventi nel loro insieme} sembravano essere… esplosi fuori controllo, in qualche modo.

Si chiese quanto il Direttore del Gioco avesse a cuore questo genere di problemi, e se aveva guadagnato o perso punti per questo motivo. Harry stesso si sentiva di aver perso un bel po’ di punti, ed era sicuro che la vecchia signora nel dipinto gli avrebbe detto che il suo era l’unico parere che contava.

E Harry si chiese anche se il Direttore del Gioco avesse inviato la professoressa Sprout. Era il pensiero più logico: la nota aveva minacciato di informare le Autorità del Gioco, e poi era apparsa la professoressa Sprout. Forse la professoressa Sprout era il Direttore del Gioco — il Preside di Casa Tassofrasso sarebbe stata l’ultima persona che chiunque avrebbe sospettato, il che avrebbe dovuta metterla vicino alla cima della lista di Harry. Aveva letto uno o due romanzi gialli, anche.

«Allora, come sto andando nel gioco?» disse ad alta voce.

Un foglio di carta volò sopra la sua testa, come se qualcuno l’avesse lanciato da dietro di lui — Harry si voltò, ma non c’era nessuno — e quando Harry si girò di nuovo in avanti, stava adagiandosi a terra.

La nota diceva:

\begin{addmargin}[3em]{3em}% 1em left, 2em right
~

\textsc{punti per lo stile: 10}

\textsc{punti per il buon senso: -3.000.000}

\textsc{bonus punti casa corvonero: 70}

\textsc{punti correnti: -2.999.871}

\textsc{turni restanti: 2}\\
\end{addmargin}

«\textit{Meno tre milioni di punti}?» disse Harry indignato al corridoio vuoto. «Mi sembra eccessivo! Voglio presentare un ricorso presso le Autorità del Gioco! E come faccio a recuperare tre milioni di punti nei prossimi due turni?»

Un’altra nota volò sopra la sua testa.

\begin{addmargin}[3em]{3em}% 1em left, 2em right
~

\textsc{ricorso: respinto}

\textsc{porre le domande sbagliate: -1.000.000.000.000}

\textsc{punti correnti: -1.000.002.999.871}

\textsc{turni restanti: 1}\\
\end{addmargin}

Harry rinunciò. Con un solo turno rimasto, tutto quello che poteva fare era provare la sua ipotesi migliore, anche se non era molto buona. «La mia soluzione è che il gioco rappresenta la vita.»

Un ultimo foglio di carta volò sopra la sua testa, con scritto:

\begin{addmargin}[3em]{3em}% 1em left, 2em right
\begin{center}
~

\textsc{tentativo fallito}

\textsc{fallito fallito fallito}

\textsc{aiiiiiiiiiieeeeeeeeeeeeee}

\textsc{punti correnti: meno infinito}

\textsc{\underline{hai perso la partita}}

\textsc{istruzione finale:}

\textit{recati nell’ufficio della professoressa McGonagall}\\
\end{center}
\end{addmargin}

L’ultima riga era di suo pugno.

Harry fissò l’ultima riga per un po’, poi scrollò le spalle. Bene. Che l’ufficio della professoressa McGonagall fosse. Se il Direttore del Gioco fosse stata \textit{lei…}

Ok, onestamente, Harry non aveva assolutamente idea di come si sarebbe sentito se la professoressa McGonagall fosse stata il Direttore del Gioco. La sua mente stava raffigurando solo il vuoto assoluto. Era, letteralmente, inimmaginabile.

Un paio di ritratti più tardi — non fu un viaggio lungo, l’ufficio della professoressa McGonagall non era lontano dalla sua aula di Trasfigurazione, almeno non di lunedì negli anni dispari — Harry era fermo fuori dalla porta del suo ufficio.

Bussò.

«Avanti», disse la voce attutita della professoressa McGonagall.

Entrò.



