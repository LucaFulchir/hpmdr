% !TeX root = Harry.tex

\chapter{Tutto ciò a cui credo è falso}
\label{capitolo:2}

\emph{«Ovviamente è stata colpa mia. Non c’è nessun altro qui che possa essere responsabile di qualcosa.»}

~\\
~\\

«Ora, giusto per essere chiari», disse Harry, «se la professoressa ti facesse levitare, Papà, quando sai che non sei stato attaccato ad alcun filo, allora quella sarebbe una prova sufficiente. Non potrai cambiare idea e dire che si tratta di un trucco da prestigiatore. Non sarebbe onesto. Se la pensi in questo modo, devi dirlo \emph{ora}, e possiamo trovare un altro esperimento, invece.»

Il padre di Harry, il professor Michael Verres-Evans, alzò gli occhi al cielo. «Sì, Harry.»

«E tu, Mamma, la tua teoria dice che la professoressa dovrebbe essere in grado di farlo, quindi se non succede, ammetterai che ti sbagli. Niente ‘la magia non funziona quando le persone non ci credono’, o cose del genere.»

La vicepreside Minerva McGonagall stava guardando Harry con un’espressione incredula. Aveva un aspetto abbastanza simile a quello di una strega, con le vesti nere e il cappello a punta, ma quando parlava suonava formale e scozzese, cosa che non andava affatto d’accordo con il suo aspetto. A prima vista sembrava che dovesse ridere fragorosamente e infilare bambini nei paioli, ma l’effetto era rovinato non appena apriva bocca. «È sufficiente, signor Potter?» chiese. «Posso procedere con la dimostrazione?»

«\emph{Sufficiente}? Probabilmente no», disse Harry. «Ma almeno sarà \emph{d’aiuto}. La prego di procedere, Vicepreside.»

«Professoressa è sufficiente», ella disse, e poi, «\emph{Wingardium Leviosa.}»

Harry osservò suo padre.

«Uh», fece Harry.

Suo padre lo guardò a sua volta. «Uh», gli fece eco.

Poi il professor Verres-Evans guardò la professoressa McGonagall. «Va bene, può rimettermi giù, ora.»

Suo padre fu premurosamente riportato a terra.

Harry si passò una mano tra i capelli. Forse era solo per quella strana parte di lui che era \emph{già} stata convinta, ma… «È un po’ una delusione», disse. «Si penserebbe che ci debba essere un qualche tipo di evento mentale ben più drammatico associato all’aggiornamento delle credenze basato su di un’osservazione di probabilità infinitesimale –» Harry si fermò. Mamma, la strega, e persino Papà gli stavano rivolgendo di nuovo \emph{quello} sguardo. «Voglio dire, allo scoprire che tutto ciò a cui credo è falso.»

Sul serio, sarebbe dovuto essere più drammatico. Il suo cervello avrebbe dovuto cestinare per intero l’attuale collezione di ipotesi sull’universo, nessuna delle quali permetteva che accadesse un evento come quello. Invece il suo cervello sembrava aver detto, \emph{Va bene, ho visto la professoressa di Hogwarts agitare la bacchetta e far alzare per aria tuo padre, e allora?}

La strega stava rivolgendo loro un sorriso benevolo, sembrando piuttosto divertita. «Desidera un’altra dimostrazione, signor Potter?»

«Non è tenuta a farlo», disse Harry. «Ha eseguito l’esperimento decisivo. Ma…» esitò. Non poteva trattenersi. In effetti, nelle circostanze correnti non \emph{doveva} trattenersi. Era giusto e corretto essere curiosi. «Cos’altro è in grado di fare?»

La professoressa McGonagall si trasformò in un gatto.

Harry balzò all’indietro senza pensarci, e arretrò così velocemente che inciampò in una pila di libri e atterrò duramente sul proprio sedere con un tonfo. Le sue mani erano scese ad attutire il colpo senza riuscirci del tutto, e sentì una fitta alla spalla quando il peso andò giù pesantemente.

Immediatamente il piccolo gatto tigrato si trasformò nuovamente in una donna vestita. «Mi dispiace, signor Potter», disse la strega, sembrando sincera, sebbene gli angoli delle sue labbra fossero rivolti verso l’alto. «Avrei dovuto avvisarla.»

Harry respirava a piccoli rantoli. La sua voce venne fuori strozzata. «\emph{Non può \textsc{farlo}!}»

«È solo una Trasfigurazione», disse la professoressa McGonagall. «Una trasformazione Animagus, per essere precisi.»

«Si è trasformata in un gatto! In un \textsc{piccolo} gatto! Ha violato la Conservazione dell’Energia! Non è solo una regola arbitraria, è implicita nella forma dell’hamiltoniana quantistica! Rigettandola distrugge l’unitarietà e poi ottiene una comunicazione superluminale! E i gatti sono \textsc{complicati}! Una mente umana non può neppure visualizzare l’anatomia di un gatto completo e, e tutta la biochimica del gatto, e che dire della sua \textit{neurologia}? Come può continuare a pensare con il cervello delle dimensioni di quello di un gatto?»

Le labbra della professoressa McGonagall si contrassero ancor di più. «Magia.»

«La magia \textit{non è sufficiente} a fare questo! Dovrebbe essere un dio!»

La professoressa McGonagall sbatté le palpebre. «È la prima volta che mi chiamano \textit{così}.»

La visione di Harry si stava offuscando, mentre il suo cervello iniziava a comprendere cosa era stato appena distrutto. L’intera idea di un universo unificato con leggi matematicamente regolari, questo era ciò che era finito giù per lo scarico; l’intera nozione di fisica. Tremila anni a risolvere i problemi grossi e complicati dividendoli in parti più piccole, a scoprire che l’armonia dei pianeti era la stessa melodia della mela che cade, a trovare che le vere leggi erano perfettamente universali e non avevano eccezioni in alcun luogo e che prendevano la forma di una matematica semplice che governava le parti più piccole, senza contare che la mente era il cervello e il cervello era fatto di neuroni, un cervello era ciò che una persona \textit{era} –

E poi una donna si trasformava in un gatto, e tutto finiva.

Un centinaio di domande combatterono per la precedenza sulle labbra di Harry, e quella vincitrice emerse: «E, e che razza di incantesimo è \textit{Wingardium Leviosa}? Chi inventa le parole di questi incantesimi, i bambini della scuola dell’infanzia?»

«Basta così, signor Potter», disse seccamente la professoressa McGonagall, anche se i suoi occhi brillavano di represso divertimento. «Se vuole imparare qualcosa sulla magia, le suggerisco di completare la sua iscrizione in modo che possa andare a Hogwarts.»

«Giusto», disse Harry un po’ stordito. Riordinò i pensieri. La Marcia della Ragione avrebbe solo dovuto ricominciare da capo, tutto lì; avevano ancora il metodo sperimentale e questa era la cosa importante. «Come faccio ad andare a Hogwarts, allora?»
Una risata strozzata sfuggì alla professoressa McGonagall, come se le fosse stata estratta con una pinzetta.

«Aspetta un attimo, Harry», disse suo padre. «Ricordi perché non sei andato a scuola finora? Che facciamo per la tua condizione?»
La professoressa McGonagall si voltò a guardare Michael. «La sua condizione? Di che si tratta?»

«Non dormo correttamente», disse Harry. Agitò la mano in segno di impotenza. «Il mio ciclo del sonno dura ventisei ore, vado sempre a dormire due ore dopo, ogni giorno. Non riesco a prendere sonno prima, e il giorno successivo vado a dormire altre due ore dopo. Alle 22, a mezzanotte, alle 2, alle 4, finché non fa un giro completo. Anche se tento di svegliarmi prima, non fa nessuna differenza e mi sento uno straccio tutto il giorno. Ecco perché non sono andato alle scuole normali fino a oggi.»

«Una delle ragioni», disse sua madre. Harry le lanciò un’occhiataccia.

McGonagall emise un lungo \textit{hmmmmm}. «Non riesco a ricordare di aver mai sentito parlare di una tale condizione, in passato…» disse lentamente. «Controllerò con Madam Pomfrey per vedere se conosce qualche rimedio.» Poi il suo viso si illuminò. «No, sono certa che questo non sarà un problema — nel tempo troverò una soluzione. Ora», e il suo sguardo si fece nuovamente acuto, «quali sono queste \textit{altre} ragioni?»

Harry indirizzò un’occhiataccia ai propri genitori. «Sono un obiettore di coscienza contro la coscrizione dei bambini, perché non ritengo di dover soffrire a causa dell’incapacità di un sistema scolastico in disintegrazione di fornire insegnanti o materiali di studio di qualità anche minimamente adeguata.»

Entrambi i genitori di Harry reagirono scoppiando a ridere, come se pensassero che fosse tutto un grosso scherzo. «Oh», disse il padre di Harry, gli occhi allegri, «è per \textit{questo} che in terza elementare hai morso una professoressa di matematica.»

«\textit{Non sapeva cosa fosse un logaritmo!}»

«Certo», concordò la madre di Harry. «Morderla è stata una reazione molto matura.»

Il padre di Harry annuì. «Una soluzione ben ponderata per affrontare il problema degli insegnanti che non conoscono i logaritmi.»

«Avevo \textit{sette anni}! Per quanto tempo ancora avete intenzione di continuare a ritirare fuori quella faccenda?»

«Lo so», disse la madre con comprensione, «mordi \textit{una} insegnante di matematica e non se ne dimenticano più, vero?»

Harry si voltò verso la professoressa McGonagall. «Ecco! Vede quello che mi tocca subire?»

«Scusatemi», disse Petunia, e attraverso la porta sul retro fuggì nel giardino, da dove le sue risate erano chiaramente udibili.

«Non, ah, non», la professoressa McGonagall sembrò avere problemi a parlare, per qualche ragione, «non dovranno esserci morsi per gli insegnanti a Hogwarts, è abbastanza chiaro, signor Potter?»

Harry la guardò con cipiglio. «Va bene, non morderò nessuno che non morda me per primo.»

Anche il professor Michael Verres-Evans dovette lasciare la stanza per un po’, dopo aver sentito quelle parole.

«Bene», disse sospirando la professoressa McGonagall, dopo che i genitori di Harry si erano ricomposti ed erano tornati. «Bene. Credo che, date le circostanze, dovrei evitare di portarla a comprare il suo corredo scolastico fino a uno o due giorni prima dell’inizio della scuola».

«Cosa? Perché? Gli altri bambini conoscono già la magia, non è vero? Devo iniziare subito a recuperare terreno!»

«Stia certo, signor Potter», replicò la professoressa McGonagall, «che Hogwarts è decisamente in grado di insegnare le basi. E io sospetto, signor Potter, che se la lasciassi da solo per due mesi con i suoi libri di scuola, anche senza una bacchetta, ritornerei in questa casa solo per trovare un cratere rigonfio di fumo rosa, una città spopolata attorno ad esso, e un’epidemia di zebre fiammeggianti che terrorizza ciò che resta dell’Inghilterra.»

La madre e il padre di Harry annuirono in perfetto unisono.

«\textit{Mamma! Papà!}»
