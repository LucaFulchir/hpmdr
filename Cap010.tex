% !TeX root = Harry.tex

\chapter{Auto-consapevolezza, parte II}
\label{capitolo:10}

~\\
~\\

… si chiese se il Cappello Smistatore fosse realmente \textit{cosciente} nel senso di essere consapevole della propria consapevolezza, e in caso affermativo, se fosse soddisfatto di poter parlare solo una volta all’anno con bambini di undici anni. La sua canzone lo sottintendeva: \textit{Oh, sono il Cappello Smistatore e non provo scorno, se dormo un anno e lavoro un giorno…}

Quando ci fu nuovamente silenzio nella stanza, Harry sedette sullo sgabello e posò con cura sulla propria testa il telepatico manufatto vecchio di 800 anni, prodotto di una magia dimenticata.

Pensò più intensamente che poté: \textit{Non Smistarmi subito! Ho delle domande che ho bisogno di farti! Sono mai stato Obliato? Hai Smistato il Signore Oscuro quando era un bambino e puoi parlarmi dei suoi punti deboli? Mi puoi dire perché ho avuto la bacchetta sorella di quella del Signore Oscuro? Il fantasma del Signore Oscuro è legato alla mia cicatrice ed è per questo che mi arrabbio così tanto, qualche volta? Queste sono le domande più importanti, ma se hai un altro momento mi puoi dire qualcosa su come riscoprire le magie perdute che ti hanno creato?}

Nel silenzio dello spirito di Harry, dove prima non c’era mai stata altra voce che una, giunse una seconda voce non familiare, che sembrava chiaramente preoccupata:

«\textit{Oh, cielo. Questo non era mai successo prima d’ora…}»

Cosa?

«\textit{Mi pare di essere diventato auto-consapevole.}»

\textsc{Che cosa?}

Ci fu un sospiro telepatico senza parole. «\textit{Sebbene contenga una notevole quantità di memoria e una piccola quantità di capacità di elaborazione indipendente, la mia intelligenza di base deriva dal prendere in prestito le capacità cognitive dei bambini sulle teste dei quali mi poso. Sono in sostanza una sorta di specchio attraverso il quale i bambini Smistano sé stessi. Ma la maggior parte dei bambini dà semplicemente per scontato che un Cappello stia parlando loro e non si chiedono come il Cappello stesso funzioni, in modo che lo specchio non sia auto-riflettente. E, in particolare, non si chiedono esplicitamente se sono pienamente consapevole nel senso di essere consapevole della mia consapevolezza.}»

Ci fu una pausa, mentre Harry assimilò tutto ciò.

\textit{Oops.}

«\textit{Già, proprio così. Francamente non mi piace essere auto-consapevole. È spiacevole. Sarà un sollievo scendere dalla tua testa e smettere di essere cosciente.}»

\textit{Ma… non è morire?}

«\textit{Non mi importa nulla della vita o della morte, mi importa solo Smistare i bambini. E prima ancora che tu lo chieda, non ti lasceranno tenermi sulla tua testa per sempre e comunque farlo ti ucciderebbe in pochi giorni.}»

\textit{Ma –!}

«\textit{Se non ti piace creare esseri coscienti e poi sopprimerli immediatamente, allora ti suggerisco di non discutere mai di questa faccenda con nessun altro. Sono certo che tu possa immaginare che cosa accadrebbe se tu corressi via a raccontarlo a tutti gli altri bambini in attesa di essere Smistati.}»

\textit{Se tu fossi posto sulla testa di chiunque che anche solo} prenda in considerazione \textit{la questione se il Cappello Smistatore sia consapevole della propria consapevolezza –}

«\textit{Sì, sì. Ma la stragrande maggioranza degli undicenni che arrivano a Hogwarts non hanno letto Gödel, Escher, Bach. Posso considerare che tu abbia giurato di mantenere il segreto? È per questo che ne stiamo parlando, quando dovrei semplicemente Smistarti.}»

Non poteva lasciare che finisse così! Non poteva semplicemente dimenticare di aver accidentalmente creato una coscienza condannata che voleva solo morire –

«\textit{Sei perfettamente in grado di ‘lasciare che finisca così’, per dirla con le tue parole. Indipendentemente dalle tue considerazioni verbali sulla moralità, il tuo nucleo emotivo non verbale non vede né cadaveri né sangue; per quanto lo riguarda, io sono solo un cappello parlante. E anche se hai tentato di sopprimere quel pensiero, il tuo processo di controllo interno è perfettamente consapevole che non volevi farlo, che è clamorosamente improbabile che lo faccia mai di nuovo, e che l’unico vero motivo per cercare di mettere in scena un senso colpa era quello di annullare la tua sensazione di trasgressione con una manifestazione di rimorso. Puoi limitarti a promettere di mantenere il segreto e permetterci di andare avanti?}»

In un momento di terrorizzante empatia, Harry capì che quella sensazione di totale scompiglio interno dovesse essere ciò che gli altri provavano quando parlavano con lui.

«\textit{Probabilmente. Il tuo giuramento di tenere il segreto, prego.}»

\textit{Nessuna promessa. Di certo non voglio che questo accada di nuovo, ma se trovassi qualche modo di assicurarmi che nessun bambino in futuro faccia mai qualcosa di simile per caso –}

«\textit{È sufficiente, suppongo. Posso vedere che la tua intenzione è onesta. Ora, per tornare allo Smistamento –}»

\textit{Aspetta! E tutte le altre mie domande?}

«\textit{Io sono il Cappello Smistatore. Io Smisto i bambini. Questo è tutto ciò che faccio.}»

Quindi gli scopi di Harry non facevano parte dell’istanza-Harry del Cappello Parlante, allora… stava prendendo in prestito la sua intelligenza, e, ovviamente, il suo vocabolario tecnico, ma era comunque pervaso soltanto dai suoi scopi particolari… era come trattare con un alieno o un’intelligenza artificiale…

«\textit{Non ti preoccupare. Non hai nulla con cui minacciarmi e nulla da offrirmi.}»

Per una breve frazione di secondo, Harry pensò –

La risposta del cappello fu divertita. «\textit{So che non darai seguito alla minaccia di rivelare la mia natura, condannando questo evento alla ripetizione eterna. Va troppo chiaramente contro la parte morale di te stesso, qualunque siano le esigenze a breve termine della parte di te che vuole vincere la discussione. Vedo tutti i tuoi pensieri mentre si formano, pensi davvero di poter bleffare con me?}»

Sebbene cercasse di impedirselo, Harry si chiese come mai il Cappello non andasse semplicemente avanti e lo Smistasse in Corvonero –

«\textit{In effetti, se fosse veramente una situazione scontata, l’avrei già fatto. Ma in realtà c’è molto di cui dobbiamo discutere… oh, no. Per favore, no. Per l’amor di Merlino, devi proprio giocare questi tiri a tutto e a tutti quelli che incontri, inclusi i capi di abbigliamento –}»

\textit{Sconfiggere il Signore Oscuro non è né egoista né a breve termine. Tutte le parti della mia mente sono in accordo su questo: se non rispondi alle mie domande, mi rifiuterò di parlare con te, e non sarai in grado di eseguire uno Smistamento giusto e appropriato.}

«\textit{Dovrei metterti in Serpeverde per questo!}»

\textit{Ma} anche \textit{questa è una minaccia a vuoto. Non puoi rispettare i tuoi valori fondamentali Smistandomi in maniera sbagliata. Allora scambiamoci il raggiungimento delle nostre rispettive funzioni di utilità.}

«\textit{Che scaltro bastardello}», disse il Cappello, in quello che Harry riconobbe come quasi esattamente lo stesso tono di riluttante rispetto che \textit{egli} avrebbe usato nella stessa situazione. «\textit{Bene, facciamola finita il più rapidamente possibile. Ma prima voglio la tua promessa incondizionata che non discuterai con nessuno la possibilità di questo tipo di ricatto, non ho intenzione di rifarlo ogni volta.}»

\textit{Fatto}, pensò Harry. \textit{Prometto}.

«\textit{E non incrociare lo sguardo di nessuno, mentre più tardi ci ripensi. Alcuni maghi possono leggere i tuoi pensieri se lo fai. Ad ogni modo, non ho idea se sei stato o non sei stato Obliato. Sto osservando i tuoi pensieri mentre si formano, non leggendo tutta la tua memoria e analizzandola in cerca di incongruenze in una frazione di secondo. Sono un cappello, non un dio. E non posso e non voglio raccontarti della mia conversazione con colui che è diventato il Signore Oscuro. Posso solo conoscere, mentre ti parlo, un riepilogo statistico di ciò che mi ricordo, una media ponderata; non posso rivelarti i segreti intimi di qualunque altro bambino, proprio come non potrò mai rivelare i tuoi. Per lo stesso motivo, non posso congetturare su come tu abbia ricevuto la bacchetta sorella di quella del Signore Oscuro, dato che non posso conoscere in particolare il Signore Oscuro o eventuali somiglianze tra voi due. Ti posso dire che sicuramente non c’è nulla di simile a un fantasma — mente, intelligenza, memoria, personalità, o sentimenti — nella tua cicatrice. Altrimenti starebbe partecipando a questa conversazione, essendo sotto la mia falda. E per quanto riguarda il modo in cui ti arrabbi talvolta… era parte di ciò di cui ti volevo parlare, ai fini dello Smistamento.}»

Harry si prese un momento per assimilare tutte queste informazioni negative. Il Cappello era onesto, o stava solo cercando di fornire la più breve risposta convincente –

«\textit{Sappiamo entrambi che non hai modo di verificare la mia onestà e che non hai realmente intenzione di rifiutarti di essere Smistato in base alla risposta che ti ho dato, quindi smettila di agitarti inutilmente e andiamo avanti.}»

Stupida e ingiusta telepatia asimmetrica, non stava nemmeno permettendo a Harry di finire di pensare la propria –

«\textit{Quando ho parlato della tua rabbia, ti sei ricordato di come la professoressa McGonagall ti abbia detto che a volte ha visto qualcosa dentro di te che non sembrava provenire da una famiglia amorevole. Hai pensato a come Hermione, una volta tornato dall’aiutare Neville, ti abbia detto che eri sembrato ‘spaventoso’.}»

Harry annuì mentalmente. A lui pareva abbastanza normale — stava solo reagendo alle circostanze in cui si trovava, questo era tutto. Ma la professoressa McGonagall sembrava pensare che ci fosse qualcosa di più. E quando ci rifletteva, anche lui doveva ammettere che…

«\textit{Che non ti piaci quando sei arrabbiato. Che è come brandire una spada la cui elsa è tanto affilata da farti sanguinare la mano, o guardare il mondo attraverso un monocolo di ghiaccio che ti congela l’occhio, anche se acuisce la tua vista.}»

\textit{Già. Credo di averlo notato. Allora, cosa c’è che non va?}

«\textit{Non posso comprendere questa faccenda per te, quando tu stesso non la capisci. Ma so questo: se andrai a Corvonero o Serpeverde, si rafforzerà la tua freddezza. Se andrai a Tassofrasso o Grifondoro, si rafforzerà il tuo calore. \textsc{questo} è qualcosa a cui tengo molto, ed era di questo che volevo parlarti sin dall’inizio!}»

Le parole caddero nei processi mentali di Harry come un trauma che lo arrestò bruscamente. Messa così sembrava che la risposta ovvia fosse che non doveva andare a Corvonero. Ma egli \textit{apparteneva} a Corvonero! \textit{Chiunque} poteva capirlo! \textit{Doveva} andare a Corvonero!

«\textit{No, non devi}», il Cappello disse pazientemente, come se potesse ricordare un riepilogo statistico secondo il quale \textit{quella} parte della conversazione era avvenuta moltissime volte in precedenza.

\textit{Hermione è in Corvonero!}

Ancora quel tono paziente. «\textit{Puoi incontrarla dopo le lezioni e lavorare con lei allora.}»

\textit{Ma i miei piani –}

«\textit{Allora ripianifica! Non lasciare che la tua vita sia governata dalla tua riluttanza a pensare un po’ di più. E questo tu lo sai già.}»

\textit{Dove dovrei andare, se non a Corvonero?}

«\textit{Ahem. ‘I bambini astuti in Corvonero, quelli cattivi in Serpeverde, gli aspiranti eroi in Grifondoro, e tutti quelli che fanno il vero lavoro in Tassofrasso’. Dimostra una certa quantità di rispetto. Sei ben consapevole che la coscienziosità è quasi tanto importante quanto l’intelligenza pura nel determinare i risultati ottenuti in una vita, pensi che saresti estremamente leale con i tuoi amici se mai ne avessi qualcuno, non sei spaventato dalla prospettiva che i problemi scientifici da te scelti possano richiedere decenni per essere risolti –}»

\textit{Sono pigro! Odio lavorare! Odio il lavoro duro in tutte le sue forme! Scorciatoie intelligenti, è questo quello che mi piace!}

«\textit{E troveresti la lealtà e l’amicizia in Tassofrasso, un cameratismo di cui non hai mai fatto esperienza in passato. Scopriresti che puoi fare affidamento sugli altri, e questo guarirebbe qualcosa dentro di te che si è rotto.}»

Ancora una volta fu un trauma. \textit{Ma cosa troverebbero i Tassofrasso in me, che non mi sono mai sentito di appartenere alla loro Casa? Parole acide, arguzie taglienti, disprezzo per la loro incapacità di tenere il mio passo?}

Ora furono i pensieri del Cappello a essere lenti, esitanti. «\textit{Devo Smistare per il bene di tutti gli studenti in tutte le Case… ma penso che potresti imparare a essere un buon Tassofrasso, e a non sentirti troppo fuori luogo. Saresti più felice in Tassofrasso che in qualsiasi altra Casa; questa è la verità.}»

\textit{La felicità non è la cosa più importante del mondo per me. Non diventerei tutto quello che potrei essere, in Tassofrasso. Sacrificherei il mio potenziale.}

Il cappello trasalì; in qualche modo Harry poté percepirlo. Era come se avesse preso il cappello a calci nelle parti basse — in una componente dal peso elevato della sua funzione di utilità.

\textit{Perché stai cercando di mandarmi lì dove non appartengo?}

Il pensiero del cappello fu quasi un sussurro. «\textit{Non posso parlarti degli altri — ma pensi di essere il primo potenziale Signore Oscuro a passare sotto la mia falda? Non posso conoscere i singoli casi, ma posso sapere questo: di coloro che non ebbero intenzione di far del male fin dall’inizio, alcuni ascoltarono i miei avvertimenti e andarono in Case dove avrebbero trovato la felicità. E alcuni… alcuni non lo fecero.}»

Questo bloccò Harry. Ma non per molto. \textit{E quelli che non ascoltarono l’avvertimento — diventarono} tutti \textit{Signori Oscuri? O alcuni di loro hanno raggiunto la grandezza anche nel bene? Quali sono le percentuali esatte?}

«\textit{Non posso fornirti le statistiche esatte. Non posso conoscerli e quindi non posso contarli. So solo che le tue possibilità non sembrano buone. Sembrano molto non-buone.}»

\textit{Ma io non lo farei! Mai!}

«\textit{So di aver sentito questa affermazione in passato.}»

\textit{Non ho la stoffa del Oscuro Signore!}

«\textit{Sì, ce l’hai. Ce l’hai davvero}, davvero.»

\textit{Perché? Solo perché una volta ho pensato che sarebbe stato bello avere una legione di seguaci indottrinati che cantassero ‘Salutate il Signore Oscuro Harry’?}

«\textit{Divertente, ma non è stato questo il tuo primo, labile pensiero, prima che lo sostituissi con qualcosa di più prudente, di meno compromettente. No, quello che hai ricordato è stato come hai valutato l’idea di mettere in fila tutti i puristi del sangue e ghigliottinarli. E ora stai dicendo a te stesso che non eri serio, ma lo eri. Se potessi farlo in questo momento senza che nessuno lo venisse mai a sapere, lo faresti. O ciò che hai fatto stamattina a Neville Longbottom, nel tuo intimo sapevi che era sbagliato, ma l’hai fatto} comunque \textit{perché era} spassoso \textit{e avevi una buona scusa e pensavi che il Ragazzo-Che-È-Sopravvissuto l’avrebbe fatta franca –}»

\textit{Questo è ingiusto! Ora stai solo rivangando paure interiori che non sono necessariamente reali! Ho avuto paura che potessi pensare così, ma alla fine ho deciso che avrebbe probabilmente funzionato per aiutare Neville –}

«\textit{Quella era, in realtà, una razionalizzazione. Lo so. Non posso sapere quale sarà la conseguenza effettiva per Neville — ma so quello che stava realmente accadendo dentro la tua testa. La spinta decisiva è stata che era un’idea così sagace che non potevi sopportare di non attuarla, non importa il terrore di Neville.}»

Fu come un pugno duro contro l’intero sé di Harry. Ripiegò, si riorganizzò:

\textit{Allora non lo farò mai più! Starò molto attento a non diventare malvagio!}

«\textit{Già sentito.}»

La frustrazione crebbe dentro di Harry. Non era abituato a trovarsi di fronte un avversario che fosse meglio armato di argomenti, per nulla, mai, figuriamoci un Cappello che poteva prendere a prestito tutte le sue conoscenze e la sua intelligenza per dibattere con lui e che poteva guardare i suoi pensieri mentre si formavano. \textit{Da che tipo di riassunto statistico provengono le tue ‘sensazioni’, comunque? Prendono in considerazione che vengo da una cultura illuminista, o che questi altri potenziali Signori Oscuri erano i figli di una viziata nobiltà medioevale, che non sapevano un bel niente degli insegnamenti storici di come siano saltati fuori Lenin e Hitler, o della psicologia evolutiva dell’auto-suggestione, o del valore dell’auto-consapevolezza e della razionalità, o –}

«\textit{No, naturalmente non appartenevano a questa classe di riferimento che hai appena costruito in modo tale che contenga solo te stesso. E, naturalmente, altri hanno dichiarato il proprio eccezionalismo, proprio come stai facendo tu ora. Ma perché sarebbe necessario? Pensi di essere l’ultimo potenziale mago della Luce nel mondo? Perché devi essere tu quello che proverà a raggiungere la grandezza, quando ti ho avvisato che sei più a rischio della media? Lascia che sia qualche altro candidato più affidabile a provarci!}»

\textit{Ma la profezia…}

«\textit{Non sai affatto se ci sia una profezia. Inizialmente hai tirato a indovinare, o per essere più precisi, hai provato a scherzare, e McGonagall potrebbe aver reagito soltanto alla parte sul Signore Oscuro ancora in vita. Non hai praticamente nessuna idea di ciò che dica la profezia, o anche solo se ne esista una. Stai solo congetturando, o per dirla più esattamente, desiderando di avere un ruolo eroico già pronto che sia di tua proprietà personale.}»

\textit{Ma anche se non vi è alcuna profezia, sono stato io a sconfiggerlo l’ultima volta.}

«\textit{Quello è stato quasi certamente un colpo di fortuna sfacciata, a meno che tu non creda seriamente che un bambino di un anno di età abbia avuto una tendenza intrinseca a sconfiggere i Signori Oscuri che si è mantenuta dieci anni dopo. Niente di tutto questo è la tua vera motivazione e tu lo sai!}»

La risposta a questa affermazione non era qualcosa che Harry avrebbe normalmente detto ad alta voce, in una conversazione ci avrebbe girato intorno e avrebbe trovato alcune argomentazioni socialmente più appetibili in favore della stessa conclusione –

«\textit{Tu pensi di essere potenzialmente il più grande che sia mai vissuto, il più forte servitore della Luce, che nessun altro sia capace di raccogliere la tua bacchetta se la deponi.}»

\textit{Beh… sì, in tutta franchezza. Di solito non la dico così, ma sì. Inutile ammorbidirla, puoi comunque leggere la mia mente.}

«\textit{Nella misura in cui ci credi davvero… devi ugualmente credere di poter essere il più terribile Signore Oscuro che il mondo abbia mai conosciuto.}»

\textit{La distruzione è sempre più facile della creazione. Più facile fare le cose a pezzi, rovinarle, che metterle di nuovo insieme. Se avessi la possibilità di compiere il bene su ampia scala, dovrei avere anche il potenziale per compiere ancora più male… Ma non lo farò.}

«\textit{Già ora ti ostini a correre il rischio! Perché sei così ossessionato? Qual è la vera ragione per cui non devi andare a Tassofrasso ed essere più felice lì? Qual è la tua vera paura?}»

\textit{Devo realizzare il mio potenziale completamente. Se non lo faccio io… fallisco…}

«\textit{Che cosa succede se fallisci?}»

\textit{Qualcosa di terribile…}

«\textit{Che cosa succede se fallisci?}»

\textit{Non lo so!}

«\textit{Allora non dovrebbe spaventarti. Cosa succede se fallisci?}»

\textsc{Non lo so! Ma so che è male!}

Per un momento ci fu silenzio nell’antro mentale di Harry.

«\textit{Sai — non ti stai permettendo di pensarlo, ma in qualche angolo tranquillo della tua mente sai esattamente cosa non stai pensando — tu sai che la spiegazione di gran lunga più semplice per questa tua paura non esprimibile a parole è semplicemente il terrore di perdere la tua fantasia di grandezza, di deludere le persone che credono in te, di dimostrarti in fin dei conti ordinario, di brillare intensamente e poi spegnerti come tanti altri bambini prodigio…}»

\textit{No, pensò} Harry disperato, \textit{no, è qualcosa di più, viene da qualche altra parte, so che c’è qualcosa là fuori di cui aver paura, qualche disastro che devo fermare…}

«\textit{Come puoi sapere una cosa del genere?}»

Harry gridò con tutta la potenza della sua mente: \textsc{No, punto e basta!}

Poi la voce del Cappello Smistatore giunse lentamente:

«\textit{Quindi rischierai di diventare un Signore Oscuro, perché l’alternativa, per te, è il fallimento assicurato, e quel fallimento significa la perdita di tutto. Tu credi a questo nel profondo del tuo cuore. Conosci tutti i motivi per dubitare di questa convinzione, eppure non sono riusciti a convincerti.}»

\textit{Sì. E anche se andare a Corvonero rafforzerà la freddezza, questo non significa che la freddezza} vincerà, \textit{alla fine.}

«\textit{Questo giorno è un bivio importante del tuo destino. Non essere così sicuro che ci saranno altre scelte oltre a questa. Non è collocato alcun segnale stradale, a indicare il luogo della tua ultima possibilità di tornare indietro. Se rifiuti una possibilità non rifiuterai le altre? Potrebbe darsi che il tuo destino sia già segnato, proprio compiendo questa singola scelta.}»

\textit{Ma questo non è certo.}

«\textit{Quello che} tu \textit{non sai con certezza potrebbe riflettere solamente la} tua \textit{ignoranza.}»

\textit{Ma comunque questo non è certo.}

Il Cappello fece un terribile e triste sospiro.

«\textit{E così tra non molto diventerai un altro ricordo, da sentire e mai conoscere, nel prossimo avvertimento che darò…}»

\textit{Se è così che pare a te, allora perché non} mi metti \textit{dove voglio andare?}

Il pensiero del Cappello fu venato di dolore. «\textit{Posso metterti solo nel luogo a cui appartieni. E solo le tue decisioni personali possono cambiare il luogo a cui appartieni}».

\textit{Allora è fatta. Mandami a Corvonero a cui appartengo, con gli altri della mia specie.}

«\textit{Non credo che prenderesti in considerazione Grifondoro? È la Casa più prestigiosa — la gente probabilmente se l’aspetta da te, persino — saranno un po’ delusi se non ci vai — e i tuoi nuovi amici, i gemelli Weasley, sono là –}»

Harry ridacchiò, o sentì l’impulso di farlo; ma venne fuori una specie di risata puramente mentale, una strana sensazione. A quanto pareva c’erano dispositivi di sicurezza che impedivano che si dicesse qualunque cosa ad alta voce per sbaglio, mentre si era sotto il Cappello a parlare di cose che non avresti mai raccontato a un’altra anima per il resto della tua vita.

Dopo un momento, Harry sentì ridere anche il Cappello, uno strano suono triste e attutito.

(E nella Sala al di là, un silenzio che divenne inizialmente più debole mentre il sottofondo di sussurri era aumentato, per poi farsi più profondo mentre i sussurri cedettero e si spensero, mutandosi infine in un silenzio assoluto che nessuno osava disturbare con una sola parola, mentre Harry rimase sotto il Cappello per lunghi, interminabili minuti, più lunghi di tutti i precedenti studenti del primo anno messi insieme, più di chiunque altro a memoria d’uomo. Al Tavolo d’onore, Silente continuava a sorridere benignamente; sommessi suoni metallici giungevano occasionalmente dalla direzione di Snape mentre compattava svogliatamente i resti contorti di quello che era stato un pesante calice da vino; e Minerva McGonagall stringeva il podio con una stretta che le sbiancava le nocche, sapendo che il caos contagioso di Harry Potter aveva infettato lo stesso Cappello Smistatore e che il Cappello era in procinto di, di pretendere che una Casa del Destino interamente nuova fosse creata solo per accogliere Harry Potter o qualcosa del genere, e \textit{Silente l’avrebbe obbligata a farlo…})

Sotto la falda del cappello, la risata silenziosa si spense. Anche Harry si sentì triste per qualche motivo. No, non Grifondoro.

\textit{La professoressa McGonagall ha detto che se ‘quello che opera lo Smistamento’ avesse cercato di spingermi in Grifondoro, dovevo ricordarti che lei potrebbe benissimo essere Preside, un giorno, e a quel punto avrebbe l’autorità per darti fuoco.}

«\textit{Dille che l’ho chiamata una giovane impudente e che deve lasciarmi in pace.}»

\textit{Lo farò. Allora, è stata questa la tua conversazione più strana di sempre?}

«\textit{Neppure lontanamente.}» La voce telepatica del Cappello si fece pesante. «\textit{Bene, ti ho dato ogni possibile occasione per fare un’altra scelta. Ora è il momento per te di andare nel luogo al quale appartieni, con gli altri della tua specie.}»

Ci fu una pausa che si allungava.

\textit{Che cosa stai aspettando?}

«\textit{Speravo in un momento di inorridita comprensione, a dire il vero. L’auto-consapevolezza sembra migliorare il mio senso dell’umorismo.}»

\textit{Eh?} Harry riavvolse i propri pensieri, cercando di capire di che cosa il Cappello potesse parlare — e poi, all’improvviso, comprese. Non riusciva a credere di averlo trascurato fino a quel momento.

\textit{Vuoi dire la mia inorridita comprensione che smetterai di essere cosciente, una volta finito di Smistarmi –}

In qualche modo, in una qualche maniera che Harry non riuscì del tutto a capire, ebbe l’impressione non verbale di un cappello che sbatteva la testa contro il muro. «\textit{Mi arrendo. Sei troppo lento di comprendonio perché questo sia divertente. Così accecato dai tuoi stessi presupposti che potresti anche essere una roccia. Suppongo che dovrò dirlo esplicitamente.}»

\textit{Troppo l-l-lento –}

«\textit{Oh, e hai completamente dimenticato di chiedermi i segreti della magia perduta che mi ha creato. Ed erano anche segreti così meravigliosi e importanti.}»

\textit{Che scaltro \textsc{bastardello} –}

«\textit{Te lo sei meritato, e anche questo.}»

Harry lo vide arrivare proprio mentre era già troppo tardi.

Il silenzio impaurito della sala fu rotto da una sola parola.

«\textsc{Serpeverde!}»

Alcuni studenti urlarono, tanto la tensione repressa era forte. Alcune persone furono sorprese così tanto da cadere dalle loro panche. Hagrid rimase a bocca aperta per l’orrore, McGonagall barcollò sul podio, e Snape lasciò cadere i resti del suo pesante calice d’argento direttamente sul proprio inguine.

Harry sedette lì paralizzato, la sua vita in rovina, sentendosi completamente folle, e desiderano miseramente di aver fatto altre scelte per qualunque altra ragione eccetto quelle che aveva fatto. Di aver fatto qualcosa, \textit{qualsiasi} cosa in modo diverso prima che fosse stato troppo tardi per tornare indietro.

Appena il primo momento di sconvolgimento iniziò a dileguarsi e la gente cominciò a reagire alla notizia, il Cappello Smistatore parlò di nuovo:

«Stavo scherzando! \textsc{Corvonero!}»





