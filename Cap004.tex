% !TeX root = Harry.tex

\chapter{L’ipotesi del mercato efficiente}
\label{capitolo:4}

\emph{«Conquista del mondo è una frase così brutta. Preferisco chiamarla ottimizzazione del mondo.»}

~\\
~\\

Cumuli di galeoni d’oro. Montagne di sicli d’argento. Pile di zellini di bronzo.

Harry rimase a bocca aperta a fissare la camera di sicurezza della sua famiglia. Aveva così tante domande da non sapere \textit{da dove} iniziare.

Appena fuori dalla porta della camera, la professoressa McGonagall lo stava osservando, apparentemente appoggiata casualmente contro il muro, ma i suoi occhi ben attenti. Beh, era comprensibile. Essere messo di fronte a un mucchio gigante di monete d’oro era una prova di carattere così pura che doveva essere archetipica.

«Queste monete sono di metallo puro?» domandò infine Harry.

«Cosa?» sibilò il goblin Griphook, che stava attendendo vicino alla porta. «Sta mettendo in dubbio l’integrità di Gringotts, signor Potter-Evans-Verres?»

«No», disse Harry con aria assente, «affatto, mi dispiace se mi sono espresso male, signore. È solo che non ho nessuna idea di come funzioni il vostro sistema finanziario. Sto chiedendo se i galeoni in generale sono fatti d’oro puro.»

«Naturalmente», disse Griphook.

«E può coniarli chiunque, o sono emessi da un monopolio che di conseguenza riscuote il signoraggio?»

«Cosa?» disse la professoressa McGonagall.

Griphook sorrise, mostrando i denti appuntiti. «Solo un folle si fiderebbe di qualcosa di diverso dall’oro dei goblin!»

«In altre parole», disse Harry, «le monete non valgono più del metallo con cui sono coniate?»

Griphook fissò Harry. La professoressa McGonagall sembrava sconcertata.

«Voglio dire, supponga che venga qui con una tonnellata di argento. Potrei farmici coniare una tonnellata di sicli?»

«In cambio di una commissione, signor Potter-Evans-Verres.» Il goblin lo guardò con occhi che brillavano. «In cambio di una certa commissione. Dove intende trovare una tonnellata di argento, mi chiedo?»

«Stavo parlando ipoteticamente», disse Harry. \textit{Per ora, ad ogni modo.} «Allora… quanto chiedereste di commissione, come frazione del peso complessivo?»

Gli occhi di Griphook erano attenti. «Dovrei consultare i miei superiori…»

«Mi dia una stima. Non considererò Gringotts vincolata ad essa.»

«La ventesima parte del metallo potrebbe ben pagare la coniazione.»

Harry annuì. «La ringrazio molto, signor Griphook.»

\textit{Quindi non solo l’economia dei maghi è quasi completamente disaccoppiata dall’economia babbana, nessuno qui ha mai sentito parlare di arbitraggio}. La più grande economia babbana aveva un tasso di cambio oro-argento fluttuante, quindi ogni volta che il rapporto babbano oro-argento si allontanava di più del 5\% dal peso di diciassette sicli per un galeone, o l’oro o l’argento sarebbero dovuti essere drenati dall’economia dei maghi fino a rendere impossibile mantenere il tasso di cambio. Portare qui una tonnellata di argento, cambiarla in sicli (e pagare il 5\%), cambiare i sicli in galeoni, portare l’oro nel mondo babbano, scambiarlo per più argento di quello con cui si era iniziato, e ripetere.

Il tasso di cambio babbano tra oro e argento non era forse intorno a cinquanta a uno? Harry non riteneva che fosse diciassette, ad ogni modo. E sembrava che le monete d’argento fossero effettivamente \textit{più piccole} delle monete d’oro.

Ma, del resto, Harry si trovava in una banca che conservava il tuo denaro \textit{letteralmente} in camere di sicurezza piene di monete d’oro e custodite da draghi, in cui dovevi entrare e portare fuori le monete dal tuo deposito ogni volta che dovevi spendere del denaro. La sottile arte di eliminare le inefficienze del mercato tramite l’arbitraggio sarebbe potuta benissimo essere incomprensibile per loro. Sarebbe stato tentato di fare commenti sprezzanti sulla crudezza del loro sistema finanziario…

\textit{Ma la cosa triste è che probabilmente il loro modo è migliore.}

D’altro canto, un solo esperto gestore di fondi speculativi sarebbe stato probabilmente in grado di prendere possesso dell’intero mondo della magia in una settimana. Harry mise da parte quella nozione, nel caso in cui si fosse trovato a corto di denaro, o avesse avuto una settimana libera.

Nel frattempo, i giganteschi cumuli di monete d’oro all’interno del deposito dei Potter sarebbero dovuti essere sufficienti per le sue necessità a breve termine.

Harry fece un passo avanti, e iniziò a raccoglier monete d’oro con una mano, e a metterle nell’altra.

Quando fu arrivato a venti, la professoressa McGonagall tossì. «Credo che quelle saranno più che sufficienti per pagare il suo corredo scolastico, signor Potter.»

«Uhm?» disse Harry, con la mente altrove. «Aspetti, sto facendo una stima di Fermi.»

«Una \textit{cosa}?» disse la professoressa McGonagall, sembrando piuttosto preoccupata.

«È una cosa matematica. Prende il nome da Enrico Fermi. Un modo di manipolare mentalmente quantità approssimative…»

Venti galeoni d’oro pesavano un decimo di chilogrammo, più o meno? E l’oro stava, quanto, diecimila sterline al chilogrammo? Quindi un galeone valeva circa cinquanta sterline… I mucchi di monete d’oro sembravano essere alti circa sessanta monete e larghi venti monete in entrambe le direzioni, e i mucchi erano piramidali, quindi all’incirca un terzo del cubo. Ottomila galeoni per mucchio, all’incirca, e c’erano circa cinque mucchi di quella grandezza, quindi quarantamila galeoni o due milioni di sterline.

Niente male. Harry sorrise con una certa cupa soddisfazione. Era un peccato che fosse proprio nel mezzo della scoperta del nuovo e meraviglioso mondo della magia, e non potesse prendersi del tempo per esplorare il nuovo e meraviglioso mondo dell’essere ricco, che una rapida stima di Fermi sostenne essere circa un miliardo di volte meno interessante.

\textit{Ad ogni modo, ho spinto un tagliaerba per una pidocchiosa sterlina per l’ultima volta.}

Harry si voltò dal gigantesco mucchio di denaro. «Mi scusi se glielo chiedo, professoressa McGonagall, ma mi pare di capire che i miei genitori fossero poco più che ventenni quando sono morti. Questa è una somma di denaro normale da tenere in deposito per una giovane coppia, nel mondo dei maghi?» Se lo fosse stata, probabilmente una tazza di tè costava cinquemila sterline. Regola numero uno dell’economia: non puoi mangiare il denaro.

La professoressa McGonagall scosse la testa. «Suo padre era l’ultimo erede di un’antica famiglia, signor Potter. È anche possibile…» La strega esitò. «Parte di questo denaro potrebbe provenire dalle taglie poste su Tu-Sai-Chi, riscuotibili dal suo uccis– ah, da chiunque l’abbia sconfitto. O quelle taglie potrebbero non essere state ancora riscosse. Non ne sono sicura.»

«Interessante…» disse Harry lentamente. «Quindi parte di questo denaro è davvero, in un certo senso, mio. Cioè, guadagnato da me. Più o meno. Forse. Anche se non ne ricordo le circostanze.» Le dita di Harry tamburellarono sulla gamba del suo pantalone. «Questo mi fa sentire meno colpevole a spenderne \textit{una frazione davvero minuscola! Non si faccia prendere dal panico, professoressa McGonagall!»}

«Signor Potter! Lei è un minore, e in quanto tale le sarà permesso di effettuare prelievi \textit{ragionevoli} dal –»

«Io sono \textit{completamente} ragionevole! Sono completamente d’accordo con la prudenza finanziaria e il controllo degli impulsi! Ma ho \textit{visto} alcune cose lungo la strada che costituirebbero acquisti \textit{sensati, da adulto}…»

Harry resse lo sguardo della professoressa McGonagall, ingaggiando con lei una lotta silenziosa.

«Tipo cosa?» disse infine la professoressa.

«Bauli il cui interno contiene più del loro esterno?»

Il volto della professoressa McGonagall divenne più severo. «Sono \textit{molto} costosi, signor Potter!»

«Sì, ma –» supplicò Harry. «Sono certo che quando sarò adulto ne vorrò uno. E \textit{posso} permettermene uno. Ovviamente, sarebbe altrettanto logico comprarlo ora invece che dopo, e usufruirne da subito. Si tratta dello stesso denaro in entrambi i casi, no? Voglio dire, ne \textit{vorrei} uno buono, con \textit{molto} spazio all’interno, abbastanza buono da non doverne prendere uno migliore dopo…» Harry lasciò in sospeso la frase speranzoso.

Lo sguardo della professoressa McGonagall non vacillò «E poi cosa \textit{terrebbe} in un baule come quello, signor Potter –»

«Libri.»

«Naturalmente», sospirò la professoressa McGonagall.

«Avrebbe dovuto dirmi \textit{molto prima} che esisteva quel genere di oggetto magico! E che me ne sarei potuto permettere uno! Ora io e mio padre saremo costretti a trascorrere i prossimi due giorni girando \textit{freneticamente} tutte le librerie di seconda mano in cerca di vecchi libri di testo, così che possa avere una biblioteca scientifica decente con me a Hogwarts – e forse una piccola collezione di fantascienza, se sarò in grado di assemblare qualcosa di dignitoso tra i cesti degli affari. O meglio ancora, renderò la cosa un po’ più dolce per lei, va bene? Mi lasci solo comprare –»

«\textit{Signor Potter}! Pensa di potermi \textit{corrompere}?»

«Che cosa? \textit{No}! Non in quel senso! Stavo dicendo, Hogwarts può tenersi alcuni dei libri che porterò, se lei pensa che qualcuno di essi possa essere una buona aggiunta alla biblioteca. Ho intenzione di prenderli a buon mercato, e \textit{io} voglio solo averli vicini da qualche parte. Va bene corrompere le persone con i \textit{libri}, giusto? Questa è una –»

«Una tradizione di famiglia.»

«Sì, proprio così.»

Il corpo della professoressa McGonagall sembrò crollare, le spalle che si abbassarono sotto le sue vesti nere. «Non posso negare il buon senso delle sue parole, anche se vorrei poterlo fare. Le permetterò di prelevare un centinaio di galeoni in più, signor Potter.» Sospirò di nuovo. «\textit{So} che mi pentirò questo, e lo sto facendo lo stesso.»

«Questo è lo spirito! E una ‘borsa mokeskin’ fa quello che penso che faccia?»

«Non altrettanto bene come un baule», disse la strega con visibile riluttanza, «ma… una borsa mokeskin con un Incantesimo di Recupero e uno dell’Estensione Impercettibile può contenere un certo numero di oggetti fino a quando non siano richiamati da colui che li ha messi dentro –»

«Sì! Ho sicuramente bisogno di una di quelle, anche! Sarebbe come una super-borsa da cintura super-mitica! La Bat-cintura della conservazione! Cosa importa il mio coltellino svizzero, potrei portarci un intero set di strumenti lì dentro! O dei \textit{libri!} Potrei avere con me i primi tre libri che stessi leggendo in ogni momento, e potrei tirarne uno fuori ovunque! Non dovrò mai più sprecare un altro minuto della mia vita! Che ne dice, professoressa McGonagall? È per promuovere la lettura tra i bambini, la miglior causa possibile.»

«… Suppongo che possa prelevare altri dieci galeoni.»

Griphook stava indirizzando a Harry uno sguardo di sincero rispetto, forse addirittura di completa ammirazione.

«E qualche spicciolo, come ha detto prima lei. Credo di ricordare di aver visto una o due cose che potrei voler custodire in quella borsa.»

«\textit{Non tiri troppo la corda, signor Potter.}»

«Ma oh, professoressa McGonagall, perché vuole rovinare la mia festa? Certamente è un giorno \textit{felice} questo, in cui scopro tutte le cose magiche per la prima volta! Perché recitare la parte dell’adulto scontroso, quando invece potrebbe sorridere e ricordare la sua infanzia innocente, osservando l’espressione di gioia sul mio giovane volto mentre acquisto alcuni giocattolini usando una percentuale esigua della ricchezza che ho guadagnato sconfiggendo il più terribile mago che la Gran Bretagna abbia mai conosciuto, non che la stia accusando di essere ingrata o altro, eppure, cosa sono alcuni giocattoli rispetto a quello?»

«\textit{Tu}», ringhiò la professoressa McGonagall. Ci fu un’espressione sul suo viso così temibile e terribile che Harry strillò e fece un passo indietro, rovesciando una pila di monete d’oro con un gran rumore tintinnante e cadendo all’indietro in un mucchio di denaro. Griphook sospirò e si mise un palmo sul viso. «Farei un grande servizio alla Gran Bretagna magica, signor Potter, se la chiudessi in questa camera di sicurezza e me ne andassi senza di lei.»

E se ne andarono senza altri incidenti.
