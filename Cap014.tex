% !TeX root = Harry.tex

\chapter{L’ignoto e l’inconoscibile}
\label{capitolo:14}

\emph{C’erano domande misteriose, ma una risposta misteriosa era una contraddizione in termini.}

~\\
~\\

«Avanti», disse la voce attutita della professoressa McGonagall.

Harry entrò.

L’ufficio della Vicepreside era pulito e ben organizzato; sul muro immediatamente adiacente alla scrivania c’era un labirinto di stipetti di legno di tutte le forme e dimensioni, la maggior parte con diversi rotoli di pergamena infilati dentro, ed era in qualche modo molto chiaro che la professoressa McGonagall sapesse esattamente cosa contenesse ciascuno stipetto, anche se nessun altro poteva dire altrettanto. Una singola pergamena si trovava sulla scrivania vera e propria, che era per il resto vuota. Dietro la scrivania c’era una porta chiusa sbarrata da diverse serrature.

La professoressa McGonagall sedeva su di uno sgabello senza schienale dietro la scrivania, e sembrava perplessa — i suoi occhi si erano spalancati, con una nota forse di apprensione, quando aveva visto Harry.

«Signor Potter?» disse la professoressa McGonagall. «Di cosa si tratta?»

La mente di Harry si svuotò. Aveva ricevuto l’istruzione di gioco di venire qui, si aspettava che lei avesse qualcosa in mente…

«Signor Potter?» ripeté la professoressa McGonagall, iniziando a sembrare un po’ infastidita.

Fortunatamente, il cervello di Harry in pieno panico ricordò proprio in quel momento che aveva qualcosa che voleva discutere con la professoressa McGonagall. Qualcosa di importante e che valeva ampiamente il suo tempo.

«Uhm…» disse Harry. «Se ci fossero incantesimi che potesse formulare per essere sicura che nessuno ci stia ascoltando…»

La professoressa McGonagall si alzò dalla sedia, chiuse la porta esterna, ed estrasse la bacchetta iniziando a pronunciare degli incantesimi.

Fu in quel momento che Harry si accorse di avere l’opportunità inestimabile e probabilmente unica di offrire alla professoressa McGonagall dello SpiriTè e non poté credere di pensare seriamente una cosa del genere e sarebbe andato tutto bene, la soda spariva dopo pochi secondi, e disse a quella parte di sé di chiudere la bocca.

Essa lo fece, e Harry iniziò a organizzare mentalmente ciò che stava per dire. Non aveva previsto di avere questa discussione così presto, ma dato che era qui…

La professoressa McGonagall terminò un incantesimo che sembrava più antico del latino, e si sedette nuovamente.

«Bene», disse con voce tranquilla. «Nessuno ci sta ascoltando.» Il suo volto era alquanto contratto.

Ah, già, si aspetta che io la ricatti riguardo la profezia.

Eh, Harry sarebbe dovuto tornare su quella questione qualche altro giorno.

«Si tratta dell’Incidente con il Cappello Smistatore», disse Harry. (La professoressa McGonagall sbatté le palpebre.) «Uhm… Penso che vi sia un incantesimo supplementare sul Cappello Smistatore, qualcosa di cui lo stesso Cappello Smistatore non è a conoscenza, che si innesca quando il Cappello Smistatore dice ‘Serpeverde’. Ho sentito un messaggio che sono certo i Corvonero non dovrebbero sentire. È giunto nel momento in cui il Cappello Smistatore veniva rimosso dalla mia testa e ho sentito interrompersi la connessione. Suonava come un sibilo e inglese allo stesso tempo», ci fu una netta inspirazione da parte di McGonagall, «e diceva: ‘Saluti da Serpeverde a Serpeverde, se volessi cercare i miei segreti, parla al mio serpente’.»

La professoressa McGonagall sedette immobile con la bocca aperta, fissando Harry come se avesse visto spuntargli altre due teste.

«Così…» disse lentamente la professoressa McGonagall, come se non potesse credere che quelle parole uscissero dalla sua bocca, «ha deciso di venire direttamente da me e parlarmene.»

«Beh, sì, naturalmente», rispose Harry. Non c’era bisogno di ammettere quanto tempo gli ci era voluto per arrivare a pensarlo. «Invece di, diciamo, mettermi alla ricerca io stesso, o parlarne con qualche altro bambino.»

«Comprendo…», disse la professoressa McGonagall. «E se, per caso, scoprisse l’ingresso alla leggendaria Camera dei Segreti di Salazar Serpeverde, un ingresso che soltanto lei potrebbe aprire…»

«Chiuderei l’ingresso e le riferirei immediatamente la cosa, così che possa essere formato un gruppo di esperti archeologi della magia», disse Harry prontamente. «Poi aprirei l’ingresso nuovamente e loro entrerebbero con molta cautela, per accertarsi che non vi sia nulla di pericoloso. Potrei entrare successivamente per dare uno sguardo, o se avessero bisogno di me per aprire qualcos’altro, ma ciò avverrebbe solo dopo che l’area fosse dichiarata sicura e che avessero scattato fotografie di come appariva il tutto prima che la gente calpestasse il loro inestimabile sito storico.»

La professoressa McGonagall rimase seduta con la bocca aperta, fissandolo come se si fosse appena trasformato in un gatto.

«È una cosa ovvia, se non sei un Grifondoro», Harry disse gentilmente.

«Credo», disse la professoressa McGonagall con una voce alquanto strozzata, «che lei sottovaluti di molto la rarità del buon senso, signor Potter.»

Quello sembrava quasi giusto. Sebbene… «Un Tassofrasso avrebbe detto la stessa cosa.»

McGonagall si interruppe, colpita. «Anche questo è vero.»

«Il Cappello Smistatore mi ha offerto Tassofrasso.»

Sbatté le palpebre come se non potesse credere alle proprie orecchie. «L’ha fatto davvero?»

«Sì.»

«Signor Potter», disse McGonagall, e ora la sua voce era grave, «cinque decenni fa è stata l’ultima volta che uno studente è morto tra le mura di Hogwarts, e io sono ora certa che cinque decenni fa sia stata l’ultima volta che qualcuno ha sentito quel messaggio.»

Un brivido attraversò Harry. «Allora starò estremante attento a non intraprendere azioni di alcun genere a riguardo senza consultarla, professoressa McGonagall.» Fece una pausa. «E posso suggerirle di radunare gli esperti migliori che può trovare e verificare se sia possibile rimuovere quell’incantesimo supplementare dal Cappello Smistatore… e se non potete farlo, forse aggiungere un altro incantesimo, un Quietus che si attivi brevemente proprio quando il Cappello è rimosso dalla testa di uno studente, potrebbe funzionare come toppa. Ecco fatto, niente più studenti morti.» Harry annuì con soddisfazione.

La professoressa McGonagall sembrò ancor più sbalordita, se possibile. «Non posso assegnarle abbastanza punti senza assegnare definitivamente la Coppa delle Case a Corvonero.»

«Uhm», disse Harry. «Uhm. Preferirei non guadagnare tutti quei punti.»

Ora la professoressa McGonagall gli stava rivolgendo uno strano sguardo. «Perché no?»

Harry ebbe una qualche difficoltà a tradurlo in parole. «Perché sarebbe semplicemente troppo triste, sa? Come… come quando stavo ancora tentando di andare a scuola nel mondo babbano, e ogni volta che c’era un lavoro di gruppo, partivo e facevo tutto da solo perché gli altri mi avrebbero solo rallentato. Mi sta bene ottenere molti punti, più di chiunque altro, persino, ma se guadagno abbastanza da essere decisivo nella vittoria della Coppa delle Case tutto da solo, allora sarebbe come se portassi Casa Corvonero tutta sulle mie spalle, e questo è troppo triste.»

«Capisco…» disse McGonagall esitante. Era chiaro che questo ragionamento non le era mai venuto in mente. «Supponiamo che le assegni solo cinquanta punti, allora?»

Harry scosse di nuovo la testa. «Non è giusto nei confronti degli altri se guadagno punti per cose da grandi di cui io posso essere parte e loro no. In che modo Terry Boot dovrebbe guadagnare cinquanta punti per aver riferito un sussurro che ha sentito dal Cappello Smistatore? Non sarebbe affatto imparziale.»

«Capisco perché il Cappello Smistatore le abbia offerto Tassofrasso», disse la professoressa McGonagall. Lo stava guardando con uno strano rispetto.

Questo fece rimanere Harry senza parole. Aveva realmente pensato di non essere degno di Tassofrasso. Che il Cappello Smistatore avesse semplicemente tentato di ficcarlo dappertutto tranne che a Corvonero, in una Casa le cui virtù egli non possedeva…

La professoressa McGonagall stava sorridendo, ora. «E se cercassi di assegnarle dieci punti, allora…?»

«Ha intenzione di spiegare da dove vengono quei dieci punti, se qualcuno dovesse chiederlo? Ci potrebbero essere molti Serpeverde, e non mi riferisco ai bambini a Hogwarts, che diventerebbero molto molto irritati se sapessero che l’incantesimo è stato rimosso da Cappello Smistatore e scoprissero che sono coinvolto nella faccenda. Dunque credo che la segretezza assoluta sia un requisito fondamentale. Non c’è bisogno di premiarmi, signora, la virtù è un premio di per sé.»

«Così sia», disse la professoressa McGonagall, «ma ho qualcos’altro di molto speciale da darle. Comprendo che le ho fatto un grosso torto nel giudicarla, signor Potter. Per favore, attenda qui.»

Si alzò, si recò alla porta posteriore chiusa, ondeggiò la bacchetta, e una sorta di tendina sfumata le spuntò attorno. Harry non poté vedere né sentire ciò che stava accadendo. Fu alcuni minuti dopo che l’offuscamento svanì e la professoressa McGonagall era lì in piedi, rivolta verso di lui, con la porta alle sue spalle che pareva non essere mai stata aperta.

E la professoressa McGonagall gli offriva una collanina che teneva in mano, una sottile catenina d’oro con al centro un circolo d’argento, all’interno del quale c’era una clessidra. Nell’altra mano c’era un opuscolo. «Questo è per lei», disse.

Uau! Stava ricevendo una specie di fantastico oggetto magico come ricompensa per una missione andata a buon fine! Apparentemente quella faccenda del rifiutare offerte di ricompense monetarie fino a quando non si riceveva un oggetto magico funzionava nella vita reale, non solo nei giochi per computer.

Harry accettò la sua nuova collanina, sorridendo. «Che cos’è?»

La professoressa McGonagall fece un respiro. «Signor Potter, questo è un oggetto che è normalmente prestato solo a bambini che si siano già dimostrati altamente responsabili, allo scopo di aiutarli con degli orari scolastici complicati.» McGonagall esitò, come se stesse per aggiungere qualcosa. «Devo sottolineare, signor Potter, che la vera natura di questo oggetto è segreta e che non deve rendere nota la sua esistenza a nessun altro studente, o lasciarsi vedere mentre lo usa. Se questo non è accettabile per lei, può restituirmelo ora.»

«So mantenere i segreti», Harry disse. «Quindi, cosa fa?»

«Per quanto concerne gli altri studenti, questo è un Cancelletto ruotante ed è usato per curare una malattia magica non contagiosa e molto rara chiamata Duplicazione Spontanea. Lo indossi sotto i tuoi vestiti, e se non ha alcuna ragione di mostrarlo a qualcuno, non ha neppure alcuna ragione di considerarlo un tremendo segreto. I Cancelletti ruotanti non sono interessanti. Comprende, signor Potter?»

Harry annuì, il suo sorriso che si allargava. Percepiva l’opera di un Serpeverde competente. «E che cosa fa realmente?»

«È un Giratempo. Ogni giro della clessidra la manda indietro nel tempo di un’ora. Così, se lo usasse per andare indietro nel tempo due ore ogni giorno, dovrebbe essere in grado di andare a dormire sempre alla stessa ora.»

La sospensione dell’incredulità di Harry fu completamente spazzata via.

Mi sta dando una macchina del tempo per curare il mio disturbo del sonno.

Mi sta dando una macchina del tempo per curare il mio disturbo del sonno.

mi sta dando una macchina del tempo per curare il mio disturbo del sonno.

«Ehehehehhheheh…» fece la bocca di Harry. Stava ora tenendo la collanina lontana da sé come se fosse una bomba innescata. Beh, no, non come una bomba innescata, questo paragone non coglieva neppure lontanamente la gravità della situazione. Harry tenne la collanina lontana da sé come se fosse una macchina del tempo.

Dica, professoressa McGonagall, lo sa che la materia ordinaria col tempo invertito è uguale all’antimateria? Ebbene sì! Lo sa che un chilogrammo di antimateria che si scontrasse con un chilogrammo di materia si annichilirebbe in un’esplosione equivalente a 43 milioni di tonnellate di tnt? Si rende conto che io stesso peso 41 chilogrammi e che l’esplosione risultante lascerebbe un gigantesco cratere fumante lì dove c’era la scozia?

«Mi scusi», riuscì a dire Harry, «ma questo sembra davvero davvero davvero davvero pericoloso!» la voce di Harry non raggiunse il tono di uno strillo, non avrebbe potuto strillare abbastanza forte da fare giustizia a questa situazione, quindi era inutile provarci.

La professoressa McGonagall gli rivolse uno sguardo di affettuosa tolleranza. «Sono contenta che lei stia prendendo la cosa sul serio, signor Potter, ma i Giratempo non sono così pericolosi. Non li affideremmo a dei bambini se lo fossero.»

«Certo», disse Harry. «Ahahahaha. Ovviamente non dareste delle macchine del tempo a dei bambini se fossero pericolose, ma cosa mi era venuto in mente? Quindi, giusto per chiarezza, starnutire su questo oggetto non mi manderà nel Medioevo dove travolgerò Gutenberg con un carro tirato da cavalli impedendo l’Illuminismo? Perché, sa com’è, odio quando mi succede.»

Le labbra di McGonagall si stavano contorcendo nel modo in cui lo facevano quando cercava di non sorridere. Offrì a Harry l’opuscolo che teneva in mano, ma Harry stava reggendo con cautela la collanina con entrambe le mani e fissando la clessidra per assicurarsi che non stesse per ruotare. «Non si preoccupi», disse McGonagall dopo una pausa momentanea, quando fu chiaro che Harry non aveva nessuna intenzione di muoversi, «questo non può assolutamente accadere, signor Potter. Il Giratempo non può essere usato per muoversi più di sei ore all’indietro nel tempo. Non può essere usato più di sei volte in qualunque giorno.»

«Ah, bene, questo è un bene. E se qualcuno mi urta il Giratempo non si romperà e non intrappolerà l’intero castello di Hogwarts in un ciclo di giovedì che si ripetono senza fine.»

«Beh, in effetti sono fragili…» disse McGonagall. «E credo davvero di ricordare che accadano cose strane se si rompono. Ma nulla di simile!»

«Forse», disse Harry quando poté parlare nuovamente, «dovreste fornire le vostre macchine del tempo di una sorta di guscio protettivo, invece di lasciare esposta la clessidra, in modo da prevenire che ciò accada.»

McGonagall sembrò alquanto colpita. «Questa è un’idea eccellente, signor Potter. Ne informerò il Ministero-»

Ecco fatto, ora è ufficiale, il Parlamento l’ha ratificato, ogni singola persona nel mondo della magia è completamente stupida.

«E sebbene io odi buttarla in filosofia», Harry tentò disperatamente di abbassare la voce a un volume inferiore a quello di un urlo, «qualcuno ha mai pensato alle implicazioni dell’andare indietro di sei ore e fare qualcosa che muti il passato, cosa che in effetti cancellerebbe tutte le persone coinvolte e le sostituirebbe con versioni differenti –»

«Oh, ma non si può cambiare il passato!» la professoressa McGonagall lo interruppe. «Santo cielo, signor Potter, pensa che questi oggetti sarebbero concessi agli studenti se questo fosse possibile? Che succederebbe se qualcuno provasse a cambiare i risultati dei propri esami?»

Harry si prese un momento per esaminare la questione. Le sue mani rilassarono, almeno un po’, la ferrea stretta sulla catenina della clessidra. Come se non stesse tenendo in mano una macchina del tempo, solo una testata nucleare innescata.

«Dunque…» disse Harry lentamente. «Si è scoperto che l’universo… del tutto casualmente è autoconsistente, in qualche modo, anche se comprende i viaggi nel tempo. Se io e il mio futuro me stesso interagissimo, allora osserverei gli stessi eventi come entrambi i me stesso, anche se, al mio primo passaggio, il mio futuro me stesso starebbe agendo sulla base di una conoscenza completa di cose che, dal mio punto di vista, non sono ancora accadute…» la voce di Harry si spense per l’inadeguatezza semantica del suo lessico.

«Corretto, suppongo», disse la professoressa McGonagall. «Sebbene i maghi siano consigliati di evitare di vedere i sé stessi passati. Se sta frequentando due lezioni nello stesso momento e ha bisogno di incrociare il percorso con sé stesso, per esempio, la prima versione dovrebbe farsi da parte e chiudere gli occhi in un certo istante — ha già un orologio, bene — cosicché il sé stesso del futuro possa passare. È tutto nell’opuscolo.»

«Ahahahaa. E che succede se qualcuno ignora quel consiglio?»

La professoressa McGonagall contrasse le labbra. «Mi risulta che questo possa rivelarsi alquanto sconcertante.»

«E non crea, diciamo, un paradosso che distrugge l’intero universo.»

Sorrise tollerante. «Signor Potter, credo che mi ricorderei di aver sentito una cosa simile se fosse mai accaduta.»

«non è tranquillizzante! non avete mai sentito parlare del principio antropico? e quale idiota ha costruito una di queste cose tanto per cominciare?»

La professoressa McGonagall rise di cuore. Era un suono piacevole e lieto che sembrava sorprendentemente fuori posto su quel volto severo. «Sta avendo un altro momento ‘si è mutata in un gatto’, dico bene, signor Potter? Probabilmente non vuol sentirselo dire, ma è alquanto simpaticamente dolce.»

«Mutarsi in un gatto non si avvicina neppure a questo. Sa, fino a ora avevo questo terribile pensiero soppresso da qualche parte nei recessi della mia mente che l’unica risposta rimasta fosse che il mio intero universo fosse una simulazione al computer come nel libro Simulacron 3 ma ora anche questo è escluso perché questo giocattolino non è calcolabile secondo turing! Una macchina di Turing potrebbe simulare il ritorno a un momento preciso del passato e calcolare un futuro differente partendo da lì, una macchina a oracolo potrebbe fare affidamento sul comportamento di terminazione di macchine di livello inferiore, ma quello che sta dicendo è che la realtà in qualche modo si calcola in maniera auto-regolare in un unico passaggio usando informazioni che non sono… accadute… ancora…»

La comprensione colpì Harry con la forza di un maglio da demolizioni.

Ora tutto aveva un senso. Tutto aveva finalmente un senso.

«ecco come funziona lo spiritè! Ma certo! L’incantesimo non forza l’avverarsi di eventi divertenti, ti fa solamente provare l’impulso di bere giusto poco prima che cose divertenti accadano comunque! Che stupido che sono, avrei dovuto accorgermene quando ho sentito l’impulso di bere lo SpiriTè prima del secondo discorso di Silente, non l’ho fatto, e poi invece mi sono strozzato con la saliva — bere lo SpiriTè non causa l’evento comico, l’evento comico fa sì che tu beva lo SpiriTè! Ho notato che i due eventi erano correlati e ho dato per scontato che lo SpiriTè dovesse essere la causa e l’evento comico dovesse essere l’effetto perché ritenevo che l’ordine temporale vincolasse il nesso di causalità e che i grafici causali dovessero essere aciclici ma tutto si spiega quando disegni le frecce causali che vanno indietro nel tempo!»

La comprensione colpì Harry con la forza di un secondo maglio da demolizioni.

Questa volta riuscì a tenerla per sé, emettendo solo un sommesso suono di strangolamento come quello di un gattino morente, mentre comprendeva chi aveva messo la nota sul suo letto quella mattina.

Gli occhi della professoressa McGonagall si accesero. «Dopo il diploma, o forse anche prima, lei deve davvero insegnare alcune di queste teorie babbane a Hogwarts, signor Potter. Sembrano molto affascinanti, anche se sono completamente sbagliate.»

«Glehhahhh…»

La professoressa McGonagall gli offrì alcune ulteriori amenità, richiese alcune ulteriori promesse alle quali Harry acconsentì, disse qualcosa riguardo il non parlare con i serpenti ovunque qualcuno potesse sentirlo, gli ricordò di leggere l’opuscolo, e poi in qualche modo Harry si ritrovò fuori dal suo ufficio con la porta fermamente chiusa dietro di sé.

«Gaahhhrrrraa…» fece Harry.

Sì, la sua mente era esplosa.

Non ultimo per il fatto che, se non fosse stato per lo Scherzo, avrebbe potuto non ottenere affatto il Giratempo.

Oppure la professoressa McGonagall glielo avrebbe dato lo stesso, solo più tardi quello stesso giorno, in un momento in cui fosse riuscito a chiederle del suo disturbo del sonno o a parlarle del messaggio del Cappello Smistatore? E a quel punto, egli avrebbe voluto giocare uno scherzo a sé stesso che lo avrebbe portato a ottenere il Giratempo anticipatamente? In modo che l’unica possibilità autoconsistente fosse quella in cui lo Scherzo fosse iniziato addirittura prima che si fosse svegliato quella mattina…?

Harry si trovò a considerare, per la prima volta nella sua vita, che la risposta alla sua domanda potesse essere letteralmente inconcepibile. Che siccome il suo stesso cervello conteneva neuroni in grado di muoversi solo in avanti nel tempo, non c’era nulla che il suo cervello potesse fare, nessuna operazione che potesse compiere, che fosse il coniugato dell’operazione di un Giratempo.

Fino a quel momento, Harry aveva tenuto fede all’ammonimento di E.T. Jaynes secondo cui se eri ignorante riguardo a un fenomeno, questo era un fatto riguardante il tuo stato mentale, non il fenomeno in sé; che la tua incertezza era un fatto che riguardava te, non la cosa su cui eri incerto; che l’ignoranza esisteva nella mente, non nella realtà; che una mappa vuota non corrispondeva a un territorio vuoto. C’erano domande misteriose, ma una risposta misteriosa era una contraddizione in termini. Un fenomeno poteva essere misterioso per qualche particolare persona, ma non esistevano fenomeni misteriosi di per sé. Venerare un sacro mistero equivaleva a venerare la propria ignoranza.

Così Harry aveva esaminato la magia e si era rifiutato di farsi intimidire. Le persone non avevano una sensibilità storica, imparavano chimica e biologia e astronomia e pensavano che queste materie fossero state da sempre oggetto della scienza, che non fossero mai state misteriose. Le stelle erano state dei misteri, una volta. Lord Kelvin aveva definito la natura della vita e la biologia — la risposta dei muscoli alla volontà umana e la generazione degli alberi dai semi – un mistero «infinitamente al di là» della portata della scienza. (Non solo appena al di là, sia chiaro, ma infinitamente al di là. Lord Kelvin aveva certamente provato un’enorme scarica emotiva dal non sapere una certa cosa.) Ogni mistero che fosse stato risolto era stato un rompicapo sin dagli albori della specie umana fino al momento in cui qualcuno l’aveva risolto.

Ora, per la prima volta, si trovava davanti alla prospettiva di un mistero che minacciava di essere permanente. Se il Tempo non operava secondo reti causali acicliche allora Harry non capiva cosa significassero causa ed effetto; e se Harry non comprendeva causa ed effetto allora non capiva di che genere di sostanza fosse invece fatta la realtà; ed era assolutamente possibile che la sua mente umana non potesse mai capire, perché il suo cervello era composto da antiquati neuroni tempo-lineari, e quello si era rivelato essere un sottoinsieme impoverito della realtà.

Dal lato positivo, lo SpiriTè, che era sembrato onnipotente e incredibile, si era dimostrato dotato di una spiegazione più semplice. A cui non aveva pensato semplicemente perché la verità era completamente al di fuori dello spazio delle sue ipotesi o di qualunque cosa per comprendere la quale il suo cervello si fosse evoluto. Ma ora era riuscito a capirlo, probabilmente. Cosa che era più o meno incoraggiante. Più o meno.

Harry diede uno sguardo al proprio orologio. Erano quasi le 11, era andato a letto all’una della notte precedente, quindi in condizioni normali quella notte sarebbe andato a dormire alle 3. Per poter andare a dormire alle 22 e svegliarsi alle 7, sarebbe dovuto tornare indietro di cinque ore in tutto. Il che voleva dire che se avesse voluto tornare al dormitorio intorno alle 6 del mattino, prima che chiunque fosse sveglio, avrebbe dovuto sbrigarsi e…

Anche in retrospettiva Harry non capiva come fosse riuscito a portare a termine metà delle componenti dello Scherzo. Da dove erano spuntate fuori le torte?

Harry stava iniziando ad avere seriamente paura dei viaggi nel tempo.

D’altro canto, doveva ammettere che era stata un’opportunità unica. Uno scherzo che potevi giocare a te stesso una sola volta nella vita, entro sei ore da quando scoprivi l’esistenza dei Giratempo.

Anzi, era ancora più sconcertante, ora che Harry ci pensava. Il Tempo gli aveva servito lo Scherzo al completo come un fatto compiuto, eppure era, chiaramente, opera sua. Concetto ed esecuzione e stile di scrittura. Ogni singola parte, anche quelle che non capiva.

Beh, stava perdendo tempo e c’erano appena trenta ore in una giornata. Harry sapeva alcune delle cose che doveva fare, e avrebbe indovinato il resto, come la torta, mentre ci lavorava su. Non c’era motivo di scoraggiarsi. Non poteva ottenere nulla bloccato lì nel futuro.

\begin{figure}[h!]
        \includegraphics[scale=0.4]{boccino.png}
        \centering
\end{figure}

Cinque ore prima, Harry si stava intrufolando nel suo dormitorio con le vesti alzate sul capo a mo’ di mascheramento, giusto nel caso in cui qualcuno fosse già sveglio e lo vedesse contemporaneamente all’Harry che giaceva nel suo letto. Non voleva essere costretto a spiegare a qualcuno il suo piccolo problema di Duplicazione Spontanea.

Fortunatamente sembrava che tutti stessero dormendo.

E sembrava anche esserci una scatola, avvolta in carta rossa e verde con un fiocco giallo acceso, appoggiata vicino al suo letto. L’immagine perfetta e stereotipata del regalo di Natale, sebbene non fosse Natale.

Harry strisciò dentro il più sommessamente possibile, nel caso qualcuno avesse messo il proprio Quietus sul minimo.

C’era una busta attaccata alla scatola, chiusa da cera chiara e liscia senza sigillo impresso.

Harry aprì cautamente la busta, e prese la lettera al suo interno.

La lettera diceva:

Questo è il Mantello dell’Invisibilità di Ignotus Peverell, tramandato dai suoi discendenti, i Potter. A differenza di mantelli inferiori e incantesimi, ha il potere di tenerti nascosto, non semplicemente invisibile. Tuo padre me lo prestò per studiarlo poco prima che morisse, e confesso che ne ho fatto molto buon uso negli anni.

In futuro dovrò accontentarmi della Disillusione, temo. È tempo che il Mantello torni a te, suo erede. Avevo pensato di fartene dono a Natale, ma desiderava tornare in mano tua prima di allora. Sembra che si aspetti che tu ne abbia bisogno. Usalo bene.

Senza dubbio stai già pensando a ogni sorta di meraviglioso scherzo, come quelli che tuo padre praticò ai suoi tempi. Se ogni sua malefatta fosse nota, tutte le donne in Grifondoro si riunirebbero per profanare la sua tomba. Non cercherò di impedire che la storia si ripeta, ma stai estremamente attento a non farti scoprire. Se Silente intravvedesse la possibilità di possedere uno dei Doni della Morte, non se la farebbe scappare fino all’ultimo dei suoi giorni.

Un Buon Natale di cuore a te.

La nota non era firmata.

\begin{figure}[h!]
        \includegraphics[scale=0.4]{boccino.png}
        \centering
\end{figure}

«Aspettate», disse Harry fermandosi all’improvviso mentre gli altri ragazzi stavano per lasciare il dormitorio Corvonero. «Scusatemi, c’è qualcos’altro che devo fare col mio baule. Vi seguirò a colazione tra un paio di minuti.»

Terry Boot guardò Harry accigliato. «Faresti meglio a non pensare di mettere le mani nella nostra roba.»

Harry alzò una mano. «Giuro che non intendo fare nulla del genere ad alcuna delle tue cose, che intendo solamente accedere a oggetti che io stesso posseggo, che non ho intenzione di giocare scherzi o altre cose discutibili contro nessuno di voi, e che non prevedo che queste intenzioni cambino prima che io scenda nella Sala Grande per colazione.»

Terry aggrottò la fronte. «Aspetta, cosa –»

«Non ti preoccupare», disse Penelope Clearwater, che era lì per guidarli. «Non c’erano scappatoie. Ben formulata, Potter, dovresti fare l’avvocato.»

Harry Potter sbatté le palpebre. Ah, già, un prefetto Corvonero. «Grazie», disse. «Credo.»

«Quando cercherai di raggiungere la Sala Grande, ti perderai.» Penelope lo disse come fosse un fatto semplice e indiscutibile. «Appena ciò avviene, chiedi a un ritratto come raggiungere il primo piano. Rivolgiti a un altro ritratto nell’istante in cui sospetti di esserti perso di nuovo. Specialmente se sembra che tu salga sempre più in alto. Se arrivi più in alto di quanto dovrebbe essere l’intero castello, fermati e aspetta le squadre di soccorso. Altrimenti ti rivedremo tra quattro mesi e sarai più vecchio di cinque mesi e vestito con un perizoma e ricoperto di neve e questo se rimani all’interno del castello.»

«Ricevuto», disse Harry deglutendo con difficoltà. «Uhm, non dovreste dirlo subito a tutti gli studenti?»

Penelope sospirò. «Cosa, tutto quello? Ci vorrebbero settimane. Lo imparerete strada facendo.» Si girò per andarsene, seguita dagli altri studenti. «Se non ti vedo a colazione entro trenta minuti, Potter, darò inizio alle ricerche.»

Quando tutti se ne furono andati, Harry attaccò la nota al suo letto – aveva già scritto quella e tutte le altre note, lavorando nel livello sotterraneo del suo baule prima che tutti si svegliassero. Poi entrò con cautela nel campo del Quietus e tirò via il Mantello dell’Invisibilità dalla forma dormiente di Harry-1.

E solo per un impulso malizioso, Harry mise il Mantello nella borsa di Harry-1, sapendo che perciò era già nel suo.

\begin{figure}[h!]
        \includegraphics[scale=0.4]{boccino.png}
        \centering
\end{figure}

«Posso provvedere affinché il messaggio sia consegnato a Cornelion Flubberwalt», disse il dipinto di un uomo dall’aria aristocratica e, in effetti, con un naso perfettamente normale. «Ma posso chiedere da dove proviene originariamente?»

Harry alzò le spalle con ingegnosa impotenza. «Mi è stato detto che è stato pronunciato da una voce cavernosa che è risuonata da un vuoto nell’aria stessa, un vuoto che si era aperto su di un feroce abisso.»

\begin{figure}[h!]
        \includegraphics[scale=0.4]{boccino.png}
        \centering
\end{figure}

«Ehi!» disse Hermione in tono indignato dal suo posto dall’altro lato del tavolo della colazione. «Quello è il dolce di tutti! Non puoi prendere un’intera torta e mettertela nella borsa!»

«Non sto prendendo una torta, ne sto prendendo due. Scusatemi tutti, devo andare!» Harry ignorò le grida di sdegno e lasciò la Sala Grande. Aveva bisogno di arrivare alla classe di Erbologia un po’ in anticipo.

\begin{figure}[h!]
        \includegraphics[scale=0.4]{boccino.png}
        \centering
\end{figure}

La professoressa Sprout lo guardò acutamente. «E come fai lei a sapere cosa stanno pensando di fare i Serpeverde?»

«Non posso rivelare la mia fonte», disse Harry. «Anzi, devo chiederle di far finta che questa conversazione non sia mai avvenuta. Faccia come se fosse casualmente incappata in loro mentre era in giro per una faccenda o qualcosa del genere. Correrò avanti appena Erbologia termina. Penso di poter distrarre i Serpeverde mentre lei arriva. Non è facile spaventarmi o intimidirmi, e non penso che oseranno ferire seriamente il Ragazzo-Che-È-Sopravvissuto. Anche se… non le sto chiedendo di correre per i corridoi, ma apprezzerei se non si attardasse lungo la strada.»

La professoressa Sprout lo guardò per un lungo istante, poi la sua espressione si addolcì. «Faccia attenzione, Harry Potter. E… grazie.»

«Abbia solo cura di non arrivare tardi», disse Harry. «E ricordi, quando arriverà, lei non si aspetterà di vedermi e questa conversazione non è mai avvenuta.»

\begin{figure}[h!]
        \includegraphics[scale=0.4]{boccino.png}
        \centering
\end{figure}

Fu orribile, vedersi strattonare Neville fuori dal circolo dei Serpeverde. Neville aveva avuto ragione, aveva usato troppa forza, assolutamente troppa forza.

«Ciao», disse Harry gelidamente. «Io sono il Ragazzo-Che-È-Sopravvissuto.»

Otto ragazzi del primo anno, per lo più della stessa altezza. Uno di loro aveva una cicatrice sulla fronte e non si stava comportando come gli altri.

Oh volesse un potere darci il piccolo dono

Di veder noi stessi come gli altri ci vedono!

Ci libererebbe da molti abbagli,

E stupide credenze –

La professoressa McGonagall aveva ragione. Il Cappello Smistatore aveva ragione. Era chiaro una volta che lo vedevi dall’esterno.

C’era qualcosa di sbagliato in Harry Potter.



