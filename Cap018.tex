% !TeX root = Harry.tex

\chapter{Gerarchie di dominanza}
\label{capitolo:18}

\emph{«Sembra il genere di cose che potrei fare, non è vero?»}

~\\
~\\


Era l’ora di colazione di venerdì mattina. Harry prese un altro grosso morso del suo pane tostato e poi cercò di ricordare al suo cervello che trangugiare la colazione non l’avrebbe portato prima nei sotterranei. Ad ogni modo, avevano un’intera ora di studio tra la colazione e l’inizio di Pozioni.

Ma, sotterranei! A Hogwarts! Nell’immaginazione di Harry stavano già nascendo immagini delle voragini, degli stretti ponti, dei sostegni per le torce e delle macchie di muschio luminescente. Ci sarebbero stati ratti? Ci sarebbero stati \textit{draghi?}

«Harry Potter», disse una voce tranquilla dietro di lui.

Harry si guardò alle spalle e si trovò a contemplare Ernie Macmillan, elegantemente vestito con abiti dal bordo giallo e dall’espressione un po’ preoccupata.

«Neville ha pensato che avrei dovuto avvertirti», Ernie disse a voce bassa. «Credo che abbia ragione. Fai attenzione al Maestro di Pozioni nella nostra lezione di oggi. I Tassofrasso più grandi ci hanno detto che il professor Snape può essere davvero maligno con le persone che non gli piacciono, e non gli piace la maggior parte delle persone che non siano Serpeverde. Se gli dici qualcosa di impertinente… potrebbe finire davvero male per te, da quello che ho sentito. Tieni solo la testa bassa e non dargli alcun motivo per notarti.»

Ci fu una pausa mentre Harry elaborava quanto ascoltato, e poi sollevò le sopracciglia. (Harry avrebbe voluto alzare un solo sopracciglio, come Spock, ma non ne era mai stato capace.) «Grazie», disse Harry. «Potresti avermi appena risparmiato parecchi guai.»

Ernie annuì, e si voltò per tornare al tavolo dei Tassofrasso.

Harry riprese a mangiare il suo pane tostato.

Fu circa quattro morsi dopo che qualcuno disse «Scusami», e Harry si voltò per vedere un Corvonero più grande, dall’aspetto un po’ preoccupato –

Qualche tempo dopo, Harry stava finendo il suo terzo piatto di pancetta. (Aveva imparato a mangiare molto a colazione. Poteva sempre mangiare poco a pranzo, se non avesse finito per usare il Giratempo.) E c’era ancora un’altra voce dietro di lui che diceva «Harry?»

«Sì», disse Harry stancamente, «Cercherò di non attirare l’attenzione del professor Snape –»

«Oh, questo è impossibile», disse Fred.

«Assolutamente impossibile», disse George.

«Così abbiamo fatto cuocere una torta dagli elfi domestici», disse Fred.

«Abbiamo intenzione di metterci una candelina per ogni punto che perdi per Corvonero», disse George.

«E fare una festa per te al tavolo dei Grifondoro durante il pranzo», disse Fred.

«Ci auguriamo che ti tiri su il morale, dopo», concluse George.

Harry inghiottì il suo ultimo boccone di pancetta e si voltò. «Va bene», disse Harry. «Non avevo intenzione di chiederlo, dopo il professor Binns, davvero non volevo, ma se il professor Snape è \textit{così} terribile, perché non è stato licenziato?»

«Licenziato?» disse Fred.

«Vuoi dire, mandato via?» disse George.

«Sì», rispose Harry. «È quello che si fa con i cattivi maestri. Li si licenzia. Poi si ingaggia un insegnante migliore al loro posto. Non avete sindacati o contratti a vita qui, giusto?»

Fred e George erano accigliati più o meno nello stesso modo in cui lo sarebbero stati gli anziani di una tribù di cacciatori-raccoglitori se avessero provato a spiegare loro il calcolo infinitesimale.

«Non lo so», disse Fred dopo un po’. «Non ci avevo mai pensato.»

«Nemmeno io», disse George.

«Sì», disse Harry, «me lo dicono spesso. Ci vediamo a pranzo, ragazzi, e non prendetevela con me se non ci saranno candeline su quella torta.»

Fred e George scoppiarono a ridere entrambi, come se Harry avesse detto qualcosa di divertente, gli rivolsero un inchino e si diressero verso Grifondoro.

Harry si girò di nuovo verso il tavolo della colazione e afferrò un pasticcino. Il suo stomaco era già pieno, ma aveva la sensazione che quella mattina avrebbe utilizzato molte calorie.

Mentre mangiava il suo pasticcino, Harry pensò al peggior insegnante che avesse conosciuto fino a quel momento, il professor Binns di Storia. Il professor Binns era un fantasma. Da quello che Hermione aveva detto dei fantasmi, non sembrava probabile che fossero pienamente consapevoli di sé. Non c’erano state scoperte famose fatte da fantasmi, o anche solo un lavoro originale, non importa chi fossero stati in vita. I fantasmi tendevano ad avere difficoltà a ricordare il secolo attuale. Hermione aveva detto che erano come ritratti fortuiti, impressi nella materia circostante dalla scarica di energia psichica che aveva accompagnato la morte improvvisa di un mago.

Harry aveva incontrato alcuni insegnanti stupidi durante le sue incursioni abortite nella normale istruzione babbana — suo padre era stato molto più esigente quando si era trattato di selezionare studenti universitari come precettori, ovviamente — ma il corso di storia era stato la prima volta che aveva incontrato un insegnante che letteralmente non era senziente.

E si vedeva, anche. Harry aveva rinunciato dopo cinque minuti e iniziato a leggere un libro di testo. Quando era stato chiaro che il «professor Binns» non avrebbe obiettato, Harry aveva messo la mano nella borsa e preso i tappi per le orecchie.

I fantasmi non prendevano stipendio? Era per questo? O era letteralmente impossibile licenziare qualcuno a Hogwarts \textit{anche se era morto?}

Ora pareva che il professor Snape si stesse comportando in maniera assolutamente tremenda con tutti coloro che non fossero Serpeverde e non era neppure \textit{venuto in mente} a qualcuno di rescindere il suo contratto.

E il Preside aveva dato fuoco a una gallina.

«Scusami», disse una voce preoccupata da dietro di lui.

«Giuro», disse Harry senza girarsi, «che questo posto è più brutto di circa l’otto e mezzo percento di quanto papà dica sia Oxford.»

\begin{figure}[h!]
        \includegraphics[scale=0.4]{boccino.png}
        \centering
\end{figure}

Harry pestava i piedi lungo i corridoi di pietra, apparentemente offeso, infastidito, e infuriato allo stesso tempo.

«Sotterranei!» sibilò Harry. «\textit{Sotterranei!} Questi non sono sotterranei! Questa è una cantina! Una \textit{cantina!}»

Alcune ragazze Corvonero lo guardarono come se fosse strano. I ragazzi erano tutti abituati a lui, ormai.

Sembrava che il livello in cui si trovava l’aula di Pozioni fosse detto dei «sotterranei» solo perché era sottoterra e un po’ più freddo del corpo principale del castello.

A \textit{Hogwarts! A Hogwarts!} Harry aveva atteso tutta la vita e ora doveva attendere \textit{ancora} e se c’era qualunque posto \textit{sulla faccia della Terra} che avesse sotterranei decenti sarebbe dovuto essere Hogwarts! Harry doveva forse costruire il proprio castello se voleva vedere un po’ di abisso senza fondo?

Poco dopo arrivarono all’aula di Pozioni vera e propria e Harry si allietò considerevolmente.

L’aula di Pozioni conservava strane creature, galleggianti in enormi vasi sugli scaffali che coprivano ogni centimetro di muro tra gli armadi. Harry era andato abbastanza avanti nelle sue letture da poter realmente identificare alcune delle creature, come il Zabriskan Fontema. Sebbene il ragno di cinquanta centimetri \textit{sembrasse} un’Acromantula, era troppo piccolo per esserne uno. Aveva provato a chiedere a Hermione, ma non sembrava molto interessata a guardare neppure vicino a dove stava indicando.

Harry stava osservando una grande palla di polvere con occhi e piedi quando l’assassino sfrecciò nella stanza.

Questo fu il primo pensiero che attraversò la mente di Harry quando vide il professor Severus Snape. C’era qualcosa di silenzioso e mortale nel modo in cui l’uomo si mosse tra i banchi dei bambini. I suoi abiti erano sciatti, i capelli unti e macchiati. Qualcosa in lui ricordava Lucius, anche se i due non si assomigliavano neppure lontanamente, e si aveva l’impressione che dove Lucius ti avrebbe ucciso con eleganza impeccabile, quest’uomo ti avrebbe semplicemente ucciso.

«Sedetevi», disse il professor Severus Snape. «Ora».

Harry e pochi altri bambini che erano stati in piedi a chiacchierare tra di loro scattarono verso i banchi. Harry aveva pensato di finire accanto a Hermione, ma in qualche modo si trovò seduto nel più vicino banco vuoto accanto a Justin Finch-Fletchley (era una Sessione doppia, Corvonero e Tassofrasso), cosa che lo poneva due banchi a sinistra di Hermione.

Severus si sedette dietro la cattedra, e senza la minima introduzione, disse «Hannah Abbott.»

«Eccomi», disse Hannah in una voce piuttosto tremante.

«Susan Bones.»

«Presente.»

E andò avanti così, nessuno che osava dire una parola fuori posto, finché:

«Ah, sì. Harry Potter. La nostra nuova… \textit{celebrità.}»

«La celebrità è presente, \textit{signore.}»

Metà della classe trasalì, e alcuni dei più intelligenti improvvisamente sembrarono voler correre fuori dalla porta a lezione ancora in corso.

Severus sorrise come in anticipazione e chiamò il successivo nome sulla sua lista.

Harry fece un sospiro mentale. Tutto era successo troppo velocemente perché potesse fare qualcosa a riguardo. Oh, bene. Chiaramente quest’uomo non lo aveva già in simpatia, quale che fosse il motivo. E pensandoci, molto meglio che questo professore di Pozioni se la prendesse con lui piuttosto che, diciamo, con Neville e Hermione. Harry era molto più bravo a difendersi. Sì, probabilmente tutto sarebbe andato per il meglio.

Quando l’appello fu terminato, Severus fece passare il suo sguardo sulla classe al completo. I suoi occhi erano vuoti come un cielo notturno senza stelle.

«Siete qui», disse Severus con una voce sommessa che gli studenti sul retro si sforzarono di udire, «per imparare la sottile scienza e l’esatta arte del preparare pozioni. Poiché qui c’è poco dello sciocco sventolar di bacchette, molti di voi difficilmente crederanno che questa sia magia. Non mi aspetto che comprendiate realmente la bellezza del calderone che ribolle dolcemente con i suoi vapori scintillanti, il delicato potere dei liquidi che si insinuano nelle vene umane», questo detto in un tono piuttosto carezzevole e compiaciuto, «stregando la mente e irretendo i sensi», tutto stava decisamente diventando sempre più inquietante e angosciante. «Io posso insegnarvi come imbottigliare la fama, fermentare la gloria, persino spillare la morte — se non foste un grande branco di stupidi come quelli cui normalmente insegno.»

In qualche modo Severus sembrò notare lo scetticismo dipinto sul volto di di Harry, o quanto meno i suoi occhi saltarono improvvisamente verso il posto dove si trovava seduto Harry.

«Potter!» disse bruscamente il Professore di Pozioni. «Che cosa ottengo se aggiungo radice di asfodelo in polvere in un infuso di assenzio?»

Harry sbatté le palpebre. «\textit{Era in Filtri magici e pozioni?}» disse. «Ho appena finito di leggerlo, e non mi ricordo niente che usasse l’assenzio –»

La mano di Hermione si alzò e Harry le lanciò un’occhiataccia che gliela fece sollevare ancora più alto.

«Tsk, tsk», fece Severus. «La fama, chiaramente, non è tutto.»

«Davvero?» disse Harry. «Ma ci ha appena detto che ci insegnerà a imbottigliare la fama. Dica, come \textit{funziona} esattamente? La si beve e ci si trasforma in una celebrità?»

Tre quarti della classe sussultarono.

La mano di Hermione stava scendendo lentamente verso il basso. Beh, non era una sorpresa. Poteva essere la sua rivale, ma non era il tipo di ragazza che sarebbe stata al gioco dopo che fosse diventato chiaro che il professore stava deliberatamente cercando di umiliarlo.

Harry stava tentando con forza di mantenere il controllo del suo temperamento. La prima risposta che aveva attraversato la sua mente era stata ‘Abracadabra’.

«Proviamo di nuovo», disse Severus. «Potter, dove cercheresti se ti avessi detto di trovarmi un bezoario?»

«Neanche questo è nel libro di testo», disse Harry, «ma in un libro babbano ho letto che un trichinobezoario è una massa di peli solidificati trovata in uno stomaco umano, e i Babbani credevano che fosse in grado di curare qualsiasi veleno –»

«Sbagliato», disse Severus. «Un bezoario si trova nello stomaco di una capra, non è fatto di capelli, ed è in grado di curare la maggior parte dei veleni, ma non tutti.»

«Non ho \textit{detto} che sarebbe stato in grado di farlo, ho detto che era quello che ho letto in un libro babbano –»

«Qui nessuno è interessato ai suoi \textit{patetici} libri babbani. Ultimo tentativo. Qual è la differenza, Potter, tra lo Strozzalupo e il Risigallo?»

Quella fu la goccia che fece traboccare il vaso.

«Sa», disse Harry gelidamente, «in uno dei miei libri babbani piuttosto \textit{affascinanti}, si descrive uno studio in cui delle persone riuscivano a farsi passare per molto intelligenti ponendo domande su fatti qualunque di cui solo loro erano a conoscenza. A quanto pare gli osservatori notavano solo che chi faceva le domande sapeva le risposte e chi le riceveva no, e non riuscivano ad adattare il loro giudizio all’iniquità dello schema sottostante. Quindi, professore, è in grado di dirmi quanti elettroni ci sono nell’orbitale più esterno di un atomo di carbonio?»

Il sorriso di Severo si allargò. «Quattro», disse. «È un fatto inutile che nessuno dovrebbe preoccuparsi di mettere per iscritto, comunque. E per tua informazione, Potter, asfodelo e assenzio fanno una pozione soporifera così potente che è nota come la Bevanda della Morte Vivente. Per quanto riguarda lo Strozzalupo e il Risigallo, sono la stessa pianta, anche nota col nome di Aconito, come avresti saputo se avessi letto \textit{Mille erbe e funghi magici}. Pensavi di non aver bisogno di aprire il libro prima di venire qui, eh, Potter? Tutti gli altri dovrebbero copiare quanto detto, in modo da non essere così ignoranti come lui.» Severus fece una pausa, sembrando alquanto soddisfatto di sé stesso. «E questo varrà… cinque punti? No, facciamo dieci punti tondi da Corvonero per insolenza.»

Hermione rantolò, insieme a diversi altri.

«Professor Severus Snape», Harry profferì. «Non so cosa io possa aver fatto per meritare la sua ostilità. Se ha qualche problema con me di cui non sono al corrente, suggerisco che noi –»

«Stia zitto, Potter. Altri dieci punti da Corvonero. Voi altri, aprite i vostri libri a pagina 3.»

C’era solamente una sensazione leggera, solamente una sensazione molto debole di bruciore nella parte posteriore della gola di Harry, e nessuna umidità nei suoi occhi. Se piangere non era un strategia efficace per distruggere questo Professore di Pozioni, allora era inutile piangere.

Lentamente, Harry raddrizzò la schiena. Tutto il suo sangue sembrava essere defluito e sostituito da azoto liquido. Sapeva che aveva provato a controllare il suo temperamento, ma non sembrava in grado di ricordare perché.

«Harry», sussurrò freneticamente Hermione due banchi più in là, «fermati, per favore, è tutto a posto, non li conteremo –»

«Parla durante la lezione, Granger? Tre –»

«Dunque», disse una voce più fredda di zero gradi Kelvin, «cosa si deve fare per presentare un reclamo formale contro un professore molesto? Bisogna parlare con la Vicepreside, scrivere una lettera al Consiglio Direttivo… le dispiacerebbe spiegare come funziona?»

La classe era completamente paralizzata.

«Detenzione per un mese, Potter», disse Severus, sorridendo ancor più largamente.

«Mi rifiuto di riconoscere la sua autorità di insegnante e non sconterò alcuna detenzione data da lei.»

Le persone smisero di respirare.

Il sorriso di Severus svanì. «Allora sarai –» la sua voce si interruppe bruscamente.

«Espulso, stava per dire?» Harry, d’altra parte, stava ora accennando un sorriso. «Ma poi è sembrato dubitare della sua capacità di attuare la sua minaccia, o temerne le conseguenze. Io, al contrario, non ho dubbi né timori alla prospettiva di trovare un’altra scuola con professori meno molesti. O forse dovrei pagare dei precettori privati, come sono solito fare, e ricevere un’educazione che sia alla mia piena velocità di apprendimento. Ho abbastanza denaro nel mio deposito. Qualcosa a proposito di taglie su di un Signore Oscuro che ho sconfitto. Ma ci \textit{sono} insegnanti a Hogwarts che mi piacciono abbastanza, quindi credo che sarebbe più semplice se trovassi il modo di sbarazzarmi di lei, invece.»

«Sbarazzarsi di me?» disse Severus, anch’egli con un accenno di sorriso, ora. «Che idea divertente. Come pensa di farlo, Potter?»

«Mi par di capire che ci sia stato un certo numero di lamentele contro di lei da parte degli studenti e dei loro genitori», una congettura, ma piuttosto sicura, «cosa che lascia solo la questione del perché non se ne sia già andato. Forse le finanze di Hogwarts non possono permettersi un vero Professore di Pozioni? Potrei dare il mio contributo. Sono certo che troverebbero un professore di ben altro livello se offrissero il doppio del suo salario corrente.»

Due poli di ghiaccio irraggiavano un inverno congelante attraverso l’aula.

«Scoprirà», Severus disse mollemente, «che il Consiglio Direttivo non è minimamente interessato alla sua offerta.»

«Lucius…» disse Harry. «\textit{Ecco} perché lei è ancora qui. Forse dovrei fare una chiacchierata con Lucius a riguardo. Credo che desideri incontrarmi. Mi chiedo se ho qualcosa che lo interessa.»

Hermione scosse la testa freneticamente. Harry lo notò con la coda dell’occhio, ma la sua attenzione era tutta su Severus.

«Lei è un ragazzo molto sciocco», disse Severus. Non sorrideva affatto, adesso. «Non ha niente che Lucius desideri più della mia amicizia. E se l’avesse, ho altri alleati.» La sua voce si fece dura. «E trovo sempre più improbabile che lei non sia stato Smistato in Serpeverde. Com’è riuscito a rimanere fuori dalla mia Casa? Ah, sì, perché il Cappello Smistatore ha detto che stava \textit{scherzando}. Per la prima volta nella storia. Di cosa stava parlando realmente col Cappello Smistatore, Potter? Possiede qualcosa che voleva?»

Harry fissò lo sguardo freddo di Severus e ricordò che il Cappello Smistatore lo aveva avvertito di non incrociare lo sguardo di chiunque mentre stava pensando a — Harry abbassò lo sguardo sulla scrivania di Severus.

«Sembra stranamente riluttante a guardarmi negli occhi, Potter!»

Una scossa di comprensione improvvisa — «Quindi è stato contro di \textit{lei} che il Cappello Smistatore mi ha messo in guardia!»

«Cosa?» disse la voce di Severus, che sembrò sinceramente sorpreso, anche se naturalmente Harry non lo guardò in faccia.

Harry si alzò dal suo banco.

«Si segga, Potter», disse una voce arrabbiata da qualche luogo in cui non stava guardando.

Harry lo ignorò, e guardò l’aula. «Non ho alcuna intenzione di lasciare che un insegnante non professionale rovini la mia permanenza a Hogwarts», disse Harry mortalmente calmo. «Penso che mi congederò da questo corso, e o assumerò un tutore per insegnarmi Pozioni mentre sono qui, o se il Consiglio è davvero sotto controllo, studierò durante l’estate. Se qualcuno di voi decidesse che non gli interessa essere vittima del bullismo di quest’uomo, le mie sessioni saranno aperte anche a lui.»

«\textit{Si segga, Potter!}»

Harry attraversò la stanza e afferrò la maniglia della porta.

Non girò.

Harry si voltò lentamente, e colse l’immagine di Severus che sorrideva malignamente prima di ricordarsi di guardare altrove.

«Apra questa porta.»

«No», disse Severus.

«Mi sta facendo sentire minacciato», disse una voce così gelida che non sembrava affatto quella di Harry, «e questo è un errore.»

La voce di Severus rise. «Cosa intende fare, ragazzino?»

Harry misurò sei lunghi passi dalla porta, finché non fu in piedi vicino all’ultima fila di banchi.

Poi si mise perfettamente eretto e alzò la mano destra in un unico e terribile movimento, le dita pronte a schioccare.

Neville urlò e si tuffò sotto il banco. Altri bambini si ritrassero o alzarono istintivamente le braccia per proteggersi.

«\textit{Harry no!}» strillò Hermione. «Qualunque cosa avessi intenzione di fargli, non farlo!»

«Siete diventati tutti \textit{pazzi?}» abbaiò la voce di Severus.

Lentamente, Harry abbassò la mano. «Non avevo intenzione di fargli del male, Hermione», disse Harry, la sua voce un po’ più bassa. «Stavo per far saltare in aria la porta.»

Anche se ora che Harry ci stava pensando, era proibito Trasfigurare oggetti da bruciare, il che significava che andare indietro nel tempo e convincere Fred o George a trasfigurare una quantità precisamente misurata di esplosivi sarebbe potuta non essere realmente un buona idea…

«\textit{Silencio}», disse la voce di Severus.

Harry cercò di dire «Che cosa?» e scoprì che non gli usciva alcun suono.

«Questa è diventata una farsa. Credo che le sia stato permesso di mettersi sufficientemente nei guai per un solo giorno, Potter. Lei è lo studente più molesto e indisciplinato che abbia mai incontrato, e non ricordo quanti punti Corvonero abbia in questo momento, ma sono sicuro di poterli a spazzare via tutti. Dieci punti da Corvonero. Dieci punti da Corvonero. Dieci punti da Corvonero! Cinquanta punti da Corvonero! Ora si segga e osservi il resto della classe fare lezione!»

Harry mise la mano in tasca e cercò di dire ‘pennarello’, ma ovviamente nessuna parola uscì fuori. Per un breve momento, questo lo fermò; poi gli venne in mente di formare la sequenza p-e-n-n-a-r-e-l-l-o con i movimenti delle dita, e questo funzionò. b-l-o-c-c-o ed ebbe il blocco di carta. Harry si diresse a un banco vuoto, non quello dove era originariamente seduto, e scarabocchiò un breve messaggio. Strappò quel foglio di carta, mise via il pennarello e il blocco in una tasca della sua veste per un accesso più rapido, e alzò il suo messaggio, non verso Severus, ma verso il resto della classe.

\begin{addmargin}[3em]{3em}% 1em left, 2em right
~

\textsc{Me ne vado}

\textsc{qualcun altro ha}

\textsc{bisogno di uscire?}\\
\end{addmargin}

«Lei è folle, Potter», disse Severus con freddo disprezzo.

A parte quello, nessuno parlò.

Harry indirizzò un ironico inchino verso la cattedra, si avvicinò alla parete, e con un unico movimento fluido spalancò la porta di uno stanzino, ci entrò e sbatté la porta dietro di sé.

Ci fu il suono ovattato di qualcuno che schioccava le dita, e poi più nulla.

In classe, gli studenti si guardarono l’un l’altro perplessi e impauriti.

Il volto del Maestro di Pozioni era ora completamente infuriato. Attraversò la stanza a grandi passi terribili e spalancò la porta dello stanzino.

Lo stanzino era vuoto.

\begin{figure}[h!]
        \includegraphics[scale=0.4]{boccino.png}
        \centering
\end{figure}

Un’ora prima, Harry si mise in ascolto all’interno dello stanzino chiuso. Non c’era alcun suono all’esterno, e neppure alcun motivo di rischiare.

\textsc{M-a-n-t-e-l-l-o}, formarono le sue dita.

Una volta resosi invisibile, con molta attenzione e circospezione aprì la porta dello stanzino e sbirciò fuori. In classe non sembrava esserci nessuno.

La porta non era chiusa a chiave.

Fu quando Harry si trovò fuori da quel luogo pericoloso e dentro il corridoio, sicuro e invisibile, che un po’ di quella rabbia sparì e si rese conto di quello che aveva appena fatto.

Quello che aveva appena fatto.

Il volto invisibile di Harry si paralizzò per l’orrore assoluto.

Si era inimicato un insegnante tre ordini di grandezza al di là di qualsiasi cosa avesse mai gestito prima. Aveva minacciato di andarsene da Hogwarts e avrebbe potuto dover dar seguito alla minaccia. Aveva perso tutti i punti che Corvonero aveva e poi aveva usato il Giratempo…

La sua immaginazione gli mostrò i suoi genitori che gli urlavano contro dopo che era stato espulso, e la professoressa McGonagall delusa, ed era troppo doloroso e non poteva sopportarlo e \textit{non riusciva a pensare a un modo per salvarsi} –

Il pensiero che Harry si concesse era che se arrabbiarsi lo aveva infilato in tutti quei guai, allora forse quando fosse stato nuovamente infuriato avrebbe trovato una via d’uscita, le cose sembravano in qualche modo più chiare quando era arrabbiato.

E il pensiero che Harry non si concesse era che non riusciva proprio ad affrontare questa prospettiva, se non era arrabbiato.

Così ricacciò tutti i suoi pensieri e ricordò la bruciante umiliazione –

\textit{Tsk, tsk. La fama, chiaramente, non è tutto.}

\textit{Dieci punti da Corvonero per insolenza.}

Il freddo calmante ripulì le sue vene come la risacca di un’onda riflessa da un ostacolo, e Harry si permise di espirare.

Va bene. Torna ad essere sano di mente, ora.

Si sentiva veramente un po’ deluso dal sé stesso non arrabbiato per essere crollato così e aver voluto solo uscire dai guai. Il professor Severus Snape era un problema di tutti. L’Harry-Normale l’aveva dimenticato e aveva desiderato un modo per proteggere sé stesso. E chi se ne frega di tutte le altre vittime? La questione non era come proteggere sé stesso, la questione era come distruggere questo Professore di Pozioni.

\textit{Quindi questo è il mio lato oscuro, vero? C’è un po’ di pregiudizio in questo termine, il mio lato illuminato sembra più egoista e codardo, per non dire confuso e in preda al panico.}

E ora che stava pensando chiaramente, fu parimenti chiaro cosa fare dopo. Aveva già dato a sé stesso un’ora in più per prepararsi, e avrebbe potuto prendersi fino a cinque ore ulteriori, se necessario…

\begin{figure}[h!]
        \includegraphics[scale=0.4]{boccino.png}
        \centering
\end{figure}

Minerva McGonagall attendeva nell’ufficio del Preside.

Silente sedeva nel suo trono imbottito dietro la scrivania, vestito con quattro strati di vesti formali color lavanda. Minerva sedeva su una sedia di fronte a lui, dall’altro lato Severus in un’altra sedia. Di fronte a loro tre c’era uno sgabello di legno vuoto.

Stavano aspettando Harry Potter.

\textit{Harry}, Minerva pensò disperata, \textit{avevi promesso che non avresti morso alcun insegnante!}

E nella sua mente riuscì a visualizzare molto chiaramente la replica, il volto arrabbiato di Harry e la sua risposta indignata: \textit{ho detto che non avrei morso chiunque non mi avesse morso per primo!}

Qualcuno bussò alla porta.

«Avanti!» chiamò Silente.

La porta si aprì, e Harry Potter entrò. Minerva si lasciò quasi scappare un forte gemito. Il ragazzo sembrava glaciale, calmo, e assolutamente nel pieno controllo di sé stesso.

«Buon gior-» la voce di Harry si interruppe improvvisamente. La sua bocca rimase aperta.

Minerva seguì lo sguardo di Harry, e vide che stava fissando Fawkes, appollaiato sul trespolo dorato. Fawkes sventolò le sue brillanti ali rosso-dorate come il tremolio di una fiamma, e abbassò la testa in un cenno misurato al ragazzo.

Harry si voltò a guardare Silente.

Silente gli fece l’occhiolino.

Minerva percepì che si era persa qualcosa.

Un’incertezza improvvisa attraversò il volto di Harry. La sua freddezza vacillò. Nei suoi occhi si mostrò la paura, poi la rabbia, e poi il ragazzo fu nuovamente calmo.

Un brivido scese lungo la schiena di Minerva. C’era qualcosa che non andava.

«Accomodati» disse Silente. Il suo volto era di nuovo serio.

Harry si sedette.

«Allora, Harry» disse Silente. «Ho sentito una relazione di questa giornata da parte del professor Snape. Ti andrebbe di dirmi cosa è successo con parole tue?»

Harry indirizzò un’occhiata sprezzante a Severus. «Non è complicato», disse il ragazzo, sorridendo appena. «Ha provato a comportarsi da bullo con me nello stesso modo in cui molesta tutti quelli qui a scuola che non appartengono a Serpeverde dal giorno che Lucius ve l’ha imposto. Per quanto riguarda gli altri dettagli, chiedo una conversazione privata con lei su questo argomento. Da uno studente che sta denunciando il comportamento abusivo di un professore difficilmente ci si può aspettare che parli con franchezza di fronte a quello stesso professore, dopo tutto.»

Questa volta Minerva non poté impedirsi di gemere forte.

Severus, semplicemente, rise.

E il volto del Preside divenne severo. «Signor Potter», disse, «non si parla di un professore di Hogwarts in questi termini. Ho paura che lei stia agendo sulla base di un terribile equivoco. Il professor Severus Snape ha la mia piena fiducia, ed è al servizio di Hogwarts dietro mia richiesta, non di Lucius Malfoy.»

Ci fu silenzio per alcuni momenti.

Quando il ragazzo parlò nuovamente, la sua voce era di ghiaccio. «Mi sono perso qualcosa?»

«Un certo numero di cose, signor Potter», disse il Preside. «Dovrebbe capire, per iniziare, che lo scopo di questo incontro è discutere come punirla per gli eventi di questa mattina.»

«Quest’uomo ha terrorizzato la sua scuola per anni. Ho parlato agli studenti e raccolto testimonianze per essere certo che ce ne fossero abbastanza per una campagna giornalistica, allo scopo di unire i genitori contro di lui. Alcuni degli studenti più giovani piangevano mentre me ne parlavano. Io ho quasi pianto ascoltandoli! \textit{Lei ha permesso a questo prevaricatore di agire liberamente? Ha fatto questo ai suoi studenti? Perché?}»

Minerva deglutì un groppo in gola. Aveva — pensato la stessa cosa, talvolta, ma per qualche ragione non era mai del tutto –

«Signor Potter», disse il Preside, la sua voce ora severa, «questo incontro non riguarda il professor Snape. Riguarda lei e il suo disprezzo per la disciplina scolastica. Il professor Snape ha suggerito, e io ho concordato, che tre mesi interi di detenzione saranno appropriati –»

«Respinto», disse Harry gelidamente.

Minerva era senza parole.

«Questa non è una richiesta, signor Potter», disse il Preside. Tutta la forza dello sguardo del mago fu concentrata sul ragazzo. «Questa è la sua punizio-»

«Lei mi spiegherà perché ha permesso a quest’uomo di fare del male ai bambini affidati alle sue cure, e se la sua spiegazione non sarà sufficiente, allora inizierò la mia campagna giornalistica con \textit{lei} come bersaglio.»

Il corpo di Minerva ondeggiò per la forza di quel colpo, per la pura lesa maestà.

Anche Severus sembrò scioccato.

«Questo, Harry, sarebbe estremamente imprudente», disse Silente lentamente. «Io sono il pezzo principale che sulla scacchiera si oppone a Lucius. Se facessi una cosa del genere lo rafforzeresti notevolmente, e non pensavo che fosse questo lo schieramento che avevi scelto.»

Il ragazzo rimase fermo per un lungo momento.

«Questa conversazione diventa privata», disse Harry. La sua mano scattò in direzione di Severus. «Lo mandi via.»

Silente scosse la testa. «Harry, non ti ho detto che Severus Snape ha la mia piena fiducia?»

Il viso del ragazzo mostrò lo sconcerto. «Il bullismo di quest’uomo la rende vulnerabile! Non sono l’unico che potrebbe dare inizio a una campagna giornalistica contro di lei! Questo è folle! Perché sta agendo così?»

Silente sospirò. «Mi dispiace, Harry. Ha a che fare con cose che non sei, in questo momento, pronto a sentire.»

Il ragazzo fissò Silente. Poi si voltò a guardare Severus. Poi di nuovo Silente.

«Questa è follia», disse il ragazzo lentamente. «Lei non l’ha tenuto a freno perché pensa che lui sia \textit{parte dello schema}. Che Hogwarts abbia bisogno di un Maestro di Pozioni cattivo per essere una vera e propria scuola di magia, così come ha bisogno di un fantasma per insegnare Storia.»

«Sembra il genere di cose che potrei fare, non è vero?» disse Silente, sorridendo.

«Inaccettabile», rispose Harry in tono piatto. Il suo sguardo era ormai freddo e buio. «Non tollererò bullismo o abusi. Avevo considerato molti modi possibili di affrontare questo problema, ma la farò semplice. O se ne va quest’uomo, o lo faccio io.»

Minerva rimase nuovamente a bocca aperta. Qualcosa di strano balenò negli occhi di Severus.

Ora anche lo sguardo di Silente stava diventando freddo. «L’espulsione, signor Potter, è la minaccia finale che può essere usata contro uno studente. Non viene abitualmente utilizzata dagli studenti come minaccia nei confronti del Preside. Questa è la migliore scuola di magia in tutto il mondo, e un’educazione qui non è un’opportunità data a tutti. Ha l’impressione che Hogwarts non possa andare avanti senza di lei?»

E Harry rimase seduto, un accenno di sorriso.

Un orrore improvviso balenò nella mente di Minerva. Sicuramente Harry non avrebbe –

«Lei dimentica», disse Harry, «che non è l’unico che può vedere gli schemi ricorrenti. \textit{Questa sta diventando una faccenda privata. Ora lo mandi –}» Harry fece nuovamente un gesto verso Severus, e poi si fermò a metà frase e a metà gesto.

Minerva poté vederlo sul viso di Harry, il momento in cui si ricordò.

Dopo tutto era stata lei a dirglielo.

«Signor Potter», disse il Preside, «ancora una volta, Severus Snape ha la mia piena fiducia.»

«Gliel’ha detto», sussurrò il ragazzo. «Lei è completamente folle.»

Silente non reagì all’insulto. «Detto cosa?»

«\textit{Che il Signore Oscuro è vivo.}»

«In nome di Merlino, di che cosa sta parlando, Potter?» esclamò Severus coi toni più puri dello stupore e dell’indignazione.

Harry lo guardò brevemente, con un sorriso cupo. «Oh, quindi \textit{è} un Serpeverde, allora», disse Harry. «Stavo cominciando a dubitarne.»

E poi ci fu silenzio.

Infine Silente parlò. La sua voce era dolce. «Harry, di \textit{che cosa} stai parlando?»

«Mi dispiace, Albus», sussurrò Minerva.

Severus e Silente si voltarono a guardarla.

«Non me l’ha detto la professoressa McGonagall», disse Harry con un tono sbrigativo e meno calmo di prima. «L’ho indovinato io. Gliel’ho detto, anche io posso vedere gli schemi ricorrenti. L’ho indovinato, e lei ha controllato la sua reazione proprio come ha fatto Severus. Ma la sua reazione è stata di un’ombra inferiore alla perfezione, e ho potuto capire che era controllata e non genuina.»

«E gli ho detto», disse Minerva, la voce un po’ tremante, «che tu e io, e Severus eravamo gli unici a saperlo.»

«Cosa che ha fatto come una concessione a me, per impedirmi di andarmene in giro a fare domande, come avevo minacciato di fare se non avesse parlato», disse Harry. Il ragazzo ridacchiò brevemente. «Avrei dovuto davvero prendere uno di voi e dirgli che mi aveva detto tutto, per vedere se gli sarebbe sfuggito qualcosa. Probabilmente non avrebbe funzionato, ma sarebbe valsa la pena provarci.» Il ragazzo sorrise di nuovo. «La minaccia è ancora valida e mi aspetto di essere informato \textit{pienamente} prima o poi.»

Severus le stava indirizzando uno sguardo di disprezzo. Minerva alzò il mento e lo sopportò. Sapeva che se l’era meritato.

Silente si appoggiò al trono imbottito. I suoi occhi erano freddi come mai Minerva li aveva visti dal giorno in cui suo fratello era morto. «E tu minacci di abbandonarci a Voldemort, se non ci adeguassimo ai tuoi desideri.»

Il tono di Harry fu tagliente. «Sono spiacente di informarvi che non siete il centro dell’universo. Non sto minacciando di abbandonare la Gran Bretagna magica. Sto minacciando di abbandonare \textit{voi}. Io non sono un piccolo e mite Frodo. Questa è la \textit{mia} impresa e se volete farne parte dovrete giocare secondo le \textit{mie} regole.»

Il volto di Silente era ancora freddo. «Sto cominciando a dubitare della sua adeguatezza come eroe, signor Potter.»

Lo sguardo che Harry restituì era altrettanto gelido. «Sto cominciando a dubitare della sua adeguatezza come mio Gandalf, \textit{signor Silente}. Almeno Boromir fu un errore comprensibile. Cosa ci fa questo \textit{Nazgûl} nella mia Compagnia?»

Minerva era completamente persa. Guardò Severus, per vedere se stava seguendo, e notò che aveva girato il suo volto lontano dal campo visivo di Harry e stava sorridendo.

«Suppongo», disse Silente lentamente, «che dal suo punto di vista si tratti di una domanda ragionevole. Allora, signor Potter, se il professor Snape la lasciasse in pace d’ora in poi, sarà questa l’ultima volta che la questione si pone, o la troverò qui ogni settimana con una nuova richiesta?»

«Lasciare in pace \textit{me?}» Il tono di Harry era indignato. «Non sono la sua unica vittima e certo non la più vulnerabile! \textit{Avete dimenticato quanto siano indifesi i bambini? Quanto vengono feriti?} D’ora in poi Severus tratterà \textit{ogni} studente di Hogwarts con adeguata e professionale cortesia, oppure si troverà un altro Maestro di Pozioni, oppure un altro eroe!»

Silente cominciò a ridere. Una risata a piena gola, calda, divertita, come se Harry avesse appena fatto un numero comico di fronte a lui.

Minerva non osava muoversi. I suoi occhi tremarono e vide che Severus era ugualmente immobile.

Il volto di Harry divenne ancora più freddo. «Lei mi fraintende, Preside, se pensa che questo sia uno scherzo. Questa non è una richiesta. Questa è la sua punizione.»

«Signor Potter –» disse Minerva. Non sapeva nemmeno quello che stava per dire. Semplicemente non poteva ignorare quella risposta.

Harry fece un gesto per zittirla e continuò a parlare a Silente. «E se questo le sembra scortese», disse, la sua voce ora un po’ meno dura, «non è sembrato meno scortese quando lei l’ha detto a me. Non direbbe una cosa del genere a chiunque considerasse un vero essere umano invece di un bambino a lei subordinato, e la tratterò con la stessa cortesia con cui lei tratta me –»

«Oh, è vero, è proprio vero, questa è la mia punizione se mai ce ne sia stata una! È \textit{ovvio} che sei qui a ricattarmi per salvare i tuoi compagni, non per salvare te stesso! Non riesco a immaginare perché ho pensato il contrario!» Ora Silente rideva ancora più fragorosamente. Batté il pugno sul tavolo tre volte.

Lo sguardo di Harry divenne incerto. Il suo viso si voltò verso Minerva, rivolgendosi a lei per la prima volta. «Mi scusi», disse Harry. La sua voce sembrò vacillare. «Ha bisogno di prendere le sue medicine o cosa?»

«Ah…» Minerva non aveva idea di cosa potesse rispondere.

«Bene», disse Silente. Si asciugò le lacrime che si erano formate nei suoi occhi. «Mi scusi. Mi dispiace per l’interruzione. La prego, continui con il ricatto.»

Harry aprì la bocca, poi la richiuse. Ora sembrava un po’ incerto. «Ah… deve anche smettere di leggere nelle menti degli studenti.»

«Minerva», disse Severus, la sua voce letale, «tu –»

«È stato il Cappello Smistatore ad avvertirmi», disse Harry.

«\textit{Cosa?}»

«Non posso dire nient’altro. Ad ogni modo, penso sia tutto qui. Ho finito.»

Silenzio.

«E ora?» disse Minerva, quando divenne chiaro che nessun altro stava per dire qualcosa.

«E ora?» Silente le fece eco. «Beh, e ora l’eroe vince, ovviamente.»

«\textit{Cosa?}» dissero Severus, Minerva, e Harry.

«Beh, certamente sembra che ci abbia stretti in un angolo», disse Silente, sorridendo felice. «Ma Hogwarts \textit{ha realmente} bisogno di un Maestro di Pozioni cattivo, o non sarebbe una vera e propria scuola di magia, non credete? Quindi che ne pensi se il professor Snape sarà tremendo solo nei confronti degli studenti dal loro quinto anno e in poi?»

«\textit{Cosa?}» dissero nuovamente tutti e tre.

«Se sono le vittime più vulnerabili che ti preoccupano. Forse hai ragione, Harry. Forse io \textit{ho} dimenticato nel corso dei decenni che cosa vuol dire essere un bambino. Quindi cerchiamo un compromesso. Severus continuerà ad aggiudicare ingiustamente punti a Serpeverde e a imporre una disciplina lassista sulla sua Casa, e sarà terribile con gli studenti non appartenenti a Serpeverde dal loro quinto anno in poi. Con tutti gli altri sarà spaventoso, ma non molesto. Prometterà di leggere le menti solo quando la sicurezza di uno studente lo richiedesse. Hogwarts avrà il suo Maestro di Pozioni cattivo, e le vittime più vulnerabili, come dici tu, saranno al sicuro.»

Minerva McGonagall era scioccata più di quanto fosse mai stata in vita sua. Lanciò un’occhiata incerta a Severus, il cui volto era rimasto completamente neutro, come se non riuscisse a decidere che tipo di espressione avrebbe dovuto assumere.

«Presumo che sia accettabile», disse Harry. La sua voce suonò un po’ strana.

«Non puoi parlare sul serio», disse Severus, la sua voce inespressiva come il suo volto.

«Sono molto favorevole», disse lentamente Minerva. Era tanto favorevole che il suo cuore batteva all’impazzata sotto le sue vesti. «Ma cosa possiamo dire agli studenti? Non si saranno lamentati quando Severus è stato… terribile con tutti, ma –»

«Harry può dire agli altri studenti che ha scoperto un terribile segreto di Severus e ha esercitato un piccolo ricatto» disse Silente. «È vero, dopo tutto; ha scoperto che Severus stava leggendo le menti, e di certo ci ha ricattati.»

«Questa è follia!» esplose Severus.

«Bwah ah ah!» disse Silente.

«Ah…» disse Harry incerto. «E se qualcuno mi chiedesse perché quelli dal quinto anno in su sono rimasti fregati? Non li biasimerei se si arrabbiassero, e quella parte dell’accordo non è esattamente idea mia –»

«Racconta loro», rispose Silente, «che non sei stato tu a suggerire il compromesso, che era tutto ciò che hai potuto ottenere. E poi rifiutati di dire altro. Anche questo è vero. È una specie di arte, ci farai la mano con la pratica.»

Harry annuì lentamente. «E i punti che ha sottratto a Corvonero?»

«Non devono essere restituiti.»

Era stata Minerva a parlare.

Harry la guardò.

«Mi dispiace, signor Potter». \textit{Era} dispiaciuta, ma doveva essere fatto. «Ci \textit{devono} essere delle conseguenze per la sua cattiva condotta o questa scuola cadrà a pezzi.»

Harry scrollò le spalle. «Accettabile», disse senza emozione. «Ma in futuro Severus non colpirà i miei rapporti con la Casa sottraendomi punti, né sprecherà il mio tempo prezioso con detenzioni. Se ritenesse che il mio comportamento richieda una punizione, potrà comunicare le sue preoccupazioni alla professoressa McGonagall.»

«Harry», disse Minerva, «continuerai a restare soggetto alla disciplina scolastica, o sarai al di sopra della legge, come è stato Severus?»

Harry la guardò. Un certo calore toccò il suo sguardo, poco prima di essere represso. «Continuerò a essere uno studente ordinario per ogni membro del personale che non sia pazzo o cattivo, a condizione che non sia messo sotto pressione da altri che lo sono.» Harry guardò brevemente Severus, poi si voltò verso Silente. «Lasciate stare Minerva, e sarò un normale studente di Hogwarts in sua presenza. Nessun particolare privilegio o immunità.»

«Bellissimo», disse Silente sinceramente. «Parli come un vero eroe.»

«E inoltre», ella disse, «il signor Potter deve scusarsi pubblicamente per le sue azioni di oggi.»

Harry le rivolse un’altra occhiata. Questa era un po’ scettica.

«La disciplina della scuola è stata gravemente danneggiata dalle sue azioni, signor Potter», disse Minerva. «Deve essere ripristinata.»

«Penso, professoressa McGonagall, che lei sopravvaluti considerevolmente l’importanza di ciò che chiama disciplina scolastica, rispetto ad avere il corso di Storia insegnato da un insegnante vivente o a non torturare i vostri studenti. Il mantenimento dello \textit{status quo} gerarchico e l’applicazione delle sue regole sembrano sempre molto più saggi, morali e importanti quando si è in alto e si fa rispettare tale applicazione di quando si è in basso, e posso citare degli studi in tal senso, se necessario. Potrei andare avanti per ore su questo punto, ma finirò qui.»

Minerva scosse la testa. «Signor Potter, lei sottovaluta l’importanza della disciplina perché lei non ne ha bisogno –» Fece una pausa. Non era venuto fuori in maniera giusta, e Severus, Silente e persino Harry le stavano rivolgendo delle strane occhiate. «Per imparare, intendo. Non tutti i bambini sono in grado di imparare in assenza di autorità. E saranno gli altri bambini a essere danneggiati, signor Potter, se crederanno che il suo sia un esempio da seguire.»

Le labbra di Harry s’incurvarono in un sorriso contorto. «La prima e l’ultima risorsa è la verità. La verità è che non mi sarei dovuto arrabbiare, che non avrei dovuto disturbare la lezione, che non avrei dovuto fare ciò che ho fatto e che ho dato un cattivo esempio a tutti. La verità è anche che Severus Snape si è comportato in modo indegno di un professore di Hogwarts, e che d’ora in poi starà molto più attento a non ferire i sentimenti degli studenti dal quarto anno in giù. Entrambi potremmo alzarci e dire la verità. Potrei accettare questo.»

«Nei tuoi sogni, Potter!» esclamò Severus.

«Dopotutto», disse Harry, sorridendo cupamente, «se gli studenti vedessero che le regole valgono per \textit{tutti…} anche per i professori, non solo per i poveri studenti indifesi che non ricevono altro che sofferenza dal sistema… beh, gli effetti positivi sulla disciplina scolastica sarebbero \textit{straordinari.}»

Ci fu una breve pausa, e poi Silente ridacchiò. «Minerva sta pensando che hai più ragione di quanto avresti il diritto di avere.»

Lo sguardo di Harry corse via da Silente, giù verso il pavimento. «\textit{Lei} sta leggendo la \textit{sua} mente?»

«Il buon senso è spesso confuso con la Legilimanzia», disse Silente. «Discuterò di questo con Severus, e non ti sarà chiesta alcuna scusa a meno che lui non si scusi allo stesso modo. E ora dichiaro chiusa questa faccenda, almeno fino all’ora di pranzo.» Fece una pausa. «Sebbene, Harry, temo che Minerva desiderasse parlarti di un ulteriore affare. E questo non è il risultato di alcuna pressione da parte mia. Minerva, ti dispiace?»

Minerva si alzò dalla sedia e quasi cadde. C’era troppa adrenalina nel suo sangue, il suo cuore stava battendo troppo velocemente.

«Fawkes», disse Silente, «accompagnala, per favore.»

«Io non –» ella iniziò a dire.

Silente la guardò, ed ella si azzittì.

La fenice si librò per la stanza come una lingua di fuoco che scatta d’improvviso, e atterrò sulla sua spalla. Ella sentì il calore attraverso le proprie vesti, espandersi attraverso tutto il suo corpo.

«La prego di seguirmi, signor Potter», disse ora fermamente, e uscirono dalla porta.

\begin{figure}[h!]
        \includegraphics[scale=0.4]{boccino.png}
        \centering
\end{figure}

Erano in piedi sulle scale ruotanti, scendendo in silenzio.

Minerva non sapeva cosa dire. Non conosceva la persona che stava al suo fianco.

E Fawkes iniziò il suo canto.

Era tenero, e soffuso, come avrebbe suonato un caminetto se avesse avuto una melodia, e inondò la mente di Minerva, alleviando, calmando, lenendo ciò che toccava…

«\textit{Che cos’è quello?}» sussurrò Harry al suo fianco. La sua voce era variabile, tremolante, dal tono cangiante.

«La canzone della fenice», rispose Minerva, non realmente cosciente di ciò che diceva, la sua attenzione era tutta per quella musica strana, calma. «Anch’essa guarisce.»

Harry rivolse il volto dall’altra parte, ma ella fu in grado di cogliere un’occhiata di agonia.

La discesa sembrò richiedere un tempo molto lungo, o forse fu solo che la musica sembrò richiedere un tempo molto lungo, e quando uscirono dal varco lasciato dal gargoyle, stava fermamente reggendo la mano di Harry tra le proprie.

Quando il gargoyle tornò al suo posto, Fawkes lasciò la sua spalla, e discese repentinamente per librarsi di fronte a Harry.

Harry fissò Fawkes come qualcuno ipnotizzato dalla luce cangiante di un incendio.

«Cosa devo fare, Fawkes?» sussurrò Harry. «Non avrei potuto proteggerli, se non fossi stato arrabbiato.»

Le ali della fenice continuarono sbattere, continuarono a librarla sul posto. Non c’era alcun suono a parte il battito delle ali. Poi ci fu un lampo come un fuoco che divampa e si spegne, e Fawkes era sparito.

Entrambi sbatterono le palpebre, come svegliandosi da un sogno, o forse come riaddormentandosi nuovamente.

Minerva guardò in giù.

L’intenso e giovane volto di Harry Potter guardò in su verso di lei.

«Le fenici sono persone?» disse Harry. «Voglio dire, sono abbastanza intelligenti per contare come persone? Potrei parlare con Fawkes se sapessi come?»

Minerva sbatté le palpebre. Poi, le sbatté ancora. «No», disse Minerva, la sua voce esitante. «Le fenici sono creature di potente magia. Quella magia dà alla loro esistenza il peso di un significato che un semplice animale non potrebbe possedere. Essi sono il fuoco, la luce, la guarigione, la rinascita. Ma in fin dei conti, no.»

«Dove posso trovarne una?»

Minerva si chinò e lo abbracciò. Non aveva voluto farlo, ma non sembrava avere molta scelta a riguardo.

Quando si alzò trovò difficile parlare. Ma doveva chiederlo. «Che cosa è successo oggi, Harry?»

«Non conosco le risposte a nessuna delle domande importanti. A parte questo, preferirei davvero non pensarci per un po’.»

Minerva gli prese nuovamente la mano tra le proprie, e fecero il resto del percorso in silenzio.

Fu solo un breve viaggio, poiché naturalmente l’ufficio della Vicepreside era vicino all’ufficio del Preside.

Minerva sedette dietro la sua scrivania.

Harry sedette davanti alla scrivania.

«Dunque», sussurrò Minerva. Avrebbe dato qualsiasi cosa per non farlo, o per non essere lei a doverlo fare, o affinché fosse in qualsiasi momento tranne quello. «C’è una questione di disciplina scolastica. Dalla quale lei non è esente.»

«Vale a dire?» disse Harry.

Non lo sapeva. Non l’aveva ancora capito. Si sentì stringere la gola. Ma c’era un lavoro da fare e non si sarebbe sottratta.

«Signor Potter», disse la professoressa McGonagall, «ho bisogno di vedere il suo Giratempo, per favore.»

Tutta la pace della fenice scomparve dal suo volto in un istante e Minerva si sentì come se l’avesse appena pugnalato.

«\textit{No!}» disse Harry. La sua voce era in preda al panico. «Ne ho bisogno, non sarò in grado di frequentare Hogwarts, non sarò in grado di dormire!»

«Sarà in grado di dormire», rispose lei. «Il Ministero ha consegnato il guscio protettivo per il suo Giratempo. Lo incanterò affinché si apra solo tra le 21 e la mezzanotte.»

Il volto di Harry si contorse. «Ma — ma io –»

«Signor Potter, quante volte ha usato il Giratempo da lunedì? Quante ore?»

«Io…» tergiversò Harry. «Un momento, mi faccia contare –» Diede un’occhiata al suo orologio.

Minerva sentì una scarica di tristezza. L’aveva pensato. «Non è stato solo due per giorno, allora. Sospetto che se chiedessi ai suoi compagni di dormitorio, scoprirei che lei ha faticato per restare sveglio sufficientemente a lungo per andare a dormire a un’ora ragionevole, e che si è svegliato sempre più presto la mattina. Dico bene?»

Il volto di Harry disse tutto ciò che ella aveva bisogno di sapere.

«Signor Potter», disse dolcemente, «ci sono studenti ai quali non si possono affidare i Giratempo, perché ne diventano dipendenti. Diamo loro una pozione che allunga il ciclo del sonno della quantità necessaria, ma finiscono con l’usare il Giratempo per qualcosa di più che il semplice frequentare le lezioni. E quindi dobbiamo riprenderceli. Signor Potter, lei ha iniziato a usare il Giratempo come la sua soluzione per tutto, e spesso una soluzione molto stupida. L’ha usato per riprendersi una Ricordella. È sparito da uno stanzino in un modo evidente per gli altri studenti, invece di tornare indietro nel tempo dopo che ne fosse uscito e chiesto che io o qualcun altro venissimo ad aprirle la porta.»

Dal volto di Harry fu chiaro che non aveva pensato a quella possibilità.

«E cosa più importante», continuò, «sarebbe dovuto semplicemente restare seduto durante la lezione del professor Snape. E osservare. E andarsene alla fine della lezione. Come avrebbe fatto se non avesse posseduto un Giratempo. Ci sono studenti ai quali non si possono affidare i Giratempo, signor Potter. Lei è uno di loro. Mi dispiace.»

«Ma ne ho \textit{bisogno!}» sbottò Harry. «E se ci fossero dei Serpeverde che mi minacciano e dovessi scappare? Mi tiene \textit{al sicuro} –»

«Ogni altro studente in questo castello corre lo stesso rischio, e le assicuro che sopravvivono. Nessuno studente è morto in questo castello negli ultimi cinquant’anni. Signor Potter, mi consegnerà il suo Giratempo e lo farà ora.»

Il volto di Harry si contorse per l’agonia, ma tirò fuori il Giratempo da sotto le vesti e glielo diede.

Dalla sua scrivania, Minerva tirò fuori uno dei gusci di protezione che erano stati inviati a Hogwarts. Montò il guscio intorno alla clessidra ruotante del Giratempo, e poi posò la bacchetta sul guscio per completare l’incantesimo.

«\textit{Questo non è giusto!}» strillò Harry. «Ho salvato metà Hogwarts dal professor Snape oggi, è giusto che io sia punito per questo? Ho visto lo sguardo sul suo viso, \textit{odiava} quello che lui stava facendo!»

Minerva non parlò per qualche istante. Stava completando l’incantesimo.

Quando ebbe finito e alzò lo sguardo, sapeva che il suo viso era severo. Forse era la cosa sbagliata da fare. E poi del resto forse era la cosa giusta da fare. C’era un bambino ostinato davanti a lei, e questo non voleva dire che l’universo fosse guasto.

«\textit{Giusto}, signor Potter?» sbottò. «Ho dovuto presentare \textit{due relazioni} al Ministero sull’uso pubblico di un Giratempo in \textit{due giorni consecutivi!} Sia \textit{estremamente} grato se le viene permesso di mantenere il Giratempo anche in forma limitata! Il Preside li ha chiamati per perorare personalmente e se non fosse stato il Ragazzo-Che-È-Sopravvissuto anche questo non sarebbe stato sufficiente!»

Harry la guardò a bocca aperta.

Ella sapeva che egli stava vedendo il volto arrabbiato della professoressa McGonagall.

Gli occhi di Harry si riempirono di lacrime.

«Mi, dispiace», sussurrò, la voce ora soffocata e rotta. «Mi dispiace, di averla, delusa…»

«Sono dispiaciuta anch’io, signor Potter», disse severamente, e gli consegnò il Giratempo appena vincolato. «Può andare.»

Harry si girò e corse via dal suo ufficio, singhiozzando. Sentì i suoi piedi picchiettare via lungo il corridoio, poi il suono terminò quando la porta si chiuse.

«Sono dispiaciuta anch’io, Harry», sussurrò alla stanza silenziosa. «Sono dispiaciuta anch’io.»

\begin{figure}[h!]
        \includegraphics[scale=0.4]{boccino.png}
        \centering
\end{figure}

Quindici minuti dall’inizio dell’ora di pranzo.

Nessuno parlava con Harry. Alcuni dei Corvonero gli indirizzavano sguardi di rabbia, altri di simpatia, alcuni degli studenti più giovani ebbero persino sguardi di ammirazione, ma nessuno gli stava parlando. Anche Hermione non aveva provato ad avvicinarsi.

Fred e George si erano cautamente appropinquati. Non avevano detto niente. L’offerta era chiara, come il suo essere facoltativa. Harry aveva detto loro che sarebbe venuto al momento del dolce, non prima. Avevano annuito e si erano allontanati di fretta.

Era probabilmente l’aspetto del tutto inespressivo del volto di Harry che ci stava riuscendo.

Gli altri probabilmente pensavano che stesse controllando la rabbia, o lo sgomento. Sapevano, perché avevano visto Flitwick venirlo a prendere, che era stato convocato nell’ufficio del Preside.

Harry stava cercando di non sorridere, perché se avesse sorriso, avrebbe iniziato a ridere, e se avesse iniziato a ridere, non si sarebbe fermato fino a quando le simpatiche persone con la divisa bianca non fossero venute a portarlo via.

Era troppo. Era tutto davvero troppo. Harry era quasi passato al Lato Oscuro, il suo lato oscuro aveva fatto cose che sembravano in retrospettiva folli, il suo lato oscuro aveva vinto una vittoria impossibile che sarebbe potuta essere vera e sarebbe potuta essere un puro capriccio di un Preside folle, il suo lato oscuro aveva protetto i suoi amici. Non riusciva più a sopportarlo. Aveva bisogno che Fawkes cantasse di nuovo per lui. Aveva bisogno di usare il Giratempo per procurarsi un’ora di tranquillità per riprendersi, ma quella non era più un’opzione e la perdita era come un buco nella sua esistenza, ma non riusciva a pensarci perché altrimenti si sarebbe potuto mettere a ridere.

Venti minuti. Tutti gli studenti che avevano intenzione di mangiare a pranzo erano arrivati, quasi nessuno se n’era andato.

Il picchiettio di un cucchiaio risuonò attraverso la Sala Grande.

«Se posso avere la vostra attenzione, prego», disse Silente. «Harry Potter ha qualcosa che vuole condividere con noi.»

Harry prese un respiro profondo e si alzò. Si avvicinò al Tavolo d’onore, con tutti gli occhi che lo fissavano.

Si girò e guardò verso le quattro tavolate.

Stava diventando sempre più difficile non sorridere, ma Harry mantenne il suo volto inespressivo mentre ripeteva il suo breve discorso mandato a memoria.

«La verità è sacra», disse Harry impassibile. «Uno dei miei averi più cari è una spilla che recita ‘Di’ la verità, anche se la tua voce trema’. Questa, dunque, è la verità. Ricordatelo. Non sto dicendo questo perché sono stato obbligato a farlo, lo sto dicendo perché è vero. Ciò che ho fatto durante la lezione del professor Snape è stato folle, stupido, infantile e una violazione imperdonabile delle regole di Hogwarts. Ho disturbato la classe e privato i miei compagni del loro insostituibile tempo di apprendimento. Tutto perché non sono stato in grado di controllare il mio temperamento. Spero che non uno di voi segua mai il mio esempio. Certamente io intendo provare a non seguirlo più.»

Molti degli studenti che fissavano Harry avevano ora espressioni solenni e infelici sui loro volti, come quelle che si potrebbero vedere a una cerimonia che segni la fine di un campione sconfitto. Nella zona più giovane della tavola Grifondoro quell’espressione era quasi universale.

Finché Harry non alzò la mano.

Non l’alzò in alto. Sarebbe potuto sembrare pretenzioso. Certamente non la sollevò verso Severus. Semplicemente, Harry alzò la mano al livello del petto, e schioccò dolcemente le dita, un gesto che fu visto più che udito. Era possibile che la maggior parte del Tavolo d’onore non l’avesse visto affatto.

Quell’apparente gesto di ribellione conquistò gli improvvisi sorrisi degli studenti più giovani e dei Grifondoro, e i freddi sogghigni di superiorità dei Serpeverde, e le occhiate accigliate e preoccupate di tutti gli altri.

Harry mantenne il volto impassibile. «Grazie», disse. «Questo è tutto.»

«Grazie, signor Potter», disse il Preside. «E ora il professor Snape ha qualcosa da condividere con noi anche lui.»

Severus si alzò tranquillamente dal suo posto al Tavolo d’onore. «È stato portato alla mia attenzione», disse, «che le mie azioni hanno giocato una parte nel provocare la palesemente imperdonabile rabbia del signor Potter, e nella conseguente discussione ho compreso di aver dimenticato quanto siano facilmente feriti i sentimenti dei giovani e degli immaturi –»

Ci fu il suono di molte persone che emettevano contemporaneamente dei rantoli soffocati.

Severus continuò come se non avesse sentito. «L’aula di Pozioni è un posto pericoloso, e ritengo ancora che una rigida disciplina sia necessaria, ma per l’avvenire sarò più consapevole della… fragilità emotiva… degli studenti del quarto anno e ancor più giovani. La mia sottrazione di punti a Corvonero è ancora valida, ma revoco la detenzione del signor Potter. Grazie.»

Ci fu un singolo battito di mani proveniente dalla direzione dei Grifondoro e, più veloce di un fulmine, la bacchetta di Severus fu nella sua mano e «\textit{Quietus!}» zittì il responsabile.

«Continuerò a pretendere disciplina e rispetto in \textit{tutte} le mie classi», disse freddamente Severus, «e chiunque scherzasse con me se ne pentirà.»

Si sedette.

«Grazie anche a te!» disse allegramente il preside Silente. «Continuate!»

E Harry, ancora impassibile, iniziò a tornare alla propria sedia in Corvonero.

Ci fu un’esplosione di conversazioni. Due parole erano chiaramente identificabili all’inizio. La prima era un «Cosa –» iniziale, con cui cominciavano diverse frasi come «Cos’è successo –» e «Cosa diavolo –» La seconda era «\textit{Scourgify!}», mentre gli studenti ripulivano il cibo caduto e le bevande che essi stessi avevano versato, dalle tovaglie e dalle vesti.

Alcuni studenti stavano apertamente piangendo. Così anche la professoressa Sprout.

Al tavolo Grifondoro, dove una torta attendeva con cinquantuno candeline non ancora accese, Fred sussurrò «Penso che questo sia davvero troppo per noi, George.»

E da quel giorno in poi, non importa cosa Hermione provasse a dire a tutti, fu una leggenda accettata di Hogwarts che Harry Potter potesse far accadere assolutamente qualunque cosa schioccando le dita.



